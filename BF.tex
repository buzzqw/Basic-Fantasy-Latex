\documentclass[a4paper,twoside,openany,10pt]{book}
\usepackage{quoting}
\usepackage{tcolorbox}
\usepackage{tikz}
\usetikzlibrary{shadows}
\usepackage{multicol}
\usepackage{tocloft}
\usepackage{lmodern}
\usepackage{caption}
\usepackage[utf8]{inputenc}
\usepackage[T1]{fontenc} 
\usepackage{setspace}
\usepackage[a4paper]{geometry}
\geometry{verbose,tmargin=2.5cm,bmargin=2.5cm,lmargin=2cm,rmargin=2cm}  %std
\setcounter{secnumdepth}{-1}
\usepackage{booktabs}
\usepackage{url}
\usepackage[english]{babel}
\usepackage{setspace} 
\usepackage{graphicx}
\usepackage{makeidx}
\usepackage[allfiguresdraft]{draftfigure}  %remove the '%' at start line for disabling images
\usepackage{colortbl}
\usepackage{multirow}
\usepackage{titlesec}
\usepackage{bookmark}
\usepackage{ragged2e}
\usepackage{wrapfig}
\usepackage{fancyhdr}
\usepackage{tcolorbox}
\tcbuselibrary{skins}
\tcbset{colback=brown!10, fonttitle=\scshape}
\usepackage{imakeidx}
\usepackage{cancel}
\usepackage{fancyhdr}
\pagestyle{fancy}
\fancyhf{} 
\fancyhead[LE,RO]{\leftmark}
\fancyhead[RE,LO]{}
\fancyfoot[C]{\thepage}
\renewcommand{\sectionmark}[1]{\markboth{#1}{}}
\usepackage{tabularx}
\usepackage{pdfpages}
\usepackage{hyperref}
\usepackage{tikz}
\usepackage[absolute,overlay]{textpos}
\usepackage{etoolbox}
\raggedbottom
\usepackage{array}
\usepackage[framemethod=TikZ]{mdframed}
\makeindex[columns=3, title=Index, intoc=true]
\setcounter{tocdepth}{3}

\setlength{\parskip}{3pt}



\usepackage{fontspec}
\setmainfont[Path=./, BoldItalicFont=Soutane Bold Italic.ttf, ItalicFont=Soutane Italic.ttf, BoldFont=Soutane Bold.ttf, Ligatures=TeX, Scale=0.94]{Soutane Regular.ttf} 


\begin{document}
	
\justifying
\thispagestyle{empty}

\begin{center}
\includegraphics[width=1\textwidth]{Pictures132/100000010000079E000008905C4F2EF6901E81AA.png}
\end{center}

\begin{center}

{{\huge Copyright © 2006-2023 Chris Gonnerman}}\bigskip

{\LARGE All Rights Reserved. See next page for license information}.\bigskip

{\LARGE \href{https://www.basicfantasy.org}{\textbf{www.basicfantasy.org}}}\bigskip
 
\end{center}

\pagebreak

\thispagestyle{empty}

\begin{center}
\textbf{Dedicated to Gary Gygax, Dave Arneson, Tom Moldvay, David Cook, and Stephen Marsh and to my daughter Taylor, my first and best inspiration}
\end{center}

\addvspace{1.5cm}

{\large \textbf{Basic Fantasy Role-Playing Game}}\\

{\large 4th Edition, Release 132}

\textbf{Copyright © 2006-2023 Chris Gonnerman -- All Rights Reserved}
\hfill
\includegraphics{Pictures132/100000000000012C00000067E942582712CE89A7.png}\\

All textual materials in this document, as well as all maps, floorplans, diagrams, charts, and forms included herein, are distributed under the terms of the \textbf{Creative Commons Attribution-ShareAlike 4.0 International License}.
\begin{wrapfigure}{r}{0.145\textwidth}
	\includegraphics{Pictures132/100000010000012C0000006A5E49E94B5B9C7E4A.png}
\end{wrapfigure}

Most other artwork presented is property of the original artist and is used with permission. Note that you may not publish or otherwise distribute this work as is without permission of the original artists; you must remove all non-licensed artwork before doing so.\\

The full license text can be viewed at:  \href{https://creativecommons.org/licenses/by-sa/4.0/} https://creativecommons.org/licenses/by-sa/4.0/\\

Some text used in the development of this document was retrieved from Wikipedia, and was originally licensed under the Creative Commons Attribution-ShareAlike 3.0 license.\\

\textbf{Contributors}: Ray Allen, William D. Smith, Jr., Nick Bogan, Evan Moore, Stuart Marshall, Emiliano Marchetti, Antonio Eleuteri, Luigi Castellani, Michael Hensley, Nazim N. Karaca, Arthur Reyes, Todd Roe, Jim Bobb, R. Kevin Smoot, Rachel Ghoul, Tom Hoyt, James Michael Spahn, Matt Sluis, Russ Westbrook, Martin Serena, Joe Ludlum, Aaron G. Motta, and Gabe Fua\\

\textbf{Cover Art}: Erik Wilson\\

\textbf{Artwork}: Erik Wilson, Steve Zieser, Matt Finch, Dan Dalton, Luigi Castellani, Nick Bogan, Mike Hill, Kevin Cook, Sean Stone, Brian "Glad" Thomas, Tomas Arfert, Andy "ATOM" Taylor, Jason Braun, Martin "Wulfgarn" Siesto, Brian DeClercq, Martin Serena, Cory "Shonuff" Gelnett, Alexander Cook, Bruce Ripple, Gabe Fua, John Simcoe, LuckyCrafts, Jeremy Putnam, Francisco Chavez, William Henry Dvorak, Tony Grant Gittoes, \,Jose Kercado, Jody Claunch, Terance Crosby, S. Ender Thiel, Colin Richards, John Fredericks, Piotr Klimkowicz, Evan Griffiths, Hadrien Riel‑Salvatore, and Benedikt Noir\\

\textbf{Proofreading}: Tonya Allen, Daryl Burns, James Roberts, Serge Petitclerc, Benedict Wolf, Onno Tasler, Peter Cook, Derrick "Omote" Landwehr, Wes Brown, Troy Gravil, Garrett Rooney, K. David Ladage, James Lemon, Martin Serena, Joe Carruthers, Jonathan Nichol, Alister Fa, Joel Davis, Hadrien Riel-Salvatore, Mark Mealman, Alan Vetter, Jim Michnowicz, \,Timothy P. Fox, Piotr Klimkowicz, Daniel Collins, Thorin Schmidt, and feveredmonk\\

\textbf{Playtesters}: Taylor Gonnerman, Alan Jett, Mike Brantner, Steve Zieser, Allan Zieser, Jonathon Foster, 
\begin{wrapfigure}{r}{0.45\textwidth}
	\includegraphics{Pictures132/100000010000039A000002265CD8B96CCD315FCD.png}
\end{wrapfigure}
Adam Young, Michael Young, Jason Schmidt, Doug Wilson, Jessica Abramson, Tonya Allen, Bryan Christian, Chuck Schoonover, Natalie Schoonover, Brianna Schoonover, Jason Brentlinger, Chris Wolfmeyer, Josh Eaton, Audra Brentlinger, Tim McAfee, Ike Borden, Cody Drebenstedt, Joseph BierFauble, Emily Drebenstedt, John Lopez, Pedro Pablo Miron Pozo, Robert Odom, Sergio I. Nemirovsky, Will E. Sanders, Brian Scalise, Timothy J. Kuhn, and Jeanne Mayer Mitchell\\

\textbf{Latex porting}: Andres Zanzani

\pagebreak

\thispagestyle{empty}

\onecolumn

\setcounter{page}{1}

\begin{multicols}{2}

\tableofcontents

\end{multicols}


\pagebreak

\thispagestyle{empty}

\includegraphics[width=1\textwidth]{Pictures132/100000010000079E00000A8CA31D778A0C29ECC8.png}

\pagebreak



\section{PART 1: INTRODUCTION}\label{part-1-introduction}\index{Part 1: Introduction}

\textit{It was our third foray into the dungeons beneath the ancient fortress in the middle of the river. We were on the second level down from the ruins, standing before the great bronze doors beyond which we believed lay the tomb of an ancient barbarian chieftain. I hadn' t believed the tales of the old drunk at the tavern back at Morgansfort, but for some reason Apoqulis, the Cleric, believed him. Turned out his stories were true... mostly, anyway.}\\

\textit{I held a torch for Barthal, the Thief, as he tried briefly to pick the lock. He turned around and said, "It must be held by magic. The lock won' t even wiggle."}\\

\textit{Morningstar, the Elf, smiled. "I have just the thing," she said, drawing from her backpack the scroll we took from the goblins. She unrolled it and began to read, and though I couldn' t understand her words I could see the characters burning away as she read them, little wisps of smoke as from a candle rising up from each in turn. Seeing that she was nearly through, I turned my attention to the lock. I' m not sure what I was expecting, but the little puff of dust that came from it as she finished didn' t seem like much. She turned to Barthal and said, "Try again."}\\

\textit{I' m tempted to say that Barthal bent to his work, but he' s a Halfling; at just over three feet tall he could look straight into the lock without stooping a bit. I must have looked impatient, as Apoqulis leaned over to me and said, "Be still, Darion, he' ll be through in a moment or two."}\\

\textit{Then I heard a loud click, and Barthal turned to me with a smile. "It' s open, my friend. After you!" I handed him the torch, then stepped to the doors, sword drawn, and Morningstar joined me, likewise ready. I steeled myself and opened the doors...}\\

\textit{Beyond lay a stone sarcophagus, resting atop a raised platform. Strewn about the floor were many human skeletons. Apoqulis made a sign with his hand that I didn' t recognize; then we walked in carefully, trying not to trip over the bones. I noticed among the bones several bronze swords, covered in verdigris. I stepped to the sarcophagus. "The lid is likely very heavy," I said. "Come, Morningstar, rather than lift it, let' s turn it about so we can see what treasures lie inside."}\\

\textit{Morningstar called "Wait!" but it was too late... I had already laid hands upon the sarcophagus. The bones on the floor began to rattle, then rose up and assembled themselves in a mockery of life. Without delay they picked up their swords from the floor and began to attack us. I would have to wait until later to kick myself, I mused, as I put my back against the sarcophagus and began to fight the monsters...}\\

\begin{multicols}{2}

\subsection{What is This?}\label{what-is-this}\index{What is This?}

The \textbf{Basic Fantasy Role-Playing Game} is a rules-light game system written with inspiration from early role-playing game systems. It is intended for those who are fans of "old-school" game mechanics. Basic Fantasy RPG is simple enough for children in perhaps second or third
grade to play, yet still has enough depth for adults as well. 

\subsection{What is a Role-Playing Game?}\label{what-is-a-role-playing-game}\index{What is a Role-Playing Game?}

In the almost 50 years since the first role-playing game appeared, much has changed. Most people have at least heard the names of one or two such games, and many, many people have played. Still, there are those who have not tried RPGs; if you are one of those people, this part is for you.\\

Role-playing games are played by a number of players, commonly two to eight, and a Game Master, or GM (often called something else,

\includegraphics[width=0.45\textwidth]{Pictures132/1000000000000384000003EAE022F8A71177AEE1.png}

but the job remains the same regardless of the title). Each player generally plays one character, called a player character or PC, while the Game Master is responsible for running the world, creating and managing the towns, nations, ruins, non-player characters (or NPCs), monsters, treasure, and all other things that aid or challenge the players. Dice are often used to determine the success or failure of most actions that take place in the game; Basic Fantasy RPG uses polyhedral dice, described below, for this purpose.\\

In effect, role-playing games are just grown-up games of pretend. If you remember playing pretend as a child, you may recall having some difficulty deciding whose idea should have precedence... if one child plays a knight and the other a dragon, who will win? Surely the knight doesn' t win every time. Role-playing games have rules to determine such things. These rules can range from the very free-form and simple to the very complex and detailed.

This game attempts to walk the line between simple and complex, free-form and detailed. Too much detail and complexity slows the game down as players and GM spend much time leafing through the rules and little time actually playing. Free-form games with simple resolution systems demand more mental agility from the participants, and are much more dependent on the good judgment of the Game Master to maintain balance. Basic Fantasy Role-Playing Game falls between these two extremes, having rules for the most common activities and guidelines to help the Game Master judge the unexpected.

\subsection{What Do I Need to Play?}\label{what-do-i-need-to-play}\index{What Do I Need to Play?}

If you are to be a player, you should have a pencil, some notebook paper, and a set of dice. Someone in your player group probably needs to have some graph paper (4 or 5 squares per inch is best) for drawing maps. You can use preprinted character sheets (such as those available on the Basic Fantasy RPG website) if you wish, but notebook paper works fine.

If you are the Game Master, you need all of the above. If this is your first time as GM, or you have limited preparation time, you might wish to use a pre-written adventure (called a module) rather than create one yourself. Several modules are distributed for free on the \textbf{basicfantasy.org} website; many of the modules available on the website are specifically designed for use with a party of new players. Adventure modules written for other game systems may also be used, but the Game Master may need to spend some time "converting" such a module before beginning play.

\bigskip

\subsection{Using the Dice}\label{using-the-dice}\index{Using the Dice}

\begin{wrapfigure}{r}{1.77cm}
	\includegraphics[width=1.77cm]{Pictures132/100000000000009600000096BC48CF6CC7B6355F.png}
\end{wrapfigure}
The 20 sided die, or \textbf{d20}, is one of the most important dice in the
game: it is used to resolve attack rolls and saving throws (concepts
that will be explained later).  In general, the die is rolled, modifiers added or subtracted, and if the total result equals or exceeds a target number, the roll is a success; otherwise it has failed.

\begin{wrapfigure}{r}{1.77cm}
\includegraphics[width=1.77cm]{Pictures132/10000001000000960000009681DEC06E30C6F7A5.png}
\end{wrapfigure}
The 10 sided die, or \textbf{d10}, is used to generate numbers from 1 to 10; it is numbered 0 to 9, but a roll of 0 is normally counted as 10. A pair of d10' s are also used together to generate numbers from 1 to 100, where a roll of 00 is counted as 100. The two dice should be different colors, and the player must declare which is the tens die and which is the ones die before rolling them! (Or, the player may have a die marked with double digits, as shown.) Rolling two d10' s in this way is called a \textbf{percentile roll}, or \textbf{d\%}.

\begin{wrapfigure}{r}{1.77cm}
	\includegraphics[width=1.77cm]{Pictures132/100000000000009600000096E8E25ABAD8A98CD1.png}
\end{wrapfigure}
These rolls are generally against target numbers, but for the roll to be a success, the result must be equal to or less than the target number. So for example, a character using a Thief ability (described later) with a 30\% chance of success rolls the dice: if the result is 01 to 30, the roll is a success.
\begin{wrapfigure}{r}{1.77cm}
\includegraphics[width=1.77cm]{Pictures132/100000000000009600000096932760312B100120.png}
\end{wrapfigure}

The 4 sided die, or \textbf{d4}, is a special case. It is not so much rolled as "flipped" and the number which is upright is the result of the roll.
\begin{wrapfigure}{r}{1.77cm}
\includegraphics[width=1.77cm]{Pictures132/100000000000009600000096E4B42447A0D2B4F1.png}	
\end{wrapfigure}
Note that d4' s are made in two different styles, as shown; regardless of which style you have, the number rolled is the one which is upright on all visible sides.

The other dice normally used have 6, 8, and 12 sides, and are called \textbf{d6}, \textbf{d8}, and \textbf{d12}. d6' s may be made with either numbers or pips; it makes no difference which type you choose. 

\includegraphics[width=1.77cm]{Pictures132/1000000000000096000000968F9C401B0CD723D4.png}
\includegraphics[width=1.77cm]{Pictures132/100000000000009600000096CAF533E329238E58.png}
\includegraphics[width=1.77cm]{Pictures132/1000000000000096000000965D55FF3D29269F24.png}

When multiple dice are to be rolled and added together, it' s noted in the text like this: \textbf{2d6} (roll two d6 dice and add them together), or  \textbf{3d4} (roll three d4 dice and add them together). A modifier may be noted as a "plus" value, such as \textbf{2d8+2} (roll two d8 dice and add them together, then add 2 to the total).

\end{multicols}

\pagebreak

\section{PART 2: PLAYER CHARACTERS}\label{part-2-player-characters}\index{Part 2: Player Characters}

\begin{multicols}{2}

\subsection{How to Create a Player}

\label{how-to-create-a-player-character}\index{How to create a player character}

First, you will need a piece of paper to write down the character' s statistics on. You may use a preprinted character sheet if one is available, or you may simply use a piece of notebook paper. An example character is shown below. You should use a pencil to write down all information, as any statistic may change during play.\\

Roll 3d6 for each ability score, as described in the \textbf{Character Abilities} section, and write the results after the names of the abilities. Write down the scores in the order you roll them; if you are unhappy with the scores you have rolled, ask your Game Master for advice, as they may allow some form of point or score exchanging.\\

Write down the ability score bonus (or penalty) for each score beside the score itself, as shown on the table on the next page.\\

Choose a race and class for your character. Your character must meet the Prime Requisite minimum for a class, as described in the \textbf{Character Classes} section on page \hyperlink{character-classes}{\pageref{character-classes}}, in order to be a member of that class. Also note that there are minimum (and maximum) ability requirements for the various races which must be met, as described under \textbf{Character Races} on page \hyperlink{character-races}{\pageref{character-races}}.\\

\begin{flushleft}
	\includegraphics[width=0.45\textwidth]{Pictures132/10000000000003CF0000042C6D899C0463712BAF.png}
\end{flushleft}\medskip

Write down the special abilities of your race and class choices, as described below. If you have chosen to play a Magic-User, ask your Game Master what spell or spells your character knows; it' s up to the Game Master to decide this, but they may allow you to choose one or more spells yourself.\\

Note on your character sheet that your character has zero (0) experience points (or XP); also you may want to note the number needed to advance to second level, as shown in the table for your class.\\

Roll the hit die appropriate for your class, adding your Constitution bonus or penalty, and note the result as your hit points on your character sheet. Note that, should your character have a Constitution penalty, the penalty will not lower any hit die roll below 1 (so if your Character has a -2 penalty for Constitution, and you roll a 2, the total is adjusted to 1).\\

Roll for your starting money. Generally your character will start with 3d6 times 10 gold pieces, but ask the Game Master before rolling.\\

Now, purchase equipment for your character, as shown in the \textbf{Cost of Weapons and Equipment} section on page  \hyperlink{cost-of-weapons-and-equipment}{\pageref{cost-of-weapons-and-equipment}}. Write your purchases on your character sheet, and note how much money remains afterward. Make sure you understand the weapon and armor restrictions for your class and race before making your purchases.\\

Since you now know what sort of armor your character is wearing, you should note your Armor Class on your character sheet. Don' t forget to add your Dexterity bonus or penalty to the figure.\\

Look up your character' s attack bonus (from the table on page \hyperlink{evasion-and-pursuit}{\pageref{evasion-and-pursuit}} of the \textbf{Encounter} section) and note it on your character sheet. Don' t add your ability bonuses (or penalties) to this figure, as you will add a different bonus (Strength or Dexterity) depending on the sort of weapon you use in combat (i.e. melee or missile weapon).\\

Also look up your saving throws (as found on page \hyperlink{saving-throw-tables-by-class}{\pageref{saving-throw-tables-by-class}}) and note them on your character sheet. Adjust the saving throw figures based on your race, if your character is not a Human (again, see \textbf{Character Races} on page \hyperlink{character-races}{\pageref{character-races}}). Please note that the saving throw bonuses for other races are presented as "plus" values, to be added to the die roll; for convenience, you may simply subtract them from the saving throw numbers on the character sheet instead. \\

Finally, if you haven' t done so already, name your character. This often takes longer than all the other steps combined.

\subsection{Character Abilities}\label{character-abilities}\index{Character Abilities}

Each character will have a score ranging from 3 to 18 in each of the following abilities. A bonus or penalty is associated with each score, as shown on the table below. Each class has a \textbf{Prime Requisite} ability score, which must be at least 9 in order for the character to become a member of that class; also, there are required minimum and maximum scores for each character race other than Humans, as described under \textbf{Character Races} on page \hyperlink{character-races}{\pageref{character-races}}.\\

\begin{tabular*}{0.93\linewidth}{@{\extracolsep{\fill}}ll}
\textbf{Ability Score} & \textbf{Bonus/Penalty} \\\toprule
3 & -3 \\\hline
4-5 & -2 \\\hline
6-8 & -1 \\\hline
9-12 & 0 \\\hline
13-15 & +1 \\\hline
16-17 & +2 \\\hline
18 & +3 \\\bottomrule
\end{tabular*}\\

\textbf{Strength:} As the name implies, this ability measures the character' s raw physical power. Strength is the Prime Requisite for Fighters. Apply the ability bonus or penalty for Strength to all attack and damage rolls in melee (hand to hand) combat. Note that a penalty here will not reduce damage from a successful attack below one point in any case (see \textbf{How to Attack} on page \hyperlink{how-to-attack}{\pageref{how-to-attack}} and \textbf{Damage} on page \hyperlink{damage}{\pageref{damage}}, both in the \textbf{Combat} section,
for details).

\textbf{Intelligence:} This is the ability to learn and apply knowledge. Intelligence is the Prime Requisite for Magic-Users. The ability bonus for Intelligence is added to the number of languages the character is able to learn to read and write; if the character has an Intelligence penalty, they cannot read more than a word or two, and will only know their native language.

\textbf{Wisdom: } A combination of intuition, willpower and common sense. Wisdom is the Prime Requisite for Clerics. The Wisdom bonus or penalty may apply to some saving throws vs. magical attacks, particularly those affecting the target' s will.

\textbf{Dexterity: } This ability measures the character' s quickness and balance as well as aptitude with tools. Dexterity is the Prime Requisite for Thieves. The Dexterity bonus or penalty is applied to all attack rolls with missile (ranged) weapons, to the character' s Armor Class value, and to
the character' s Initiative die roll.

\textbf{Constitution: } A combination of general health and vitality. Apply the Constitution bonus or penalty to each hit die rolled by the character. Note that a penalty here will not reduce any hit die roll to less than 1 point.

\textbf{Charisma: } This is the ability to influence or even lead people; those with high Charisma are well-liked, or at least highly respected. Apply the Charisma bonus or penalty to reaction rolls. Also, the number of retainers a character may hire, and the loyalty of those retainers, is affected by Charisma.

\subsection{Hit Points and Hit Dice}\label{hit-points-and-hit-dice}\index{Hit Points and Hit Dice}

When a character is injured, they lose hit points from their current total. Note that this does not change the figure rolled, but rather reduces the current total; healing will restore hit points, up to but not exceeding the rolled figure.

When their hit point total reaches 0, your character may be dead. This may not be the end for the character; don' t tear up the character sheet.

First level characters begin play with a single hit die of the given type, plus the Constitution bonus or penalty, with a minimum of 1 hit point. Each time a character gains a level, the player should roll another hit die and add the character' s Constitution bonus or penalty, with the result again being a minimum of 1 point. Add this amount to the character' s maximum hit points figure. Note that, after 9th level, characters receive a fixed number of hit points each level, as shown in the advancement table for the class, and no longer add the Constitution bonus or penalty.

\subsection{Languages}\label{languages}\index{Languages}

All characters begin the game knowing their native language. In most campaign worlds, Humans all (or nearly all) speak the same language, often called "Common." Each non-Human race has its own language, i.e. Elvish, Dwarvish, or Halfling, and members of these races begin play knowing both their own language and Common (or the local Human language if it isn' t called Common).

Characters with Intelligence of 13 or higher may choose to begin the game knowing one or more languages other than those given above; the number of additional languages that may be learned is equal to the Intelligence bonus (+1, +2, or +3). Characters may choose to learn other non-Human languages, including those of creatures such as Orcs, Goblins, etc. The GM will decide which languages may be learned. The player may choose to leave one or more bonus language "slots" open, to be filled during play. Some Game Masters may even allow player characters to learn exotic languages such as Dragon; also, "dead" or otherwise archaic languages might be allowed to more scholarly characters.

\end{multicols}

\pagebreak


\begin{center}
	\subsection{Character Races}
\end{center}\label{character-races}\hypertarget{character-races}{}\index{Character Races}

\begin{multicols}{2}
	

\subsection{Dwarves}\label{dwarves}\index{Dwarves}

\begin{center}
	\includegraphics[width=0.45\textwidth]{Pictures132/100000000000027600000384496EA2F62B886958.png}
\end{center}

\textbf{Description: } Dwarves are a short, stocky race; both male and female Dwarves stand around four feet tall and typically weigh around 120 pounds. Their long hair and thick beards are dark brown, gray or black. They take great pride in their beards, sometimes braiding or forking them. They have a fair to ruddy complexion. Dwarves have stout frames and a strong, muscular build. They are rugged and resilient, with the capacity to endure great hardships. Dwarves are typically practical, stubborn and courageous. They can also be introspective, suspicious and possessive. They have a lifespan of three to four centuries.

\textbf{Restrictions: } Dwarves may become Clerics, Fighters, or Thieves. They are required to have a minimum Constitution of 9. Due to their generally dour dispositions, they may not have a Charisma higher than 17. They may not employ Large weapons more than four feet in length (specifically, two-handed swords, polearms, and longbows).

\textbf{Special Abilities:} All Dwarves have Darkvision (see page \hyperlink{darkvision}{\pageref{darkvision}}) with a 60'{} range, and are able to detect slanting passages, stonework traps, shifting walls and new construction on a roll of 1-2 on 1d6; a search must be performed before this roll may be made.

\textbf{Saving Throws: } Dwarves save at +4 vs. Death Ray or Poison, Magic Wands, Paralysis or Petrify, and Spells, and at +3 vs. Dragon Breath.

\columnbreak

\subsection{Elves}\label{elves}\index{Elves}

\begin{center}
	\includegraphics[width=0.47\textwidth]{Pictures132/100000000000030C0000041A48B8B2F7EE8F7CE3.png}
\end{center}

\textbf{Description:}
Elves are a slender race, with both males and females standing around five feet tall and weighing around 130 pounds. Most have dark hair, with little or no body or facial hair. Their skin is pale, and they have pointed ears and delicate features. Elves are lithe and graceful. They have keen eyesight and hearing. Elves are typically inquisitive, passionate, self-assured, and sometimes haughty. Their typical lifespan is a dozen centuries or more.

\textbf{Restrictions: } Elves may become Clerics, Fighters, Magic-Users or Thieves; they are also allowed to combine the classes of Fighter and Magic-User, and of Magic-User and Thief (see Combination Classes, below). They are required to have a minimum Intelligence of 9. Due to their generally delicate nature, they may not have a Constitution higher than 17. Elves never roll larger than six-sided dice (d6) for hit points.

\textbf{Special Abilities: }All Elves have Darkvision (see page \hyperlink{darkvision}{\pageref{darkvision}}) with a 60'{} range. They are able to find secret doors more often than normal (1-2 on 1d6 rather than the usual 1 on 1d6). An Elf is so observant that one has a 1 on 1d6 chance to find a secret door with a cursory look. Elves are immune to the paralyzing attack of ghouls. Also, they are less likely to be surprised in combat, reducing the chance of surprise by 1 in 1d6.

\textbf{Saving Throws: } Elves save at +1 vs. Paralysis or Petrify, and
+2 vs. Magic Wands and Spells.

\subsection{Halflings}\label{halflings}\index{Halflings}

\begin{center}
	\includegraphics[width=0.35\textwidth]{Pictures132/100000000000021C000003844CF40402A90AE745.png}
\end{center}\medskip

\textbf{Description:}
Halflings are small, slightly stocky folk who stand around three feet tall and weigh about 60 pounds. They have curly brown hair on their heads and feet, but rarely have facial hair. They are usually fair skinned, often with ruddy cheeks. Halflings are remarkably rugged for their small size. They are dexterous and nimble, capable of moving quietly and remaining very still. They usually go barefoot. Halflings are typically outgoing, unassuming and good-natured. They live about a hundred years.

\textbf{Restrictions: } Halflings may become Clerics, Fighters or Thieves. They are required to have a minimum Dexterity of 9. Due to their small stature, they may not have a Strength higher than 17. Halflings never roll larger than six-sided dice (d6) for hit points regardless of class. Halflings may not use Large weapons, and must wield Medium weapons with both hands.

\textbf{Special Abilities:} Halflings are unusually accurate with all sorts of ranged weapons, gaining a +1 attack bonus when employing them. When attacked in melee by creatures larger than man-sized, Halflings gain a +2 bonus to their Armor Class. Halflings are quick-witted, adding +1 to Initiative die rolls. In their preferred forest terrain, they are able to hide very effectively; so long as they remain still there is only a 10\% chance they will be detected. Even indoors, in dungeons or in non-preferred terrain they are able to hide such that there is only a 30\% chance of detection. Note that a Halfling Thief will roll only once, using either the Thief ability or the Halfling ability, whichever is better.

\textbf{Saving Throws:} Halflings save at +4 vs. Death Ray or Poison, Magic Wands, Paralysis or Petrify, and Spells, and at +3 vs. Dragon Breath.

\subsection{Humans}\label{humans}\index{Humans}

\begin{center}
	\includegraphics[width=0.35\textwidth]{Pictures132/100000000000023A00000384E68DA27234952A4A.png}
\end{center}\medskip

\textbf{Description:} Humans come in a broad variety of shapes and sizes; the Game Master must decide what sorts of Humans live in the game world. An average Human male in good health stands around six feet tall and weighs about 175 pounds, while females average five feet nine inches and weigh around 145 pounds. Most Humans live around 75 years.

Restrictions: Humans may be any single class. They have no minimum or maximum ability score requirements.

\textbf{Special Abilities:} Humans learn unusually quickly, gaining a bonus of 10\% to all experience points earned.

Saving Throws: Humans are the "standard," and thus have no saving throw bonuses.


\subsection{Combination Classes}
\label{combination-classes}\index{Combination Classes}

To become a member of a combination class, a character must meet the requirements of both classes. Combination class characters use the best attack bonus and the best saving throw values of their original two classes, but must gain experience equal to the combined requirements of both base classes to advance in levels. Elves are the only characters eligible to be a member of one of these combination classes:

\textbf{Fighter/Magic-User:} These characters may both fight and cast magic spells; further, they are allowed to cast magic spells while wearing armor. These characters roll six-sided dice (d6) for hit points.

\textbf{Magic-User/Thief:} Members of this combination class may cast spells while wearing leather armor. These characters roll four-sided dice (d4) for hit points.

\end{multicols}

\pagebreak

\subsection{Character Classes}\label{character-classes}\hypertarget{character-classes}{}  \index{Character Classes}

\begin{multicols}{2}

\subsection{Cleric}\label{cleric}\index{Cleric}

\begin{tabular*}{0.93\linewidth}{@{\extracolsep{\fill}}lllllllll}

	& \textbf{Exp}. & & \multicolumn{6}{c}{\textbf{Spells}} \\
	\textbf{LV} & \textbf{Points} & \textbf{HD} & \textbf{1} & \textbf{2} & \textbf{3} & \textbf{4} & \textbf{5} & \textbf{6} \\\toprule
	1 & 0 & 1d6 & - & - & - & - & - & - \\\hline
	2 & 1,500 & 2d6 & 1 & - & - & - & - & - \\\hline
	3 & 3,000 & 3d6 & 2 & - & - & - & - & - \\\hline
	4 & 6,000 & 4d6 & 2 & 1 & - & - & - & - \\\hline
	5 & 12,000 & 5d6 & 2 & 2 & - & - & - & - \\\hline
	6 & 24,000 & 6d6 & 2 & 2 & 1 & - & - & - \\\hline
	7 & 48,000 & 7d6 & 3 & 2 & 2 & - & - & - \\\hline
	8 & 90,000 & 8d6 & 3 & 2 & 2 & 1 & - & - \\\hline
	9 & 180,000 & 9d6 & 3 & 3 & 2 & 2 & - & - \\\hline
	10 & 270,000 & 9d6+1 & 3 & 3 & 2 & 2 & 1 & - \\\hline
	11 & 360,000 & 9d6+2 & 4 & 3 & 3 & 2 & 2 & - \\\hline
	12 & 450,000 & 9d6+3 & 4 & 4 & 3 & 2 & 2 & 1 \\\hline
	13 & 540,000 & 9d6+4 & 4 & 4 & 3 & 3 & 2 & 2 \\\hline
	14 & 630,000 & 9d6+5 & 4 & 4 & 4 & 3 & 2 & 2 \\\hline
	15 & 720,000 & 9d6+6 & 4 & 4 & 4 & 3 & 3 & 2 \\\hline
	16 & 810,000 & 9d6+7 & 5 & 4 & 4 & 3 & 3 & 2 \\\hline
	17 & 900,000 & 9d6+8 & 5 & 5 & 4 & 3 & 3 & 2 \\\hline
	18 & 990,000 & 9d6+9 & 5 & 5 & 4 & 4 & 3 & 3 \\\hline
	19 & 1,080,000 & 9d6+10 & 6 & 5 & 4 & 4 & 3 & 3 \\\hline
	20 & 1,170,000 & 9d6+11 & 6 & 5 & 5 & 4 & 3 & 3 \\\bottomrule
\end{tabular*}\medskip


Clerics are those who have devoted themselves to the service of a deity, pantheon or other belief system. Most Clerics spend their time in mundane forms of service such as preaching and ministering in a temple; but there are those who are called to go abroad from the temple and serve their deity in a more direct way, smiting undead monsters and aiding in the battle against evil and chaos. Player character Clerics are assumed to be among the latter group.\\

Clerics fight about as well as Thieves, but not as well as Fighters. They are hardier than Thieves, at least at lower levels, as they are accustomed to physical labor that the Thief would deftly avoid. Clerics can cast spells of divine nature starting at 2\textsuperscript{nd} level, and they have the power to Turn the Undead, that is, to drive away undead monsters by means of faith alone (refer to page \hyperlink{turning-the-undead}{\pageref{turning-the-undead}} in the \textbf{Encounter} section for details).\\

The Prime Requisite for Clerics is Wisdom; a character must have a Wisdom score of 9 or higher to become a Cleric. They may wear any armor, but may only use blunt weapons (specifically including warhammer, mace, maul, club, quarterstaff, and sling).

\subsection{Fighter}\label{fighter}\index{Fighter}

\begin{tabular*}{0.93\linewidth}{@{\extracolsep{\fill}}lll}
& \textbf{Exp.} & \\
\textbf{Level} & \textbf{Points} & \textbf{Hit Dice} \\\toprule
1 & 0 & 1d8 \\\hline
2 & 2,000 & 2d8 \\\hline
3 & 4,000 & 3d8 \\\hline
4 & 8,000 & 4d8 \\\hline
5 & 16,000 & 5d8 \\\hline
6 & 32,000 & 6d8 \\\hline
7 & 64,000 & 7d8 \\\hline
8 & 120,000 & 8d8 \\\hline
9 & 240,000 & 9d8 \\\hline
10 & 360,000 & 9d8+2 \\\hline
11 & 480,000 & 9d8+4 \\\hline
12 & 600,000 & 9d8+6 \\\hline
13 & 720,000 & 9d8+8 \\\hline
14 & 840,000 & 9d8+10 \\\hline
15 & 960,000 & 9d8+12 \\\hline
16 & 1,080,000 & 9d8+14 \\\hline
17 & 1,200,000 & 9d8+16 \\\hline
18 & 1,320,000 & 9d8+18 \\\hline
19 & 1,440,000 & 9d8+20 \\\hline
20 & 1,560,000 & 9d8+22 \\\bottomrule
\end{tabular*}\medskip


Fighters include soldiers, guardsmen, barbarian warriors, and anyone else for whom fighting is a way of life. They train in combat, and they generally approach problems head-on, weapon in hand.\\

Not surprisingly, Fighters are the best at fighting of all the classes. They are also the hardiest, able to take more punishment than any other class. Although they are not skilled in the ways of magic, Fighters can nonetheless use many magic items, including but not limited to magical weapons and armor.\\

The Prime Requisite for Fighters is Strength; a character must have a Strength score of 9 or higher to become a Fighter. Members of this class may wear any armor and use any weapon.

\addvspace{0.8cm}

\includegraphics[width=0.45\textwidth]{Pictures132/10000000000003CF000000EAA383880B1FF0D466.png}


\subsection{Magic-User}\label{magic-user}\index{Magic-User}

\begin{tabular*}{0.93\linewidth}{@{\extracolsep{\fill}}lllllllll}
& \textbf{Exp}. & & \multicolumn{6}{c}{\textbf{Spells}} \\
\textbf{LV} & \textbf{Points} & \textbf{HD} & \textbf{1} & \textbf{2} & \textbf{3} & \textbf{4} & \textbf{5} & \textbf{6} \\\toprule
1 & 0 & 1d4 & 1 & - & - & - & - & - \\\hline
2 & 2,500 & 2d4 & 2 & - & - & - & - & - \\\hline
3 & 5,000 & 3d4 & 2 & 1 & - & - & - & - \\\hline
4 & 10,000 & 4d4 & 2 & 2 & - & - & - & - \\\hline
5 & 20,000 & 5d4 & 2 & 2 & 1 & - & - & - \\\hline
6 & 40,000 & 6d4 & 3 & 2 & 2 & - & - & - \\\hline
7 & 80,000 & 7d4 & 3 & 2 & 2 & 1 & - & - \\\hline
8 & 150,000 & 8d4 & 3 & 3 & 2 & 2 & - & - \\\hline
9 & 300,000 & 9d4 & 3 & 3 & 2 & 2 & 1 & - \\\hline
10 & 450,000 & 9d4+1 & 4 & 3 & 3 & 2 & 2 & - \\\hline
11 & 600,000 & 9d4+2 & 4 & 4 & 3 & 2 & 2 & 1 \\\hline
12 & 750,000 & 9d4+3 & 4 & 4 & 3 & 3 & 2 & 2 \\\hline
13 & 900,000 & 9d4+4 & 4 & 4 & 4 & 3 & 2 & 2 \\\hline
14 & 1,050,000 & 9d4+5 & 4 & 4 & 4 & 3 & 3 & 2 \\\hline
15 & 1,200,000 & 9d4+6 & 5 & 4 & 4 & 3 & 3 & 2 \\\hline
16 & 1,350,000 & 9d4+7 & 5 & 5 & 4 & 3 & 3 & 2 \\\hline
17 & 1,500,000 & 9d4+8 & 5 & 5 & 4 & 4 & 3 & 3 \\\hline
18 & 1,650,000 & 9d4+9 & 6 & 5 & 4 & 4 & 3 & 3 \\\hline
19 & 1,800,000 & 9d4+10 & 6 & 5 & 5 & 4 & 3 & 3 \\\hline
20 & 1,950,000 & 9d4+11 & 6 & 5 & 5 & 4 & 4 & 3 \\\bottomrule
\end{tabular*}\medskip

Magic-Users are those who seek and use knowledge of the arcane. They do magic not as the Cleric does, by faith in a greater power, but rather through insight and understanding.\\

Magic-Users are the worst of all the classes at fighting; hours spent studying massive tomes of magic do not lead a character to become strong or adept with weapons. They are the least hardy, equal to Thieves at lower levels but quickly falling behind.\\

The Prime Requisite for Magic-Users is Intelligence; a character must have an Intelligence score of 9 or higher to become a Magic-User. The only weapons they become proficient with are the dagger and the walking staff (or cudgel). Magic-Users may not wear armor of any sort nor use a shield as such things interfere with spellcasting.\\

A first level Magic-User begins play knowing \textbf{read magic }and one other spell of first level. These spells are written in a spellbook provided by their master. The GM may roll for the spell, assign it as they see fit, or allow the player to choose it, at their option. See the \textbf{Spells} section for more details.\\

\columnbreak

\subsection{Thief}\label{thief}\index{Thief}

\addvspace{0.5cm}

\begin{tabular*}{0.93\linewidth}{@{\extracolsep{\fill}}lll}
& \textbf{Exp}. & \\
\textbf{Level} & \textbf{Points} & \textbf{Hit Dice} \\\toprule
1 & 0 & 1d4 \\\hline
2 & 1,250 & 2d4 \\\hline
3 & 2,500 & 3d4 \\\hline
4 & 5,000 & 4d4 \\\hline
5 & 10,000 & 5d4 \\\hline
6 & 20,000 & 6d4 \\\hline
7 & 40,000 & 7d4 \\\hline
8 & 75,000 & 8d4 \\\hline
9 & 150,000 & 9d4 \\\hline
10 & 225,000 & 9d4+2 \\\hline
11 & 300,000 & 9d4+4 \\\hline
12 & 375,000 & 9d4+6 \\\hline
13 & 450,000 & 9d4+8 \\\hline
14 & 525,000 & 9d4+10 \\\hline
15 & 600,000 & 9d4+12 \\\hline
16 & 675,000 & 9d4+14 \\\hline
17 & 750,000 & 9d4+16 \\\hline
18 & 825,000 & 9d4+18 \\\hline
19 & 900,000 & 9d4+20 \\\hline
20 & 975,000 & 9d4+22 \\\bottomrule
\end{tabular*}\medskip

Thieves are those who take what they want or need by stealth, disarming traps and picking locks to get to the gold they crave; or "borrowing" money from pockets, beltpouches, etc. right under the nose of the "mark" without the victim ever knowing.\\

Thieves fight better than Magic-Users but not as well as Fighters. Avoidance of honest work leads Thieves to be less hardy than the other classes, though they do pull ahead of the Magic-Users at higher levels.\\

The Prime Requisite for Thieves is Dexterity; a character must have a Dexterity score of 9 or higher to become a Thief. They may use any weapon, but may not wear metal armor as it interferes with stealthy activities, nor may they use shields of any sort. Leather armor is acceptable, however.\\

Thieves have a number of special abilities, described below. One turn (ten minutes) must usually be spent to use any of these abilities, as determined by the GM. The GM may choose to make any of these rolls on behalf of the player to help maintain the proper state of uncertainty. Also note that the GM may apply situational adjustments (plus or minus percentage points) as they see fit; for instance, it' s obviously harder to climb a wall slick with slime than one that is dry, so the GM might apply a penalty of 20\% for the slimy wall.


\end{multicols}

\pagebreak

\subsection{Thief Abilities}\label{thief-abilities}\index{Thief Abilities}

\begin{tabularx}{0.95\textwidth}{@{}XXXXXXXX@{}}


%\multirow{2}{*}\textbf{Thief} & \textbf{Open} & \textbf{Remove} & \textbf{Pick} & \textbf{Move}& \textbf{Climb}s & & \\
%\textbf{Level}&\textbf{Locks}&\textbf{Traps}&\textbf{Pockets}&\textbf{Silently}&\textbf{Walls}&\textbf{Hide}&\textbf{Listen}\\

\textbf{Thief Level} & \textbf{Open Locks} & \textbf{Remove Traps } & \textbf{Pick Pockets} & \textbf{Move Silently}& \textbf{Climb Walls}s &Hide & Listen \\\toprule
1 & 25 & 20 & 30 & 25 & 80 & 10 & 30 \\\hline
2 & 30 & 25 & 35 & 30 & 81 & 15 & 34 \\\hline
3 & 35 & 30 & 40 & 35 & 82 & 20 & 38 \\\hline
4 & 40 & 35 & 45 & 40 & 83 & 25 & 42 \\\hline
5 & 45 & 40 & 50 & 45 & 84 & 30 & 46 \\\hline
6 & 50 & 45 & 55 & 50 & 85 & 35 & 50 \\\hline
7 & 55 & 50 & 60 & 55 & 86 & 40 & 54 \\\hline
8 & 60 & 55 & 65 & 60 & 87 & 45 & 58 \\\hline
9 & 65 & 60 & 70 & 65 & 88 & 50 & 62 \\\hline
10 & 68 & 63 & 74 & 68 & 89 & 53 & 65 \\\hline
11 & 71 & 66 & 78 & 71 & 90 & 56 & 68 \\\hline
12 & 74 & 69 & 82 & 74 & 91 & 59 & 71 \\\hline
13 & 77 & 72 & 86 & 77 & 92 & 62 & 74 \\\hline
14 & 80 & 75 & 90 & 80 & 93 & 65 & 77 \\\hline
15 & 83 & 78 & 94 & 83 & 94 & 68 & 80 \\\hline
16 & 84 & 79 & 95 & 85 & 95 & 69 & 83 \\\hline
17 & 85 & 80 & 96 & 87 & 96 & 70 & 86 \\\hline
18 & 86 & 81 & 97 & 89 & 97 & 71 & 89 \\\hline
19 & 87 & 82 & 98 & 91 & 98 & 72 & 92 \\\hline
20 & 88 & 83 & 99 & 93 & 99 & 73 & 95 \\\bottomrule
\end{tabularx}

\begin{multicols}{2}
	


The numbers above are percentages; instructions for making these rolls are in \textbf{Using the Dice} on page \hyperlink{using-the-dice}{\pageref{using-the-dice}}.

\textbf{Open Locks }allows the Thief to unlock a lock without a proper key. It may only be tried once per lock. If the attempt fails, the Thief must wait until they have gained another level of experience before trying again.

\textbf{Remove Traps }is generally rolled twice: first to detect the trap, and second to disarm it. The GM will make these rolls as the player won' t know for sure if the character is successful or not until someone actually tests the trapped (or suspected) area.

\textbf{Pick Pockets }allows the Thief to lift the wallet, cut the purse, etc. of a victim without being noticed. If the roll fails, the Thief didn' t get what they wanted; but further, the intended victim (or an onlooker, at the GM' s option) will notice the attempt if the die roll is more than two times the target number (or if the die roll is 00).

\textbf{Move Silently}, like Remove Traps, is always rolled by the GM. The Thief will usually believe they are moving silently regardless of the die roll, but opponents they are trying to avoid will hear the Thief if the roll is failed.

\textbf{Climb Walls} permits the Thief to climb sheer surfaces with few or no visible handholds. This ability should normally be rolled by the player. If the roll fails, the Thief falls from about halfway up the wall or other vertical surface. The GM may require multiple rolls if the distance climbed is more than 100 feet. See \textbf{Falling Damage }on page \hyperlink{falling-damage}{\pageref{falling-damage}} for the consequences of failing this roll.

\textbf{Hide} permits the Thief to hide in any shadowed area large enough to contain their body. Like Move Silently, the Thief always believes they are being successful, so the GM makes the roll. A Thief hiding in shadows must remain still for this ability to work.

\textbf{Listen} is generally used to listen at a door, or to try to listen for distant sounds in a dungeon. The GM must decide what noises the Thief might hear; a successful roll means only that a noise could have been heard. The GM should always make this roll for the player. Note that the Thief and their party must try to be quiet in order for the Thief to use this ability.

Finally, Thieves can perform a \hypertarget{Sneakux20Attackux20Thiefux20Ability}{}{}\textbf{Sneak
Attack} any time they are behind an opponent in melee and it is likely the opponent doesn' t know the Thief is there. The GM may require a Move Silently or Hide roll to determine this. The Sneak Attack is made with a +4 attack bonus and does double damage if it is successful. A Thief usually can' t make a Sneak Attack on the same opponent twice in any given combat.

The Sneak Attack can be performed with any melee (but not missile) weapon, or may be performed bare-handed (in which case \textbf{subduing damage} is done, as explained on page \hyperlink{subduing-damage}{\pageref{subduing-damage}}). Also, the Sneak Attack can be performed with the "flat of the blade;" the bonuses and penalties cancel out, so the attack has a +0 attack bonus and does normal damage; the damage done in this case is subduing damage.

\end{multicols}

\pagebreak

\subsection{Cost of Weapons and Equipment}

\label{cost-of-weapons-and-equipment}\hypertarget{cost-of-weapons-and-equipment}{} \index{Cost of Weapons and Equipment}

\begin{multicols}{2}
	
\subsection{Money}\label{money}\index{Money}

Monetary values are usually expressed in gold pieces. In addition to gold coins, there are coins made of platinum, silver, electrum (an alloy of gold and silver), and copper. They are valued as follows:

1 platinum piece (pp) = 5 gold pieces (gp)

1 gold piece (gp) = 10 silver pieces (sp)

1 electrum piece (ep) = 5 silver pieces (sp)

1 silver piece (sp) = 10 copper pieces (cp)

For game purposes, assume that one gold piece weighs 1/20th of a pound, and that ten coins will "fit" in a cubic inch of storage space (this isn' t literally accurate, but works well enough when applied to a box or chest).

First level characters generally begin the game with 3d6 x 10 gp, though the GM may choose some other amount.

\subsection{Equipment}\label{equipment}\index{Equipment}

This list represents common adventuring equipment at average prices. Prices and availability may vary. Weights are expressed in pounds. Items marked * weigh very little; ten such items weigh one pound. Items marked ** have almost no weight and should not usually be counted.\\

\begin{tabular*}{0.93\linewidth}{@{\extracolsep{\fill}}lll}
\textbf{Item} & \textbf{Price} & \textbf{Weight} \\\toprule
Backpack (Standard or Halfling) & 4 gp & * \\\hline
Belt Pouch & 1 gp & * \\\hline
Bit and bridle & 15 sp & 3 \\\hline
Candles, 12 & 1 gp & * \\\hline
Chalk, small bag of pieces & 2 gp & * \\\hline
Cloak & 2 gp & 1 \\\hline
Clothing, common outfit & 4 gp & 1 \\\hline
Glass bottle or vial & 1 gp & * \\\hline
Grappling Hook & 2 gp & 4 \\\hline
Holy Symbol & 25 gp & * \\\hline
Holy Water, per vial & 10 gp & * \\\hline
Horseshoes \& shoeing & 1 gp & 10 \\\hline
Ink, per jar & 8 gp & ½ \\\hline
Iron Spikes, 12 & 1 gp & 1 \\\hline
Ladder, 10 ft. & 1 gp & 20 \\\hline
Lantern & 5 gp & 2 \\\hline
Lantern, Bullseye & 14 gp & 3 \\\hline
Lantern, Hooded & 8 gp & 2 \\\hline
Manacles (without padlock) & 6 gp & 4 \\\hline
Map or scroll case & 1 gp & ½ \\\hline
Mirror, small metal & 7 gp & * \\\hline
Oil (per flask) & 1 gp & 1 \\\hline
Padlock (with 2 keys) & 12 gp & 1 \\\bottomrule
\end{tabular*}

\includegraphics[width=0.45\textwidth]{Pictures132/1000000000000358000004B00F13D4D9AB1079B8.png}
\vfill

\begin{tabular*}{0.93\linewidth}{@{\extracolsep{\fill}}lll}
\textbf{Item} & \textbf{Price} & \textbf{Weight} \\\toprule
Paper (per sheet) & 1 gp & ** \\\hline
Pole, 10'{} wooden & 1 gp & 10 \\\hline
Quill & 1 sp & ** \\\hline
Quill Knife & 1 gp & * \\\hline
Quiver or Bolt case & 1 gp & 1 \\\hline
Rations, Dry, one week & 10 gp & 14 \\\hline
Rope, Hemp (per 50 ft.) & 1 gp & 5 \\\hline
Rope, Silk (per 50 ft.) & 10 gp & 2 \\\hline
Sack, Large & 1 gp & * \\\hline
Sack, Small & 5 sp & * \\\hline
Saddle, Pack & 5 gp & 15 \\\hline
Saddle, Riding & 10 gp & 35 \\\hline
Saddlebags, pair & 4 gp & 7 \\\hline
Spellbook (128 pages) & 25 gp & 1 \\\hline
Tent, Large (ten men) & 25 gp & 20 \\\hline
Tent, Small (one man) & 5 gp & 10 \\\hline
Thieves'{} picks and tools & 25 gp & 1 \\\hline
Tinderbox, flint and steel & 3 gp & 1 \\\hline
Torches, 6 & 1 gp & 1 \\\hline
Whetstone & 1 gp & 1 \\\hline
Whistle & 1 gp & ** \\\hline
Wineskin/Waterskin & 1 gp & 2 \\\hline
Winter blanket & 1 gp & 3 \\\bottomrule
\end{tabular*}

\subsection{Weapons}\label{weapons}\index{Weapons}

\begin{tabular*}{0.93\linewidth}{@{\extracolsep{\fill}}lllll}

\textbf{Weapon} & \textbf{Price} & \textbf{Size} & \textbf{W.} & \textbf{Dmg}. \\\toprule
\textbf{Axes} & & & & \\\hline
Hand Axe & 4 gp & S & 5 & 1d6 \\\hline
Battle Axe & 7 gp & M & 7 & 1d8 \\\hline
Great Axe & 14 gp & L & 15 & 1d10 \\\hline
\textbf{Bows} & & & & \\\hline
Shortbow & 25 gp & M & 2 & \\\hline
Shortbow Arrow & 1 sp & & * & 1d6 \\\hline
Silver† Shortb. Arrow & 2 gp & & * & 1d6 \\\hline
Longbow & 60 gp & L & 3 & \\\hline
Longbow Arrow & 2 sp & & * & 1d8 \\\hline
Silver† Longb. Arrow & 4 gp & & * & 1d8 \\\hline
Light Crossbow & 30 gp & M & 7 & \\\hline
Light Quarrel & 2 sp & & * & 1d6 \\\hline
Silver† Light Quarrel & 5 gp & & * & 1d6 \\\hline
Heavy Crossbow & 50 gp & L & 14 & \\\hline
Heavy Quarrel & 4 sp & & * & 1d8 \\\hline
Silver Heavy Quarrel & 10 gp & & * & 1d8 \\\hline
\textbf{Daggers} & & & & \\\hline
Dagger & 2 gp & S & 1 & 1d4 \\\hline
Silver Dagger & 25 gp & S & 1 & 1d4 \\\hline
\textbf{Swords} & & & & \\\hline
Shortsword & 6 gp & S & 3 & 1d6 \\\hline
Longsword / Scimitar & 10 gp & M & 4 & 1d8 \\\hline
Two-Handed Sword & 18 gp & L & 10 & 1d10 \\\hline
\textbf{Hammers / Maces} & & & & \\\hline
Warhammer & 4 gp & S & 6 & 1d6 \\\hline
Mace & 6 gp & M & 10 & 1d8 \\\hline
Maul & 10 gp & L & 16 & 1d10 \\\hline
\textbf{Other Weapons} & & & & \\\hline
Club / Cudgel & 2 sp & M & 1 & 1d4 \\\hline
Walking Staff & 2 sp & M & 1 & 1d4 \\\hline
Quarterstaff & 2 gp & L & 4 & 1d6 \\\hline
Pole Arm & 9 gp & L & 15 & 1d10 \\\hline
Sling & 1 gp & S & * & \\\hline
Bullet & 1 sp & & * & 1d4 \\\hline
Stone & n/a & & * & 1d3 \\\hline
Spear & 5 gp & M & 5 & \\\hline
Thrown (one handed) & & & & 1d6 \\\hline
Melee (one handed) & & & & 1d6 \\\hline
Melee (two handed) & & & & 1d8 \\\bottomrule
\end{tabular*}\medskip

* These items weigh little individually. Ten of these items weigh one
pound.

† Silver tip or blade, for use against lycanthropes.

\subsection{Weapon Size}\label{weapon-size}\index{Weapon Size}

Humans and Elves must wield Large weapons with both hands, but may use Small or Medium weapons in one hand. Halflings may not use Large weapons at all, and must use Medium weapons with both hands. Dwarves, due to their stocky, powerful builds, are able to use Medium weapons one-handed and some Large weapons in two hands, but Large weapons more than four feet in length are prohibited (specifically, two-handed swords, polearms, and longbows). Some weapons must be used with both hands by design (such as bows and crossbows) but the maximum size limits still apply.

The GM should apply similar limitations to weapon-armed monsters; for instance, goblins are about the same size as Halflings, and thus are similarly limited.

\subsection{Missile Weapon Ranges}\label{missile-weapon-ranges}\index{Missile Weapon Ranges}

\begin{tabular*}{0.93\linewidth}{@{\extracolsep{\fill}}llll}


\multirow{2}{*} {\textbf{Weapon}} & \textbf{Short}& \textbf{Medium} & \textbf{Long}\\\toprule
&\textbf{(+1)}&\textbf{(0)}&\textbf{(-2)}\\\hline
Longbow & 70 & 140 & 210 \\\hline
Shortbow & 50 & 100 & 150 \\\hline
Heavy Crossbow & 80 & 160 & 240 \\\hline
Light Crossbow & 60 & 120 & 180 \\\hline
Dagger & 10 & 20 & 30 \\\hline
Hand Axe & 10 & 20 & 30 \\\hline
Oil or Holy Water & 10 & 30 & 50 \\\hline
Sling & 30 & 60 & 90 \\\hline
Spear & 10 & 20 & 30 \\\hline
Warhammer & 10 & 20 & 30 \\\bottomrule
\end{tabular*}\\

Missile weapon ranges are given in feet. In the wilderness, substitute yards for feet. If the target is as close as or closer than the Short range figure, the attacker receives a +1 attack bonus. If the target is further away than the Medium range figure, but not beyond the Long range figure, the attacker receives a ‑2 attack penalty.

\subsection{Armor and Shields}\label{armor-and-shields}\index{Armor and Shields}

\begin{tabular*}{0.93\linewidth}{@{\extracolsep{\fill}}llll}
\textbf{Armor Type} & \textbf{Price} & \textbf{Weight} & \textbf{AC} \\\toprule
No Armor & 0 gp & 0 & 11 \\\hline
Leather Armor & 20 gp & 15 & 13 \\\hline
Chain Mail & 60 gp & 40 & 15 \\\hline
Plate Mail & 300 gp & 50 & 17 \\\hline
Shield & 7 gp & 5 & +1 \\\bottomrule
\end{tabular*}

\subsection{Beasts of Burden}\label{beasts-of-burden}\index{Beasts of Burden}

Note: Statistics for the animals below are on page
\hyperlink{beasts-of-burden-1}{\pageref{beasts-of-burden-1}}.

\medskip

\begin{tabular*}{0.93\linewidth}{@{\extracolsep{\fill}}ll}
\textbf{Item} & \textbf{Price} \\\toprule
Horse, Draft & 120 gp \\\hline
Horse, War & 200 gp \\\hline
Horse, Riding & 75 gp \\\hline
Pony* & 40 gp \\\hline
Pony, War* & 80 gp \\\bottomrule
\end{tabular*}\\

* Due to their small stature, Dwarves and Halflings generally ride ponies rather than horses.

\end{multicols}

\pagebreak

\subsection{Explanation of Equipment}\label{explanation-of-equipment}\index{Explanation of Equipment}

\begin{multicols}{2}


A \textbf{Backpack }will hold a maximum 40 pounds or 3 cubic feet of goods. Some items may be lashed to the outside, and thus count toward the weight limit but not the volume limit. A Halfling' s backpack holds at most 30 pounds and/or 1½ cubic feet, but costs the same as a full-sized item.\smallskip

A \textbf{Candle} will shed light over a 5'{} radius, with dim light extending 5'{} further. A normal candle will burn about 3 turns per inch of height.\smallskip

\textbf{Chalk} is useful for "blazing a trail" through a dungeon or ruin.\smallskip

\textbf{Holy Water} is explained in the Encounter section.\smallskip

\textbf{Iron Spikes} are useful for spiking doors closed (or spiking them open) and may be used as crude pitons in appropriate situations.\smallskip

A \textbf{Lantern} will provide light covering a 30'{} radius; dim light will extend about 20'{} further. A lantern will consume a flask of oil in 18+1d6 turns. A \textbf{Hooded Lantern} allows the light to be hidden or revealed as the user pleases; in all other ways it performs as an ordinary lantern. A \textbf{Bullseye Lantern} projects a cone of light 30'{} long and 30'{} wide at the widest point, with dim light extending an additional 20'{} beyond that point. This type of lantern is generally hooded.\smallskip

A \textbf{Map or Scroll Case} is a tubular oiled leather case used to carry maps, scrolls, or other paper items. The case will have a water-resistant (but not waterproof) cap which slides over the end, and a loop to allow the case to be hung from a belt or bandolier. A standard scroll case can hold up to 10 sheets of paper, or a single scroll of up to seven spells.\smallskip

A \textbf{Mirror} is useful in a dungeon environment for many reasons; for instance, it is the only way to look at a Medusa without being turned to stone. Mirrors are also useful for looking around corners, and can be used outdoors to send signals using reflected sunlight.\smallskip

A \textbf{Quiver} is an open container used to hold arrows. A Bolt Case is a similar sort of container for crossbow bolts. In either case, the standard capacity is 20 missiles. The length of a quiver or bolt case must match the length of the ammunition for it to be useful; therefore, there are longbow and shortbow quivers and light and heavy crossbow bolt cases. The price is the same for all types.\smallskip

\textbf{Dry Rations} may consist of dry bread, hard cheese, dried fruit, nuts, beans, jerky, or any other food which will not "go bad" in less than about a month (if not longer). Dry rations are generally sold in quantities sufficient for one character for a week, and are packaged in waxed or oiled cloth to protect them.\smallskip

\textbf{Hemp Rope} is ½ inch in diameter and has a breaking strength of 1,600 pounds. Safe working load for a rope is normally one-quarter of the breaking strength. One or more knots in a rope cut the breaking strength in half. This does not affect the safe working load, because knots are figured into the listed one-quarter ratio.\smallskip

\textbf{Silk Rope} is about 3/8 inch in diameter and has a breaking strength of 1,600 pounds, although it weighs considerably less than hemp rope. The notes regarding rope strength given for hemp rope, above, apply here also.\smallskip

A \textbf{Large Sack} will hold at most 40 pounds or 4 cubic feet of goods.\smallskip

A \textbf{Small Sack} will hold at most 20 pounds or 2 cubic feet of goods.\smallskip

A pair of \textbf{Saddlebags} will hold at most 10 pounds or 1 cubic foot of goods (divided evenly between both bags).\smallskip

\textbf{Thieves' Picks and Tools} are required for the use of Thief abilities such as opening locks and removing traps. These abilities may not be usable without appropriate tools, or may be used at a penalty at the option of the Game Master.\smallskip

A \textbf{Tinderbox }is generally purchased with a \textbf{flint and steel}; the flint, a piece of hard rock, is struck vigorously against a C-shaped piece of high-carbon steel. When done correctly, hot sparks will fly from the flint and steel into the tinder, hopefully starting a fire. The best tinder is a dried piece of prepared tinder fungus, carried in the tinderbox to keep it dry; char cloth, hemp rope, or even very dry grass can substitute if prepared tinder fungus is not available. The time required to start a fire should be determined by the GM according to the prevailing conditions; under ideal conditions, starting a fire with a flint, steel and tinder takes about a turn.\smallskip

A \textbf{Torch} sheds light over a 30'{} radius, with dim light extending about 20'{} further, and burns for 1d4+4 turns. Of course, a torch is also useful for setting flammable materials (such as cobwebs or oil) alight.\smallskip

A \textbf{Whetstone} is used to sharpen and maintain edged weapons such as swords, daggers, and axes.\smallskip

\textbf{Wineskin/Waterskin} is a container for drinking water or wine; though generally water is taken into a dungeon or wilderness environment. The standard waterskin holds one quart of liquid, which is the minimum amount required by a normal character in a single day. If adventuring in the desert or other hot, dry areas, a character may need as much as ten times this amount. Note that the given 2 pound weight is for a full skin; an empty skin has negligible weight.

\end{multicols}

\pagebreak

\subsection{Vehicles}\label{vehicles}\index{Vehicles}

The following tables give details of various land and sea vehicles. Game Masters should feel free to create their own vehicles, in which case the table can be used for guidance. Some of the statistics given below are explained in detail later.

\subsection{Land Transportation}\label{land-transportation}\index{Land Transportation}

\begin{tabular*}{1\linewidth}{@{\extracolsep{\fill}}lllllll}
\textbf{Vehicle} &\textbf{ Length x width*} & \textbf{Weight} & \textbf{Cargo} & \textbf{Movement} & \textbf{Hardness / HP} & \textbf{Cost (gp)}\\\toprule
Chariot & 15'{} x 6'{} & 300 & 750 lbs & 60'{} (10') & 10 / 10 & 400 \\\hline
Coach & 30'{} x 8'{} & 1,000 & 2,000 lbs & 40'{} (15') & 6 / 12 & 1,500 \\\hline
Wagon & 35'{} x 8'{} & 2,000 & 4,000 lbs &20'{} (15') & 6 / 16 & 500 \\\bottomrule
\end{tabular*}

* Includes hitched horses or mules.

\subsection{Water Transportation}\label{water-transportation}\index{Water Transportation}

\begin{tabular*}{1\linewidth}{@{\extracolsep{\fill}}lllllllll}
\textbf{Vehicle} & \textbf{Length x Width} & \textbf{Cargo} & \textbf{Crew} & \textbf{Movement/Day} & \textbf{Miles HP} & \textbf{Hardness} & \textbf{Cost (go)} & \\\toprule
Canoe & 15' x 4'& ½ ton & 1 & 40' (5') & 30 & 4 / 4 & 50 &\\\hline
Caravel & 55'x 15'& 75 tons & 10 & 20'(20') & 42 & 8 / 75 & 10,000& \\\hline
Carrack & 60' x 20' & 135 tons & 20 &30' (30') & 48 & 10 / 120 & 20,000 &\\\hline
Galley, Small & 100'x 15'& 210 tons & 90 & 20' (20') & 36 / 24 & 8 / 75 & 15,000 &\\\hline
Galley, Large & 120'x 20'& 375 tons & 160 & 30' (25') & 42 / 24 & 10 / 120 & 30,000 &\\\hline
Longship & 110'x 15' & 10 tons & 70 & 30' (25') & 42 / 24 & 9 / 110 & 25,000& \\\hline
Raft/Barge & per 10' x 10' & 1 ton & 2 & 40' (10') & 18 & 6 / 12 & 100 &\\\hline
Riverboat & 50' x 20' & 50 tons & 10 & 20'  (20') & 30 & 8 / 30 & 3,500 &\\\hline
Rowboat & 15' x 6' & 1 ton & 1 & 30' (10') & 24 & 6 / 8 & 60 &\\\hline
Sailboat & 40' x 8' & 5 tons & 1 & 40'(15') & 36 & 7 / 20 & 2,000 &\\\bottomrule
\end{tabular*}\vfill

\begin{multicols}{2}
	

\subsection{Notes Regarding Vehicles}\label{notes-regarding-vehicles}\index{Notes Regarding Vehicles}

The\textbf{ Crew} figure given reflects the minimum number of sailors and/or rowers needed to operate the ship. Officers are not counted among these numbers, and of course it is always a good idea to hire extra sailors and/or rowers to ensure that any casualties will not slow down the ship.\\

Cargo for wagons is given in pounds, while for ships it is given in tons. If the ship sails night and day, each passenger requires living space equivalent to one ton of cargo; in addition, provisions for one man for one month occupy 1/10 of a ton of space.\\

Movement is given separately here in yards (see Time and Scale on page \hyperlink{time-and-scale}{\pageref{time-and-scale}} for an explanation) as well as miles per day. The encounter movement of ships is not directly related to the long-distance travel rate, since the crew must work hard to make the ship move quickly in combat, and this level of effort cannot be maintained day and night.\\

The parenthesized figure represents Maneuverability; as explained in the Encounter section on page \hyperlink{maneuverability}{\pageref{maneuverability}}.\\

See \textbf{Attacking a Vehicle} on page \hyperlink{attacking-a-vehicle}{\pageref{attacking-a-vehicle}} of the \textbf{Encounter }section, for details on the \textbf{Hardness }and \textbf{HP }statistics.\\

A \textbf{chariot }requires a single horse, generally a warhorse, to pull it. Both \textbf{coaches }and\textbf{ wagons} require at least a pair of draft horses to pull them. \\

A \textbf{caravel} is a highly maneuverable sailing ship with two or three masts. Though superficially similar to the larger carrack, caravels are capable of sailing up rivers, a task for which the larger ship is ill suited.\\

A \textbf{carrack} is a large, ocean-going sailing ship with three or four masts.\\

\textbf{Galleys} are equipped with both sails and oars; the second listed movement rate for galleys is the rowing speed. A small galley will have around 20 rows of oars, with each oar pulled by two men (for a total of 80 rowers) while a large galley will have around 35 rows of oars (for a total of 140 rowers). Galleys are generally much more maneuverable than sailing ships such as the carrack or caravel, and may be outfitted with rams.\\

The \textbf{longship} commonly used by northern raiders is very similar to the large galley. However, where more civilized nations have specialist rowers, sailors, and marines, the crew of a longship is more generalized; most crewmen will be qualified for all of these tasks.

\end{multicols}

\pagebreak

\subsection{Siege Engines}\label{siege-engines}\index{Siege Engines}

These are weapons used to attack strongholds, or sometimes ships. Their cost may be up to twice as high in a remote location. A siege engine that throws missiles (a ballista, onager, or trebuchet) must have a trained artillerist to fire it; this is the character who makes the attack rolls for the weapon. Missile-throwing engines have attack penalties, detailed below. Note: siege engines are not generally usable against individuals or monsters; the GM may make exceptions for very large monsters like giants or dragons. Review the rules in the \textbf{Stronghold} section on page \hyperlink{attacking-a-vehicle}{\pageref{attacking-a-vehicle}} for details regarding attacking fortified buildings such as castles, towers, fortresses, and so on.\\

\begin{tabular*}{1\linewidth}{@{\extracolsep{\fill}}llllllll}
\multirow{2}{*}{\textbf{Weapon}}&\textbf{Cost}&\textbf{Rate}&\textbf{Attack}&\textbf{Damage}&\textbf{Short}&\textbf{Medium} & \textbf{Long}\\\toprule
&&\textbf{of fire}&\textbf{Penalty}&&\textbf{Range (+1)}&\textbf{Range (+0)}&\textbf{Range (-2)}\\\hline
Ballista & 100 gp & 1/4 & -3 & 2d8 & 50' & 100' & 150' \\\hline
Battering Ram & 200 gp & 1/3 & +0 & 2d8 & N/A & N/A & N/A \\\hline
Onager & 300 gp & 1/6 & -6 & 2d12 & 100'{} &200'{} & 300'{} \\\hline
Screw & 200 gp & N/A & N/A & 1d8 & N/A & N/A & N/A \\\hline
Sow & 100 gp & N/A & N/A & N/A & N/A & N/A & N/A \\\hline
Trebuchet & 400 gp & 1/10 & -8 & 3d10 & N/A & 300' & 400'{} \\\bottomrule
\end{tabular*}\vfill

\begin{multicols}{2}

\bigskip

\textbf{Ballista:} This is effectively a very large crossbow that may fire a spear-like bolt or a large stone. It is usually mounted on a tripod or wagon, but may also be mounted on a ship. When firing bolts, a ballista cannot damage brick or stone. A ballista requires a crew of three to operate.\\

\textbf{Battering Ram:} These are usually operated under a \textbf{sow}. They require a crew of eight or more\\.

\textbf{Screw:} This device may be used to attack a stronghold, by means of boring through the walls. A crew of at least eight is required to operate it. It is only used at the base of a wall, and it is usually operated under a \textbf{sow}.\\

\textbf{Sow:} This is a kind of portable roof, used for protection while performing slower attacks on a fortified building. Those under a sow will be harder to hit, receiving at least a +6 bonus to Armor Class against ranged attacks while taking cover under it. The sow itself has a hardness of 9 and 50 hit points.\

\textbf{Onager:} This weapon throws a stone with a fairly flat trajectory. Operating an onager requires a crew of four.\\

\textbf{Trebuchet:} This mighty weapon uses a counterweight to fling a stone on a high, arcing path. It cannot fire at targets within 200 yards. If it is aimed at a target that is more than 20 feet higher than the weapon, there is an additional --2 attack penalty. A trebuchet requires a crew of eight to operate.

\end{multicols}

\vfill

\includegraphics[width=0.95\textwidth]{Pictures132/100000000000071000000400408A09D14F87532D.png}

\pagebreak

\section{PART 3: SPELLS}\label{part-3-spells}\index{Part 3: Spells}

\begin{multicols}{2}
	
The number of spells of each level which a Cleric or Magic-User may cast
per day is shown on the appropriate table in the \textbf{Characters} section starting on page \hyperlink{part-2-player-characters}{\pageref{part-2-player-characters}}. Each morning spellcasters prepare spells to replace those they have used. Clerics pray, while Magic-Users must study their spellbooks. Spells prepared but not used persist from day to day; only those actually cast must be replaced. A spellcaster may choose to dismiss a prepared spell (without casting it) in order to prepare a different spell of that level.\\

Spellcasters must have at least one hand free, and be able to speak, in order to cast spells; thus, binding and gagging a spellcaster is an effective means of preventing them from casting spells. In combat, casting a spell usually takes the same time as making an attack. If a spellcaster is attacked (even if not hit) or must make a saving throw (whether successful or not) on the Initiative number on which they are casting a spell, the spell is spoiled and lost. As a specific exception, two spell casters releasing their spells at each other on the same Initiative number will both succeed in their casting; one caster may disrupt another with a spell only if they have a better Initiative, and choose to delay casting the spell until \emph{right before} the other caster.

Some spells are reversible; such spells are shown with an asterisk after the name.

\subsection{Cleric Spells}\label{cleric-spells}\index{Cleric Spellstry}\index{Cleric Spells}

Clerics receive their spells through faith and prayer. Each day, generally in the morning, a Cleric must pray for at least three turns in order to prepare spells. Of course, the Cleric may be expected to pray more than this in order to remain in their deity' s good graces.

Because they gain their spells through prayer, a Cleric may prepare any spell of any level they are able to cast. In some cases the Cleric' s deity may limit the availability of certain spells; for instance, a deity devoted to healing may refuse to grant reversed healing spells.\\

{\large \textbf{First Level Clerical Spells}}\index{First Level Clerical Spells}

\begin{tabularx}{0.45\textwidth}{@{}ll@{}}
1 & Cure Light Wounds* \\\toprule
2 & Detect Evil* \\\hline
3 & Detect Magic \\\hline
4 & Light* \\\hline
5 & Protection from Evil* \\\hline
6 & Purify Food and Water \\\hline
7 & Remove Fear* \\\hline
8 & Resist Cold \\\bottomrule
\end{tabularx}\\\bigskip

{\large \textbf{Second Level Clerical Spells}}\index{Second Level Clerical Spells}

\begin{tabularx}{0.45\textwidth}{@{}ll@{}}
1 & Bless* \\\toprule
2 & Charm Animal \\\hline
3 & Find Traps \\\hline
4 & Hold Person \\\hline
5 & Resist Fire \\\hline
6 & Silence 15'{} radius \\\hline
7 & Speak with Animals \\\hline
8 & Spiritual Hammer \\\bottomrule
\end{tabularx}\\\bigskip

{\large \textbf{Third Level Clerical Spells}}\index{Third Level Clerical Spells}

\begin{tabularx}{0.45\textwidth}{@{}ll@{}}
1 & Continual Light* \\\toprule
2 & Cure Blindness* \\\hline
3 & Cure Disease* \\\hline
4 & Growth of Animals \\\hline
5 & Locate Object \\\hline
6 & Remove Curse* \\\hline
7 & Speak with Dead \\\hline
8 & Striking \\\bottomrule
\end{tabularx}\\\bigskip

{\large \textbf{Fourth Level Clerical Spells}}\index{Fourth Level Clerical Spells}

\begin{tabularx}{0.45\textwidth}{@{}ll@{}}
1 & Animate Dead \\\toprule
2 & Create Water \\\hline
3 & Cure Serious Wounds* \\\hline
4 & Dispel Magic \\\hline
5 & Neutralize Poison* \\\hline
6 & Protection from Evil 10' radius* \\\hline
7 & Speak with Plants \\\hline
8 & Sticks to Snakes \\\bottomrule
\end{tabularx}\\\bigskip

{\large \textbf{Fifth Level Clerical Spells}}\index{Fifth Level Clerical Spells}

\begin{tabularx}{0.45\textwidth}{@{}ll@{}}
1 & Commune \\\toprule
2 & Create Food \\\hline
3 & Dispel Evil \\\hline
4 & Insect Plague \\\hline
5 & Quest* \\\hline
6 & Raise Dead* \\\hline
7 & True Seeing \\\hline
8 & Wall of Fire \\\bottomrule
\end{tabularx}\\\bigskip

{\large \textbf{Sixth Level Clerical Spells}}\index{Sixth Level Clerical Spells}

\begin{tabularx}{0.45\textwidth}{@{}ll@{}}
1 & Animate Objects \\\toprule
2 & Blade Barrier \\\hline
3 & Find the Path \\\hline
4 & Heal* \\\hline
5 & Regenerate \\\hline
6 & Restoration \\\hline
7 & Speak with Monsters \\\hline
8 & Word of Recall \\\bottomrule
\end{tabularx}

\pagebreak

\subsection{Magic-User Spells}\label{magic-user-spells}\index{Magic-User Spells}\index{Magic-User Spells}

Magic-Users cast spells through the exercise of knowledge and will. They prepare spells by study of their spellbooks; each Magic-User has their own spellbook containing the magical formulae for each spell the Magic-User has learned. Spellbooks are written in a magical script that can only be read by the one who wrote it, or through the use of the spell \textbf{read magic}. All Magic-Users begin play knowing \textbf{read magic}, and it is so ingrained that it can be prepared without a spellbook.\\

A Magic-User may only prepare spells after resting (i.e. a good night' s sleep), and needs one turn per each three spell levels to do so (rounding fractions up). Spells prepared but not used on a previous day are not lost. For example, a 3\textsuperscript{rd} level Magic-User preparing all three of their available spells (two 1\textsuperscript{st} level and one 2\textsuperscript{nd} level) is preparing a total of 4 levels of spells, and thus needs 2 turns (4 divided by 3 and rounded up).\\

Rules for the acquisition of new spells are found in the Game
Master' s section on page \hyperlink{acquisition-of-spells}{\pageref{acquisition-of-spells}}.\\

{\large \textbf{First Level Magic-User Spells}}\\\index{First Level Magic-User Spells}

\begin{tabularx}{0.45\textwidth}{@{}ll@{}}
1 & Charm Person \\\toprule
2 & Detect Magic \\\hline
3 & Floating Disc \\\hline
4 & Hold Portal \\\hline
5 & Light* \\\hline
6 & Magic Missile \\\hline
7 & Magic Mouth \\\hline
8 & Protection from Evil* \\\hline
9 & Read Languages \\\hline
10 & Shield \\\hline
11 & Sleep \\\hline
12 & Ventriloquism \\\bottomrule
\end{tabularx}\bigskip

{\large \textbf{Second Level Magic-User Spells}}\\\index{Second Level Magic-User Spells}

\begin{tabularx}{0.45\textwidth}{@{}ll@{}}
1 & Continual Light* \\\toprule
2 & Detect Evil* \\\hline
3 & Detect Invisible \\\hline
4 & Invisibility \\\hline
5 & Knock \\\hline
6 & Levitate \\\hline
7 & Locate Object \\\hline
8 & Mind Reading \\\hline
9 & Mirror Image \\\hline
10 & Phantasmal Force \\\hline
11 & Web \\\hline
12 & Wizard Lock \\\bottomrule
\end{tabularx}\bigskip

{\large \textbf{Third Level Magic-User Spells}}\\\index{Third Level Magic-User Spells}

\begin{tabularx}{0.45\textwidth}{@{}ll@{}}
1 & Clairvoyance \\\toprule
2 & Darkvision \\\hline
3 & Dispel Magic \\\hline
4 & Fireball \\\hline
5 & Fly \\\hline
6 & Haste* \\\hline
7 & Hold Person \\\hline
8 & Invisibility 10` radius \\\hline
9 & Lightning Bolt \\\hline
10 & Protection from Evil 10' radius* \\\hline
11 & Protection from Normal Missiles \\\hline
12 & Water Breathing \\\bottomrule
\end{tabularx}\bigskip

{\large \textbf{Fourth Level Magic-User Spells}}\\\index{Fourth Level Magic-User Spells}

\begin{tabularx}{0.45\textwidth}{@{}ll@{}}
1 & Charm Monster \\\toprule
2 & Confusion \\\hline
3 & Dimension Door \\\hline
4 & Growth of Plants* \\\hline
5 & Hallucinatory Terrain \\\hline
6 & Ice Storm \\\hline
7 & Massmorph \\\hline
8 & Remove Curse* \\\hline
9 & Polymorph Other \\\hline
10 & Polymorph Self \\\hline
11 & Wall of Fire \\\hline
12 & Wizard Eye \\\bottomrule
\end{tabularx}\bigskip

{\large \textbf{Fifth Level Magic-User Spells}}\\\index{Fifth Level Magic-User Spells}

\begin{tabularx}{0.45\textwidth}{@{}ll@{}}
1 & Animate Dead \\\toprule
2 & Conjure Elemental \\\hline
3 & Cloudkill \\\hline
4 & Feeblemind* \\\hline
5 & Hold Monster \\\hline
6 & Magic Jar \\\hline
7 & Passwall \\\hline
8 & Telekinesis \\\hline
9 & Teleport \\\hline
10 & Wall of Stone \\\bottomrule
\end{tabularx}\bigskip

\textbf{{\large Sixth Level Magic-User Spells}}\\\index{Sixth Level Magic-User Spells}

\begin{tabularx}{0.45\textwidth}{@{}ll@{}}
1 & Anti-Magic Shell \\\toprule
2 & Death Spell \\\hline
3 & Disintegrate \\\hline
4 & Flesh to Stone* \\\hline
5 & Geas* \\\hline
6 & Invisible Stalker \\\hline
7 & Lower Water \\\hline
8 & Projected Image \\\hline
9 & Reincarnate \\\hline
10 & Wall of Iron \\\bottomrule
\end{tabularx}

\end{multicols}

\pagebreak

\subsection{All Spells, in Alphabetical Order}\index{All Spells, in Alphabetical Order} \label{all-spells-in-alphabetical-order}

\begin{multicols}{2}

\begin{flushleft}\index{Animate Dead}
\begin{tabularx}{0.45\textwidth}{@{}m{3.4cm}m{5.5cm}@{}} 
\textbf{Animate Dead} & Range:	30'\\
Cleric 4, Magic-User 5 & Duration: special\\	
\end{tabularx}\end{flushleft}

\label{animate-dead-range-30}


The casting of this spell causes the mortal remains of one or more deceased creatures to arise as animated skeletons or zombies. Such undead monsters persist until slain, and obey the verbal commands of the caster.

A single casting of this spell may animate a number of hit dice of undead equal to twice the caster' s level of ability, and no more. Animated skeletons have hit dice equal to the number the creature had in life; for skeletons of members of player character races, this means one hit die, regardless of the character level of the deceased. Zombies have one more hit die than the creature had in life.

The sort of monsters created depend on the condition of the remains. A reasonably intact corpse may only arise as a zombie, while similarly in act skeletal remains may only be animated as a skeleton. The caster chooses which remains are animated when casting the spell, in any case where there are more bodies than the caster can animate.

No character may normally control more hit dice of undead than 4 times their level, regardless of how many times this spell is cast.

\smallskip\begin{flushleft} \index{Animate Objects}
\begin{tabularx}{0.45\textwidth}{@{}m{3.2cm}m{5.5cm}@{}} 
\textbf{Animate Objects} & Range: 100'+10'/level\\
Cleric 6 & Duration: 1 round/level\\	
\end{tabularx}\end{flushleft}


This spell allows the caster to animate objects that are normally inanimate, such that they may move and even appear to be alive. Objects to be animated may not be in the possession (worn or carried) of any creature, and must be non-magical in nature. The caster can animate one object per level, up to a maximum of 25 lbs. per caster level (i.e. 300 lbs. at 12\textsuperscript{th} level, 325 lbs. at 13\textsuperscript{th} level, and so on).

Such objects are normally used to attack the enemies of the caster, and the GM must rule on their effectiveness in combat. In general, animated objects attack using the same attack bonus as the caster. Small or lightweight objects do no more than 1d4 damage per hit, while larger and/or heavier objects do 1d6 or 1d8 (at the GM' s discretion). As a special case, weapons which are animated do damage using the normal die roll for the type, but only up to a maximum 1d8. Animated objects have a movement rate of 10', and generally must move in contact with the ground (walking, hopping, slithering, or bouncing, however seems most appropriate to the GM).

\smallskip\begin{flushleft} \index{Anti-Magic Shell}
	\begin{tabularx}{0.45\textwidth}{@{}m{3.5cm}m{5.5cm}@{}} 
		\textbf{Anti-Magic Shell} & Range: 10'\\
		Magic-User 6 & Duration: 1 turn/level\\	
	\end{tabularx}\end{flushleft}


Within a 10'{} radius around the caster, all magic is negated for the full duration of the spell. Magical attacks will not affect the caster, magic items and spells within the radius are suppressed, and the caster cannot perform further magic until the spell has expired.

\smallskip\begin{flushleft} \index{Blade Barrier}
\begin{tabularx}{0.45\textwidth}{@{}m{3.2cm}m{5.5cm}@{}} 
\textbf{Blade Barrier} & Range: 90'\\
Cleric 6 & Duration: 1 round/level\\	
\end{tabularx}\end{flushleft}


This spell creates a barrier of flying, spinning, flashing blades. The caster may choose a barrier up to 20'{} high which extends up to 20'{} long per level of caster, or a ring-shaped barrier up to 20'{} high with a radius of up to 5'{} per each two full caster levels (so 30'{} at level 12 or 13, 35'{} at level 14 or 15, and so on).

Should any creature pass through the wall, it suffers 1d6 points of damage per caster level, with a maximum damage of 15d6. A successful save vs. Death Ray reduces damage by half.

The caster may choose to conjure the wall in an area where creatures are already located; if this is done, all such creatures are harmed as if they passed through the wall, but a successful save vs. Death Ray in this case indicates that the creature has avoided all damage by jumping to one side or the other of the wall as chosen by the creature.  Any ranged attacks passing through a blade barrier suffer a penalty of -4 on the attack roll.

\smallskip\begin{flushleft} \index{Bless*}
	\begin{tabularx}{0.45\textwidth}{@{}m{3.5cm}m{5.5cm}@{}} 
		\textbf{Bless*} & Range: 50'\\
		Cleric 2 & Duration: 1 minute/level\\	
	\end{tabularx}\end{flushleft}


This spell gives the caster and their allies within a 50'{} radius a bonus of +1 on all attack rolls, morale checks (for monsters or NPCs allied with the caster), and saving throws vs. any kind of magical \textbf{fear}. Casters of the 7\textsuperscript{th} or higher level grant a bonus of +2 to attacks and saves vs. \textbf{fear}, but the morale bonus remains +1.

The reverse of \textbf{bless }is called \textbf{bane}. The caster' s enemies within a 50'{} radius become fearful and uncertain, causing them to suffer a penalty of -1 on attack rolls, morale checks, and saving throws vs. any kind of magical \textbf{fear}. Casters of the 7\textsuperscript{th} or higher level apply a penalty of -2 to attacks and saves vs. \textbf{fear}, but the morale penalty remains -1.

\smallskip\begin{flushleft} \index{Charm Animal}
	\begin{tabularx}{0.45\textwidth}{@{}m{3.5cm}m{5.5cm}@{}} 
		\textbf{Charm Animal} & Range: 60'\\
		Cleric 2 & Duration: level+1d4 rounds\\	
	\end{tabularx}\end{flushleft}

This spell allows the caster to charm one or more animals, in much the same fashion as \textbf{charm person}, at a rate of 1 hit die per caster level. The caster may decide which individual animals out of a mixed group are to be affected first; excess hit dice of effect are ignored. No saving throw is allowed, either for normal or giant-sized animals, but creatures of more fantastic nature (as determined by the GM) are allowed a save vs. Spells to resist. When the duration expires, the animals will resume normal activity immediately.

This spell does not grant the caster any special means of communication with the affected animals; if combined with \textbf{speak with animals}, this spell becomes significantly more useful.

\smallskip\begin{flushleft} \index{Charm Monster}
	\begin{tabularx}{0.45\textwidth}{@{}m{3.5cm}m{5.5cm}@{}} 
		\textbf{Charm Monster} & Range: 30'\\
		Magic-User 4 & Duration: special\\	
	\end{tabularx}\end{flushleft}

This spell has an effect similar to \textbf{charm person}, but it is able to affect living creatures of any size or type. Undead monsters are unaffected, as are constructs such as golems and any creature which is functionally mindless (such as any kind of jelly). This spell can affect 3d6 hit dice of creatures of 3 or fewer hit dice, or one creature of more than 3 hit dice. Saving throws are made just as for \textbf{charm person}.

\smallskip\begin{flushleft} \index{Charm Person}
	\begin{tabularx}{0.45\textwidth}{@{}m{3.5cm}m{5.5cm}@{}} 
		\textbf{Charm Monster} & Range: 30'\\
		Magic-User 1 & Duration: special\\	
	\end{tabularx}\end{flushleft}

This spell causes a humanoid (including all character races as well as creatures such as orcs, goblins, gnolls, and so on) of 4 hit dice or less to perceive the caster as a close friend, love interest, or at the very least as its trusted ally. Normal characters (PC or NPC) may be affected regardless of level of ability.

A save vs. Spells will negate the effect. If hostilities have already commenced or the target otherwise feels threatened by the caster, it receives a bonus of +5 on its saving throw. 

The caster does not directly control the target; rather, orders must be given verbally, in writing, or by means of gestures. Obviously, verbal orders will only work if the target and caster share a spoken language, and the same limitation applies to written orders. Also note that the exact perception of the caster by the affected individual is not under the control of the caster; the GM should decide how the subject of this spell perceives its relationship to the caster.

Commands that go against the target' s basic nature or ask it to attack its own allies or friends grant it a fresh saving throw with a bonus of +5 on the roll. Even if the target fails this save it may still choose to do something else when commanded to perform an unwanted action. Of course, if the caster is attacked, the charmed creature will act to protect its "friend" (though that could mean attacking its own allies, which might cause the target to instead attempt to carry off the caster to a "safe" place).

The target receives a new saving throw each day if it has an Intelligence of 13 or greater, every week if its Intelligence is 9-12, or every month if its Intelligence is 8 or less; the GM must rule on the equivalent intelligence of humanoid monsters.\medskip


\smallskip\begin{flushleft} \index{Clairvoyance}
	\begin{tabularx}{0.45\textwidth}{@{}m{3.5cm}m{5.5cm}@{}} 
		\textbf{Clairvoyance} & Range: 60'\\
		Magic-User 3 & Duration: 12 turns\\	
	\end{tabularx}\end{flushleft}

This spell enables the caster to see into another area through the eyes of a living creature in that area. The caster must specify the direction and approximate distance, up to a maximum of 60'{} away. If there is no appropriate creature in that area, the spell fails. No saving throw is allowed, and the target creature is unaware that it is being so used. The caster may choose another subject creature after at least a turn has passed, enabling multiple locations to be viewed. If the subject creature moves out of range, contact is lost, though the caster may be able to choose another target in this case.


\begin{flushleft}	
	\includegraphics{Pictures132/10000000000003CF000003D46F2E64E7334A3E9F.png}
\end{flushleft}

\smallskip\begin{flushleft} \index{Cloudkill}
	\begin{tabularx}{0.45\textwidth}{@{}m{3.5cm}m{5.5cm}@{}} 
		\textbf{Cloudkill} & Range: 100'+10'/level\\
		Magic-User 5 & Duration: 6 rounds/level\\	
	\end{tabularx}\end{flushleft}

This spell creates a 20' x20' x20'{} cloud of poison gas which moves at a rate of 10'{} per round under the control of the caster (so long as they concentrate on it). The gas kills outright any creatures of 3 or fewer hit dice or levels it comes in contact with; creatures having 4 or more hit dice or levels must save vs. Poison or die. The cloud persists for the entire duration even if the caster ceases to concentrate upon it.

\smallskip\begin{flushleft} \index{Commune}
	\begin{tabularx}{0.45\textwidth}{@{}m{3.5cm}m{5.5cm}@{}} 
		\textbf{Commune} & Range: self\\
		Cleric 5 & Duration: 1 round/level\\	
	\end{tabularx}\end{flushleft}

This spell puts the caster in contact with his patron deity or an extraplanar servant thereof, who answers one yes-or-no question per caster level. The ritual to cast this spell takes 1 turn to complete. The being contacted may or may not be omniscient, and further, though the being is technically allied with the caster, it may still not answer questions clearly or completely. These details are left to the GM' s discretion.

\smallskip\begin{flushleft} \index{Confusion}
	\begin{tabularx}{0.45\textwidth}{@{}m{3.5cm}m{5.5cm}@{}} 
		\textbf{Confusion} & Range: 280'+10'\\
		Magic-User 4 & Duration: 2 rounds\\
		&+1/level\\	
	\end{tabularx}\end{flushleft}

This spell causes up to 3d6 living creatures within a 30'{} radius circle around the target point to become confused\emph{, }making them unable to independently determine what they will do. A saving throw vs. Spells is allowed to resist the effect. Roll on the following table on each subject's Initiative number each round to see what the subject does.\medskip

\begin{tabularx}{0.45\textwidth}{@{}lX@{}}
\textbf{d10} & \textbf{Behavior} \\\hline
1 & Act normally. \\\hline
2 & Move toward the caster, and attack if possible. \\\hline
3-5 & Take no action except possibly to babble. \\\hline
6-7 & Move swiftly away from the caster. \\\hline
8-10 & Attack the nearest creature, regardless of whether it is a friend
or foe. \\\hline
\end{tabularx}\medskip

If the target cannot perform the action indicated, the GM should move down the table (going back to the top if the table runs out) until an action is found that the target can perform. If a confused creature is attacked, it returns the attack on its next initiative number (later in this round or in the next round if it has already acted) regardless of what is rolled on the table.

\smallskip\begin{flushleft} \index{Conjure Elemental}
	\begin{tabularx}{0.45\textwidth}{@{}m{3.7cm}m{5.5cm}@{}} 
		\textbf{Conjure Elemental} & Range: 240'\\
		Magic-User 5 & Duration: special\\	
	\end{tabularx}\end{flushleft}

A portal to one of the Elemental Planes is opened, allowing the Magic-User to summon an elemental from that plane. Review the \textbf{Elemental} entry in the \textbf{Monsters} section on page \hyperlink{elemental}{\pageref{elemental}} for further details regarding the types available and their statistics. At most one elemental of each type may be summoned by the caster in a given day. Once the elemental appears, it serves the conjurer indefinitely, provided the caster concentrates on nothing but controlling the creature; spell casting, combat, or movement over half the normal rate results in loss of concentration. The conjurer, while in control of an elemental, can dismiss it to its native plane at will (doing so on their Initiative if in combat). If the Magic-User loses concentration, control of the summoned Elemental is lost and cannot be regained. The creature then seeks to attack the conjurer and all others in its path. Only \textbf{dispel magic} or \textbf{dispel evil} will banish the elemental once control has been lost. An elemental may, of course, choose to return to its home plane on its own; such creatures will not choose to remain on the material plane for long.

\smallskip\begin{flushleft} \index{Continual Light*}
	\begin{tabularx}{0.45\textwidth}{@{}m{3.5cm}m{5.5cm}@{}} 
		\textbf{Continual Light*} & Range: 360'\\
		Cleric 3, Magic-User 2 & Duration: 1 year/level\\
	\end{tabularx}\end{flushleft}

This spell creates a spherical region of light, as bright as full daylight up to a 30'{} radius, with light of lesser intensity to a radius of 60'. Continual light can be cast on an object, into the air, or at a creature, just as with the \textbf{light} spell, up to a maximum range of 360' from the caster. The spell remains in effect for one year per level of the caster.

As with \textbf{light}, this spell can be used to blind a creature if cast on its visual organs. Creatures targeted by this spell are allowed a save vs. Death Ray; if the save is made, the spell is cast into the air just behind the target creature. A penalty of -4 is applied to the blinded creature' s attack rolls if the saving throw fails.

The reversed spell, \textbf{continual darkness}, causes complete absence of light in the area of effect, overpowering normal light sources. Continual darkness may be used to blind in the same way as continual light.

\smallskip\begin{flushleft} \index{Create Food}
	\begin{tabularx}{0.45\textwidth}{@{}m{3.5cm}m{5.5cm}@{}} 
		\textbf{Create Food} & Range: 10'\\
		Cleric 5 &Duration: permanent\\
	\end{tabularx}\end{flushleft}

This spell creates food sufficient to feed up to 3 men or one horse per caster level for one day. This food remains edible for just a day, but this can be extended by a day by casting \textbf{purify food and water} on it (and this can be done repeatedly, keeping the food good indefinitely). Food created by this spell is nourishing and satisfying, but is rather bland.

\smallskip\begin{flushleft} \index{Create Water}
	\begin{tabularx}{0.45\textwidth}{@{}m{3.5cm}m{5.5cm}@{}} 
		\textbf{Create Water} & Range: 10'\\
		Cleric 4 &Duration: permanent\\
	\end{tabularx}\end{flushleft}

This spell creates one gallon of clean water per caster level. One or more vessels to contain the water must be available at the time of casting. Note: Water weighs about 8 pounds per gallon, and one cubic foot of water is roughly 8 gallons.

\smallskip\begin{flushleft} \index{Cure Blindness*}
	\begin{tabularx}{0.45\textwidth}{@{}m{3.5cm}m{5.5cm}@{}} 
		\textbf{Cure Blindness*} & Range: touch\\
		Cleric 3 &Duration: instantaneous\\
	\end{tabularx}\end{flushleft}

With this spell the caster can cure a creature suffering blindness (whether caused by injury or by magic, including \textbf{light} or \textbf{continual light}). Blindness caused by a curse cannot be cured by this spell.

Reversed, this spell becomes \textbf{cause blindness}, which causes a living creature touched to become blind. A successful melee attack roll is required to touch the victim, and no Saving Throw is allowed. Blinded creatures suffer the penalties described in \textbf{Deafness and Blindness} on page \hyperlink{deafness-and-blindness}{\pageref{deafness-and-blindness}}.

\smallskip\begin{flushleft} \index{Cure Disease*}
	\begin{tabularx}{0.45\textwidth}{@{}m{3.5cm}m{5.5cm}@{}} 
		\textbf{Cure Disease*} & Range: touch\\
		Cleric 3 &Duration: instantaneous\\
	\end{tabularx}\end{flushleft}

Cure disease cures all diseases and kills all parasites afflicting the target creature. A magical or otherwise special disease (as defined by the GM) may be unaffected by this spell, or may require a caster of a certain minimum level to cure it. Also note, just because a disease or parasite is removed does not mean that the victim cannot be infected anew should that affliction be encountered again.\\

The reverse form of this spell, \textbf{cause disease}, causes a living creature touched by the caster to suffer from a debilitating and potentially deadly disease for the next 1d10 days. A successful melee attack roll is required to touch the victim, and no Saving Throw is allowed. The target suffers a -2 penalty to attack rolls, Armor Class and Saving Throws. While the victim is sick they cannot benefit from natural healing of damage or Constitution point losses, nor prepare spells or move at running speed. At the end of each day in which an infected character performed any form of exertion (for example by fighting, traveling, working, or doing magical research), the character must roll a saving throw vs. Spells or suffer 1d6 points of damage. Once the spell duration has elapsed the affected character, if still alive, is free of the illness and can start healing damage and recovering lost Constitution points, and can again prepare spells.


\begin{flushleft}
	\includegraphics{Pictures132/10000001000003CF0000044B5B23146B812F11B1.png}
\end{flushleft}

\smallskip\begin{flushleft} \index{Cure Light Wounds*}
	\begin{tabularx}{0.45\textwidth}{@{}m{3.5cm}m{5.5cm}@{}} 
		\textbf{Cure Light Wounds } & Range: touch\\
		Cleric 1 &Duration: instantaneous\\
	\end{tabularx}\end{flushleft}

With this spell the caster heals 1d6+1 hit points of damage by laying
their hand upon the injured creature.

The reverse form of this spell, \textbf{cause light wounds}, causes 1d6+1 damage to the creature affected by it. A successful attack roll is required in this case, and the spell is wasted if the attack roll fails.

Undead are affected by this spell and its reverse in opposite fashion; they are injured by \textbf{cure light wounds} and healed by \textbf{cause light wounds}.

\smallskip\begin{flushleft} \index{Cure Serious Wounds*}
	\begin{tabularx}{0.45\textwidth}{@{}m{3.5cm}m{5.5cm}@{}} 
		\textbf{Cure Serious Wounds} & Range: touch\\
		Cleric 4 &Duration: instantaneous\\
	\end{tabularx}\end{flushleft}

This spell works exactly like \textbf{cure light wounds,} save that it heals 2d6 points of damage, plus 1 point per caster level. The reverse, \textbf{cause serious wounds}, also works exactly like \textbf{cause light wounds}, except that it inflicts 2d6 points of damage, +2 points per caster level.\medskip


\smallskip\begin{flushleft} \index{Darkvision}
	\begin{tabularx}{0.45\textwidth}{@{}m{3.5cm}m{5.5cm}@{}} 
		\textbf{Darkvision} & Range: touch\\
Magic-User 3 &Duration: 1 hour/level\\
	\end{tabularx}\end{flushleft}

The subject receives Darkvision with a range of 60' for the duration of the spell. (See page \hyperlink{darkvision}{\pageref{darkvision}} for details.)

\smallskip\begin{flushleft} \index{Death Spell}
	\begin{tabularx}{0.45\textwidth}{@{}m{3.5cm}m{5.5cm}@{}} 
		\textbf{Death Spell} & Range: 240'\\
Magic-User 6 & Duration: instantaneous\\
	\end{tabularx}\end{flushleft}

This spell will kill 3d12 hit dice or levels of creatures in a 30'{} radius sphere centered wherever the caster wishes (within the range limit). Excess levels of effectiveness are lost. Each creature affected is allowed to save vs. Death Ray; those that fail the save die immediately. Creatures of 8 or more hit dice or levels are immune to the spell, as are undead monsters, golems, and any other "creature" that is not truly alive.

\smallskip\begin{flushleft} \index{Detect Evil*}
	\begin{tabularx}{0.45\textwidth}{@{}m{3.5cm}m{5.5cm}@{}} 
		\textbf{Detect Evil*} & Range: 60'\\
Cleric 1, Magic-User 2 & Duration: 1 round/level\\
	\end{tabularx}\end{flushleft}

This spell allows the caster to detect evil; specifically, the caster can detect creatures with evil intentions, magic items with evil enchantments, and possibly extraplanar creatures of evil nature. Normal characters, even "bad" characters, cannot be detected by this spell, as only overwhelming evil is detectable. The caster sees the "evil" creatures or objects with a definite glow around them, but the glow cannot be seen by anyone else.

The exact definition of evil is left for the GM to decide. Note that items such as ordinary traps or poisons are not "evil," and thus not detectable by this spell.

Reversed, this spell becomes \textbf{detect good}, which works just as described above with respect to detecting "good" enchantments, angelic creatures, and so on.

\smallskip\begin{flushleft} \index{Detect Invisible}
	\begin{tabularx}{0.45\textwidth}{@{}m{3.5cm}m{5.5cm}@{}} 
		\textbf{Detect Invisible} & Range: 60'\\
Magic-User 2 & Duration: 1 turn/level\\
	\end{tabularx}\end{flushleft}

By means of this spell the caster is able to see invisible characters, creatures or objects within the given range, seeing them as bright transparent outlines or shapes.

\smallskip\begin{flushleft} \index{Detect Magic}
	\begin{tabularx}{0.45\textwidth}{@{}m{3.5cm}m{5.5cm}@{}} 
		\textbf{Detect Magic} & Range: 60'\\
Cleric 1, Magic-User 1 &Duration: 2 turns\\
	\end{tabularx}\end{flushleft}

The caster of this spell is able to detect enchanted or enspelled objects or creatures within the given range by sight, seeing them surrounded by a pale glowing light. Only the caster sees the glow. Invisible creatures or objects are not detected by this spell, but the emanations of the invisibility magic will be seen as an amorphous glowing fog, possibly allowing the caster (only) to attack the invisible creature at an attack penalty of only -2.

\smallskip\begin{flushleft} \index{Dimension Door}
	\begin{tabularx}{0.45\textwidth}{@{}m{3.5cm}m{5.5cm}@{}} 
		\textbf{Dimension Door} & Range: 10'\\
Magic-User 4 &Duration: instantaneous\\
	\end{tabularx}\end{flushleft}

The caster of this spell or any single target creature within range is transported instantly, to any location within 200 feet plus 20 feet per caster level. The caster may give distance and direction or may choose to visualize the target location, and the target will be transported unerringly to that place. An unwilling target may save vs. Spells to avoid being transported. Anything worn or carried by the caster or target creature will be transported also, including another character or creature if the transportee can lift it. If the target area is within a solid object, the spell fails automatically.

\smallskip\begin{flushleft} \index{Disintegrate}
	\begin{tabularx}{0.45\textwidth}{@{}m{3.5cm}m{5.5cm}@{}} 
		\textbf{Disintegrate} & Range: 60'\\
		Magic-User 6 &Duration: instantaneous\\
	\end{tabularx}\end{flushleft}

This spell causes a green laser-like beam of light to spring from the caster' s pointing finger. Any single creature or object (up to a 10x10x10 foot cube of material) will be completely disintegrated. The target is allowed a save vs. Spells to resist. This spell can target just one creature per casting, so if the target makes its save the spell is wasted.

Equipment worn or carried by a disintegrated creature is not affected, and will naturally fall to the floor as if dropped.


\smallskip\begin{flushleft} \index{Dispel Evil}
	\begin{tabularx}{0.45\textwidth}{@{}m{3.5cm}m{5.5cm}@{}} 
		\textbf{Dispel Evil} & Range: touch\\
Cleric 5 &Duration: 1 round/level\\
	\end{tabularx}\end{flushleft}

This powerful spell aids the caster in dealing with creatures from the nether planes, hereafter called "evil creatures."

First, the caster' s Armor Class is improved by 4 points when attacked by evil creatures; this effect lasts until the spell ends, whether by expiration of the duration or through the employment of one of the following effects.

With a successful attack roll the caster can drive an evil creature back to the nether planes; the caster must of course be engaged with the creature (i.e. in melee range). The creature may make a save vs. Spells to resist. Any successful attack roll by the caster for this purpose ends the spells effects, whether or not the creature saves.

Alternately, by touching the affected creature, item, or area, the caster can immediately dispel any one spell or similar magical effect cast by the evil creature. There is no roll for this effect, except if the target is an unwilling creature in which case an attack roll is needed to actually touch it. The only spells which cannot be ended this way are those which are immune to \textbf{dispel magic}. Successfully performing the touch ends the spell, regardless of whether or not any evil magic is actually ended.

\smallskip\begin{flushleft} \index{Dispel Magic}
	\begin{tabularx}{0.45\textwidth}{@{}m{3.5cm}m{5.5cm}@{}} 
		\textbf{Dispel Magic} & Range: 120'\\
Cleric 4, Magic-User 3 & Duration: instantaneous\\
	\end{tabularx}\end{flushleft}


This spell can be used to end ongoing spells or similar magical effects cast on a creature or object, or to end such effects within a cubic area 20' on a side. The caster must choose whether to dispel magic on a creature or object, or to affect an area.

If dispel magic is targeted at a creature, all spell effects (including ongoing potion effects) may be canceled. If cast upon an area, all such effects within the area may be canceled. Any spell or effect having a caster level equal to or less than the \textbf{dispel magic} caster' s level is ended automatically. Those created by higher level casters might not be canceled; there is a 5\% chance the dispel magic will fail for each level the spell effect exceeds the caster level. For example, a 10\textsuperscript{th} level caster dispelling magic created by a 14\textsuperscript{th} level caster has a 20\% chance of failure.

Some spells cannot be ended by dispel magic; this specifically includes any curse, including those created by \textbf{bestow curse} (the reverse of \textbf{remove curse}) as well as by cursed items.



\smallskip\begin{flushleft} \index{Feeblemind*}
	\begin{tabularx}{0.45\textwidth}{@{}m{3.5cm}m{5.5cm}@{}} 
		\textbf{Feeblemind*} & Range: 180'\\
		Magic-User 5 &Duration: permanent\\
	\end{tabularx}\end{flushleft}

This spell allows the caster to inflict a terrible curse on a living creature, reducing both Intelligence and Charisma to just 1 point each. A saving throw vs. Spells is allowed to resist this effect, but if the target creature is a spellcaster a penalty of -4 is applied to the saving throw.

Once feebleminded, the victim of this spell can no longer cast spells, speak or understand any language, or indeed communicate at all as their mind can no longer understand even such simple things as pointing or beckoning. The victim still knows their friends and allies and will follow them and try to help or protect them.

This effect can be removed with a \textbf{heal} spell, or with the reversed form of this spell \textbf{restoremind}.


\smallskip\begin{flushleft} \index{Find Traps}
	\begin{tabularx}{0.45\textwidth}{@{}m{3.5cm}m{5.5cm}@{}} 
		\textbf{Find Traps} & Range: 30'\\
Cleric 2 &Duration: 3 turns\\
	\end{tabularx}\end{flushleft}


This spell permits the caster to detect a variety of traps, both mechanical and magical. When the caster moves within 30' of a trap, they will see it glow with a faint greenish-blue aura. The caster is not, however, able to detect certain natural hazards such as quicksand, a sinkhole, or unsafe walls of natural rock. The spell also does not bestow the caster with the knowledge needed to disarm the trap, nor any details about its type or nature.

\smallskip\begin{flushleft} \index{Find the Path}
	\begin{tabularx}{0.45\textwidth}{@{}m{3.5cm}m{5.5cm}@{}} 
		\textbf{Find the Path} & Range: touch\\
		Cleric 6 &Duration: 1 turn/level\\
	\end{tabularx}\end{flushleft}

This spell grants the caster, or another subject the caster touches, the ability to find the best and shortest route to a destination selected by the caster. The caster must have some knowledge about the location; any location the caster has ever visited can be so located, as well as locations described to the caster. Even knowing the name of a location (if it has a name) is enough for this spell to function.

This spell works as well indoors or even underground as it does in the outdoors. However, \textbf{find the path} works with respect to locations, not objects or creatures. It can only lead someone to a destination that is on the same plane of existence as the caster when cast; this can be very limiting if the caster (and subject if any) is in a closed or pocket universe of some kind.

The subject can sense the approximate direction of the destination, but also knows the best path to get there, sensing each turning and knowing which paths or corridors to follow when choices appear (but not sooner, unfortunately). The spell even overcomes things such as secret doors, which the subject will sense and know how to open, and grants the knowledge of passwords or codes at the correct moments.

Note that the spell will always choose the best path. For example, if the shortest way requires the use of a key which the subject does not have, the spell will choose a different, probably longer way if possible. If this is not possible, the subject will be led to the locked door and will understand that they must find a way to pass through on their own.

The spell ends when the subject arrives at the destination or when the duration expires, whichever comes first. If the duration expires before arrival, the subject may still remember the approximate direction to the destination but will no longer know the way.


\smallskip\begin{flushleft} \index{Fireball}
	\begin{tabularx}{0.45\textwidth}{@{}m{3.5cm}m{5.5cm}@{}} 
		\textbf{Fireball} & Range: 100'+10'/level\\
Magic-User 3 &Duration: instantaneous\\
	\end{tabularx}\end{flushleft}

Casting this spell causes a tiny glowing ember about the size of a pea to fly forth from the caster' s pointing finger, by which the direction of flight is indicated. The ember flies as fast as an arrow and explodes into flames filling a 20' radius sphere when it reaches a distance chosen by the caster (up to its maximum range), or sooner if it impacts any solid or liquid surface. Those within the area of the flames suffer 1d6 points of damage per caster level, with a saving throw vs. Spells allowed for half damage.

No roll is normally required when casting this spell, but if the caster wishes to project the ember through a slit or other small opening they must roll a missile attack (without range adjustments) to hit it; failure indicates the fireball hits an adjacent surface and detonates there.

Combustible objects or substances within the area of effect will generally catch fire and possibly be destroyed (as determined by the GM). Metals with low melting points such as gold, silver, lead, copper, or bronze may be softened briefly and thus deformed, but the flames do not persist long enough to actually melt the metal items unless they are very small (such as a thin gold neck chain for example).

While the fireball exerts little if any force or pressure when exploding, it may still destroy weak or flammable barriers; if such a thing happens (in the determination of the GM, of course), the fireball completes its expansion in the space beyond the destroyed barrier.

\smallskip\begin{flushleft} \index{Flesh to Stone*}
	\begin{tabularx}{0.45\textwidth}{@{}m{3.5cm}m{5.5cm}@{}} 
		\textbf{Flesh to Stone*} & Range: 30'/level\\
Magic-User 6 & Duration: permanent\\
	\end{tabularx}\end{flushleft}

This spell causes a living creature, along with all gear it wears or carries, to be petrified. This is the same effect caused by the gaze of a medusa or the touch of a cockatrice. A saving throw vs. Petrification is allowed to resist the spell.

The petrified victim becomes dormant; its mind cannot be contacted in any way, and it is neither properly alive nor dead. Damage inflicted on its statue-like form will apply to its fleshy form if it is ever restored; simply breaking off an arm or leg is probably enough to result in death due to blood loss. The GM should determine the effects of any such injury.

The reverse spell, \textbf{stone to flesh}, acts as a counterspell for \textbf{flesh to stone}, restoring the creature just as it was when it was petrified. It does nothing if applied to ordinary stone which is not the result of \textbf{flesh to stone} or a similar petrification effect.

\smallskip\begin{flushleft} \index{Floating Disc}
	\begin{tabularx}{0.45\textwidth}{@{}m{3.5cm}m{5.5cm}@{}} 
		\textbf{Floating Disc} & Range: 0\\
Magic-User 1 &Duration: 5 turns +1/level\\
	\end{tabularx}\end{flushleft}


The casting of this spell causes an invisible platform of magical force to appear. It is about the size of a shield, about 3 feet in diameter and an inch deep at its center. It can support a maximum of 500 pounds of weight. (Note that water weighs about 8 pounds per gallon.)

The disc floats level to the ground, at about the height of the caster' s waist. It remains still when within 10' of the caster, and follows at the caster' s movement rate if they move away from it. The floating disc can be pushed as needed to position it but will be dispelled if somehow moved more than 10 feet from the caster. When the spell duration expires, the disc disappears from existence and drops whatever was supported to the surface beneath.

The disc must be loaded so that the items placed upon it are properly supported, or they will (of course) fall off. For example, the disc can support just over 62 gallons of water, but the water must be in a barrel or other reasonable container that can be placed upon the disc. Similarly, a pile of loose coins will tend to slip and slide about, and some will fall off with every step the caster takes; but a large sack full of coins, properly tied, will remain stable.


\smallskip\begin{flushleft} \index{Fly}
	\begin{tabularx}{0.45\textwidth}{@{}m{3.5cm}m{5.5cm}@{}} 
		\textbf{Fly} & Range: touch\\
		Magic-User 3 &Duration:  1 turn/level\\
	\end{tabularx}\end{flushleft}

The subject of this spell can fly at a speed equal to their normal ground movement rate (as adjusted by encumbrance). The subject can ascend at half their movement rate and descend at twice normal movement rate, with the same maneuverability as the subject has when moving on the ground. Flying under the effect of this spell requires no more concentration than walking, so the subject can attack or cast spells normally. The subject of a fly spell can neither charge nor run, nor carry more weight than their normal maximum load.

If the spell duration expires while the subject is still airborne, the magic fails slowly such that the subject descends at a rate of 120 feet per round for 1d10 rounds. If the subject reaches the ground in that time they land safely; if not, the subject falls the rest of the distance and suffers normal falling damage. However, if this spell is ended by \textbf{dispel magic} or similar outside forces, the subject falls immediately. For any such fall, see \textbf{Falling Damage }on page \hyperlink{falling-damage}{\pageref{falling-damage}} for details of the consequences.

\smallskip\begin{flushleft} \index{Geas*}
	\begin{tabularx}{0.45\textwidth}{@{}m{3.5cm}m{5.5cm}@{}} 
		\textbf{Geas*} & Range: 5' per level\\
		Magic-User 6 &Duration:  special\\
	\end{tabularx}\end{flushleft}

By means of this spell the caster compels a living creature to perform some specific action or services, or alternately to avoid performing some action. The target creature must be able to hear and understand the caster, or it cannot be affected. This spell will automatically fail if used to compel a creature to engage in some obviously self-destructive action.

A saving throw vs. Spells will allow an unwilling target to resist a geas when it is first cast. However, the target may choose to accept the geas, typically as part of a bargain with the caster for some service. Once subjected to this spell the subject must obey the instructions given by the caster indefinitely, though if the geas is to perform some action the spell effectively ends when that action has been completed.

For every 24 hours that the subject chooses not to obey the geas (or is prevented from obeying it), it suffers a penalty of -2 to each of its ability scores, up to a maximum penalty of -8. No ability score can be reduced to less than 3 by this effect. If the subject resumes obeying the geas, all such penalties are removed after 24 hours.

If the task assigned to the subject of this spell is open-ended or otherwise unable to be completed, the subject is still compelled to try to perform the task, but the spell will end in no more than one day per caster level.

Very clever creatures may be able to subvert the instructions given; the GM must decide on the results of any such attempts.

This spell can be ended by \textbf{remove curse} cast by a character of higher level than the caster of the geas, but is never affected by \textbf{dispel magic}.


\smallskip\begin{flushleft} \index{Growth of Animals}
	\begin{tabularx}{0.45\textwidth}{@{}m{3.5cm}m{5.5cm}@{}} 
		\textbf{Growth of Animals} & Range: 60'+10'/level\\
Cleric 3 & Duration: 1 turn/level\\
	\end{tabularx}\end{flushleft}

This spell causes an animal to grow to double its normal size and eight times its normal weight. The affected creature will do double normal damage with all physical attacks, and its existing natural Armor Class increases by 2. The animal' s carrying capacity is also doubled. Unfriendly animals may save vs. Spells to resist this spell; normally, domesticated animals will not attempt to resist it, though they may become confused or panicky afterward (at the GM' s discretion).

This spell does not give the caster control of the animal. Gear worn or carried by the animal are also enlarged but not altered in any other way. If removed from the creature such items resume normal size instantly. Any magical properties of enlarged items are not changed.

\smallskip\begin{flushleft} \index{Growth of Plants*}
	\begin{tabularx}{0.45\textwidth}{@{}m{3.5cm}m{5.5cm}@{}} 
		\textbf{Growth of Plants*} & Range: 120'\\
Magic-User 4 &Duration: permanent\\
	\end{tabularx}\end{flushleft}

This spell causes normal vegetation of any sort within range to become thick and overgrown. The area of effect is determined by the caster, but cannot exceed 1,000 square feet (a 10' x100' area or equivalent) per 5 caster levels. The plants become densely entangled, and characters or creatures wishing to move through must hack or force their way through it. All movement within the affected area is reduced to no more than 5' per round for less than giant sized creatures; giant sized creatures are reduced to half normal movement rate.

This spell cannot take effect in an area that does not already have some plants present. Any sort of animated plant creature affected by this spell is allowed a saving throw vs. Spells to resist, but if this save fails it is affected as with \textbf{growth of animals} above.

The reverse form, \textbf{shrink plants}, may be used to render overgrown areas passable. The area of effect is identical to the normal version. Animated plant creatures are normally unaffected by this spell but if such creatures have already been enlarged by \textbf{growth of plants} this spell will reverse the effect. In the latter case, a saving throw vs. Spells is allowed to resist.

Both forms of this spell are permanent until countered, either by the reverse of the spell or by \textbf{dispel magic}.


\smallskip\begin{flushleft} \index{Hallucinatory Terrain}
	\begin{tabularx}{0.45\textwidth}{@{}m{3.5cm}m{5.5cm}@{}} 
		\textbf{Hallucinatory Terrain} &  Range: 400'+40'/level\\
Magic-User 4  & Duration: 12 turns/level\\
	\end{tabularx}\end{flushleft}

This spell makes one 10 yard cube per level of outdoor terrain appear to be a different type (i.e. field into forest, grassland into desert, or the like). This spell requires a full turn to cast.

The affected terrain looks, sounds, and smells like another type of normal terrain. The magic does not affect creatures nor any sort of fabricated item; such things retain their appearance and visibility. A save vs. Spells is allowed to see through the illusion, but only if the creatures or characters viewing the area actively attempt to do so. 

\smallskip\begin{flushleft} \index{Haste*}
	\begin{tabularx}{0.45\textwidth}{@{}m{3.5cm}m{5.5cm}@{}} 
		\textbf{Haste*} & Range: 30'+10'/level\\
		Magic-User 3  & Duration: 1 round/level\\
	\end{tabularx}\end{flushleft}

This spell accelerates the actions of 1 creature per caster level. The affected creatures move and act twice as fast as normal, having double their normal movement rates and making twice the normal attacks per round, for the duration of the spell. Spellcasting is not accelerated, nor is the use of magic items such as wands, which may still be used just once per round. Multiple haste or speed effects don't combine; only apply the most powerful or longest lasting effect.

In combat, the subject of this spell performs a full round of actions (movement followed by attacking) on their normal Initiative; after all combatants have acted, the subject may perform another round of actions as before. If two or more combatants are under the effect of this spell, use the same Initiative numbers to determine the order of their actions in each of their "extra" rounds.

Reversed, \textbf{haste} becomes \textbf{slow}; affected creatures move at half speed, attacking half as often (generally, every other round) and making half a normal move each round. Naturally, target creatures may save vs. Spells to avoid the effect. \textbf{Haste} and \textbf{slow} counter and dispel each other.

\smallskip\begin{flushleft} \index{Heal*}
	\begin{tabularx}{0.45\textwidth}{@{}m{3.5cm}m{5.5cm}@{}} 
		\textbf{Heal*} & Range: touch\\
Cleric 6 &Duration: permanent\\
	\end{tabularx}\end{flushleft}

This spell enables the caster to almost totally cure wounds, diseases, and other afflictions of a single living creature. With a touch the caster cures blindness, deafness, \textbf{confusion}, disease, \textbf{feeblemind}, choking, nausea, and poison. All but 1d4 hit points are restored to the subject, if it has any injuries. Ability points losses are fully restored, if and only if those losses would be recoverable without this spell; permanent losses to ability scores are not recovered. Energy drain is also not affected by this spell.

The reversed spell, \textbf{harm}, injures the creature touched so horribly that it is left with only 1d4 hit points. The caster must succeed at a normal attack roll in this case; failure means the spell is wasted. Note that, if the victim has fewer hit points remaining than the number rolled, they will take at least one point of damage (and this is the only case in which \textbf{harm} may kill a creature).

\textbf{Heal} and \textbf{harm} normally only affect living creatures. If applied to an undead creature, the effects are reversed such that \textbf{heal} becomes \textbf{harm} and vice versa.


\smallskip\begin{flushleft} \index{Hold Monster}
	\begin{tabularx}{0.45\textwidth}{@{}m{3.5cm}m{5.5cm}@{}} 
		\textbf{Hold Monster} & Range: 180'\\
Magic-User 5 &Duration: 2d8 turns\\
	\end{tabularx}\end{flushleft}

This spell functions like \textbf{hold person}, except that it affects any living creature that fails its save vs. Spells.


\begin{flushleft}
	\includegraphics[width=0.47\textwidth]{Pictures132/10000000000003D8000003C062C9EF18F79A1F0C.png}
\end{flushleft}


\smallskip\begin{flushleft} \index{Hold Person}
	\begin{tabularx}{0.45\textwidth}{@{}m{3.5cm}m{5.5cm}@{}} 
		\textbf{Hold Person} & Range: 180'\\
Cleric 2, Magic-User 3 & Duration: 2d8 turns\\
	\end{tabularx}\end{flushleft}

This spell will render any living (not undead) humanoid creature (as defined in \textbf{charm person}) paralyzed. Creatures larger than ogres will not be affected by this spell. Targets of the spell are conscious and able to breathe, but cannot move, act, or speak in any way. A successful save vs. Spells is allowed to resist the effect. The spell may be cast at a single person, who makes their save at -2, or at a group, in which case 1d4 of the creatures in the group may be affected.

A paralyzed swimmer can't swim and may drown. A character who has somehow gained wings and is then paralyzed while airborne will not be able to move its wings to fly and will thus fall.


\smallskip\begin{flushleft} \index{Hold Portal}
	\begin{tabularx}{0.45\textwidth}{@{}m{3.5cm}m{5.5cm}@{}} 
		\textbf{Hold Portal} & Range: 100'+10'/level\\
Magic-User 1 & Duration: 1 round/level\\
	\end{tabularx}\end{flushleft}

This spell secures a portal such as a door, gate, window, or shutter made of normal non-magical building materials; the portal behaves as if securely locked for the duration of the spell. The door may be opened early only by means of \textbf{knock} or a successful casting of \textbf{dispel magic}, or by literally destroying the door (which may well require more time than the duration of this spell allows).



\smallskip\begin{flushleft} \index{Ice Storm}
	\begin{tabularx}{0.45\textwidth}{@{}m{3.5cm}m{5.5cm}@{}} 
		\textbf{Ice Storm} & Range: 300'+30'/level\\
		Magic-User 4 & Duration: 1 round\\
	\end{tabularx}\end{flushleft}

This spell causes a powerful storm of sleet and hail to fall in a 20' radius around the target spot for a full round. This effect causes 5d6 points of damage to every creature within the area, with a save vs. Spells allowed to reduce damage by half. The ice storm fills a vertical volume of 40', so creatures higher than that distance above the target spot are unaffected. Any creature naturally resistant to cold takes half damage (or one-quarter damage if it makes its save).

Visibility within the storm is very poor, such that most creatures will attack with a penalty of -2 on the attack roll. Walking movement is reduced to half speed, and running is not possible within the storm; anyone who tries will fall down and be forced to remain on the ground until the spell ends. Flying is likewise impossible, and any flying creature within the area will be driven to the ground. A penalty of -20\% is applied to any Listen rolls made by those in the storm' s area.

When the spell ends, the hail and ice deposited by the storm disappears as if it had never existed.

\smallskip\begin{flushleft} \index{Insect Plague}
	\begin{tabularx}{0.45\textwidth}{@{}m{3.5cm}m{5.5cm}@{}} 
		\textbf{Insect Plague} & Range: 300'+30'/level\\
Cleric 5 &Duration: 1 round/level\\
	\end{tabularx}\end{flushleft}

This spell summons one swarm of locusts per three caster levels, with a maximum of 6 swarms. See \textbf{Insect Swarm} on page \hyperlink{insect-swarm}{\pageref{insect-swarm}} for the effects of a swarm. As explained there, a normal swarm of insects occupies a contiguous area equal to three 10 foot cubes; all swarms summoned by this spell must be contiguous with each other such that they effectively form a single huge swarm, though the caster may stretch them out in a line, form them into a block, or indeed arrange them in some serpentine form if desired.

The summoned swarms persist for the duration above, or until they are slain, whichever comes first. The caster may summon them in areas where creatures are already located. Once summoned, the swarms are stationary until they disappear or are slain, and they attack any creatures who are within their area.


\smallskip\begin{flushleft} \index{Invisibility}
	\begin{tabularx}{0.45\textwidth}{@{}m{3.5cm}m{5.5cm}@{}} 
		\textbf{Invisibility} & Range: touch\\
Magic-User 2 & Duration: special\\
	\end{tabularx}\end{flushleft}

This spell causes the creature touched (who may be the caster) to become invisible, undetectable by normal vision or Darkvision. Invisible creatures may be detected by those with non-visual sensory abilities.

All items worn or carried by the target when the spell is cast become invisible as well. If the target lays down an invisible item, it instantly becomes visible again; on the other hand, items picked up do not automatically become invisible. If the target places a visible item entirely inside its invisible clothing, backpack, pouch, or other container so that if the target were visible the item could not be seen, it will become invisible just as if it were held when the spell was cast.

Note that casting this spell upon another makes the target invisible to the caster as well as everyone else. A party of invisible characters will likely experience problems with running into or tripping over each other.

The spell lasts at most 24 hours if not ended sooner. It ends instantly if the target attacks an opponent or casts any spell. Other actions do not normally end the spell. The target may end the spell at will. 

Objects that shed light may be made invisible but the light itself will remain visible, and the source of that light (and the character carrying it) can thus be discovered. Rain, dust, paint, and any other visible substance thrown or applied to an invisible creature will also render it detectable.

\smallskip\begin{flushleft} \index{Invisibility 10' Radius}
	\begin{tabularx}{0.45\textwidth}{@{}m{3.5cm}m{5.5cm}@{}} 
		\textbf{Invisibility 10' Radius} & Range: touch\\
		Magic-User 3 & Duration:  1 turn/level\\
	\end{tabularx}\end{flushleft}

The target of this spell becomes invisible just as with \textbf{invisibility}, but in addition all other creatures within 10 feet of the target at the time of casting (referred to as members of the group) also become invisible. Unlike with normal invisibility, all members can see each other and themselves normally.

If any member negates the effect by attacking or casting a spell, it becomes visible but all other members remain invisible. However, if the original subject takes an action that negates the invisibility, it ends for the entire group. If a member negates its own invisibility, that member will no longer be able to see the still-invisible group members. Moving further than 10 feet from the target also causes a group member to become visible; returning to the area does not reinstate the invisibility.


\smallskip\begin{flushleft} \index{Invisible Stalker}
	\begin{tabularx}{0.45\textwidth}{@{}m{3.5cm}m{5.5cm}@{}} 
		\textbf{Invisible Stalker} & Range: 0\\
		Magic-User 6 & Duration: special\\
	\end{tabularx}\end{flushleft}

The caster summons an invisible stalker to do their bidding. The spell persists until \textbf{dispel evil} is cast on the creature, it is slain, or the task is fulfilled. The GM is advised to review the monster entry for the invisible stalker (found on page \hyperlink{invisible-stalker}{\pageref{invisible-stalker}}) when this spell is used, as they may not always perform reliably.

\smallskip\begin{flushleft} \index{Knock}
	\begin{tabularx}{0.45\textwidth}{@{}m{3.5cm}m{5.5cm}@{}} 
		\textbf{Knock} & Range: 30'\\
		Magic-User 2 & Duration: special\\
	\end{tabularx}\end{flushleft}

This spell can undo a single means of securing a door, chest, or shackle. It negates \textbf{hold portal}, dispelling it; a \textbf{wizard lock} spell is deactivated for 1 turn but is not dispelled. A lock can be opened, a stuck door forced, a door barred or bolted can be unbarred or unbolted. These effects do not automatically restore themselves, except for the \textbf{wizard lock}. Knock cannot raise a portcullis or open any other portal which requires a mechanism to open (such as a winch for example).

Each casting undoes a single method of securing the door or item, so that for example a door which is both locked and stuck can be unlocked but will still be stuck; a second casting of \textbf{knock} will open such a door.

\smallskip\begin{flushleft} \index{Levitate}
	\begin{tabularx}{0.45\textwidth}{@{}m{3.5cm}m{5.5cm}@{}} 
		\textbf{Levitate} & Range: touch\\
		Magic-User 2 & Duration: 1 turn/level\\
	\end{tabularx}\end{flushleft}

When this spell is cast, the caster or any one creature or object they touch begins to levitate. Levitation allows the caster to cause the subject to rise or fall at a rate of as much as 20 feet per round. 

If a target creature is unwilling, the caster must roll a successful attack roll in order to touch the target, and even if this roll succeeds the target is allowed a save vs. Spells to resist the magic. Likewise, if the target is an object being held by an unwilling creature, a similar attack roll and saving throw will be required. Unattended objects are automatically affected when touched.

The caster controls the levitating subject mentally. If the caster ceases to control the subject, it remains at its current altitude until the spell ends, at which point it falls.

Levitation does not provide any means of horizontal movement, but neither is the levitated subject fixed in place. It can be moved by pushing or by towing with a rope, for example. A creature being levitated can pull itself along a cliff face or across a ceiling, or pull itself with a rope fixed to a solid object. Movement of this sort is usually at half normal walking speed. The subject cannot, however, change its own altitude.

If a levitating creature attempts to attack with most weapons it will find that it is unstable. The first such attack is made at a penalty of -1 on the attack roll, and this penalty worsens by an additional -1 each round to a maximum penalty of -5. Should the creature refrain from making an attack for a full round, it is reduced to the -1 penalty level on the next round. This does not usually affect spellcasting while levitated, however.


\smallskip\begin{flushleft} \index{Light*}
	\begin{tabularx}{0.45\textwidth}{@{}m{3.4cm}m{5.5cm}@{}} 
		\textbf{Light*} & Range: 120'\\
Cleric 1, Magic-User 1 & Duration: 6 turns\\
&+ 1/level\\
	\end{tabularx}\end{flushleft}

This spell creates a light equal to torchlight which illuminates a 30' radius area well (with dim light extending for an additional 20') around the target location or object. This effect is stationary when cast in an area, but it can be cast on a movable object or even onto a character or creature.

Reversed, \textbf{light} becomes \textbf{darkness}, creating an area of darkness just as described above. This darkness blocks out Darkvision and negates mundane light sources. Wherever both spells overlap they cancel out, leaving only normal illumination in the overlapping area. 

A light spell may be cast to dispel the darkness spell of an equal or lower level caster (and vice versa), leaving neither spell active; likewise, a darkness spell can cancel the light spell of an equal or lower level caster.

Either version of this spell may be used to blind an opponent by means of casting it on the target' s ocular organs. The target is allowed a saving throw vs. Death Ray to avoid the effect, and if the save is made the spell does not take effect at all. A \textbf{light} or \textbf{darkness} spell cast to blind does not have the given area of effect (that is, no light or darkness is shed around the victim).

\medskip

\begin{flushleft}
\includegraphics[width=0.47\textwidth]{Pictures132/10000000000003D8000005D99FEE0CF60942656A.png}
\end{flushleft}

\smallskip\begin{flushleft} \index{Lightning Bolt}
	\begin{tabularx}{0.45\textwidth}{@{}m{3.5cm}m{5.5cm}@{}} 
		\textbf{Lightning Bolt} & Range: 100'+10'/level\\
Magic-User 3 &Duration: instantaneous\\
	\end{tabularx}\end{flushleft}

Casting this spell causes a bright thin spark about the thickness of a string to fly forth from the caster' s pointing finger, which indicates the direction. The spark stretches as fast as an arrow' s flight until it reaches the caster' s chosen distance (but not more than the range given above) or strikes a solid or liquid surface, at which point it explodes into a full-fledged bolt of lightning extending another 60 feet further. The lightning bolt passes through an area 5 feet wide, arcing and jumping, so that while it is not actually 5 feet wide, for game purposes it is treated as if it were.

Those within the area of the lightning bolt suffer 1d6 points of damage per caster level, with a saving throw vs. Spells allowed for half damage.

If a lightning bolt is targeted at a body of water it explodes at the point of impact, as described above, but within the water volume struck it expands like a \textbf{fireball} to a maximum radius of 20 feet instead of performing as it does in the air. If a lightning bolt is cast underwater it explodes in the same way, but at the tip of the caster' s outstretched finger.

The lightning bolt sets fire to combustibles and damages objects in its path. It can damage metals with a low melting point; metals such as gold, silver, lead, copper, or bronze may be softened briefly and thus deformed, but the lightning does not persist long enough to actually melt the metal items unless they are very small (such as a thin gold neck chain for example).

If the damage caused to an interposing barrier shatters or breaks through it, the bolt may continue beyond the barrier if the spell's range permits; otherwise, it may reflect from the barrier back toward the caster, or in a random direction at the GM' s option. Creatures already affected by the lightning bolt do not take additional damage if struck by the reflection of the same bolt.



\smallskip\begin{flushleft} \index{Locate Object}
	\begin{tabularx}{0.45\textwidth}{@{}m{3.5cm}m{5.5cm}@{}} 
		\textbf{Locate Object} & Range: 360'\\
Cleric 3, Magic-User 2 &Duration: 1 round/level\\
	\end{tabularx}\end{flushleft}

This spell grants the caster knowledge of the location of an object. The caster must know the object well or be able to clearly imagine it. (Viewing an accurate drawing or painting will suffice for the latter option.) A general item can be located; if more than one such item is in range, the spell will lead the caster to the nearest one.

Unique or unusual items can only be located if the caster has first-hand knowledge (not merely through divination such as \textbf{clairvoyance} or a \textbf{crystal ball}). The spell cannot be used to locate creatures of any sort. A layer of lead or gold no thicker than foil surrounding the item will prevent it from being located.


\smallskip\begin{flushleft} \index{Lower Water}
	\begin{tabularx}{0.45\textwidth}{@{}m{3.5cm}m{5.5cm}@{}} 
		\textbf{Lower Water} & Range: 20'/level\\
Magic-User 6 & Duration: 1 turn/level\\
	\end{tabularx}\end{flushleft}

Using this spell the caster lowers the level of the water in a river, a lake, or even the sea by up to 2 feet per caster level for the given duration, but to no less than 1 inch deep. The spell affects an area with radius of at most 10 feet per level centered on the caster' s chosen location (within the range given).

There will be a steep slope up to the surface of any area of un-lowered water that falls outside the spell' s radius. Ships which were already in the area of effect as well as any that dare to enter it will be unable to climb the slope and thus unable to leave the area (if not outright beached by the spell in the first place).

This spell has the effect of \textbf{slow} (the reverse of \textbf{haste}) when cast upon water elementals and other creatures formed from water; a save vs. Spells is allowed, with success negating the effect. It cannot be cast upon any other kind of creatures.


\smallskip\begin{flushleft} \index{Magic Jar}
	\begin{tabularx}{0.45\textwidth}{@{}m{3.5cm}m{5.5cm}@{}} 
		\textbf{Magic Jar} & Range: 60'\\
Magic-User 5 &Duration: special\\
	\end{tabularx}\end{flushleft}

This spell allows the caster to attempt to possess the body of another living creature. The caster begins by placing their spirit into a gem, jewel, or large crystal of some sort within the spell range, called the \emph{magic jar}. The caster needs to know where the magic jar is located, but does not have to be able to see it. While the caster' s spirit is outside their body, that body appears to all intents and purposes to be dead, but does not undergo decay as a normal dead body would.

Each round after entering the magic jar the caster' s spirit can attempt to take control of the body of a living creature within the spell range; the target is allowed a save vs. Spells to resist. If the saving throw fails, the spirit of the target (now called the \emph{host}) is trapped in the magic jar and the caster' s spirit takes possession of the host' s body. Possession of a creature by means of this spell is blocked by \textbf{protection from evil }or a similar ward. 

If the attempt to possess a victim fails, the caster' s spirit remains in the magic jar, and that target creature is immune to further attempts for the duration of the spell. The caster may make an attempt to possess another target on the following round, if desired.

If on the other hand the possession attempt is successful, the caster may remain in control of the host' s body for as long as desired. The caster' s spirit may return to the magic jar at any time when it is within spell range, restoring the host' s spirit to its own body. The caster may not return to the same host' s body again for the remainder of the duration of this spell, but may attempt to possess it again on a subsequent casting.

Whenever the caster' s spirit is in the magic jar, it may choose to return to its own body if it is within spell range. If the body is not in range, the caster' s spirit may become trapped if no vulnerable creatures are in range.

Whenever the caster' s spirit returns to their own body, the spell ends. There is a 50\% chance that the magic jar will shatter when the spell ends (if the spell did not end because it was broken), becoming worthless and unusable.

When the caster' s spirit possesses a host, the caster has access to the physical abilities of the host' s body, including Strength, Dexterity, and Constitution, while still retaining their own Intelligence, Wisdom, and Charisma. The caster has access only to their own knowledge, class and level, attack bonus, saving throws, spell casting ability, and any other purely mental capabilities.

The caster does not gain access to supernatural or otherwise extraordinary powers of the host, and may not be able to perform physical actions that the caster has never done before (such as flying, if a winged body has been possessed). The caster does have access to the host' s sensory capabilities, such as Darkvision, the enhanced sense of smell of a dog, and so on.

If the caster' s spirit is in the magic jar, and the jar is broken (whether by \textbf{dispel magic} or physical damage), the caster' s spirit will return to its own body if it is in spell range; if not, the caster' s spirit departs (i.e. the caster dies). In either case, the spell ends.

If the caster' s spirit is driven from the host body by \textbf{dispel evil}, and the magic jar is in range of the host body, the caster' s spirit returns to the jar and the host' s spirit returns to its body. The caster will not be able to possess the same host again for the remaining duration of the spell. If the magic jar is not in range of the host body, the caster' s spirit departs, the host' s spirit is freed from the jar and also departs, and the
host' s body dies. 

If the host' s spirit is in the magic jar, and the jar is broken while in spell range of the host body, the caster' s spirit departs, the host' s spirit returns to its body, and the spell ends. If the jar is broken while out of range of the host' s body, the host' s spirit departs, the caster' s spirit is stranded in the host body. Note here that the spell has not ended. \textbf{Dispel evil} can still be used to drive the caster' s spirit from the body, which departs as noted, ending the spell. A stranded caster may use another casting of this spell (with another magic jar, of course) to return to their own body, which of course kills the host' s body.

\smallskip\begin{flushleft} \index{Magic Missile}
	\begin{tabularx}{0.45\textwidth}{@{}m{3.5cm}m{5.5cm}@{}} 
		\textbf{Magic Missile} & Range: 100'+10'/level\\
Magic-User 1 & Duration: instantaneous\\
	\end{tabularx}\end{flushleft}

This spell causes a magical arrow of energy to fly from the caster' s finger and unerringly hit its target, inflicting 1d6+1 points of damage. The target must be at least partially visible to the caster, and no saving throw is normally allowed. It' s not possible to target a specific part of the target. Inanimate objects are not affected by this spell.

For every three caster levels beyond 1\textsuperscript{st}, an additional missile is fired: two at 4\textsuperscript{th} level, three at 7\textsuperscript{th}, four at 10\textsuperscript{th}, and the maximum of five missiles at 13\textsuperscript{th} level or higher. When multiple missiles are fired in this way, the caster can target one or several creatures as desired, as long as all are visible to the caster at the same time. All such targets must be designated before any damage is rolled.

\smallskip\begin{flushleft} \index{Magic Mouth}
	\begin{tabularx}{0.45\textwidth}{@{}m{3.5cm}m{5.5cm}@{}} 
		\textbf{Magic Mouth} & Range: 30'\\
Magic-User 1 & Duration: special\\
	\end{tabularx}\end{flushleft}

This spell places a simple form of programmed illusion on a non-living object within range. When triggered, the spell causes the illusion of a mouth to appear on the object and a message to be said aloud. The enchantment can remain in place indefinitely, but is expended when triggered (i.e. the message is normally delivered only once).

The message recounted may be up to three words per caster level in length. The caster may insert pauses in the message, but the entire message must be delivered in a time period of no more than a turn. The voice of the spell can be made to speak at any volume attainable by a normal human. It will sound enough like the caster' s own voice to be recognized by a close associate of the caster, but not identical.

The illusionary mouth moves as if actually speaking the message being delivered, and remains visible during pauses. If placed on an artistic depiction of a creature with a mouth (such as a painting or statue), the spell can be made to appear to animate the mouth of the object.

This spell cannot be used to activate magic items which have command words, nor to activate any other magical effects.

The caster must choose the conditions under which this spell is triggered. The conditions may be as complicated or simple as desired, but must depend only on sight and hearing; the spell has no other sensory capabilities. The spell also has no particular intelligence, and can be fooled by disguises or illusions. The spell does have the capability to effectively see in normal darkness, but not in any sort of magical darkness, and it cannot detect invisible creatures nor see through doors, walls, or even opaque curtains. Likewise, stealth or magical silence are effective in preventing audible triggers. Finally, the spell cannot detect a character' s class, level, ability scores, or any other feature not obvious to a normal NPC.

Triggers have an effective sensory range of 10 feet per caster level; sounds, sights, or actions outside that range will never trigger the spell.


\smallskip\begin{flushleft} \index{Massmorph}
	\begin{tabularx}{0.45\textwidth}{@{}m{3.5cm}m{5.5cm}@{}} 
		\textbf{Massmorph} & Range: 100'+10'/level\\
Magic-User 4 &Duration: 1 hour/level\\
	\end{tabularx}\end{flushleft}

With this spell the caster causes 1d4+1 man-sized (or smaller) creatures per four caster levels to appear as if they are natural effects of the terrain (for example, trees in a forest, stalagmites in a cave, coral underwater, boulders in a cavern, etc.). All creatures to be affected must be within a 120' radius of the caster at the time the spell is cast. Only those creatures the caster wishes to hide are affected, and then only if they are willing to be concealed. The caster may choose to be included among the affected creatures.

Those affected are thus concealed from other creatures passing through the area for so long as they remain still. If an affected creature chooses to move or attack, the illusion is dispelled for that creature, but those who remain still continue to be hidden. The caster may end the spell early if they wish by speaking a single word. The illusion can also be ended by \textbf{dispel magic}.


\smallskip\begin{flushleft} \index{Mind Reading}
	\begin{tabularx}{0.45\textwidth}{@{}m{3.5cm}m{5.5cm}@{}} 
		\textbf{Mind Reading} & Range: 60'\\
Magic-User 2 & Duration: 1 turn/level\\
	\end{tabularx}\end{flushleft}


This spell, sometimes incorrectly called \textbf{ESP}, permits the caster to hear (and possibly see, if the target visualizes anything) the surface thoughts of one or more living creatures within range. The caster must designate a direction or select a visible target, and then concentrate for a turn in order to "hear" the thoughts. Each turn the caster may choose to "listen" in a different direction. The caster may stop listening, then resume again later, so long as the duration has not expired. The target creature is not normally aware of being spied upon in this way, though any creature already under the effect of this spell or any similar form of telepathy will instantly know and be able to sense the direction of the caster.

Rock more than 2 inches thick or even a very thin covering of lead or gold will block the spell. All undead creatures are immune to this effect, as are mindless creatures and constructs such as golems.


\smallskip\begin{flushleft} \index{Mirror Image}
	\begin{tabularx}{0.45\textwidth}{@{}m{3.5cm}m{5.5cm}@{}} 
		\textbf{Mirror Image} & Range: self\\
		Magic-User 2 & Duration: 1 turn/level\\
	\end{tabularx}\end{flushleft}

This spell allows the caster to create multiple illusory duplicates (called \emph{figments}) which seem to swirl and move around and through each other as well as the caster more or less constantly, making it impossible for most creatures to determine which is the real one. A total of 1d4 images plus one image per three caster levels (maximum eight images total) are created.

The figments mimic the caster' s actions, going through the motions of casting spells, drinking potions, levitating, and so on, just as the caster does. Figments always look exactly like the caster.

Any opponent who attacks or casts spells directly on the caster will always hit a figment instead. Attacking a figment destroys it, whether or not the attack roll is successful, as does any attack spell directed at one. Area-effect spells are not cast directly on the caster, and thus appear to affect all figments exactly as they affect the caster; for instance, if the caster is subjected to a \textbf{fireball}, all figments will appear to be injured just as the caster was.


\smallskip\begin{flushleft} \index{Neutralize Poison*}
	\begin{tabularx}{0.45\textwidth}{@{}m{3.4cm}m{5.5cm}@{}} 
		\textbf{Neutralize Poison*} & Range: touch\\
Cleric &4 Duration: instantaneous\\
	\end{tabularx}\end{flushleft}

This spell neutralizes any poison or venom in the creature or object touched. A creature suffering from poison or venom suffers no further effects from it.

If cast upon a creature slain by poison in the last 10 rounds, the creature is revived with 1 hit point. If cast upon a poisonous object (weapon, trap, etc.) the poison is rendered permanently ineffective.

Reversed, this spell becomes \textbf{poison}. The caster must make a successful attack roll; if the attack is a success, the target must save vs. Poison or die. The caster' s touch remains poisonous for 1 round per caster level, or until discharged (i.e. only one creature can be affected by the reversed spell).


\smallskip\begin{flushleft} \index{Passwall}
	\begin{tabularx}{0.45\textwidth}{@{}m{3.5cm}m{5.5cm}@{}} 
		\textbf{Passwall} & Range: 30'\\
Magic-User 5 &Duration: 3 turns\\
	\end{tabularx}\end{flushleft}

Passwall creates a passage through wooden, plaster, or stone walls, but not through metal or other harder materials. The passage is up to 10 feet deep plus an additional 10 feet per three caster levels above 9\textsuperscript{th} (20 feet at 12\textsuperscript{th}, 30 feet at 15\textsuperscript{th}, 40 feet at 18\textsuperscript{th}). If the wall's thickness is more than the depth of the passage created, then a single passwall simply makes a niche or short tunnel. Several passwall spells can then form a continuing passage to breach very thick walls. When passwall ends (due to duration, \textbf{dispel magic}, or caster' s choice), creatures within the passage are ejected out the nearest exit.

\smallskip\begin{flushleft} \index{Phantasmal Force}
	\begin{tabularx}{0.45\textwidth}{@{}m{3.5cm}m{5.5cm}@{}} 
		\textbf{Phantasmal Force} & Range: 180'\\
Magic-User 2 &Duration: concentration\\
	\end{tabularx}\end{flushleft}

With this spell the caster visualizes and projects the illusion of an object, creature (or small group of creatures), or other effect. The caster can project an illusion up to a maximum size of 20' x20' x20'. The illusion is purely visual, with no other sensory features. The image is not static, but can be animated as the caster desires so long as all images remain within the area of effect. The illusion persists so long as the caster concentrates upon it.

If used to create the illusion of one or more creatures, they will have an Armor Class of 11 and will disappear if hit in combat. Damage done by monsters, spells, etc. simulated by this spell is not real; those "killed" or otherwise apparently disabled will wake up uninjured (at least by this spell) after 2d8 rounds. The illusory damage done will be equal to the normal damage for any attack form simulated.

Attempting to animate more creatures than the caster' s level grants viewing creatures with at least average Intelligence an immediate save vs. Spells to recognize the creatures as illusions; those making the save will be unaffected by any actions taken by the illusions from that point on. A similar save may be granted by the GM any time they feel the illusion is likely to be seen through, especially if the player describes an illusion which seems improbable or otherwise poorly conceived.

\smallskip\begin{flushleft} \index{Polymorph Other}
	\begin{tabularx}{0.45\textwidth}{@{}m{3.5cm}m{5.5cm}@{}} 
		\textbf{Polymorph Other} & Range: 30'\\
		Magic-User 4 &Duration: permanent\\
	\end{tabularx}\end{flushleft}

This spell allows the caster to change one living creature which is not incorporeal or gaseous into another form of living creature. The assumed form can't have more hit dice than caster's level, or be incorporeal or gaseous. Unlike \textbf{polymorph self}, the transformed target also gains the behavioral and mental traits, any physical attacks, special, supernatural or spell-like abilities of the new form, in addition to the physical capabilities and statistics of such. If the new form is substantially less intelligent, the target may not remember its former life.

The target creature will have the same number of hit points it previously had, regardless of the hit dice of the form assumed. A creature with the ability to transform or change shape such as a doppleganger is changed, but can assume a different form after a single round.

Equipment worn or carried will be dropped if the new form is unable to wear or carry the items. If any such items would be constricting or physically harmful to the new form, the transformation slows and alters such that they are dropped without damage to the items nor harm to the target creature. If the GM determines that any items cannot be removed in this way, they must decide on the exact results.

Unwilling targets which successfully save vs. Paralysis are not affected. The spell is permanent until dispelled or the creature is slain, at which time the target resumes its original form.

\smallskip\begin{flushleft} \index{Polymorph Self}
	\begin{tabularx}{0.45\textwidth}{@{}m{3.5cm}m{5.5cm}@{}} 
		\textbf{Polymorph Self} & Range: self\\
Magic-User 4 & Duration: 1 hour/level\\
	\end{tabularx}\end{flushleft}

This spell allows the caster to change into a different form of living creature. The form assumed may not have more hit dice than the caster has levels, nor be incorporeal or gaseous.

The caster assumes the physical nature of the assumed form while retaining their mental and spiritual characteristics. They gain the Armor Class and all physical attacks possessed by the form but does not gain any special, supernatural or spell-like abilities. Dragon breath is a special ability, for instance, so were the caster to assume the form of a dragon they could use the dragon' s normal claw, bite, and tail swipe attacks, but not the dragon' s breath.

If the form assumed is capable of speaking and making appropriate gestures (as determined by the GM) the caster may use their own spells in the assumed form.

Equipment worn or carried will be dropped if the new form is unable to wear or carry the items. If any such item would be constricting or physically harmful to the new form, the transformation slows and alters such that the item is dropped without damage to the items nor harm to the target creature. If the GM decides that any such item cannot be removed in this way, the spell fails.

The caster can remain transformed up to one hour per level of ability, or may choose to end the spell before that point if they wish.

\smallskip\begin{flushleft} \index{Projected Image}
	\begin{tabularx}{0.45\textwidth}{@{}m{3.5cm}m{5.5cm}@{}} 
		\textbf{Projected Image} & Range: 240'\\
Magic-User 6 & Duration: 6 turns\\
	\end{tabularx}\end{flushleft}

This spell creates a quasi-real, illusory version of the caster. This illusory projected image looks, sounds, and smells like the caster, in addition to mimicking gestures and actions (including speech, which is projected from the caster to the illusory image as if by a form of \textbf{ventriloquism}). Any further spells cast seem to originate from the illusion, not the actual caster.

A line of sight between the caster and their illusory self must be maintained or the spell ends. Any effect or action that breaks the line of sight dispels the image, as does the illusionary caster being struck in combat. Note that this spell grants no special sensory powers to the caster; for example, if the illusory self is positioned so as to be able to see something the caster can' t directly see, the caster does not see it. Also, all spell ranges are still figured from the caster' s actual position, not the illusory self' s position.

\smallskip\begin{flushleft} \index{Protection from Evil*}
	\begin{tabularx}{0.45\textwidth}{@{}m{3.5cm}m{5.5cm}@{}} 
		\textbf{Protection from Evil*} & Range: touch\\
Cleric 1, Magic-User 1 & Duration: 1 turn/level\\
	\end{tabularx}\end{flushleft}

This spell protects the caster or a creature touched by the caster (the "subject") from evil; specifically, the spell wards against summoned creatures, creatures with significantly evil intentions, and extraplanar creatures of evil nature. A magical barrier with a radius of just 1 foot is created around the subject. The barrier moves with the subject, and provides three specific forms of magical protection against attacks or other effects attempted by the affected creatures against the subject. 

First, the subject receives a bonus of +2 to their Armor Class, and a similar bonus of +2 on all saving throws.

Second, the barrier blocks all attempts to \textbf{charm }or otherwise control the subject, or to possess the subject (such as with \textbf{magic jar}). Such attempts simply fail during the duration of this spell. Note however that a creature who receives this protection \emph{after} being possessed is not cured of the possession.

Third, any and all summoned creatures and extraplanar creatures of evil nature are unable to physically touch the subject. Attacks by such creatures using their natural weapons simply fail. This effect is canceled if the subject performs any form of physical attack (even with a ranged weapon) on any affected creature, but the other features of the spell continue in force.

Reversed, this spell becomes \textbf{protection from good}. It functions in all ways as described above, save that "good" creatures are kept away, rather than "evil" ones.


\smallskip\begin{flushleft} \index{Protection from Evil 10' Radius*}
	\begin{tabularx}{0.45\textwidth}{@{}m{4.7cm}m{5.5cm}@{}} 
		\textbf{Protection from Evil 10' Radius*} & Range: touch\\
Cleric 4, Magic-User 3 & Duration: 1 turn/level\\
	\end{tabularx}\end{flushleft}

This spell functions exactly as \textbf{protection from evil}, but with a 10' radius rather than a 1' radius. All within the radius receive the protection; those who leave and then re-enter, or who enter after the spell is cast, receive the protection as well.

Reversed, this spell becomes \textbf{protection from good 10' radius}, and functions exactly as the reversed form of \textbf{protection from evil}, except that it covers a 10' radius around the target rather than the normal 1' radius.

\smallskip\begin{flushleft} \index{Protection from Normal Missiles}
	\begin{tabularx}{0.45\textwidth}{@{}m{4cm}m{5.5cm}@{}} 
		\textbf{Protection from Normal Missiles} & Range: self\\
Magic-User 3 & Duration: 1 turn/level\\
	\end{tabularx}\end{flushleft}

The caster is completely protected from small sized, non-magical missile attacks. Therefore, magic arrows, hurled boulders, or other such are not blocked, but any number of normal arrows, sling bullets, crossbow bolts, thrown daggers, etc. will be fended off. Note that normal missiles projected by magic bows count as magical missiles for the purposes of this spell.



\smallskip\begin{flushleft} \index{Purify Food and Water}
	\begin{tabularx}{0.45\textwidth}{@{}m{4cm}m{5.5cm}@{}} 
		\textbf{Purify Food and Water} & Range: 10'\\
		Cleric 1 & Duration: instantaneous\\
	\end{tabularx}\end{flushleft}

With this spell the caster makes contaminated food or water pure and safe to eat or drink. Poison is neutralized and spoilage is reversed by this spell. The spell does not protect against future decay, however, nor does it affect magic potions (including, unfortunately, \textbf{Potions of Poison}). Unholy water, if it exists in your campaign, is ruined by the casting of this spell. The spell affects about 2 pounds of food and/or drink per caster level; note that a quart of water or similar drink weighs just over 2 pounds.


\smallskip\begin{flushleft} \index{Quest*}
	\begin{tabularx}{0.45\textwidth}{@{}m{3.5cm}m{5.5cm}@{}} 
		\textbf{Quest*} & Range: 5'/level\\
Cleric 5 &Duration: special\\
	\end{tabularx}\end{flushleft}

By means of this spell the caster compels a living creature to perform some specific action or services, or alternately to avoid performing some action. The target creature must be able to hear and understand the caster, or it cannot be affected. This spell will automatically fail if used to compel a creature to engage in some obviously self-destructive action.

A saving throw vs. Spells will allow an unwilling target to resist a quest when it is first cast. However, the target may choose to accept the quest, typically as part of a bargain with the caster to perform some service. Once subjected to this spell, the subject must obey the instructions given by the caster indefinitely, though if the quest is to perform some action the spell effectively ends when that action has been completed.

For every 24 hours that an affected creature chooses not to obey the quest (or is prevented from obeying it), it suffers 3d6 points of damage. This damage is limited, in that it will not kill the target; if the damage is enough to do so, roll 1d4 for the number of hit points the affected creature retains (similar to the spell \textbf{harm}, the reverse of \textbf{heal}).


\begin{flushleft}
	\includegraphics[width=0.47\textwidth]{Pictures132/10000000000003D80000032ECB76FE75AC098DB9.png}
\end{flushleft}

If the task assigned to the subject of this spell is open-ended or otherwise unable to be completed, the subject is still compelled to try to perform the task, but the spell will end in no more than one day per caster level.

Very clever creatures may be able to subvert the instructions given; the GM must decide on the results of any such attempts.

A quest (and all effects thereof) can be ended by a \textbf{remove curse} spell from a caster two or more levels higher than the caster of the quest, or by a wish, or by the reverse of this spell. \textbf{Dispel magic} does not affect a \textbf{quest }spell.



\smallskip\begin{flushleft} \index{Raise Dead*}
	\begin{tabularx}{0.45\textwidth}{@{}m{3.5cm}m{5.5cm}@{}} 
		\textbf{Raise Dead*} & Range: touch\\
		Cleric 5 & Duration: instantaneous\\
	\end{tabularx}\end{flushleft}

This spell restores life to a deceased humanoid (as defined in \textbf{charm person}). The caster can only raise a being that has not been dead for more days than the caster has levels. The spirit of the target of this spell must be willing to return. If the target' s spirit is trapped or contained in any way, the spell will fail. It will also fail if the target died of old age, as the body simply has no life left in it. Similarly, undead creatures are not affected by this spell as they can no longer be returned to life in any normal sense.

The body of the target must be adequately intact to support life, but all wounds no matter how major are healed. Body parts missing when the target is raised are still missing afterward. Normal poison and normal disease are cured in the process of raising the subject, but magical diseases and curses are not undone.

Creatures brought back from the dead always suffer some loss or penalty from the ordeal. Characters lose one level of ability permanently (i.e. it does not accrue a negative level, but rather loses an actual level, being reduced to the minimum number of experience points required for the previous level). First level characters are reduced to Normal Man status; if the character was already a Normal Man they lose a point of Constitution. These losses are permanent, though of course the character may gain levels in the normal fashion. (Characters reduced to Normal Man status must gain 1,000 XP to return to 1\textsuperscript{st} level).

Monstrous humanoids (orcs, goblins, and the like) lose one hit die, or are reduced to ½ hit dice if the monster has just one to start with. Such humanoids who already have ½ hit dice are reduced to a single hit point. These losses are generally permanent, though the GM may allow such creatures in service to a player character to recover by gaining 1,000 XP per hit die the creature would be returning to (so a lizard man who has been reduced to 1 hit die must earn 2,000 XP to return to its original 2 hit dice); treat such a creature as being a retainer for this purpose.

Upon being raised, the target has 1 hit point per level or hit die (using its current reduced figure, of course), with a minimum of 1 hit point. A character who died with spells prepared has none prepared upon being raised.

The reverse of this spell, \textbf{slay living}, will kill instantly the creature touched (which may be of any sort, not just a humanoid) unless a save vs. Spells is made. If the saving throw is successful, 2d6 points of damage is dealt to the victim instead. An attack roll is required to apply this spell in combat. 

\smallskip\begin{flushleft} \index{Read Languages}
	\begin{tabularx}{0.45\textwidth}{@{}m{3.5cm}m{5.5cm}@{}} 
		\textbf{Read Languages} & Range: 0\\
Magic-User 1 & Duration: special\\
	\end{tabularx}\end{flushleft}

This spell grants the caster the ability to read almost any written language. It may be cast in one of three modes:

In the first mode, the spell allows the caster to read any number of written works in a variety of languages. This mode lasts for 1 turn per caster level.

In the second mode, the spell allows the caster to read any one book or tome; this mode lasts 3 hours per caster level. 

In the third mode, the spell allows the caster to read any one non-magical scroll or other single-sheet document; this mode is permanent.

This spell does not work on any sort of magical text, such as spell scrolls or spellbooks; see \textbf{read magic}, below, for the correct spell to use in such cases.

The spell grants the ability to read the texts, but does not in any way hasten the reading nor grant understanding of concepts the caster doesn' t otherwise have the ability to understand. Also, for this spell to function, there must be at least one living creature that can read the given language somewhere on the same plane. The knowledge is not copied from that creature' s mind; rather, it is the existence of the knowledge that enables the spell to function. 

\smallskip\begin{flushleft} \index{Read Magic}
	\begin{tabularx}{0.45\textwidth}{@{}m{3.5cm}m{5.5cm}@{}} 
		\textbf{Read Magic} & Range: 0\\
Magic-User 1 &Duration: permanent\\
	\end{tabularx}\end{flushleft}

When cast upon any magical text, such as a spellbook or magic-user spell scroll, this spell enables the caster to read that text. Casting this spell on a cursed text will generally trigger the curse. All Magic-Users begin play knowing this spell, and it can be prepared even if the Magic-User loses access to their spellbook.

\smallskip\begin{flushleft} \index{Regenerate}
	\begin{tabularx}{0.45\textwidth}{@{}m{3.5cm}m{5.5cm}@{}} 
		\textbf{Regenerate} & Range: touch\\
Cleric 6 & Duration: permanent\\
	\end{tabularx}\end{flushleft}

This is the most powerful of healing spells, able to cause lost or destroyed body parts, even internal organs, of a living creature to grow back and heal. Severed body parts can be put back in place and will reattach fully in a round (or one round per body part if multiple parts are to be reattached), but regrowing any number of lost body parts requires a full turn. In addition, the spell heals 3d8 points of damage just as if it were a normal \textbf{cure wounds} spell.

\smallskip\begin{flushleft} \index{Reincarnate}
	\begin{tabularx}{0.45\textwidth}{@{}m{3.5cm}m{5.5cm}@{}} 
		\textbf{Reincarnate} & Range: touch\\
Magic-User 6 & Duration: instantaneous\\
	\end{tabularx}\end{flushleft}

By touching the body of a deceased humanoid (as defined in \textbf{charm person}), the caster brings them back in an entirely new body. The whole body is not needed; in fact, even the smallest fragment of body is sufficient, so long as that fragment was part of the body at the time of the target' s death.

The caster can only reincarnate a being that has not been dead for more than a week. The spirit of the target of this spell must be willing to return. If the target' s spirit is trapped or contained in any way, the spell will fail.

Roll on the following table to determine the new form of the target creature:\\

\begin{tabular}[]{@{}ll|ll@{}}
\textbf{d\%} & \textbf{New Form}&\textbf{d\%} & \textbf{New Form}\\hline
01 & Bugbear &47-60 & Halfling\\\hline
02-15 & Dwarf &61-88 & Human\\\hline
16-29 & Elf &89-91 & Kobold\\\hline
30 & Gnoll &92-93 & Lizard Man\\\hline
31-39 & Gnome &94-98 & Orc \\\hline
40-46 & Goblin &99-00 & Choice*\\hline
\end{tabular}\\

\medskip

If "choice" is rolled for a player character being reincarnated, the player is allowed to choose the new form from among those on the table above. If an NPC is being reincarnated, the GM may choose or roll again.

When the spell is cast, a new body forms in a nearby location selected by the caster. The body forms over a period of 6 turns (i.e. an hour), first as a misty outline, then becoming more solid moment by moment until it takes its first breath and awakens. The new form is a young adult, unless the target was younger than that when they died, in which case the new body is the same age as the deceased body.

The target creature' s new body has obviously suffered none of the harm that may have befallen the old one, and is completely healthy (at least, to start with).

The target remembers their previous life, and retains the same class (if possible for the new form) as well as its Intelligence, Wisdom, and Charisma. Strength, Dexterity, and Constitution scores should be rerolled. (If the character' s ability scores are outside the allowable range for the new form, they should be adjusted up or down by the GM as needed.) The target loses one level (or hit die); this is a real reduction, not a negative level, and is not subject to magical \textbf{restoration}. The target' s hit points should be rerolled completely, as this is an entirely new body. If the target was 1st level, instead of a hit point reduction its new Constitution score is reduced by 2.

Characters turned into non-character humanoids (such as an elf who returns as a kobold) will require adjudication by the GM; if the restored target is a player character, the GM is counseled to give as much leeway to the player as possible with the character' s new form. Conversely, non-character humanoids who return as characters will need all ability scores rolled; such characters will usually be fighters with a level equal to the target' s previous hit dice minus 1, or as Normal Men if the target' s previous hit dice were 1 or less.

Undead creatures are not affected by this spell; such creatures can no longer be returned to life in any normal sense.

\smallskip\begin{flushleft} \index{Remove Curse*}
	\begin{tabularx}{0.45\textwidth}{@{}m{3.4cm}m{5.5cm}@{}} 
		\textbf{Remove Curse*} & Range: 30'\\
Cleric 3, Magic-User 4 &Duration: instantaneous\\
	\end{tabularx}\end{flushleft}

This spell removes any and all ordinary curses afflicting a creature. It does not generally remove the curse from a magic item such as a sword or suit of armor, but a character afflicted by a cursed item of this type will be freed of it long enough to discard the item (a turn, at least). 

Some special curses are more difficult to remove, and may require a caster of a certain minimum level. A very few curses created by godlike beings cannot be removed by this spell at all.

The reverse of this spell, \textbf{bestow curse}, allows the caster to place a curse on the subject. A save vs. Spells is allowed to resist. The caster must choose one of the following three effects:

\begin{itemize}
\item
  --4 decrease to an ability score (minimum 1).
\item
  --4 penalty on attack rolls and saves.
\item
Each round of combat, the target has a 50\% chance to act normally; otherwise, it takes no action.
\end{itemize}

The caster may also invent their own curse, but it should be no more powerful than those described above. The curse thus bestowed cannot be dispelled, but it can be removed with a \textbf{remove curse} spell.

\smallskip\begin{flushleft} \index{Remove Fear*}
	\begin{tabularx}{0.45\textwidth}{@{}m{3.5cm}m{5.5cm}@{}} 
		\textbf{Remove Fear*} & Range: 120'\\
Cleric 1 & Duration: instantaneous\\
& (2 turns) \\
	\end{tabularx}\end{flushleft}

This spell will calm the creature touched. If the target creature is currently subject to any sort of magical fear, it is allowed a new save vs. Spells to resist that fear, at a bonus of +1 per level of the caster.

The reverse of this spell, \textbf{cause fear}, causes one target creature within 120' to become frightened; if the target fails to save vs. Spells, it flees for 2 turns. Creatures with 6 or more hit dice are immune to this effect.

\smallskip\begin{flushleft} \index{Resist Cold}
	\begin{tabularx}{0.45\textwidth}{@{}m{3.5cm}m{5.5cm}@{}} 
		\textbf{Resist Cold} & Range: touch\\
		Cleric 1 & Duration: 1 round/level\\
	\end{tabularx}\end{flushleft}

This spell makes the caster, or any living creature the caster touches, completely immune to normal cold. The spell also gives protection against magical or otherwise superior cold such as the breath of an Ice Dragon or the \textbf{ice storm} spell. Specifically, the spell gives the protected creature a bonus of +3 on all saving throws against such effects, and reduces any damage suffered by half (so that for example a successful save vs. the Ice Dragon' s breath would reduce damage to just one-fourth normal, and even if the saving throw fails the protected creature only takes half damage).


\smallskip\begin{flushleft} \index{Resist Fire}
	\begin{tabularx}{0.45\textwidth}{@{}m{3.5cm}m{5.5cm}@{}} 
		\textbf{Resist Fire} & Range: touch\\
		Cleric 2 & Duration: 1 round/level\\
	\end{tabularx}\end{flushleft}

This spell makes the caster, or any living creature the caster touches, completely immune to normal heat or fire. The spell also gives protection against magical or otherwise superior heat or fire such as the breath of a Mountain Dragon or the \textbf{fireball} spell. Specifically, the spell gives the protected creature a bonus of +3 on all saving throws against such effects, and reduces any damage suffered by half (so that for example a successful save vs. the \textbf{fireball }spell would reduce damage to just one-fourth normal, and even if the saving throw fails the protected creature only takes half damage).

\smallskip\begin{flushleft} \index{Restoration}
	\begin{tabularx}{0.45\textwidth}{@{}m{3.5cm}m{5.5cm}@{}} 
		\textbf{Restoration} & Range: touch\\
Cleric 6 &Duration: permanent\\
	\end{tabularx}\end{flushleft}

Each casting of the spell removes a single negative level from a creature who has suffered energy drain. At 16th level, two negative levels may be removed. See the rules for Energy Drain (in the \textbf{Encounter} section on page \hyperlink{energy-drain}{\pageref{energy-drain}}) for more details.

Alternately, this spell can be used to restore drained ability score points. If applied to a character who has temporarily lost ability points, it will restore up to 2d4 lost points to any one drained ability immediately. If applied to a character who has suffered permanent loss of ability points, 1 point can be restored.

This spell cannot restore any levels lost permanently, such as those lost due to death as described for the spells \textbf{raise dead} and \textbf{reincarnate}.

\smallskip\begin{flushleft} \index{Shield}
	\begin{tabularx}{0.45\textwidth}{@{}m{3.5cm}m{5.5cm}@{}} 
		\textbf{Shield} & Range: self\\
Magic-User 1 & Duration: 5 rounds+1/level\\
	\end{tabularx}\end{flushleft}

This spell creates an invisible shield made of magical force which floats in front of the caster, protecting them from various attacks. The spell totally blocks \textbf{magic missile} attacks directed at the caster, and improves the caster' s Armor Class by +3 vs. melee attacks and +6 vs. missile weapons. The Armor Class benefits do not apply to attacks originating from behind the caster, but \textbf{magic missiles} are warded off from all directions.

\smallskip\begin{flushleft} \index{Silence 15' Radius}
	\begin{tabularx}{0.45\textwidth}{@{}m{3.5cm}m{5.5cm}@{}} 
		\textbf{Silence 15' Radius} & Range: 360'\\
Cleric 2 & Duration: 2 rounds/level\\
	\end{tabularx}\end{flushleft}

This spell creates a spherical area with a 15 foot radius where no sound will pass. No one within the affected area can make nor hear any sound. Neither does sound issue from the affected area; those outside cannot hear those inside. This effect blocks verbal communication, of course, as well as spell casting.

This effect can be cast in a fixed area, upon an item (making it portable), or upon a creature. An unwilling target receives a save vs. Spells to negate the spell. If an item in another creature's possession is targeted, that creature also receives a save vs. Spells to negate.

This spell can be used to protect against any kind of attack or magic where the victims must be able to hear the attacker, for such attacks cannot pass into or out of the affected area.

\smallskip\begin{flushleft} \index{Sleep}
	\begin{tabularx}{0.45\textwidth}{@{}m{3.5cm}m{5.5cm}@{}} 
		\textbf{Sleep} & Range: 90'\\
		Magic-User 1 & Duration: 5 rounds/level\\
	\end{tabularx}\end{flushleft}

This spell puts several creatures of 3 or fewer hit dice, or a single 4 hit die creature, into a magical slumber. Creatures of 5 or more hit dice are not affected. The caster chooses a point of origin for the spell (within the given range, of course), and those creatures within 30' of the chosen point may be affected. Each creature in the area of effect is allowed a save vs. Spells to resist.

Victims of this spell can always be hit if attacked. Injuring such a creature will cause it to awaken, and it may begin fighting back or efending itself on the very next round. Slapping or shaking such a creature will awaken it in 1d4 rounds, but normal noises will not.

Sleep does not affect unconscious creatures, constructs, or undead creatures.

When the duration elapses, the sleeping creatures normally wake up immediately; however, if they are made very comfortable and the surroundings are quiet, the affected creatures may continue sleeping normally at the GM' s option.

\smallskip\begin{flushleft} \index{Speak with Animals}
	\begin{tabularx}{0.45\textwidth}{@{}m{3.5cm}m{5.5cm}@{}} 
		\textbf{Speak with Animals} & Range: special\\
		Cleric 2 & Duration: 1 turn/4 levels\\
	\end{tabularx}\end{flushleft}

This spell allows the caster to speak to and understand any single animal (normal or giant sized, but not magical or monstrous) that is in sight of the caster and able to hear them. The caster may change which animal they are speaking with at will, once per round. The spell doesn't alter the animal's reaction or attitude towards the caster; a standard reaction roll should be made to determine this. The GM should ensure that the animal' s manner of speaking reflects its intelligence and nature.


\smallskip\begin{flushleft} \index{Speak with Dead}
	\begin{tabularx}{0.45\textwidth}{@{}m{3.5cm}m{5.5cm}@{}} 
		\textbf{Speak with Dead} & Range: 10'\\
		Cleric 3 & Duration: 3 rounds/level\\
	\end{tabularx}\end{flushleft}

With this spell the caster causes the corpse of an intelligent creature to become animated and to answer the caster' s questions. It does not matter how long the corpse has been dead, but it must be essentially intact with at least a complete mouth in order to answer questions.

The corpse will answer at most one question per two caster levels, but if the duration expires any remaining questions are lost. The corpse only knows what it knew when it was alive; this includes the languages it knew in life. Thus, the caster must share a language with the deceased in order to get any questions answered at all.

The answers given are drawn from knowledge "imprinted" on the corpse during life; the caster does not in any case actually communicate with the spirit of the deceased creature. The corpse cannot retain any information given to it, and does not even remember any previous instances of communication via this spell.

The answers given may not be useful for various reasons. If the corpse knew the caster when it was alive, or if the caster is a member of a group thedeceased disliked, it may choose to lie or mislead the caster. If the caster asks the corpse questions of a personal nature, or questions that indicate the caster may be working against whatever interests the corpse had in life, it will almost certainly seek to mislead the caster.

If the corpse has been roused by this spell within the last seven days, the spell will fail. Undead creatures (including the remains of defeated undead creatures) cannot be affected by this spell.


\smallskip\begin{flushleft} \index{Speak with Monsters}
	\begin{tabularx}{0.45\textwidth}{@{}m{3.5cm}m{5.5cm}@{}} 
		\textbf{Speak with Monsters} & Range: special\\
		Cleric 6 & Duration: 1 turn/5 levels\\
	\end{tabularx}\end{flushleft}

This spell allows the caster to speak to and understand any single living monster that is in sight of the caster and able to hear them. The caster may change which monster they are speaking with at will, once per round. Others able to understand the language spoken by the target monster (if any) will be able to understand the caster. The spell doesn't alter the monster's reaction or attitude towards the caster. Mindless monsters, plant creatures and undead are unaffected by this spell.


\smallskip\begin{flushleft} \index{Speak with Plants}
	\begin{tabularx}{0.45\textwidth}{@{}m{3.5cm}m{5.5cm}@{}} 
		\textbf{Speak with Plants} & Range: 20'\\
		Cleric 4 & Duration: 1 turn\\
	\end{tabularx}\end{flushleft}

This spell allows the caster to speak to and understand any single plant (either normal plant or animate plant creature). The GM should remember that normal plants have a limited sense of their surroundings, and most never move from the place where they sprouted. The spell doesn't alter the plant's reaction or attitude towards the caster; however, normal plants will generally communicate freely with the caster, as they have nothing else of importance to do. Plant creatures will tend to be slightly more intelligent, and a reaction roll should be used to determine how such creatures respond to the caster' s words.

\smallskip\begin{flushleft} \index{Spiritual Hammer}
	\begin{tabularx}{0.45\textwidth}{@{}m{3.5cm}m{5.5cm}@{}} 
		\textbf{Spiritual Hammer} & Range: 30'\\
		Cleric 2 &  1 round/level\\
	\end{tabularx}\end{flushleft}

This spell causes a warhammer made of magical force to appear, attacking any foe chosen by the Cleric within range once per round. The weapon moves about as if wielded by a person of about the caster' s stature, but no such person is present. It deals 1d6 hit points of damage per strike, +1 point per three caster levels (maximum of +5). It uses the caster's normal attack bonus, striking as a magical weapon, and thus can inflict damage upon creatures that are only hit by magic weapons. If the Cleric loses sight of the weapon, causes it to move out of the spell range, or ceases to direct it, the hammer disappears. The weapon is immune to any normal attack, but can be destroyed by \textbf{disintegrate}, \textbf{dispel magic}, or a \textbf{rod of cancellation}.

\smallskip\begin{flushleft} \index{Sticks to Snakes}
	\begin{tabularx}{0.45\textwidth}{@{}m{3.5cm}m{5.5cm}@{}} 
		\textbf{Sticks to Snakes} & Range: 120'\\
Cleric 4 & Duration: 6 turns\\
	\end{tabularx}\end{flushleft}

This spell transforms normal wooden sticks into 1d4 hit dice worth of normal (not giant) snakes per every four caster levels. (Types of snakes are detailed in the \textbf{Monsters} section, starting on page \hyperlink{snake-pit-viper-and-rattlesnake}{\pageref{snake-pit-viper-and-rattlesnake}}.) The snakes follow the commands of the caster. When slain, dispelled, or the spell expires, the snakes return to their original stick form. Magical "sticks" such as enchanted staves cannot be affected.

\smallskip\begin{flushleft} \index{Striking}
	\begin{tabularx}{0.45\textwidth}{@{}m{3.5cm}m{5.5cm}@{}} 
		\textbf{Striking} & Range: touch\\
		Cleric 3 & Duration: 1 round/level\\
	\end{tabularx}\end{flushleft}

This spell bestows upon one weapon the ability to deal 1d6 points of additional damage. This extra damage is applied on each successful attack for the duration of the spell. It provides no attack bonus, but if cast on a normal weapon, the spell allows monsters only hit by magical weapons to be affected; only the 1d6 points of magical damage applies to such a monster, however.

\smallskip\begin{flushleft} \index{Telekinesis}
	\begin{tabularx}{0.45\textwidth}{@{}m{3.5cm}m{5.5cm}@{}} 
		\textbf{Telekinesis} & Range: self\\
Magic-User 5 &Duration: 3 turns\\
	\end{tabularx}\end{flushleft}

This spell permits the caster to move objects or creatures by concentration alone; the caster can move such things weighing up to 50 pounds per caster level at a rate of up to 20 feet per round. Creatures targeted by this spell are allowed a saving throw vs. Death Ray to resist, whether it is the creature itself being affected or an object in its possession.

In order to use this power the caster must maintain concentration, moving no more than normal movement (no running), making no attacks and casting no further spells. If concentration is lost (whether intentionally or not), the power may be used again on the next round but the target of the effect is allowed a new saving throw.

\medskip

\begin{flushleft}
	\includegraphics[width=0.47\textwidth]{Pictures132/10000000000003D80000026AB418155D17C263FB.png}
\end{flushleft}

\smallskip\begin{flushleft} \index{Teleport}
	\begin{tabularx}{0.45\textwidth}{@{}m{3.5cm}m{5.5cm}@{}} 
		\textbf{Teleport} & Range: self\\
Magic-User 5 &Duration: instantaneous\\
	\end{tabularx}\end{flushleft}

The caster of this spell is instantly transported to another location up to 100 miles away per level of ability. The spell transports the caster only within their current plane of existence. Other creatures (passengers) and inanimate objects (cargo) may be transported along with the caster, up to a maximum of 300 pounds plus 100 pounds per level above 10th. The caster must be in contact with all objects and/or creatures to be transported (although creatures to be transported may be in contact with one another, with at least one of those creatures in contact with the caster). Unwilling creatures are allowed a saving throw vs. Spells to resist the spell, and the caster may need to make an attack roll to make contact with such a creature. Likewise, a successful save vs. Spells will prevent items in a creature' s possession from being teleported.

The spell is directed by the mind of the caster, who must visualize the destination area; failure to visualize it properly can cause the spell to fail in a variety of ways, and destinations heavily saturated with magical energy (as defined by the Game Master) will cause the spell to fail automatically.

To determine the results of this spell, choose the appropriate column on the table below, then roll d\%.\\

\begin{tabular}[]{@{}llll@{}}
\multirow{2}{*}\textbf{Knows} & \textbf{Knows}& \textbf{Saw} & \textbf{Spell Result}\\
\textbf{Well} & \textbf{Somewhat}&\textbf{Once}&\\\hline
01 & 01-02 & 01-03 & Disaster \\\hline
02 & 03-07 & 04-13 & Wrong Place \\\hline
03 & 08-13 & 14-25 & Fell Short \\\hline
04-00 & 14-00 & 26-00 & Success! \\\hline
\end{tabular}\medskip

\textbf{Knows Well}applies when the caster has visited the destination frequently and/or spent a substantial amount of time there; generally, the caster should have spent at least 7 days (not necessarily in a row) at the destination within the last year to qualify for this category. Any place where the caster lived for more than a month in the last ten years, or more than a year in their life, also qualifies.

\textbf{Knows Somewhat}applies when the caster has spent substantial time in the destination area, but not enough to qualify for Knows Well. Alternately, the caster may have made an in-depth study of the area, looking at accurate drawings, maps, and floorplans or spending hours listening to descriptions from one or more people who Know Well the destination.

\textbf{Saw Once}applies when the caster has visited a place for as much as a day, but no more, or when the caster has attempted the study required for the second definition of Knows Somewhat but has failed to acquire enough information (in the GM' s opinion).

\textbf{If the caster attempts to travel to a location that does not exist},or perhaps once existed but has been destroyed or otherwise changed so much that the caster would not recognize it, roll 2d20 instead d\% for the result of the casting. In this case, if Success! is rolled the spell simply fails and no one is transported anywhere.

\textbf{Success!} means exactly what it says. The caster, passengers, and cargo arrive safely exactly where the caster intended.

\textbf{Fell Short }indicates that the caster, passengers, and cargo arrive safely 1d8x10\% of the way to the intended destination. Note that arriving "safely" does not mean that the destination is safe, but only the trip.

\textbf{Wrong Place }means that the caster, passengers, and cargo arrive at some place that resembles the intended destination. This means that the caster appears in the closest similar place within range, as decided by the GM. If no such area exists within the spell's range, the spell simply fails instead.

\textbf{Disaster} indicates that the caster, passengers, and cargo have encountered dimensional turbulence and have crashed, becoming separated (if passengers and/or cargo accompanied the caster) and being injured in the process. Each creature including the caster suffers 1d12 points of damage, and then rolls again on the same column using 2d20 instead of d\%. If another Disaster is rolled for any creature, apply another 1d12 points of damage and roll once more. Cargo objects are not normally damaged but the result must be rolled for each such item to determine where it has appeared; in this case, if Disaster is rolled the cargo item disappears forever.


\smallskip\begin{flushleft} \index{True Seeing}
	\begin{tabularx}{0.45\textwidth}{@{}m{3.5cm}m{5.5cm}@{}} 
		\textbf{True Seeing} & Range: touch\\
		Cleric 5 & Duration:  1 round/level\\
	\end{tabularx}\end{flushleft}

This spell confers on the target the ability to see all things as they actually are. The subject sees through normal and magical darkness, notices secret doors, sees the exact locations of displaced creatures or objects, sees through normal or magical disguises, sees invisible creatures or objects normally, sees through illusions, and sees the true form of transformed, changed, or transmuted things. The range of true seeing conferred is 120 feet.

True seeing, however, does not penetrate solid objects. It in no way confers X-ray vision or its equivalent. It does not negate concealment, including that caused by fog and the like. In addition, the spell effects cannot be further enhanced with known magic, so one cannot use true seeing through a \textbf{crystal ball} or in conjunction with \textbf{clairvoyance}.


\smallskip\begin{flushleft} \index{Ventriloquism}
	\begin{tabularx}{0.45\textwidth}{@{}m{3.5cm}m{5.5cm}@{}} 
		\textbf{Ventriloquism} & Range: 60'\\
Magic-User 1 &Duration: 1 turn/level\\
	\end{tabularx}\end{flushleft}

This spell causes the caster' s voice to appear to come from another location within range, for example, from a dark alcove or statue. The caster may choose a new location each round if desired, and can cause the spell to temporarily abate without ending it and then resume it again at any time within the given duration.

\smallskip\begin{flushleft} \index{Wall of Fire}
\begin{tabularx}{0.45\textwidth}{@{}m{3.4cm}m{5.5cm}@{}} 
\textbf{Wall of Fire} & Range: 180'\\
Cleric 5, Magic-User 4 & Duration: 1 round / level\\
& (or special)\\
\end{tabularx}\smallskip\end{flushleft}


This spell creates a vertical sheet of flames in an area indicated by the caster, which is either a wall of flame up to 20' in length per caster level, or a ring with a radius up to 5' per caster level. The caster may choose to make the wall smaller if desired. The wall may be up to 20' tall (as desired by the caster and/or constrained by the ceiling). The entire wall must lie within the range given above. One or both sides of the wall may be hot, as determined by the caster at the time of casting. Once created, the wall cannot be moved or changed.

Any creature within 20 feet of a hot side of the wall will suffer 1d4 points of damage each round, or 2d4 points if within 10 feet. Damage is applied on the round the spell is cast and each round thereafter. Actually passing through the wall causes 2d6 points of damage, plus 1 per caster level, even if the side the character or creature entered from was not a hot side.

Undead creatures are particularly susceptible to this spell, and suffer twice the damage described above.

If the caster evokes the wall so that it appears where creatures are, each creature takes damage as if passing through the wall; a save vs. Spells is allowed, with success indicating that damage is rolled as if the creature is within 10' of the wall.

The caster may choose to maintain the spell indefinitely (within reasonable limits of endurance) by concentration, or may cast it with the standard duration of 1 round per level, at their option.


\smallskip\begin{flushleft} \index{Wall of Iron}
	\begin{tabularx}{0.45\textwidth}{@{}m{3.5cm}m{5.5cm}@{}} 
		\textbf{Wall of Iron} & Range: 90'\\
		Magic-User 6 &Duration: permanent\\
	\end{tabularx}\end{flushleft}

Using this spell the caster creates an iron wall. The wall stands upright, and consists of up to one 10' x10' square section, one inch thick, per caster level. The caster can increase the thickness of the wall with a proportionate reduction in the area; for example, doubling the thickness halves the area. The caster may choose to make the wall smaller than the maximum size if desired. The wall may not be made less than one inch thick, and must always be created in contact with the ground or floor beneath it. It is always a flat plate with no bends, but the edges do not have to be straight; indeed, the caster can cause the wall edge to mold itself around any obstructing object very closely. The wall cannot otherwise be created such that it occupies the space of any object or creature.

If the caster wishes, the wall edges will bond to any inanimate materials they touch (stone walls, soil, furnishings, and so on). If this is not done, the wall may be unsupported (as determined by the GM based on the situation) and will thus likely fall. If left unattended, there is an equal chance it will fall in either direction in 1d6 rounds, but it can be pushed in a specific direction by any character having a minimum of Strength score of 13, or any monster with 4 or more hit dice. Several creatures can work together to do so if desired. (If the optional Ability Roll rule is being used, a Strength roll at -3 is sufficient to topple the wall.)

When the wall falls, any creatures it falls upon are likely to be injured or killed. If it is possible for a character or creature to escape the area (i.e. it has sufficient movement rate and is not otherwise prevented from doing so), it is allowed to roll a save vs. Death Ray (with Dexterity bonus added). If this save is successful, the creature or character moves by the most direct route to the nearest safe space; if the save fails or it is for some reason impossible to flee, 10d6 points of damage are inflicted on that victim. Creatures larger than ogre-sized are immune to being crushed and will simply be knocked down if the save fails.

The wall is permanent, as indicated, but being made of iron is susceptible to rust and corrosion.

\smallskip\begin{flushleft} \index{Wall of Stone}
	\begin{tabularx}{0.45\textwidth}{@{}m{3.5cm}m{5.5cm}@{}} 
		\textbf{Wall of Stone} & Range: 15' per	level\\
		Magic-User 5 &Duration: permanent\\
	\end{tabularx}\end{flushleft}

Using this spell the caster creates a stone wall. The wall is composed of up to one 10' x10' square section, one foot thick, per caster level. The caster can form this wall into almost any shape, with some restrictions. The caster can increase the thickness of the wall with a proportionate reduction in the area; for example, doubling the thickness halves the area. The wall cannot be
created such that it occupies the space of any object or creature. It
must be adequately supported by existing stone, which it will bond with
automatically, but need not be supported over its entire area. For
example, a wall of stone may be formed into a bridge over a stream or
chasm, so long as both ends of the bridge rest solidly upon (and bond
with) existing stone.

Bridges longer than 20 feet must be arched, buttressed, or both in order
to stand; this extra construction reduces the wall' s
usable volume by half, as does creating a wall with battlements,
crenelations, and similar basic structural elements. No complex
structural elements may be created by this spell.

Though made by magic, the wall is made of stone and can be broken or
damaged just as if it were ordinary stone.

The wall can be formed into a container to trap creatures, and if this
is attempted the targets of the spell are allowed to save vs. Death Ray
to avoid being trapped. If the save is successful the targets are able
to make up to one full move to a space outside the container.

\smallskip\begin{flushleft} \index{Water Breathing}
	\begin{tabularx}{0.45\textwidth}{@{}m{3.5cm}m{5.5cm}@{}} 
		\textbf{Water Breathing} & Range: touch\\
Magic-User 3 & Duration: 2 hours/level\\
	\end{tabularx}\end{flushleft}

This spell grants living creatures touched by the caster (including the caster, if desired) the ability to breath in water as a fish does. The duration may be divided evenly if the caster touches multiple targets one after another. Affected targets do not lose the ability to breathe air.

\medskip

\begin{flushleft}
\includegraphics[width=0.47\textwidth]{Pictures132/10000000000003D800000172E9821AB098EA0C2B.png}
\end{flushleft}

\smallskip\begin{flushleft} \index{Web}
	\begin{tabularx}{0.45\textwidth}{@{}m{3.5cm}m{5.5cm}@{}} 
		\textbf{Web} & Range: 10' per level\\
		Magic-User 2 &Duration: 2 turns/level\\
	\end{tabularx}\end{flushleft}

This spell creates a volume of sticky strands resembling a spider' s web but much larger and thicker. The spell fills a volume of up to 8,000 cubic feet (equivalent to eight 10' x10' x10' cubes). The webs must be attached to adjacent solid objects such as walls, pillars, and the like; any unsupported section of webbing collapses to the ground and disappears. Within this limitation, the caster may choose any arrangement of webs they wish, up to the limit of range and the given 8,000 cubic foot volume. The caster may choose to create a smaller volume if they wish.

Creatures within the web at the time the spell is cast, as well as anyone entering the area afterward, will become entangled. Each should roll a save vs. Death Ray, and any creatures who succeed at this save may move through the webbing but are reduced to one-half normal movement rate. Such creatures may not cast spells or perform normal attacks; whether other actions are possible is left to the GM to decide. Once an entangled creature leaves the area of effect of the web, it will be able to act normally again.

Those who fail the save are fully entangled and trapped. They cannot move, cast spells, or perform normal attacks or any other physical action. Speech remains possible, however. Creatures with Strength of 13 or higher (or 4 or more hit dice) may be able to break loose, however; each round, such creatures are allowed another save vs. Death Ray with results as given above. Creatures failing the initial save and having Strength of 12 or less (or fewer than 4 hit dice) are trapped until the duration expires or the webs are otherwise removed.

Attacks against an entangled creature by one outside the webbing will not normally entrap the attacker, so long as they do not need to venture into the web to reach the entangled target.

The web can be ignited; any application of fire to the webbing will cause a 10 foot cube to burn away in one round, with all 10 foot cubes adjacent to the destroyed one burning in the next round, and so on. If any part of the web becomes unsupported it will collapse and disappear as noted above.

Creatures trapped within the burning web suffer 2d4 points of damage when the cube they are trapped in burns, but they are thereafter completely free of the web.

\smallskip\begin{flushleft} \index{Wizard Eye}
	\begin{tabularx}{0.45\textwidth}{@{}m{3.5cm}m{5.5cm}@{}} 
		\textbf{Wizard Eye} & Range: 240'\\
		Magic-User 4 &Duration: 6 turns\\
	\end{tabularx}\end{flushleft}

With this spell the caster creates an invisible magical "eye" through which they can see. The eye has Darkvision with a range of 30 feet, but otherwise sees exactly as the caster would. It can be created in any place the caster can see, up to a range of 240 feet away, and thereafter can move at a rate of 40 feet per round as directed by the caster. The eye will not move more than 240 feet away from the caster under any circumstances. The eye cannot pass through solid objects, but as it is exactly the size of a normal human' s eye, it can pass through holes as small as 1 inch in diameter. The caster must concentrate to use the eye.

\smallskip\begin{flushleft} \index{Wizard Lock}
	\begin{tabularx}{0.45\textwidth}{@{}m{3.5cm}m{5.5cm}@{}} 
		\textbf{Wizard Lock} & Range: 20'\\
		Magic-User 2 &Duration: permanent\\
	\end{tabularx}\end{flushleft}

This spell magically holds shut a door, gate, window, or shutter of wood, metal, or stone. The magic affects the portal just as if it were securely closed and normally locked. The effect lasts indefinitely. \textbf{Knock} can be used to open the doorway without ending the spell, and \textbf{dispel magic} can be used to end it permanently. The caster of this spell can easily open the door or other portal without ending the spell, as can a Magic-User three or more levels higher than the caster.

\smallskip\begin{flushleft} \index{Word of Recall}
	\begin{tabularx}{0.45\textwidth}{@{}m{3.5cm}m{5.5cm}@{}} 
		\textbf{Word of Recall} & Range: self\\
Cleric 6 &Duration: instantaneous\\
	\end{tabularx}\end{flushleft}

With the utterance of a single word this spell transports the caster to a place of refuge which they designate when preparing the spell. The place must be Known Well (as explained for \textbf{teleport}) to the caster to be so designated. This spell cannot transport the caster beyond their current plane of existence.

The caster can bring along objects or creatures, not to exceed 300 pounds plus 100 pounds per level above 10\textsuperscript{th}. The caster must be in contact with all objects and/or creatures to be transported (although creatures to be transported may be in contact with one another, with at least one of those creatures in contact with the caster).

Unwilling creatures cannot be transported by this spell, nor can items in their possession. If the caster or one of their passengers is holding an object that is also being held by an unwilling creature, the latter can retain possession of the object with a successful save vs. Spells. 

\end{multicols}

\vfill

\begin{center}
	\includegraphics[width=0.75\textwidth]{Pictures132/100000000000079E000004DDEB977F3A2A1A86EF.jpg}
\end{center}

\pagebreak

\section{PART 4: THE ADVENTURE}\label{part-4-the-adventure}\index{Part 4: The Adventure}

\subsection{Time and Scale}\label{time-and-scale}\index{Time and Scale}

Time in the dungeon is measured in \textbf{game turns} (or just \emph{turns}), which are approximately 10 minutes long. When combat begins, the time scale changes to \textbf{combat round}s, which are approximately 10 seconds long. Thus, there are 60 combat rounds per game turn. \emph{Approximately}, because time is not meant to be kept exactly as it is subjective to the characters.

Distances in the dungeon are measured in feet. Outdoors, change all distance measurements (movement, range, etc.) to yards (so 100 feet becomes 100 yards) but area of effect measurements (for spells, for instance) normally remain in feet. Note that the single quote character is used as an abbreviation for feet in some places.

\subsection{Dungeon Adventures}\label{dungeon-adventures}\index{Dungeon Adventures}

\begin{multicols}{2}
	
\subsection{Carrying Capacity}\label{carrying-capacity}\index{Carrying Capacity}

Normal Human, Elven, and Dwarvish player characters are able to carry up
to 60 pounds and still be considered lightly loaded, or up to 150 pounds
and be considered heavily loaded. Halflings may carry up to 50 pounds
and be considered lightly loaded, or up to 100 pounds and be heavily
loaded. Note that armor for Halfling characters is about one-quarter as
heavy as armor for the other races.

These figures are affected by Strength; each +1 of Strength bonus adds
10\% to the capacity of the character, while each -1 deducts 20\%. Thus,
carrying capacities for normal characters are as shown below (rounded to
the nearest 5 pounds for convenience):\medskip

\begin{tabular}[]{@{}lllll@{}}

&\multicolumn{2}{c}{\textbf{Dwarf}, \textbf{Elf}, \textbf{Human}}&\multicolumn{2}{c}{\textbf{Halfling}}\\
&\multirow{2}{*}\textbf{Light}&\textbf{Heavy}&\textbf{Light}&\textbf{Heavy}\\
\textbf{\textbf{STR}}&  \textbf{Load} &  \textbf{Load} &  \textbf{Load} &  \textbf{Load}\\\toprule
3 & 25 & 60 & 20 & 40 \\\hline
4-5 & 35 & 90 & 30 & 60 \\\hline
6-8 & 50 & 120 & 40 & 80 \\\hline
9-12 & 60 & 150 & 50 & 100 \\\hline
13-15 & 65 & 165 & 55 & 110 \\\hline
16-17 & 70 & 180 & 60 & 120 \\\hline
18 & 80 & 195 & 65 & 130 \\\bottomrule
\end{tabular}\\

The carrying capacities of various domesticated animals are given in the \textbf{Monsters} section, in the entry for each type of animal.

\subsection{Movement and Encumbrance}\label{movement-and-encumbrance}\index{Movement and Encumbrance}

The movement rate of a character or creature is expressed as the number of feet it can move per combat round. The normal player character races can all move 40' per round. When exploring a dungeon, time is expressed in turns, as explained above; normal movement per turn is 3 times the movement rate per round.

This may seem slow, but this rate of movement includes such things as drawing maps, watching out for traps and monsters (though they may still surprise the party), etc. In a combat situation, on the other hand, everyone is moving around swiftly, and such things as drawing maps are not important.

A character' s movement rate is adjusted by their \textbf{Encumbrance} (the load they are carrying) as follows:\medskip

\begin{flushleft}
	\begin{tabularx}{0.45\textwidth}{@{}Xll@{}}
& \multirow{2}{*}\textbf{Lightly} & \textbf{Heavily} \\
\textbf{Armor Type} &\textbf{Loaded}&\textbf{Loaded}\\\toprule
 No Armor or Magic Leather & 40' &30' \\\hline
 Leather Armor or Magic Metal & 30' &20 \\\hline
 Metal Armor& 20' & 10' \\\bottomrule
\end{tabularx}
\end{flushleft}

Count the weight of armor worn when calculating encumbrance, because armor counts both for bulk and restrictiveness as well as for weight. Magic armor counts for its full weight but is not as bulky and restrictive as normal armor, thus granting an improved movement rate.

For animals such as horses, being heavily loaded reduces movement rate as shown below:\\

\begin{flushleft}
	\begin{tabularx}{0.47\textwidth}{@{}ll|ll@{}}
\textbf{Normal Mv}&\textbf{Heavy Load}&\textbf{Normal Mv}&\textbf{Heavy Load}\\\toprule
10' & 5' &  130' &100' \\\hline
20' & 10' &  140' &110' \\\hline
30' & 20' &  150' &120' \\\hline
40' & 30' &  160' &130' \\\hline
50'-60' & 40' &170'-180' & 140' \\\hline
70' & 50' &  190' &150' \\\hline
80' & 60' &  200' &160' \\\hline
90' & 70' &  210' &170' \\\hline
100' & 80' &  220' &180' \\\hline
110'-120' & 90'  &230'-240' & 190' \\\bottomrule
\end{tabularx}
\end{flushleft}

\subsection{Mapping}\label{mapping}\index{Mapping}

In any dungeon expedition, making maps is important. Generally one player will do this, drawing a map on graph paper as the Game Master describes each room or corridor. Absolute accuracy is usually not possible; the main thing is to ensure that the party can find its way back out of the dungeon.

\begin{flushleft}
	\includegraphics[width=0.47\textwidth]{Pictures132/10000000000003CF0000029294B1F984EE3AD86A.png}
\end{flushleft}

\subsection{Light}\label{light}\index{Light}

A torch or lantern will provide light covering a 30' radius; dim light will extend about 20' further. Normal torches burn for 1d4+4 turns, while a flask of oil in a lantern will burn for 1d6+18 turns. A candle will shed light over a 5' radius, with dim light extending 5' further. In general, taper candles such as are used for illumination will burn about 3 turns per inch of height.

\subsection{Darkvision}\label{darkvision}\index{Darkvision}

Some character races as well as almost all monsters have Darkvision, an ability which allows them to see even in total darkness. Such vision is in black and white, but otherwise like normal sight. Magical darkness obstructs Darkvision just as it does normal vision. The range of Darkvision is typically either 30' or 60'; if not given for a particular creature, assume the 60' range.

Darkvision is totally ineffective in any light greater than moonlight.


\begin{flushleft}
	\includegraphics[width=0.47\textwidth]{Pictures132/10000000000003D800000150153F4D18635896E2.png}
\end{flushleft}

\subsection{Doors}\label{doors}\index{Doors}

A stuck door can be opened on a roll of 1 on 1d6; add the character' s Strength bonus to the range, so that a character with a bonus of +2 can open a stuck door on a roll of 1-3 on 1d6.

Locked doors can be forced by rolling the same range, but on 1d10. Metal bars can sometimes be bent on a roll of this range on 1d20.

A careful character might choose to listen at a door before opening it. Thieves have a special ability, Listen, which should be applied if the listener is a Thief. For other characters, the GM rolls 1d6, with 1 indicating success. Sounds heard might include voices, footsteps, or any other sound the GM considers appropriate. Of course, the room beyond the door might really be silent; thus, the Game Master must make the roll, so that a roll of 1 in such a case will not give anything away to the players.

\subsection{Traps}\label{traps}\index{Traps}

Dungeons and ruins frequently contain traps, including spear-throwers, covered pits, etc. The GM will decide what is required to trigger a trap, and what happens when the trap is triggered. (Some guidance on this is provided in the Game Master section later in this book.) In general, there will be some way to avoid or reduce the effect of the trap being sprung. For instance, a save vs. Death Ray is often used to avoid falling into a covered pit (with Dexterity bonus added), while spear-throwers, automated crossbows, and the like are sometimes treated as if they were monsters (attacking vs. the victim' s Armor Class at some given attack bonus).

Normal characters have a chance equal to a roll of 1 on 1d6 to detect a trap if a search for one is made. Note that this is about a 16.7\% chance; Thieves have a special ability to find and remove traps, which supersedes this roll, as does the stonework trap-finding ability of Dwarves. A Dwarven Thief is a special case; apply whichever trap-detection ability is higher. In all cases, a search for traps takes at least a turn per 10' square area. A single character may only effectively search a given area for traps once, even if the character has more than one trap-detection roll "type" allowed (such as the Dwarven Thief above).

Trap detection may not be allowed if the trap is purely magical in nature; on the other hand, in such cases Magic-Users and/or Clerics may be able to detect magical traps at the given 1 in 1d6 chance, at the Game Master' s option. Note also the 2\textsuperscript{nd} level Clerical spell \textbf{find traps}, available to 4\textsuperscript{th} level and higher Clerics.

\subsection{Secret Doors}\label{secret-doors}\index{Secret Doors}

Under normal conditions, searching for secret doors takes one turn per character per 10' of wall searched. A secret door is found on a roll of 1 on 1d6; characters with 15 or higher Intelligence succeed on a roll of 1-2. Also, as noted previously, Elves add 1 to the range automatically, such that an Elf discovers secret doors on a 1-2 on 1d6, or 1-3 if the Elf has an Intelligence of 15 or higher. The GM may create secret doors that are more difficult (or easier) to detect at their option.

Multiple characters searching for secret doors ensures that any such will eventually be found; however, if the first and second searchers fail, the next searcher must take two turns to search, and all subsequent searches of the area require an hour.

Note that finding a secret door does not grant understanding of how it works. The GM may require additional rolls or other actions to be taken before the door can be opened.

\subsection{Dungeon Survival}\label{dungeon-survival}\index{Dungeon Survival}

As described previously in the \textbf{Equipment} section, normal characters must consume one day' s worth of rations (or equivalent food) and at least one quart of water per day.

Failure to consume enough food does not significantly affect a character for the first two days, after which they suffer 1 point of damage per day. Furthermore, at that point the character loses the ability to heal wounds normally, though magic will still work. Eating enough food for a day (over the course of about a day, not all at once) restores the ability to heal, and the character will resume recovering lost hit points at the normal rate.

Inadequate water affects characters more swiftly; after a single day without adequate water, the character loses 1d4 hit points, and will lose an additional 1d4 hit points per day thereafter. Healing ability is lost when the first die of damage is rolled.

\end{multicols}

\vfill

\begin{center}
	\includegraphics{Pictures132/10000000000007E90000062757832BE25B8D0FF0.png}
\end{center}

\pagebreak

\subsection{Wilderness Adventures}\label{wilderness-adventures}\index{Wilderness Adventures}

\begin{multicols}{2}

\subsection{Wilderness Movement Rates}\label{wilderness-movement-rates}\index{Wilderness Movement Rates}

The table below shows wilderness travel rates for characters or creatures based on encounter movement. Naturally, any group traveling together moves at the rate of the slowest member.\\

\begin{tabular}[]{@{}ll|ll@{}}

\textbf{Normal Mv}&\textbf{Miles/Day}&\textbf{Normal Mv}&\textbf{Miles/Day}\\\toprule
10' & 6 &  70' & 42 \\\hline
20' & 12 & 80' & 48 \\\hline
30' & 18 & 90' & 54 \\\hline
40' & 24 & 100' & 60 \\\hline
50' & 30 & 110' & 66 \\\hline
60' & 36 & 120' & 72 \\\bottomrule
\end{tabular}

\subsection{Overland Travel}\label{overland-travel}\index{Overland Travel}

The movement rates shown on the table above are figured based on an 8 hour day of travel through open, clear terrain. The terrain type will alter the rate somewhat, as shown on this table:\\

\begin{tabular*}{0.93\linewidth}{@{\extracolsep{\fill}}ll}
\textbf{Terrain} & \textbf{Adjustment} \\\toprule
Jungle, Mountains, Swamp & x1/3 \\\hline
Desert, Forest, Hills & x2/3 \\\hline
Clear, Plains, Trail & x1 \\\hline
Road (Paved) & x1 1/3 \\\bottomrule
\end{tabular*}\\\medskip

Characters may choose to perform a \emph{forced march}, traveling 12 hours per day and adding 50\% to the distance traveled. Each day of forced march after the first inflicts 1d6 points of damage on the characters (and their animals, if any). A daily save vs. Death Ray with Constitution bonus applied is allowed to avoid this damage, but after this save is failed once, it is not rolled again for that character or creature. A day spent resting "restarts" this progression.

\subsection{Becoming Lost}\label{becoming-lost}\index{Becoming Lost}

Though adventurers following roads, rivers, or other obvious landmarks are unlikely to become lost, striking out into trackless forest, windblown desert, and so on is another matter. Secretly roll a save vs. Death Ray, adjusted by the Wisdom of the party leader (i.e., whichever character seems to be leading). An Ability Roll against Wisdom may be rolled, if that optional rule is in use. The GM must determine the effects of failure.

\subsection{Waterborne Travel}\label{waterborne-travel}\index{Waterborne Travel}

Travel by water may be done in a variety of boats or ships; see \textbf{Vehicles} on page \hyperlink{vehicles}{\pageref{vehicles}} for details. Travel distances are for a 12 hour day of travel, rather than the usual 8 hours per day. Sailing ships may travel 24 hours per day if a qualified navigator is aboard, and so may be able to cover twice the normal distance per day of travel. This is in addition to the multiplier given below. If the ship stops each night, as is done by some vessels traveling along a coastline as well as those having less than the minimum number of regular crew on board, the two-times multiplier does not apply.

Movement of sailing ships is affected by the weather, as shown below. \textbf{Sailing} movement modifiers apply when sailing with the wind; sailing against the wind involves \textbf{tacking} (called "zigzagging" by landlubbers) which reduces movement rates as indicated. \\

\begin{tabular*}{0.93\linewidth}{@{\extracolsep{\fill}}ll}
\textbf{d12} & \textbf{Wind Direction} \\\toprule
1 & Northerly \\\hline
2 & Northeasterly \\\hline
3 & Easterly \\\hline
4 & Southeasterly \\\hline
5 & Southerly \\\hline
6 & Southwesterly \\\hline
7 & Westerly \\\hline
8 & Northwesterly \\\hline
9-12 & Prevailing wind direction for this locale \\\bottomrule
\end{tabular*}\\\medskip

\begin{tabularx}{0.45\textwidth}{@{}lXll@{}}
\textbf{d\%} & \textbf{Wind Conditions} & \textbf{Sail.} & \textbf{Tack.} \\\toprule
01-05 & Becalmed & x0 & x0 \\\hline
06-13 & Very Light Breeze & x1/3 & x0 \\\hline
14-25 & Light Breeze & x1/2 & x1/3 \\\hline
26-40 & Moderate Breeze & x2/3 & x1/3 \\\hline
41-70 & Average Winds & x1 & x1/2 \\\hline
71-85 & Strong Winds & x1 1/3 & x2/3 \\\hline
86-96 & Very Strong Winds & x1 1/2 & x0 \\\hline
97-00 & Gale & x2 & x0 \\\bottomrule
\end{tabularx}\\\medskip

\textbf{Becalmed: }Sailing ships cannot move. Oared ships may move at
the given rowing movement rate.

\textbf{Very Strong Winds:} Sailing against the wind (tacking) is not
possible.

\textbf{Gale:} Sailing against the wind is not possible, and ships
exposed to a gale may be damaged or sunk; apply 2d8 points of damage to
any such ship, per hour sailed.

\subsection{Traveling by Air}\label{traveling-by-air}\index{Traveling by Air}

When traveling by air, overland movement rates are doubled, and all terrain effects are ignored. Most winged creatures must maintain at least one-third normal forward movement in order to remain airborne; however, devices such as \textbf{flying carpets} generally do not have this limitation.

\end{multicols}

\subsection{Retainers, Specialists and Mercenaries}\index{Retainers, Specialists and Mercenaries} \label{retainers-specialists-and-mercenaries}

\begin{multicols}{2}
	
Player characters will sometimes want or need to hire NPCs (Non-Player Characters) to work for them. There are several categories of NPCs available for hire, as follows:

\subsection{Retainers}\label{retainers}\index{Retainers}

A retainer is a close associate of his employer. Retainers are hired for a share of treasure (typically at least 15\% of the employer' s income) plus support costs (weapons, armor, rations, and basic equipment provided by the employer). Retainers are typically very loyal and are willing to take reasonable risks; in particular, they are the only sort of hireling who will generally accompany a player character into a dungeon, lair, or ruin.

Hiring a retainer is more involved than hiring other NPCs. First, the player character must advertise for a retainer, typically by hiring a crier, posting notices in public places, or asking (and possibly paying) NPCs such as innkeepers or taverners to direct potential retainers to the player character. It is up to the Game Master to rule on what must be done, and how successful these activities are.

If the player character is successful, one or more NPCs will present themselves to be interviewed. The Game Master should play out the interview with the player, and after all offers have been made and all questions asked, a reaction roll should be made. To check the potentialretainer' s reaction, the Game Master rolls 2d6 and adds the player character' s Charisma bonus. In addition, the Game Master may apply any adjustments they feel are appropriate (a bonus of +1 for higher-than-average pay or the offer of a magic item such as a \textbf{Sword +1}, or a penalty if the player character offers poor terms). The roll is read as follows:\\

\begin{tabular*}{0.93\linewidth}{@{\extracolsep{\fill}}ll}
\textbf{Adjusted Die Roll} & \textbf{Result} \\\toprule
2 or less & Refusal, -1 on further rolls \\\hline
3-5 & Refusal \\\hline
6-8 & Try again \\\hline
9-11 & Acceptance \\\hline
12 or more & Acceptance, +1 to Loyalty \\\bottomrule
\end{tabular*}\\\medskip

Refusal, -1 on further rolls means that all further reaction rolls made toward that player character in the given town or region will be at a penalty of -1 due to unkind words said by the NPC to his fellows. If the player character tries again in a different town, the penalty does not apply.

If a \textbf{Try again} result is rolled, the potential retainer is reluctant, and needs more convincing; the player character must "sweeten" the deal in order to get an additional roll, such as by offering more pay, a magic item, etc. If the player character makes no better offer, treat \textbf{Try again} as a \textbf{Refusal} result.

\textbf{Loyalty:} All retainers have a Loyalty score, which is generally 7 plus the employer' s Charisma bonus (or penalty). The Loyalty score is used just as the Morale score of monsters or mercenaries is used.

If a Loyalty check roll made in combat is a natural 2, the Loyalty of the retainer increases by +1 point. Note that a Loyalty of 12 is fanatical... the retainer will do virtually anything the player character asks, and never flees in combat. However, the Game Master should still apply penalties when the player character instructs the retainer to do something which appears very risky, making a failed check possible.

In addition, the Game Master should roll a Loyalty check for each retainer at the end of each adventure, after\textbf{ }treasure is divided, to determine if the retainer will remain with the player character. The GM may apply adjustments to this roll, probably no more than two points plus or minus, if the retainer is particularly well or poorly paid.

Maximum Number of Retainers: A player character may hire at most 4 retainers, adjusted by the character' s Charisma bonus or penalty. Any attempts to hire more than this number of retainers will be met with automatic refusals.

Level of Retainers: Normally, potential retainers will be one-half the level of the employer (or less). So, a first level character cannot hire retainers, second level PCs can only hire first level characters, and so on. Of course, there is no way for the retainers to directly know the level of the PC employer, nor for the employer to know the level of the potential retainer; but the Game Master should usually enforce this rule for purposes of game balance. It shouldn' t be surprising that first level characters can' t hire retainers, as they have no reputation to speak of yet.

Experience for Retainers: Unlike other hired NPCs, retainers do gain experience just as other adventurers do; however, as they are under the command of a player character, only one-half of a share of XP is allocated to each retainer. See Character Advancement, below, for an example.

\subsection{Specialists}\label{specialists}\index{Specialists}

Specialists are NPCs who may be hired by player characters to perform various tasks. Specialists do not go on adventures or otherwise risk their lives fighting monsters, disarming traps, or any of the other dangerous things player characters and retainers may do. Rather, specialists perform services the player characters usually can' t perform for themselves, like designing and erecting castles, training animals, or operating ships.

A player character is limited in the number of specialists they can hire only by the amount of money they cost; Charisma does not affect this.

\hypertarget{Alchemistux20Entry}{}\label{Alchemistux20Entry}Alchemist\index{Alchemist}: \emph{ 1,000 gp per month. }These characters are generally hired for one of two reasons: to make potions, or to assist a Magic-User with magical research.

An alchemist can produce a potion, given the required materials and a sample or a written formula for the potion, in the same time and for the same cost as a Magic-User. They may also research new potions, but at twice the cost in time and materials as a Magic-User. Review the rules for \textbf{Magical Research }starting on page \hyperlink{character-races}{\pageref{character-races}} for details.

Alternately, a Magic-User seeking to create certain magic items may employ an alchemist as an assistant. In this case, the alchemist adds 15\% to the Magic-User' s chance of success.

Animal Trainer: \emph{250 to 750 gp per month.} Characters wishing to ride hippogriffs or employ carnivorous apes as guards will need the assistance of an animal trainer. The lowest cost above is for an average animal trainer, able to train one type of "normal" animal such as carnivorous apes; those able to train more than one sort of animal, or to train monstrous creatures such as hippogriffs, are more expensive to hire. The Game Master must decide how long it takes to train an animal; in some cases, animal training may take years, a fact the player characters may find inconvenient as well as expensive. A single animal trainer can train and manage no more than 5 animals at a time, though in most cases once an animal is fully trained, if it is put into service right away the animal trainer won' t be needed to handle it any longer.

Armorer (or \hypertarget{Weaponsmithux20Entry}{}{}\label{Weaponsmithux20Entry}\index{Weaponsmith}Weaponsmith): \emph{100 to 500 gp per month.} Characters hiring mercenaries, or having armed and armored followers to take care of, will need the services of an armorer. In general, for every 50 Fighters employed, one armorer is required to care for their gear. The armorer' s equipment is not included in the costs given above, but the cost to maintain his apprentices is included; most such characters will have 1d4 apprentices assisting.

Higher priced armorers or weaponsmiths may be hired to assist in making magic weapons or armor; in this case, the character hired will be a specialist, an expert in making one particular type of armor or weapon, and will command a higher price (as shown above). Such characters will rarely agree to do the mundane work of maintaining weapons and armor for a military unit.

Engineer: \emph{750 gp per month.} Any player character wishing to build a fortress, a ship, or any other mundane construction will need an engineer. Large projects may require several engineers, at the GM' s option.

Savant: \emph{1,500 gp per month.} Savants are experts in ancient and obscure knowledge. Many savants have particular interests in very limited or focused areas (for example, "Elven migrations of the 2\textsuperscript{nd} age"), but even these will know or have access to a lot of facts. The listed cost is the minimum required to maintain a savant with his library, collections, etc. If the savant' s patron asks a difficult question, there may be additional costs for materials or research to answer it.

\textbf{Ship' s Crew:} \emph{Special.} A crew for a waterborne vessel involves several types of characters. At the very least, a complement of sailors and a Captain are needed; rowers will be needed aboard galleys, and a Navigator is required aboard ships going out of sight of land.

Costs per month for each sort of character are given below:\medskip

\begin{tabular*}{0.93\linewidth}{@{\extracolsep{\fill}}ll}
\textbf{Seaman Type} & \textbf{Cost} \\\toprule
Captain & 300 gp \\\hline
Navigator & 200 gp \\\hline
Sailor & 10 gp \\\hline
Rower & 3 gp \\\bottomrule
\end{tabular*}\medskip

In general, all such characters are normal men, and are not armored; they will usually be armed with clubs, daggers, or shortswords. Player characters with appropriate backgrounds may act as Captain, but unless experienced as a ship' s captain, they will have difficulty commanding respect from the regular sailors (lower the Morale of such regular sailors by -2 if led by an inexperienced Captain).

\subsection{Mercenaries}\label{mercenaries}\index{Mercenaries}

Mercenaries are hired warriors. They are generally hired in units as small as platoons: 32 to 48 Fighters, divided into two to four squads of soldiers; each squad is led by a corporal, while the platoon is led by a lieutenant plus a sergeant. Platoons are joined together into companies, each generally consisting of two to five platoons and led by a captain with a sergeant as his assistant (called a \textbf{first sergeant}).

As mercenaries are almost always veteran troops, the average mercenary is a 1st level Fighter; 10\% of corporals and 50\% of sergeants are 2nd level. A mercenary lieutenant will generally be 2nd level, while a captain will be 2nd to 4th level and his first sergeant will be 2nd or 3rd level. Larger mercenary units will usually be beyond the reach of player characters until they have reached fairly high levels, and are left to the Game Master to detail.

Mercenaries will virtually never go into a dungeon, lair, or ruin, at least until it has been fully cleared. Rather, they are used in outdoor military engagements; high level player characters may hire mercenaries to defend or help defend their castles or other holdings.

Mercenaries housed in a player character' s stronghold require 200 square feet each but cost 25\% less per month, as this is covered by their room and board. (Elven mercenaries, however, require 500 square feet of space each in order to reduce their pay, as they demand better living conditions.) See the \textbf{Stronghold} section onpage \hyperlink{strongholds}{\pageref{strongholds}} for more details.

Statistics are given below for the most common sorts of mercenaries; the statistics are for first level characters, and should be adjusted when higher level characters are indicated (as given above). In particular, multiply the given cost of each mercenary by their level. Listed costs are in gold pieces per month.

\end{multicols}

\begin{tabular*}{0.93\linewidth}{@{\extracolsep{\fill}}llll}
\textbf{Type of Mercenary}&\textbf{Cost}&\textbf{Equipment}&\textbf{Morale}\\
Light Foot, Human&2 gp&Leather Armor, Shield, and Longsword&8\\\toprule
Light Foot, Elf&8 gp&Leather Armor, Shield, and Longsword&8\\\hline
Light Foot, Orc&1 gp&Leather Armor and Spear&7\\\hline
Heavy Foot, Human&3 gp&Chainmail, Shield, and Longsword&8\\\hline
Heavy Foot, Dwarf&6 gp&Chainmail, Shield, and Shortsword&9\\\hline
Heavy Foot, Orc&2 gp&Chainmail, Shield, and Shortsword&8\\\hline
Archer, Human&5 gp&Leather Armor, Shortbow, and Shortsword&8\\\hline
Archer, Elf&15 gp&Chainmail, Shortbow, and Shortsword&8\\\hline
Archer, Orc&3 gp&Leather Armor, Shortbow, and Shortsword&8\\\hline
Crossbowman, Human&5 gp&Chainmail, Crossbow, and Shortsword&8\\\hline
Crossbowman, Dwarf&12 gp&Platemail, Crossbow, and Shortsword&9\\\hline
Longbowman, Human&9 gp&Chainmail, Longbow, and Shortsword&8\\\hline
Longbowman, Elf&20 gp&Chainmail, Longbow, and Longsword&8\\\hline
Light Horseman, Human&10 gp&Leather Armor, Shield, Lance, and Longsword&8\\\hline
Light Horseman, Elf&22 gp&Leather Armor, Lance, Shortbow, and Longsword&8\\\hline
Medium Horseman, Human&15 gp&Chainmail, Shield, Lance, and Longsword&8\\\hline
Medium Horseman, Elf&33 gp&Chainmail, Lance, Shortbow, and Longsword&9\\\hline
Heavy Horseman, Human&20 gp&Platemail, Shield, Lance, and Longsword&8\\\hline
\end{tabular*}

\vfill

\begin{center}
	\includegraphics[width=0.75\textwidth]{Pictures132/10000000000003D8000003D88D9A9299098FC632.png}
\end{center}

\pagebreak


\subsection{Character Advancement}\label{character-advancement}\index{Character Advancement}

\begin{multicols}{2}

\subsection{Experience Points (XP)}\label{experience-points-xp}\index{Experience Points (XP)}

Experience points are given for monsters defeated, and for other challenges as the GM sees fit. The following table provides XP values for monsters. Where a monster has both a character level and hit dice given, use the larger value as the monster' s level. Non-combat challenges may be assigned a level, or a flat XP value assigned, as the GM wishes.

If asterisks appear after the hit dice listing for a monster, each asterisk adds the special ability bonus once; for example, a creature with a hit dice figure of 2** is worth 125 XP.

For monsters with more than 25 hit dice, add 750 XP to the XP Value and 25 XP to the Special Ability Bonus per additional hit die. 

NPCs should be treated as monsters of a number of hit dice equivalent to the character' s level. Add a special ability bonus for Clerics and Magic-Users if they are able to cast useful spells during the encounter.


\begin{flushleft}
	\includegraphics[width=0.47\textwidth]{Pictures132/10000000000002A3000004548C62710626AE11F6.png}
\end{flushleft}

After tallying the XP earned in a given adventure, the amount should be divided by the number of adventurers. As described above, each retainer should receive a one-half share; so a group with four player characters and a retainer is counted as having 4½ members. If 2,000 XP are earned by this group, one share is 444 XP, and the retainer receives 222 XP. 

No character may advance more than one level due to the experience points from a single adventure. For example, Barthal the Thief is 1st level and has 1,000 XP before going on an adventure; during the adventure, he earns 2,000 more XP (an amazing feat). This would make his total 3,000 XP, and he would be a 3rd level Thief. This is not allowed; instead, he advances to 2,499 XP, one short of the amount required for 3rd level, and starts his next adventure at 2nd level.\\\medskip

\begin{tabular*}{0.93\linewidth}{@{\extracolsep{\fill}}lll}
\textbf{Monster Hit Dice} & \textbf{XP Value} & \textbf{Special Ability Bonus}\\\toprule
less than 1 & 10 & 3 \\\hline
1 & 25 & 12 \\\hline
2 & 75 & 25 \\\hline
3 & 145 & 30 \\\hline
4 & 240 & 40 \\\hline
5 & 360 & 45 \\\hline
6 & 500 & 55 \\\hline
7 & 670 & 65 \\\hline
8 & 875 & 70 \\\hline
9 & 1,075 & 75 \\\hline
10 & 1,300 & 90 \\\hline
11 & 1,575 & 95 \\\hline
12 & 1,875 & 100 \\\hline
13 & 2,175 & 110 \\\hline
14 & 2,500 & 115 \\\hline
15 & 2,850 & 125 \\\hline
16 & 3,250 & 135 \\\hline
17 & 3,600 & 145 \\\hline
18 & 4,000 & 160 \\\hline
19 & 4,500 & 175 \\\hline
20 & 5,250 & 200 \\\hline
21 & 6,000 & 225 \\\hline
22 & 6,750 & 250 \\\hline
23 & 7,500 & 275 \\\hline
24 & 8,250 & 300 \\\hline
25 & 9,000 & 325 \\\bottomrule
\end{tabular*}


\end{multicols}
\pagebreak

\section{PART 5: THE ENCOUNTER}\label{part-5-the-encounter}\index{Part 2: The Encounter}

\textit{I raised my shield to fend off one of the monsters, and hewed at another with my sword, but I missed my first swing. Morningstar swung at one of the monsters and struck it, but her sword did the bony thing little harm. I saw that Apoqulis still stood by the door; of Barthal there was no sign. Fortunately, Apoqulis also had a torch.}

\textit{Apoqulis raised his holy symbol and called in a loud voice, "In the name of Tah, begone!" To my surprise, several of the monsters turned as if afraid and ran out the door, disappearing into the gloom. Unfortunately this left quite a few of them still in the room.}

\textit{Even as I saw all this I continued to hack at the monsters. It took two good blows to down the first one; it appeared that Morningstar was having similar trouble with the monsters. Then one of the skeletons hit her, just a minor wound, but still I felt good that I had invested my part of the proceeds of our last excursion in a suit of plate mail armor; I was shrugging off blows that would have harmed me were I still wearing chain mail.}

\textit{To my surprise, I saw Apoqulis down one of the monsters in a single blow, then do the same to another in his very next strike. His mace seemed to be much more effective against the monsters than our swords. As I finally managed to down a second skeleton, I heard a high-pitched yell... it was Barthal, a little ways down the hallway, and he was throwing something.}

\textit{There was a sound of glass breaking, and I felt a splash of water on my face. Several of the skeletons began to smoke, and then one of them fell in a heap. Holy water, I decided, but I didn' t have time to think about it. I just kept hacking at the skeletons.}

\textit{By the time they were all gone, I had taken a wound, and Morningstar had taken a second. We had one potion of healing left of those that Apoqulis' temple had given us; Morningstar told me to drink it, but I could tell she was in worse shape than I, so I insisted she take it.}

\textit{Then we turned back to the sarcophagus...}

\begin{multicols}{2}

\subsection{Order of Play}\label{order-of-play}\index{Order of Play}

When the party of adventurers comes in contact with potential enemies, time shifts to combat rounds (10 seconds long, as described previously). Before beginning combat, surprise is checked (see below). Unsurprised characters then roll for Initiative, and act in order of the rolls (again, as described below).

\subsection{Surprise}\label{surprise}\index{Surprise}

When surprise is possible, roll 1d6 for each side which might be surprised; most normal characters are surprised on a roll of 1-2. Surprised characters are unable to act for one round. Characters or creatures which are well hidden and prepared to perform an ambush surprise on a roll of 1-4 on 1d6. Some characters or creatures (such as Elves) are described as being less likely to be surprised; reduce the range by 1 for such creatures.

For example: Darion (a Human) and Morningstar (an Elf) open a door and come face-to-face with a party of goblins. The GM rolls 1d6 for the goblins; on a 1-2 they are all surprised. Then the GM rolls 1d6 for Darion and Morningstar. If the result is 1, both of them are surprised; if the roll is 2, only Darion is surprised. If the roll is 3 or more, neither of them are surprised.

Surprised characters or creatures stand flat-footed for one round. They still defend themselves, so there is no penalty to Armor Class, but they cannot move nor attack during the round of surprise. 

\subsection{Monster Reactions}\label{monster-reactions}\index{Monster Reactions}

When a group of player characters meet one or more monsters, it' s important to know how the monsters will react to the party. In many cases, the reaction of the monster or monsters is obvious... zombies guarding a tomb will virtually always attack intruders, for example. 

In cases where the reaction of the monsters to the party is not obvious, a \textbf{reaction roll} may be made. The Game Master rolls 2d6, adding the Charisma bonus of the "lead" character (or applying their Charisma penalty) along with any other adjustments they feel are reasonable, and consults the table below:\\

\textbf{Reaction Roll Table}\\\medskip

\begin{tabular*}{0.93\linewidth}{@{\extracolsep{\fill}}ll}
\textbf{Adjusted Die Roll} & \textbf{Result} \\\toprule
2 or less & Immediate Attack \\\hline
3-7 & Unfavorable \\\hline
8-11 & Favorable \\\hline
12 or more & Very Favorable \\\bottomrule
\end{tabular*}\\\medskip

A result of 2 or less means that the player characters have so offended the monsters that they attack immediately. An Unfavorable result means that the monsters do not like the player characters, and will attack if they may reasonably do so. A Favorable result simply means that the monsters will consider letting the player characters live if they choose to parley; it does not necessarily mean that the monsters \emph{like} the player characters. A Very Favorable result means that the monsters (or perhaps only the monster leader) do, in fact, like the player characters; this does not mean that the monsters will just hand over their treasure, but it does indicate that they may choose to cooperate with the player characters in mutually beneficial ways.

As always, interpreting the results of this roll is left to the GM, who may choose to alter the result if they believe a different result would be more enjoyable to play out than the one rolled.

\subsection{Initiative}\label{initiative}\index{Initiative}

Each combat round, 1d6 is rolled for Initiative for each character or monster. This roll is adjusted by the character' s Dexterity bonus. High numbers act first. Any characters/monsters with equal numbers act simultaneously. The GM may make single rolls for groups of identical monsters at their option.

As the GM counts down the Initiative numbers, each character or monster may act on their number. If desired, a combatant can choose to wait until a later number to act. If a player states that they are waiting for another character or monster to act, then that character' s action takes place on the same Initiative number as the creature they are waiting for, and is simultaneous just as if they rolled the same number.

A character using a weapon with a long reach (spears, for instance) may choose to attack a closing opponent on the closing opponent' s number and thus attack simultaneously with the opponent, even if the character rolled lower for Initiative.

\begin{flushleft}
	\includegraphics[width=0.47\textwidth]{Pictures132/10000000000003D800000264E1D42E3BF50CCEF4.png}
\end{flushleft}

\subsection{Combat}\label{combat}\index{Combat}

Each character or creature involved in combat may move, if desired, up to its encounter movement distance, and then attack, if any opponent is in range, when its Initiative number comes up. After attacking, a character or creature may not move again until the next round.

Opponents more than 5' apart may move freely, but once
two opposing figures are within 5' of each other, they
are "engaged" and must abide by the rules under \textbf{Disengaging From
Melee }on page \hyperlink{disengaging-from-melee}{\pageref{disengaging-from-melee}}.

\begin{flushleft}
	\includegraphics[width=0.47\textwidth]{Pictures132/10000000000003CF000000FD73150EA8294B0E88.png}
\end{flushleft}

\subsection{Running}\label{running}\index{Running}

Characters may choose to run; a running character is not normally allowed to attack (but see Charging, below). Running characters can move at double their normal encounter movement rate. Characters are allowed to run a number of rounds equal to 2 times the character' s Constitution, after which they are exhausted and may only walk (at the normal encounter rate). For monsters not having a given Constitution, allow the monster to run for 24 rounds. Exhausted characters or creatures must rest for at least a turn before running again.

\subsection{Maneuverability}\label{maneuverability}\index{Maneuverability}

\textit{The following rules may be considered optional. They are hardly needed for most dungeon adventures, but will add measurably to combat situations in the wilderness, especially in waterborne combat situations or when some or all combatants are flying.}

Characters, creatures, and vehicles of various sorts have a turning distance. This is given as a distance in feet in parentheses after their movement rate, and it determines how far they must move between facing changes when moving about in combat.

All normal player characters, and in fact most moderately sized creatures which walk on the ground, have a turning distance of 5'. If no turning distance is given for a creature, assume that it is 5'.

In general, a facing change is any turn of up to 90º (a right-angle turn); on a square-gridded map, this means turning to face directly to the right or left of the figure' s current facing. A half-turn (45º) still counts as a full facing change. If using hexes, "diagonal" movement is not available, so a facing change is the 60º turn to face toward the hex-side to the right or left of the current facing.

There are a few exceptions to this rule: 

First, any creature that does not move away from its starting position during the combat round may make as many facing changes as desired (though circumstances, such as trying to turn a horse around in a narrow corridor, may prevent this).

Incorporeal flying creatures, such as spectres, can turn freely at any point while moving.

Creatures which are running (moving at double speed) may not make facing changes of more than 60º, and their turning distance increases by 10' (or, if it is 5' normally, it increases to 10').

Also, most creatures can shift one space laterally while preserving their facing (this is called "sidestepping"), but this may only be done when moving at normal ("walking") speed, not at fast ("running") speed. "One space" means either 5' or 10', depending on the map or board being used.

\subsection{Climbing and Diving}\label{climbing-and-diving}\index{limbing and Diving}

For battles involving three dimensions, each creature or vehicle has an altitude (when flying) or depth (underwater). For air or sea battles, at least one of the creatures or vehicles should start at an altitude/depth of 0, and a new 0 level can be established at any time, to simplify play, by adjusting the altitudes of each creature or vehicle.

A winged flier can gain up to 10' of altitude after moving forward by the distance shown for its maneuverability class, and can dive (lose altitude in a controlled fashion) at up to twice the normal movement rate; if the creature does not move horizontally by at least one-third its normal speed, it will stall, being forced to dive at maximum rate for one round. Floating creatures or vehicles (balloons, fly spell, flying carpets, etc.) can climb vertically without horizontal motion up to half the normal movement rate, but such "floaters" can only descend at the normal movement rate, unless they have lost the ability to float entirely.

\subsection{Charging}\label{charging}\index{Charging}

Under some circumstances, characters or creatures may be allowed to attack at the end of a running move. This is called a\textbf{ charge}, and some specific limitations apply. The charging character or creature must have moved at least 10 feet. The movement must be in a more or less straight line toward the intended target, and the path to the target must be reasonably clear. Finally, the attacker must be using a weapon such as a spear, lance, or pole arm which is suitable for use while charging. Certain monsters, especially those with horns, are able to use natural attacks when charging. If the attacker does not have line of sight to an opponent, they can' t charge that opponent.

The attack made after the charge is made at +2 on the attack roll. The charging character or creature suffers a -2 penalty to Armor Class for the remainder of the round. If the attack hits, it does double damage.

\textbf{Set Weapon Against Charge}: Spears, pole arms, and certain other piercing weapons deal double damage when "set" (braced against the ground or floor) and used against a charging creature. For this to be done, the character or creature being charged must have equal or better Initiative; this counts as holding an action: both attacker and defender act on the attacker' s Initiative number and are therefore simultaneous.

\subsection{Disengaging From Melee}\label{disengaging-from-melee}\index{Disengaging From Melee}

When any combatant is within reach of the melee attacks of at least one enemy, that combatant is considered to be \textbf{engaged}. Such a combatant may \textbf{disengage} in one of two ways:

First, the combatant may simply \textbf{flee}, turning away from all opponents they are engaged with and moving more than half normal movement. All opponents with whom they are engaged are allowed a \textbf{"parting shot"} with a +2 bonus to attack, even if that opponent has already made all attacks for the round. Opponents who have multiple melee attacks per round make just one; for instance, a tiger with the usual "2 claws, 1 bite" routine could only claw once or bite once.

To avoid the parting shot, the combatant may choose to \textbf{withdraw}, i.e. back away by up to half normal movement. After a withdrawal, the character may still attack at any point later in the same round if an opponent is within reach.

\subsection{Evasion and Pursuit}\label{evasion-and-pursuit}\hypertarget{evasion-and-pursuit}{}\index{Evasion and Pursuit}

Sometimes a party of adventurers will want nothing more than to avoid a group of monsters (or sometimes, it' s the monsters avoiding the adventurers). If one group is surprised, and the other is not, the unsurprised group may be able to escape automatically (unless something prevents them from making an exit). Otherwise, those wanting to evade the encounter begin doing so on their Initiative numbers. Note that the rules above for \textbf{Disengaging From Melee} will naturally apply to any combatant who is in reach of an enemy.

The GM may easily play out the pursuit, following along on their map (note that the players can' t draw maps while they run headlong through the dungeon or wilderness area). Any time a character must pass through a doorway, make a hard turn, etc., the GM may require a saving throw vs. Death Ray (with Dexterity bonus added); if the save is failed, the character has fallen at that point and moves no further that round; they may stand up and make a full move (but not a double move) on their Initiative number in the next round.

If the fleeing characters or creatures are ever able to get beyond the pursuer' s sight for a full round, they have evaded pursuit... the pursuers have lost them.

\end{multicols}

\subsection{Attack Bonus Table}
\begin{tabular*}{1\linewidth}{@{\extracolsep{\fill}}ccccc}
\textbf{Fighter Level} & \textbf{Cleric or Thief Level} & \textbf{Magic-User Level} & \textbf{Monster HD}& \textbf{Attack Bonus}\\\toprule
NM & & & less than 1 & +0\\\hline
1		& 1-2 &	1-3	 & 1	& +1\\\hline
2-3		& 3-4 & 4-5  & 2 & +2\\\hline
4		& 5-6 & 6-8  & 3 & +3\\\hline
5-6		& 7-8 & 9-12 & 4 & +4\\\hline
7		& 9-11& 13-15& 5 & +5\\\hline
8-10	& 12-14&16-18& 6 & +6\\\hline
11-12	& 15-17&19-20& 7 & +7\\\hline
13-15	& 17-20&	 &8-9& +8\\\hline
16-17	&		&	& 10-11&+9\\\hline			
18-20	&		&	& 12-13&+10\\\hline
		&		&	& 14-15&+11\\\hline
		&		&	& 16-19&+12\\\hline
		&		&	& 20-23&+13\\\hline
		&		&	& 24-27&+14\\\hline
		&		&	& 28-31&+15\\\hline
		&		&	& 32 or more&+16\\\bottomrule
\end{tabular*}

\begin{multicols}{2}
	


\subsection{How to Attack}\label{how-to-attack}\index{How to Attack}

To roll "to hit," the attacker rolls 1d20 and adds their attack bonus (AB), as shown on the Attack Bonus table, as well as Strength bonus (if performing a melee attack) or Dexterity bonus (if performing a missile attack) and any other adjustments required by the situation. If the total is equal to or greater than the opponent' s Armor Class, the attack hits and damage is rolled. A natural "1" on the die roll is always a failure. A natural "20" is always a hit, if the opponent can be hit at all (for example, monsters that can only be hit by silver or magic weapons cannot be hit by normal weapons, so a natural "20" with a normal weapon will not hit such a monster).

\subsection{Attacking From Behind}\label{attacking-from-behind}\index{Attacking From Behind}

Attacks made from behind an opponent usually receive a +2 attack bonus. This does not combine with the Sneak Attack ability (as described for the Thief on page \hyperlink{attacking-from-behind}{\pageref{attacking-from-behind}}).

\subsection{Normal Men}\label{normal-men}\index{Normal Men}\index{NM}

The NM entry in the table above is for \textbf{normal men}, also known as \textbf{zero level characters}. These characters represent the artisans, shopkeepers, scullery maids, and other non-adventurer characters who will appear in the game. All such characters are NPCs, of course. As mentioned elsewhere, it is up to the GM to determine if members of non-Human character races have zero level members of their own, and the exact statistics of such characters.

Average zero-level humans have 1d4 hit points, and usually are not proficient with any weapons except bare hands. Green troops (those who have not been in battle yet) are zero-level, but have 1d6 hit points and are allowed to use any weapon allowed to a Fighter.

It is recommended not to waste time in detailing the ability score or other statistics of such characters further; they are normal, as in "average," and so very few would have extreme statistics. A blacksmith might be credited with a Strength score of 13 or more, or a savant with Intelligence of 16 or more, but in general such things need not be detailed for most of these characters.

\subsection{Monster Attack Bonus}\label{monster-attack-bonus}\index{Monster Attack Bonus}

When looking up a monster' s hit dice on the Attack Bonus Table, ignore all "plus" or "minus" values; so a monster with 3+2 hit dice, or one with 3-1, is still treated as just 3 hit dice. The exception is monsters with 1-1 hit dice or less, which are considered as being less than one hit die and have an attack bonus of +0.

\subsection{Melee Combat}\label{melee-combat}\index{Melee Combat}

Melee occurs when a character closes (approaches within the reach of their own weapon) and strikes at a foe. Melee weapons or attacks may generally only be used against foes who are engaged with the attacker (as described above).

\subsection{Missile Fire}\label{missile-fire}\index{Missile Fire}

Missile weapons may be used to attack foes at a distance. The distance the attacker is from his target affects the attack roll, as shown on the Missile Weapon Ranges table in the previous section on \textbf{Characters}. In general, opponents within Short range are attacked at +1 on the die, those beyond Short range but within Medium range are attacked at +0, and those beyond Medium but within Long range are attacked at -2. Foes beyond Long range cannot be effectively attacked.

If a character attempts to use a missile weapon against a foe who is within 5' of them (i.e. who is engaged with the shooter), a penalty of -5 is applied to the attack roll. This is due to the shooter dodging around to avoid the foe' s attacks. The only exception is if the attacker is behind the target creature and undetected, or that creature is distracted so as to not be able to attack the shooter; in these cases, apply the usual +1 bonus (+3 total bonus if attacking from behind).

\subsection{Cover and Concealment}\label{cover-and-concealment}\index{Cover and Concealment}

In certain situations, the intended target of a missile (or melee) attack may have cover or concealment of some kind. Cover is defined as "hard" protection such as that afforded by a thick tree trunk or stone wall, that is, anything that will stop or slow a missile weapon. Concealment is "soft" cover like fog or light foliage that makes the target difficult to see but does not affect the missile itself. Cover or concealment makes it more difficult to strike an intended target, and thus a penalty will be applied to the attacker' s die roll depending upon how much of the target is protected from attack. For concealment the attack penalty should range from -1 (25\% obscured) to -4 (90\% obscured). For hard cover, these penalties should be doubled. 

\subsection{Missile Weapon Rate of Fire}\label{missile-weapon-rate-of-fire}\index{Missile Weapon Rate of Fire}

In general, missile weapons are allowed a single attack per round, just as are melee weapons. However, crossbows are an exception, as reloading a crossbow between shots is time-consuming.

A light crossbow can be fired once per two rounds, and the user may not perform any other actions (including movement) during the "reloading" round. A heavy crossbow can be fired just once per three rounds, again requiring the user to spend two rounds doing nothing other than cocking and loading the weapon in order to fire it again.

Siege engines also fire less often than ordinary weapons. The rate of fire for such a weapon is presented as a fraction, indicating the number of attacks per round; for example, 1/6 means one attack every six rounds.

Of course, the user of such a weapon may drop or sling the weapon and switch to another weapon rather than reloading. Also, it is possible (especially when defending a position) to load more than one crossbow in advance and then switch weapons each round until all have been fired. In a dungeon environment this sort of strategy is unlikely, of course.

\subsection{Grenade-Like Missiles}\label{grenade-like-missiles}\index{Grenade-Like Missiles}

When throwing grenade-like missiles (flasks of oil, etc.), a successful attack roll indicates a direct hit. Otherwise, the GM will roll 1d10 and consult the diagram below to determine where the missile hit. Treat each number as representing a 10' square area.

\begin{center}
	\begin{tabular}{lcl}
	&(behind)&\\\cmidrule(lr){2-2}
	& \cellcolor[HTML]{C0C0C0}\textbf{0} & \\ \cmidrule(lr){2-2}
	\cellcolor[HTML]{C0C0C0}\textbf{7}&\cellcolor[HTML]{C0C0C0}\textbf{8}&\cellcolor[HTML]{C0C0C0}\textbf{9}\\\cmidrule(lr){1-1}\cmidrule(lr){2-2}\cmidrule(lr){3-3}
	\cellcolor[HTML]{C0C0C0}\textbf{5}&\textbf{Target}&\cellcolor[HTML]{C0C0C0}\textbf{6}\\\cmidrule(lr){1-1}\cmidrule(lr){2-2}\cmidrule(lr){3-3}
	\cellcolor[HTML]{C0C0C0}\textbf{2}&\cellcolor[HTML]{C0C0C0}\textbf{3}&\cellcolor[HTML]{C0C0C0}\textbf{4}\\\cmidrule(lr){1-1}\cmidrule(lr){2-2}\cmidrule(lr){3-3}
	&\cellcolor[HTML]{C0C0C0}\textbf{1}&\\\cmidrule(lr){2-2}
	&(in front)&\\
\end{tabular}
\end{center}

\subsection{Missiles That Miss}\label{missiles-that-miss}\index{Missiles That Miss}

With the exception of grenade-like missiles, missile weapons which miss the intended target are normally considered lost. However, if the weapon is fired into a melee where allies of the shooter are involved, and the attack misses, it may hit one of the allied creatures. The GM should decide which allies may be hit, and roll attacks against each until a hit is made or all possible targets are exhausted. These attack rolls are made with the shooter' s normal attack bonus, just as if they intended to attack the allied creature. However, the GM must make these rolls, not the player.

This rule is applied to attacks made by monsters, when appropriate. However, the GM still makes the rolls.

This rule is intentionally vague; the GM must decide when and how to apply it based on the circumstances of the battle. It is recommended that no more than three allies be "tried" in this way, but the GM may make an exception as they see fit.

\subsection{Oil}\label{oil}\index{Oil}

A flask of oil can be used as a grenade-like missile. The oil must be set afire in order to inflict damage; otherwise the oil is just slippery. Assuming some means of igniting the oil is at hand, a direct hit to a creature deals 1d8 points of fire damage, plus in the next round the target takes an additional 1d8 points of damage, unless they spend the round extinguishing the flames by some reasonable means. The GM must judge the method used; rolling on the floor (assuming it' s not oily also) or covering the flames with a wet blanket are good methods, for instance, while pouring or splashing water on burning oil does little good. In any event, a flask of burning oil only causes damage for two rounds at most.

If the oil is ignited by some sort of wick or fuse, then all other creatures within 5 feet of the point of impact receive 1d6 points of fire damage from the splash. A save vs. Death Ray is allowed to avoid this damage. If the flask does not hit the intended target (as described under Grenade-Like Missiles, above), then that creature may still take damage from the splash, and receives a saving throw. No saving throw is allowed for a creature which has received a direct hit.

A flask of oil spilled or splattered on the ground will burn for 10 rounds. Those attempting to cross the burning oil will receive 1d6 points of fire damage each round they are in it (with no saving throw in this case).

Fire-resistant creatures, including creatures having fire-based abilities, are not damaged by burning oil.

\subsection{Holy Water}\label{holy-water}\index{Holy Water}

Holy water is harmful to undead creatures. A character may throw a flask of holy water as a grenade-like missile, which will break and release the contents if thrown against the body of a corporeal creature. To use it against an incorporeal creature it must be opened and poured or splashed onto the target, generally requiring the attacker to be directly adjacent to it.

Each flask of holy water can inflict 1d8 points of damage to such a monster. In addition, each additional undead monster within 5 feet of the point of impact receives 1d6 points of damage from the splash. Holy water is only effective for one round.

\subsection{Damage}\label{damage}\index{Damage}

If an attack hits, the attacker rolls damage as given for the weapon. Melee attacks apply the Strength bonus or penalty to the damage dice, as do thrown missile weapons such as daggers or spears. Usually, attacks with bows or crossbows do not gain the Strength bonus, but sling bullets or stones do.

Also, magic weapons will add their bonuses to damage (and cursed weapons will apply their penalty). Note that, regardless of any penalties to damage, any successful hit will do at least one point of damage.

As explained elsewhere, a creature or character reduced to 0 hit points is dead.

\subsection{Subduing Damage}\label{subduing-damage}\index{Subduing Damage}

Attacks made with the "flat of the blade" for non-lethal damage are made at a -4 attack penalty and do half damage. Most weapons can be used this way; only those with penetration or slashing features on all sides cannot.

If a character is reduced to zero hit points who has taken at least some subduing damage, and the total amount of normal (killing) damage the character has suffered is not equal to or greater than their total hit points, the character becomes unconscious rather than dying. (Any further subduing damage is then considered killing damage, allowing the possibility that someone might be beaten to death.) A character knocked out in this way, but not subsequently killed, will wake up with 1 hit point in 1d4 turns, or can be awakened (with 1 hit point) by someone else after 2d10 rounds.

\subsection{Brawling}\label{brawling}\index{Brawling}

Sometimes a character will attack without a weapon, striking with a fist or foot. This is called brawling. Normal characters do 1d3 points of subduing damage with a punch, 1d4 with a kick; kicks are rolled at a -2 attack penalty. A character in no armor or leather armor cannot successfully punch or kick a character in metal armor, and in fact, if this is attempted the damage is applied to the attacker instead of the defender. The GM must decide which monsters can be successfully attacked this way. All character classes may engage in brawling; there is no "weapon" restriction in this case.

\subsection{Wrestling}\label{wrestling}\index{Wrestling}

A wrestling attack requires a successful melee attack roll, where success indicates the attacker has grabbed their opponent. This hold is maintained until the attacker releases it or the defender makes a save vs. Death Ray, which is attempted at the defender' s next action (according to Initiative). A successful wrestling attack causes the attacker to move into the same "space" as the defender (if miniature figures are used).

After achieving a hold on an opponent, the attacker can automatically inflict unarmed damage (as if striking with a fist), prevent a held opponent from speaking, use simple magic items such as rings, or take any other action the GM allows. The attacker may also attempt to acquire an item the opponent is holding (such as a weapon) or attempt to move the opponent (as described below). A held character may be voluntarily released whenever the attacker so desires.

The attacker can't draw or use a weapon or use a wand, staff, scroll or potion, escape another's wrestling attack, cast a spell, or pin another character while holding an opponent.

\textbf{Moving the Opponent:} The attacker can move up to one-half speed (bringing the defender along) with a successful attack roll, if the attacker is strong enough to carry or drag the defender.

\textbf{Acquiring an Object: } The attacker may attempt to take an item away from the defender. This requires an additional attack roll; if the roll fails, the defender may immediately attempt an attack roll (even if they have already attacked this round) which, if successful, results in the defender pinning the attacker; or, the defender may choose to escape instead of reversing the hold.

\textbf{Actions Allowed to the Defender: }The target of a successful hold is usually immobile (but not helpless) at least until their next action, as determined by Initiative. Such characters suffer a penalty of -4 to AC against opponents other than the attacker.

If the defender is significantly stronger and/or larger than the attacker, they may move at up to one-half speed, dragging the attacker along.

On the defender' s next action, they can try to escape the pin with a saving throw vs. Death Ray; the defender must apply the better of their Strength or Dexterity bonuses (or penalties) on this roll. If the escape roll succeeds, the defender finishes the action by moving into any space adjacent to the attacker.

If more than one attacker has a hold on a particular defender, a successful escape roll frees the defender from just one of those attackers.

Held characters may also use simple magic items such as rings. A character being held may not normally cast a spell, even if they have not been silenced by the attacker.

\textbf{Multiple Opponents: }Several combatants can be involved in a wrestling match. Up to four combatants can wrestle a single opponent of normal size in a given round. Creatures that are smaller than the attacker count for half, while creatures that are larger count at least double (as determined by the GM). Note that, after an opponent is pinned, other attackers benefit from the -4 AC penalty applied to the defender. However, this AC penalty is not cumulative (that is, each successful attack does not lower the defender' s AC
further).

It is also possible for another character to attack the attacker in an ongoing wrestling bout. In this case, a successful hold on the attacker grants the original defender a +4 bonus on subsequent escape rolls. 

\textbf{Wrestling With Monsters: }In general, the rules above can be used not only when character races wrestle but also when humanoid monsters are involved. The GM will decide whether or not to allow wrestling involving non-humanoid creatures on a case-by-case basis; if this is allowed, the following adjustments apply:

Creatures with extra grasping appendages (more than the usual two) gain a +1 bonus on attack rolls or saving throws for each such appendage. This includes creatures with feet capable of grasping (such as monkeys or apes, giant spiders, etc.)

Large creatures able to fly may attempt to carry off their opponents (even if the flying creature is the defender).

Wrestling attacks against creatures with touch attacks (such as wights) will cause the attacker to suffer one such attack automatically every round.

\subsection{Morale}\label{morale}\index{Morale}

NPCs and monsters don' t always fight to the death; in fact, most will try to avoid death whenever possible. Each monster listing includes the monster' s Morale score, a figure between 2 and 12. To make a Morale check, roll 2d6; if the roll is equal to or less than the Morale score, the monster or monsters are willing to stand and fight. If the roll is higher, the monster has lost its nerve. Monsters with a Morale score of 12 never fail a Morale check; they always fight to the death.

In general, Morale is checked when monster(s) first encounter opposition, and again when the monster party is reduced to half strength (by numbers if more than one monster, or by hit points if the monster is alone). For this purpose, monsters incapacitated by \textbf{sleep}, \textbf{charm}, or \textbf{hold }magic are counted as if dead.

The Game Master may apply adjustments to a monster' s Morale score at their discretion. Generally, adjustments should not total more than +2 or -2. No adjustment is ever applied to a Morale score of 12.

A monster that fails a Morale check will generally attempt to flee; intelligent monsters or NPCs may try to surrender, if the GM so desires.

Note that special rules apply to \textbf{retainers}, as explained further on page \hyperlink{retainers}{\pageref{retainers}}.

\subsection{Turning the Undead}\label{turning-the-undead}\index{Turning the Undead}

Clerics can Turn the undead, that is, drive away undead monsters by means of faith alone. The Cleric brandishes their holy symbol and calls upon the power of their divine patron. The player rolls 1d20 and tells the GM the result. Note that the player should always roll, even if the GM knows the character can' t succeed (or can' t fail), as telling the player whether or not to roll may reveal too much.

The GM looks up the Cleric' s level on the Clerics vs. Undead table, and cross-references it with the undead type or Hit Dice. (The Hit Dice row is provided for use with undead monsters not found in the Core Rules; only use the Hit Dice row if the specific type of undead monster is not on the table and no guidance is given in the monster' s description. Note that the hit dice given are not necessarily the same as the hit dice of the monster given for that column.) If the table indicates "No" for that combination, it is not possible for the Cleric to affect that type of undead monster. If the table gives a number, that is the minimum number needed on 1d20 to Turn that type of undead. If the table says "T" for that combination, that type of undead is automatically affected (no roll needed). If the result shown is a "D," then that type of undead will be Damaged (and possibly destroyed) rather than Turned.

If the roll is a success, 2d6 hit dice of undead monsters are affected; surplus hit dice are lost (so if zombies are being Turned and a roll of 7 is made, at most 3 zombies can be Turned), but a minimum of one creature will always be affected if the first roll succeeds.

\end{multicols}

\subsection{Clerics vs. Undead Table}

\begin{tabular*}{1\linewidth}{@{\extracolsep{\fill}}cccccccccc}
\textbf{Cleric}&\textbf{Skeleton}&\textbf{Zombie}&\textbf{Ghoul}&\textbf{Wight}&\textbf{Wraith}&\textbf{Mummy}&\textbf{Spectre}&\textbf{Vampire}&\textbf{Ghost}\\
\textbf{Level}&1 HD&2 HD&3 HD&4 HD&5 HD&6 HD&7 HD&8 HD&9+ HD\\\toprule
1  & 13 & 17 & 19 & No & No & No & No & No & No\\\hline
2  & 11 & 15 & 17 & 20 & No & No & No & No & No\\\hline
3  & 9  & 13 & 15 & 19 & No & No & No & No & No\\\hline
4  & 7  & 11 & 13 & 17 & 20 & No & No & No & No\\\hline
5  & 5  & 9  & 11 & 15 & 19 & No & No & No & No\\\hline
6  & 3  & 7  & 9  & 13 & 17 & 20 & No & No & No\\\hline
7  & 2  & 5  & 7  & 11 & 15 & 19 & No & No & No\\\hline
8  & T  & 3  & 5  & 9  & 13 & 18 & 20 & No & No\\\hline
9  & T  & 2  & 3  & 7  & 11 & 17 & 19 & No & No\\\hline
10 & T  & T  & 2  & 5  & 9  & 15 & 18 & 20 & No\\\hline
11 & D  & T  & T  & 3  & 7  & 13 & 17 & 19 & No\\\hline
12 & D  & T  & T  & 2  & 5  & 11 & 15 & 18 & 20\\\hline
13 & D  & D  & T  & T  & 3  & 9  & 13 & 17 & 19\\\hline
14 & D  & D  & D  & T  & 2  & 7  & 11 & 15 & 18\\\hline
15 & D  & D  & D  & T  & T  & 5  & 9  & 13 & 17\\\hline
16 & D  & D  & D  & D  & T  & 3  & 7  & 11 & 15\\\hline
17 & D  & D  & D  & D  & T  & 2  & 5  & 9  & 13\\\hline
18 & D  & D  & D  & D  & D  & T  & 3  & 7  & 11\\\hline
19 & D  & D  & D  & D  & D  & T  & 2  & 5  & 9\\\hline
20 & D  & D  & D  & D  & D  & T  & T  & 3  & 7\\\bottomrule
\end{tabular*}

\begin{multicols}{2}
	

\begin{flushleft}
	\includegraphics[width=0.47\textwidth]{Pictures132/10000000000003CF000003849F6A530E9126D37E.png}
\end{flushleft}

If a mixed group of undead (say, a wight and a pair of zombies) is to be Turned, the player still rolls just once. The result is checked against the weakest sort first (the zombies), and if they are successfully Turned, the same result is checked against the next higher type of undead. Likewise, the 2d6 hit dice are rolled only once. For example, if the group described above is to be Turned by a 2\textsuperscript{nd} level Cleric, they would first need to have rolled a 15 or higher to Turn the zombies. If this is a success, 2d6 are rolled; assuming the 2d6 roll is a 7, this would Turn both zombies and leave a remainder of 3 hit dice of effect. Wights are, in fact, 3 hit die monsters, so assuming the original 1d20 roll was a 20, the wight is Turned as well. Obviously, were it a group of 3 zombies and a wight, the 2d6 roll would have to be a total of 9 or higher to affect them all.

If a Cleric succeeds at Turning the undead, but not all undead monsters present are affected, they may try again in the next round to affect those which remain. If any roll to Turn the Undead fails, that Cleric may not attempt to Turn Undead again for one full turn. A partial failure (possible against a mixed group) counts as a failure for this purpose.

Undead monsters which are Turned flee from the Cleric and their party at maximum movement. If the party pursue and corner the Turned undead, they may resume attacking the party; but if left alone, the monsters will not return or attempt to attack the Cleric or those near them for at least 2d4 turns.

Undead monsters subject to a D (Damaged) result suffer 1d8 damage per level of the Cleric (roll once and apply the same damage to all undead monsters affected); those reduced to zero hit points are utterly destroyed, being blasted into little more than dust. Those surviving this damage are still Turned as above.

\subsection{Energy Drain}\label{energy-drain}\index{Energy Drain}

Sometimes characters are exposed to energy drain from undead or evil magic. Such energy drain is manifested in the form of "negative levels." For each negative level a victim receives, they suffer a semi-permanent loss of one hit die worth of hit points, a penalty of -1 on all attack and saving throw rolls (and any other roll made on 1d20), and -5\% to any percentile roll such as thief abilities. In addition, an affected spell caster loses access to one of their highest-level spell slots. The victim may or may not be allowed a saving throw to resist the effect (depending on the specific monster type).

If the character' s hit points are reduced to zero or less by means of energy drain, the victim is immediately slain. If the energy drain is caused by an undead monster, the victim will usually be transformed into that sort of undead (exact details vary by type of monster).

Negative levels may be removed by magic, such as the \textbf{restoration} spell. When a negative level is to be removed, divide the total number of hit points lost by the number of negative levels (rounding normally) to determine how many hit points are restored.

For example, a character suffers three negative levels of energy drain. The hit point losses rolled were 6, 5, and 2, for a total of 13 points lost. The first negative level removed restores 13 / 3 = 4.3333 hit points (which is rounded to 4 even). Now the character has two negative levels and has lost 9 hit points. The next time a negative level is removed, the character recovers 9 / 2 = 4.5 hit points, which is rounded to 5 even. Now the character has one negative level and 4 hit points lost. Removal of the last negative level will restore the remaining 4 points.

Those who have suffered energy drain generally have a gaunt, haggard look about them, noticeable by observant characters.

\subsection{Healing and Rest}\label{healing-and-rest}\index{Healing and
Rest}

Characters recover 1 hit point of damage every day, provided that normal sleep is possible. Characters who choose full bedrest regain an additional hit point each evening.

\begin{flushleft}
	\includegraphics[width=0.47\textwidth]{Pictures132/10000000000003D8000004F2C1F1610DCABC127A.png}
\end{flushleft}


Normal characters require 6 hours sleep out of every 24. Subtract from this number of hours the character' s Constitution bonus; so a character with 18 Constitution needs only 3 hours sleep per night (and a character with 3 Constitution needs 9 hours). Note that these figures are minimums; given a choice, most characters would prefer to sleep two or more hours longer.

Characters who get less than the required amount of sleep suffer a -1 penalty on all attack rolls and saving throws (as well as not receiving any hit points of healing). For each additional night where sufficient sleep is not received, the penalty becomes one point worse. Regardless of how long the character has gone without adequate sleep, the normal amount of sleep will remove these penalties.


\subsection{Constitution Point Losses}\label{constitution-point-losses}\index{Constitution Point Losses}

Any character who has lost Constitution points temporarily (such as due to a disease) may regain them with normal rest. The rate of recovery is one point per day, awarded each morning when the character awakens from a normal night' s sleep. If more than one Constitution point was lost, the character must make a save vs. Death Ray (without adjustments) to regain the final point; failure results in a permanent loss of that point.

If a Constitution loss results in a lower bonus or penalty, the character' s maximum hit points must be reduced appropriately; for instance, a character reduced from 16 to 15 Constitution goes from +2 to +1, thus losing one hit point per die rolled. If a reduction in maximum hit points reduces that figure to less than the character' s current hit points, reduce the current hit points to the new maximum hit point figure immediately.

When regaining Constitution, any increase that increases the character' s Constitution bonus results in the restoration of the hit points lost due to the reduction, added to the maximum hit point figure only. Current hit points will not be improved in this fashion, but rather must be regained by normal healing.

\subsection{Falling Damage}\label{falling-damage}\index{Falling Damage}

Characters suffer 1d6 points of damage per 10' fallen, up to a maximum 20d6. Fractional distances are rounded to the nearest whole number, so that a fall of 1-4' does no damage, 5'-14' does 1d6, etc.

\subsection{Deafness and Blindness}\label{deafness-and-blindness}\index{Deafness and Blindness}

A deafened creature can react only to what it can see or feel, is surprised on 1-3 on 1d6, and suffers a -1 penalty to its Initiative rolls. A blinded creature is surprised on 1-4 on 1d6, suffers a -4 penalty to its attack rolls, a -4 penalty to its Armor Class, and a -2 penalty to its Initiative rolls. These effects are modified when dealing with monsters having unusual sensory abilities; for example, bats may be affected by deafness as if blinded instead.

These penalties are for characters or creatures recently handicapped. Those who are normally blind or deaf may have reduced penalties at the GM' s option.

Note that the penalty for attacking an invisible opponent is the same as the penalty for attacking blind, that is, -4 on the attack roll. Do not apply this twice... a blind character attacking an invisible opponent is no worse off than if they were attacking a visible one.

\subsection{Attacking a Vehicle}\label{attacking-a-vehicle}\index{Attacking a Vehicle}

Attacks against vehicles (such as wagons or ships) are made against Armor Class 11. Each vehicle has listed Hardness and Hit Point values. Roll damage against the vehicle, and then reduce that damage by the Hardness value. Any excess damage is applied to the vehicle.

If the vehicle takes damage equal to or greater than the listed HP on one side, it is reduced to half speed due to wheel damage or a hull breach; if it takes this much again, it is immobilized, and this much damage will sink a ship.

\subsection{Repairing a Vehicle}\label{repairing-a-vehicle}\index{Repairing a Vehicle}

Damage done to a vehicle may be restored at a rate of 1d4 hit points per crew member per hour of labor. However, a vehicle can only be restored to 90\% of its maximum hit points by field repairs; a damaged ship must be put into drydock and repaired by a shipwright and his crew, while a wagon, cart or chariot will require a wagonmaker to repair them. Costs of such repairs are left to the Game Master to decide.

\subsection{Saving Throws}\label{saving-throws}\index{Saving Throws}

\textbf{Saving throws} represent the ability of a character or creature to resist or avoid special attacks, such as spells or poisons. Saving throws are made by rolling a d20 against a target number based on the character' s class and level; for monsters, a comparable class and level are provided for the purpose of determining the monster' s saving throw figures. As with attack rolls, a natural (unadjusted) roll of 20 is always a success, and a natural 1 is always a failure.

The five categories of saving throw as follows: \textbf{Death Ray or Poison}, \textbf{Magic Wands}, \textbf{Paralysis or Petrify}, \textbf{Dragon Breath}, and \textbf{Spells}. Spells and monster special attacks will indicate which category applies (when a saving throw is allowed), but in some unusual situations the Game Master will need to choose a category. One way to make this choice is to interpret the
categories metaphorically. For example, a GM might be writing an adventure wherein there is a trap that pours burning oil on the hapless adventurers. Avoiding the oil might be considered similar to avoiding Dragon Breath. Or perhaps a stone idol shoots beams of energy from its glaring eyes when approached. This attack may be considered similar to a Magic Wand, or if especially potent, a Spell. The saving throw vs. Death
Ray is often used as a "catch all" save versus many of the "ordinary" dangers encountered in a dungeon environment.

In general, saving throw rolls are not adjusted by ability score bonus or penalty figures. There are a few exceptions:

\begin{itemize}
\item
  Poison saving throws are adjusted by the character' s
  Constitution modifier.
\item
  Saving throws against illusions (such as \textbf{phantasmal force})
  are adjusted by the character' s Intelligence modifier.
\item
  Saving throws against \textbf{charm} spells (and other forms of mind
  control) are adjusted by the character' s Wisdom
  modifier.
\end{itemize}

The GM may decide on other saving throw adjustments as they see fit.

\subsection{Item Saving Throws}\label{item-saving-throws}\index{Item Saving Throws}

Area effects (such as fireball or lightning bolt spells) may damage items carried by a character as well as injuring the character. For simplicity, assume that items carried are unaffected if the character or creature carrying them makes their own saving throw. However, very fragile items (paper vs. fire, glass vs. physical impact, etc.) may still be considered subject to damage even if the bearer makes their save.

In any case where one or more items may be subject to damage, use the saving throw roll of the bearer to determine if the item is damaged or not. For example, if a character holding an open spellbook is struck by a fireball spell, they must save vs. Spells, and then save again at the same odds for the spellbook.

The GM should feel free to amend this rule as they wish; for instance, a backpack full of fragile items might be given a single saving throw rather than laboriously rolling for each and every item.

\end{multicols}

\vfill

\begin{center}
	\includegraphics[width=0.4\textwidth]{Pictures132/10000000000003C9000001F7FA8EDBBA57BEFDE0.jpg}
\end{center}

\pagebreak


\subsection{Saving Throw Tables by Class}\label{saving-throw-tables-by-class}\index{Saving Throw Tables by Class}

\begin{multicols}{2}

\bigskip

\textbf{Cleric}\label{cleric-1}\index{Cleric Saving Throw}\index{Saving Throw Cleric}\bigskip

\addvspace{1cm}

\begin{tabular*}{0.95\linewidth}{@{\extracolsep{\fill}}llllll}
\multirow{3}{*}&\textbf{Death}&&\textbf{Paralysis}&&\\
\multirow{2}{*}&\textbf{Ray or }&\textbf{Magic}&\textbf{or}&\textbf{Dragon}&\\
\textbf{Level} & \textbf{Poison} & \textbf{Wands} & \textbf{Petrify} & \textbf{Breath} &\textbf{Spell}\\\toprule
1 & 11 & 12 & 14 & 16 & 15 \\\hline
2-3 & 10 & 11 & 13 & 15 & 14 \\\hline
4-5 & 9 & 10 & 13 & 15 & 14 \\\hline
6-7 & 9 & 10 & 12 & 14 & 13 \\\hline
8-9 & 8 & 9 & 12 & 14 & 13 \\\hline
10-11 & 8 & 9 & 11 & 13 & 12 \\\hline
12-13 & 7 & 8 & 11 & 13 & 12 \\\hline
14-15 & 7 & 8 & 10 & 12 & 11 \\\hline
16-17 & 6 & 7 & 10 & 12 & 11 \\\hline
18-19 & 6 & 7 & 9 & 11 & 10 \\\hline
20 & 5 & 6 & 9 & 11 & 10 \\\bottomrule
\end{tabular*}

\addvspace{1.5cm}

\textbf{Fighter}\label{fighter-1}\index{Fighter Saving Throw}\index{Saving Throw Fighter}\bigskip

\begin{tabular*}{0.95\linewidth}{@{\extracolsep{\fill}}llllll}	 \multirow{3}{*}&\textbf{Death}&&\textbf{Paralysis}&&\\
\multirow{2}{*}&\textbf{Ray or }&\textbf{Magic}&\textbf{or}&\textbf{Dragon}&\\
\textbf{Level} & \textbf{Poison} & \textbf{Wands} & \textbf{Petrify} & \textbf{Breath} &\textbf{Spell}\\\toprule
NM & 13 & 14 & 15 & 16 & 18 \\\hline
1 & 12 & 13 & 14 & 15 & 17 \\\hline
2-3 & 11 & 12 & 14 & 15 & 16 \\\hline
4-5 & 11 & 11 & 13 & 14 & 15 \\\hline
6-7 & 10 & 11 & 12 & 14 & 15 \\\hline
8-9 & 9 & 10 & 12 & 13 & 14 \\\hline
10-11 & 9 & 9 & 11 & 12 & 13 \\\hline
12-13 & 8 & 9 & 10 & 12 & 13 \\\hline
14-15 & 7 & 8 & 10 & 11 & 12 \\\hline
16-17 & 7 & 7 & 9 & 10 & 11 \\\hline
18-19 & 6 & 7 & 8 & 10 & 11 \\\hline
20 & 5 & 6 & 8 & 9 & 10 \\\bottomrule
\end{tabular*}

\columnbreak

\textbf{Magic-User}\label{magic-user-1}\index{Magic-User Saving Throw}\index{Saving Throw Magic-User}\bigskip

\begin{tabular*}{0.95\linewidth}{@{\extracolsep{\fill}}llllll}	 \multirow{3}{*}&\textbf{Death}&&\textbf{Paralysis}&&\\
\multirow{2}{*}&\textbf{Ray or }&\textbf{Magic}&\textbf{or}&\textbf{Dragon}&\\
\textbf{Level} & \textbf{Poison} & \textbf{Wands} & \textbf{Petrify} & \textbf{Breath} &\textbf{Spell}\\\toprule
1 & 13 & 14 & 13 & 16 & 15 \\\hline
2-3 & 13 & 14 & 13 & 15 & 14 \\\hline
4-5 & 12 & 13 & 12 & 15 & 13 \\\hline
6-7 & 12 & 12 & 11 & 14 & 13 \\\hline
8-9 & 11 & 11 & 10 & 14 & 12 \\\hline
10-11 & 11 & 10 & 9 & 13 & 11 \\\hline
12-13 & 10 & 10 & 9 & 13 & 11 \\\hline
14-15 & 10 & 9 & 8 & 12 & 10 \\\hline
16-17 & 9 & 8 & 7 & 12 & 9 \\\hline
18-19 & 9 & 7 & 6 & 11 & 9 \\\hline
20 & 8 & 6 & 5 & 11 & 8 \\\bottomrule
\end{tabular*}

\addvspace{1.5cm}

\textbf{Thief}\label{thief-1}\index{Thief Saving Throw}\index{Saving Throw Thief}\bigskip

\begin{tabular*}{0.95\linewidth}{@{\extracolsep{\fill}}llllll}
\multirow{3}{*}&\textbf{Death}&&\textbf{Paralysis}&&\\
\multirow{2}{*}&\textbf{Ray or }&\textbf{Magic}&\textbf{or}&\textbf{Dragon}&\\
\textbf{Level} & \textbf{Poison} & \textbf{Wands} & \textbf{Petrify} & \textbf{Breath} &\textbf{Spell}\\\toprule
1 & 13 & 14 & 13 & 16 & 15 \\\hline
2-3 & 12 & 14 & 12 & 15 & 14 \\\hline
4-5 & 11 & 13 & 12 & 14 & 13 \\\hline
6-7 & 11 & 13 & 11 & 13 & 13 \\\hline
8-9 & 10 & 12 & 11 & 12 & 12 \\\hline
10-11 & 9 & 12 & 10 & 11 & 11 \\\hline
12-13 & 9 & 10 & 10 & 10 & 11 \\\hline
14-15 & 8 & 10 & 9 & 9 & 10 \\\hline
16-17 & 7 & 9 & 9 & 8 & 9 \\\hline
18-19 & 7 & 9 & 8 & 7 & 9 \\\hline
20 & 6 & 8 & 8 & 6 & 8 \\\bottomrule
\end{tabular*}

\end{multicols}

\vfill

\begin{center}
	\includegraphics[width=0.90\textwidth]{Pictures132/10000000000007E0000002C6B782F61C7CC1CDD4.jpg}
\end{center}


\pagebreak

\section{PART 6: MONSTERS}\label{part-6-monsters}\index{Part 6: Monsters}

\begin{multicols}{2}


\textbf{Name}: The first thing given for each monster is its name (or its most common name, if the monster is known by more than one). If an asterisk appears after the monster' s name, it indicates that the monster is only able to be hit by special weapons (such as silver or magical weapons, or creatures affected only by fire, etc.) which makes
the monster harder to defeat.

\textbf{Armor Class:} This line lists the creature's Armor Class. If the monster customarily wears armor, the first listed AC value is with that armor, and the second, in parentheses, is unarmored. Some monsters are only able to be hit (damaged) by silver or magical weapons; these are indicated with (s); some monsters may only be hit with magical weapons, indicated by (m).

\textbf{Hit Dice}: This is the creature's number of hit dice, including any bonus hit points. Monsters always roll eight sided dice (d8) for hit points, unless otherwise noted. So for example, a creature with 3+2 hit dice rolls 3d8 and adds 2 points to the total. A few monsters may be marked as having ½ hit dice; this means 1d4 points, and the creature has "less than one hit die" for attack bonus purposes.

One or two asterisks (*) may appear after the hit dice figure; where present, they indicate a Special Ability Bonus to experience points (XP) awarded for the monster. See \textbf{Character Advancement }in the \textbf{Adventure }section for more details.

If the monster' s \textbf{Attack Bonus} is different than its number of Hit Dice, for convenience the Attack Bonus will be listed in parentheses after the Hit Dice figure.

\textbf{Movement}: This is the monster' s movement rate, or rates for those monsters able to move in more than one fashion. For example, bugbears have a normal walking movement of 30', and this is all that is listed for them. Mermaids can only move about in the water, and so their movement is given as Swim 40'. Pegasi can both walk and fly, so their movement is listed as 80' Fly 160'.

In addition, a distance may appear in parentheses after a movement figure; this is the creature' s turning distance (as explained in \textbf{Maneuverability} on page \hyperlink{maneuverability}{\pageref{maneuverability}}). If a turning distance is not listed, assume 5'.

\textbf{Attacks:} This is the number (and sometimes type or types) of attacks the monster can perform. For example, goblins may attack once with a weapon, so they are marked \textbf{1 weapon}. Ghouls are marked \textbf{2~claws, 1 bite }as they can attack with both claws and also bite in one round. Some monsters have alternate attacks, such as the triceratops with an attack of \textbf{1 gore or 1 trample} which means that the creature can do a gore attack or a trample attack, but not both in the same round.

\textbf{Damage}: The damage figures caused by successful attacks by the monster. Generally this will be defined in terms of one or more die rolls.

\textbf{No. Appearing}: This is given in terms of one or more die rolls. Monsters that only appear underground and have no lairs will have a single die roll; those that have lairs and/or those that can be found in the wilderness will be noted appropriately. For example, a monster noted as "1d6, Wild 2d6, Lair 3d6" is encountered in groups of 1d6 individuals in a dungeon setting, 2d6 individuals in the wilderness, or 3d6 individuals in a lair.

Note that number appearing applies to combatants. Non-combatant monsters (juveniles, and sometimes females) do not count in this number. The text of the monster description should explain this in detail where it matters, but the GM is always the final arbiter.

\textbf{Save As}: This is the character class and level the monster uses for saving throws. Most monsters save as Fighters of a level equal to their hit dice, but this is not always the case.

\textbf{Morale}: The number that must be rolled equal to or less than on 2d6 for the monster to pass a Morale Check. Monsters having a Morale of 12 never fail morale checks, and fight until destroyed (or until they have no enemies left). Morale is explained further on page \hyperlink{morale}{\pageref{morale}}.

\textbf{Treasure Type}: This describes any valuables the monster is likely to have. See the Treasure section on page \hyperlink{part-7-treasure}{\pageref{part-7-treasure}} for more details about interpreting the letter codes which usually appear here. A monster' s treasure is normally found in its lair, unless described otherwise; sometimes for monsters who live in towns or other large groups this line will describe both the lair treasure as well as treasure carried by random individuals.

\textbf{XP}: This is the number of experience points awarded for defeating this monster. In some cases, the figure will vary; for instance, Dragons of different age categories will have different XP values. Review the Experience Points awards table in the \textbf{Adventure} section on page \hyperlink{experience-points-xp}{\pageref{experience-points-xp}} to calculate the correct figure in these cases.

\end{multicols}

\subsection{Beasts of Burden}\label{beasts-of-burden-1}\index{Beasts of Burden}

\begin{tabular*}{1\linewidth}{@{\extracolsep{\fill}}lllll}

& \textbf{Camel} & \textbf{Donkey} & \textbf{Horse, Draft} & \textbf{Horse, Riding} \\\hline
Armor Class: & 13 & 13 & 13 & 13 \\\hline
Hit Dice: & 2 & 2 & 3 & 2 \\\hline
No. of Attacks: & 1 bite, 1 hoof & 1 bite & 2 hooves & 2 hooves \\\hline
Damage: & 1 point bite, 1d4 hoof & 1d2 bite & 1d4 hoof & 1d4 hoof \\\hline
Movement: & 50' (10') {[}
40' (10') {]} & 40'
(10') & 60' (10') &
80' (10') \\\hline
No. Appearing: & Wild 2d4 & Wild 2d4 & domestic only & Wild 10d10 \\\hline
Save As: & Fighter: 2 & Fighter: 2 & Fighter: 3 & Fighter: 2 \\\hline
Morale: & 7 & 7 & 7 & 7 \\\hline
XP: & 75 & 75 & 145 & 75 \\\hline
& & & & \\\hline
& Horse, War & Mule & Pony & \\\hline
Armor Class: & 13 & 13 & 13 & \\\hline
Hit Dice: & 3 & 2 & 1 & \\\hline
No. of Attacks: & 2 hooves & 1 kick or 1 bite & 1 bite & \\\hline
Damage: & 1d6 hoof & 1d4 kick, 1d2 bite & 1d4 bite & \\\hline
Movement: & 60' (10') &
40' (10') & 40'
(10') & \\\hline
No. Appearing: & domestic only & domestic only & domestic only & \\\hline
Save As: & Fighter: 3 & Fighter: 2 & Fighter: 1 & \\\hline
Morale: & 9 & 7 & 6 (9) & \\\hline
XP: & 145 & 75 & 25 & \\\hline
\end{tabular*}\\\medskip

\begin{multicols}{2}
	
For convenience, animals commonly used to carry loads and/or characters are listed here together. Such creatures obviously have no treasure.

\textbf{Camels} are large animals found in arid environments that bear distinctive fatty deposits known as "humps" on their backs. There are two relevant species of camel described here: the far more common one-humped dromedary, and the two-humped Bactrian camel. Statistics presented above are for the dromedary; the Bactrian camel is slower and its movement is given in brackets. A light load for a camel is up to 400 pounds; a heavy load, up to 800 pounds.

\textbf{Donkeys} are hoofed mammals in the same family as the horse. They are smaller, but are strong and hardy. Burros are a similar species, and the statistics herein can be used for either; both varieties are capable of being taken into dungeons as pack animals. A light load for a donkey is up to 70 pounds; a heavy load, up to 140 pounds.

\textbf{Draft Horses} are large horses bred to be working animals doing hard tasks such as plowing and other farm labor. There are a number of breeds, with varying characteristics, but all share common traits of strength, patience, and a docile temperament. A light load for a draft horse is up to 350 pounds; a heavy load, up to 700 pounds.

\textbf{Riding Horses} are smaller horses bred and trained for riding. They cannot effectively fight while the rider is mounted. A light load for a riding horse is up to 250 pounds; a heavy load, up to 500 pounds.

\textbf{War Horses} are large, powerful horses which are both bred for their size, strength, and combat ability and trained to tolerate the sounds and stresses of battle. They are able to attack while the rider is mounted due to their training. A light load for a warhorse is up to 350 pounds; a heavy load, up to 700 pounds.

\textbf{Mules} are a domestic equine hybrid between a donkey and a horse. Mules vary widely in size, and may be of any color. They are more patient, hardier and longer-lived than horses, and are perceived as less obstinate and more intelligent than donkeys. Like donkeys, they are capable of being taken into dungeons as pack animals. A light load for a mule is up to 300 pounds; a heavy load, up to 600 pounds.

A \textbf{Pony} is a variety of small horse. Compared to a larger horse, a pony may have a thicker coat, mane and tail, with proportionally shorter legs, a wider barrel, heavier bone, a thicker neck and a shorter, broader head. Ponies can be trained for war, and the morale in parentheses above is for a war pony; this does not allow them to fight while a rider is mounted, however. A light load for a pony is up to 275 pounds; a heavy load, up to 550 pounds.

\end{multicols}

\pagebreak

\subsection{Monster Descriptions}\label{monster-descriptions}

\begin{multicols}{2}

\textbf{Ant, Giant (and Huge, Large}\label{ant-giant-and-huge-large}\index{Ant, Giant (and Huge, Large}

\begin{flushleft}
	\begin{tabularx}{0.48\textwidth}{lXXX} & \textbf{Giant} & \textbf{Huge} & \textbf{Large} \\\hline
Armor Class: & 17 & 15 & 13 \\\hline
Hit Dice: & 4 & 2 & 1 \\\hline
No. of Attacks: &\multicolumn{3}{c}{1 bite} \\\hline
Damage: & 2d6 bite & 1d10 bite & 1d6 bite \\\hline
Movement: & 60' (10') & 50' & 40' \\\hline
No. Appearing: & 2d6, Lair 4d6 & 3d6, Lair 4d8 & 4d6, Lair 4d10 \\\hline
Save As: & Fighter: 4 & Fighter: 2 & Fighter: 1 \\\hline
Morale: & \multicolumn{3}{c}{ 7 on first sighting, 12 after engaged} \\\hline
Treasure Type:&\multicolumn{3}{c}{U or special} \\\hline
XP: & 240 & 75 & 25 \\
\end{tabularx}
\end{flushleft}
\medskip

Giant ants are fantastically enlarged versions of the more common variety of ants. Normal workers are 5 to 6 feet long; queens are larger, growing up to 9 feet in length. Giant ants may be red or black; there is no statistical difference between them. Though relatively shy when first encountered, once combat begins they will fight to the death. They are known to collect shiny things, and so will sometimes have a small amount of treasure in their lair. 

Giant ants may occasionally mine shiny metals such as gold or silver; one in three (1-2 on 1d6) giant ant lairs will contain 1d100 x 1d100 gp value in relatively pure nuggets.

Large and huge ants are similar to giant ants in all ways except for size; large ants are 1 to 2 feet long, while huge ants are 3 to 4 feet in length. Though smaller, their colonies have more members, and so their lair treasures are of similar size to those found in the lairs of giant ants.

\vfill

\begin{center}
	\includegraphics[width=0.47\textwidth]{Pictures132/10000000000003CF000001C45962AA7F672B55E3.jpg}
\end{center}

\columnbreak

\subsection*{Antelope}\label{antelope}\index{Antelope}

\begin{tabularx}{0.48\textwidth}{ll}
Armor Class: & 13 \\\hline
Hit Dice: & 1 to 4 \\\hline
No. of Attacks: & 1 butt \\\hline
Damage: & 1d4 or 1d6 or 1d8 (see below) \\\hline
Movement: & 80' (10') \\\hline
No. Appearing: & Wild 3d10 \\\hline
Save As: & Fighter: 1 to 4 (as Hit Dice) \\\hline
Morale: & 5 (7) \\\hline
Treasure Type: & None \\\hline
XP: & 25 - 240 \\\hline
\end{tabularx}\medskip

The statistics above represent the swifter sorts of wild herd animals, including deer (1 hit die, usually), antelope (2 hit dice), elk (3 hit dice), and moose (4 hit dice). They are skittish and will flee if provoked, but males are more aggressive in the presence of females (use the parenthesized morale in this case).

Cattle, aurochs, and bison are not included in this category but rather can be found on page \hyperlink{cattle-including-aurochs-and-bison}{\pageref{cattle-including-aurochs-and-bison}}. 

Generally, 1 hit die herd animals inflict 1d4 points of damage on a hit, 2 and 3 hit die animals inflict 1d6, and 4 hit die animals inflict 1d8. The GM should feel free to vary from these figures as he or she sees fit; there are many types of herd animals in the world, and some are better armed than others.

\subsection*{Ape, Carnivorous}\label{ape-carnivorous}\index{Ape, Carnivorous}

\begin{tabularx}{0.48\textwidth}{ll}
Armor Class: & 14 \\\hline
Hit Dice: & 4 \\\hline
No. of Attacks: & 2 claws \\\hline
Damage: & 1d4 claw \\\hline
Movement: & 40' \\\hline
No. Appearing: & 1d6, Wild 2d4, Lair 2d4 \\\hline
Save As: & Fighter: 4 \\\hline
Morale: & 7 \\\hline
Treasure Type: & None \\\hline
XP: & 240 \\\hline
\end{tabularx}\medskip

Carnivorous apes appear much like ordinary gorillas, but are bad-tempered and aggressive. They are actually omnivores, but have a marked preference for meat. Adult females are 4½ to 5 feet tall and weigh up to 300 pounds, while males are larger, being 5½ to 6 feet tall and weighing up to 400 pounds.

\subsection*{Assassin Vine}\index{Assassin Vine}\label{assassin-vine}

See \textbf{Strangle Vine} on page \hyperlink{strangle-vine}{\pageref{strangle-vine}}.

\pagebreak

\subsection*{Aurochs}\index{Aurochs}\label{aurochs}

See \textbf{Cattle (including Aurochs and Bison)} on page
\hyperlink{cattle-including-aurochs-and-bison}{\pageref{cattle-including-aurochs-and-bison}}.

\subsection*{Barkling}\index{Barkling}\label{barkling}

\begin{tabularx}{0.48\textwidth}{ll}
Armor Class: & 15 (11) \\\hline
Hit Dice: & ½ (1d4 hit points) \\\hline
No. of Attacks: & 1 bite or 1 weapon \\\hline
Damage: & 1d4 bite, 1d4 or by weapon \\\hline
Movement: & 20' Unarmored 40' \\\hline
No. Appearing: & 3d4, Wild 4d6, Lair 5d10 \\\hline
Save As: & Normal Man \\\hline
Morale: & 7 (9) \\\hline
Treasure Type: & P, Q each; C, K in lair \\\hline
XP: & 10 \\\hline
\end{tabularx}\medskip

Barklings are diminutive furry humanoids with very dog-like faces. They stand between 2½ and 3½ feet tall and typically weigh around 45 pounds. They are pack hunters by nature, shy when encountered singly or in small groups but bold when their numbers are overwhelming. Use the higher morale figure when a barkling group outnumbers their enemies by 3 combatants to 1 or more.

Barklings can deliver a nasty bite but prefer to fight with weapons, favoring small weapons made for their stature and relative lack of strength; all such weapons do 1d4 points of damage on a hit. 

Barklings see well in the dark, having Darkvision with a range of 30 feet, but their sense of smell is where they excel; a barkling can track almost any living or corporeal undead creature by scent, even if it has been as much as a day since it passed.

In combat barklings usually wear chainmail armor which they craft themselves (as shown in the Armor Class given above).


\begin{center}
	\includegraphics[width=0.47\textwidth]{Pictures132/10000000000003D8000003C170A013A5B73BEF77.png}
\end{center}


One out of every ten barklings will be a warrior with 1 hit die (25 XP). In barkling encampments, one out of every twenty will be a chieftain of 2 hit dice (75 XP) having a +1 bonus to damage due to strength. In villages of 50 or more there will be a barkling lord of 3 hit dice (145 XP) who has +1 bonus to damage. Barklings gain a +1 bonus to their morale as long as they are led by any of their leaders.

In addition, a lair has a chance equal to 1-2 on 1d6 of a wizard being present (or 1-3 on 1d6 if a chieftain is present). A wizard is equivalent to a 1 hit die warrior barkling statistically, but has Magic-User abilities at level 1d4+1. For XP purposes, treat the wizard barkling as if it has a number of hit dice equal to its magic-user level ‑1, and assign one special ability bonus asterisk.

Barklings are sometimes confused with kobolds, for whom they have a particular hatred; calling a barkling a kobold or suggesting that the two species are related is considered a terrible insult.\\

\subsection*{Basilisk}\index{Basilisk}\label{basilisk}

\begin{tabularx}{0.48\textwidth}{lXX}
& Common & Greater* \\\hline
Armor Class: & 16 & 17 \\\hline
Hit Dice: & 6** & 8*** \\\hline
No. of Attacks: & 1 bite, 1 gaze & 1 bite, 1 gaze \\\hline
Damage: & 1d10 bite, petrification gaze & 1d12 + poison bite,petrification gaze \\\hline
Movement: & -- 20' (10') -- & \\\hline
No. Appearing: & 1d6, Wild 1d6, Lair 1d6 & 1 \\\hline
Save As: & Fighter: 6 & Fighter: 8 \\\hline
Morale: & 9 & 10 \\\hline
Treasure Type: & F & F, K \\\hline
XP: & 610 & 1,085 \\\hline
\end{tabularx}\medskip


\begin{center}
	\includegraphics[width=0.47\textwidth]{Pictures132/10000000000003CF0000034B3730E6C1E1F282EF.png}
\end{center}

A basilisk is a giant six-legged lizard-like monster that petrifies living creatures with its gaze. A basilisk has dark brown, green, or black skin on its back and a pale yellow or white belly. Adults reach a body length of 5 to 7 feet with a tail of roughly equal length, and a weight of 250 to 400 pounds. There is no particular difference in size between males and females.

Any living creature meeting the gaze of a basilisk must save vs. Petrify or be turned to stone instantly. In general, any creature surprised by the basilisk will meet its gaze. Those who attempt to fight the monster while averting their eyes suffer penalties of -4 to attack and -2 to AC. It is possible to use a mirror to fight the monster, in which case the penalties are -2 to attack and no penalty to AC. If a basilisk sees its own reflection in a mirror it must save vs. Petrify or be turned to stone; a petrified basilisk loses its power to petrify. Basilisks instinctively avoid mirrors or other reflective surfaces, even drinking with their eyes closed, but if an attacker can manage to surprise the monster with a mirror it may see its reflection.

The greater basilisk appears identical to the common variety, save that it is larger, having a body length of about 8 feet with a 7 to 9 foot long tail and weighing between 400 and 750 pounds. The skin of the greater basilisk is toxic to the touch, such that any living creature bitten by one or who touches one with bare skin must save vs. Poison or die. This effect persists even after the monster is dead, typically for about 2d20 hours; the only way to tell if the effect has subsided is to touch the corpse, an obviously bad idea.

\subsection*{Bat (and Bat, Giant)}\index{Bat (and Bat, Giant)}\label{bat-and-bat-giant}


\begin{flushleft}
	\begin{tabularx}{0.48\textwidth}{lXX}
& Bat & Giant Bat \\\hline
Armor Class: & 14 & 14 \\\hline
Hit Dice: & 1 Hit Point & 2 \\\hline
No. of Attacks: & 1 special & 1 bite \\\hline
Damage: & Confusion & 1d4 \\\hline
Movement: & 30' Fly 40' &10' Fly 60' (10') \\\hline
No. Appearing: & 1d100, Wild $\sim$1d100, Lair $\sim$1d100 & 1d10,Wild 1d10, Lair $\sim$1d10 \\\hline
Save As: & Normal Man & Fighter: 2 \\\hline
Morale: & 6 & 8 \\\hline
Treasure Type: & None & None \\\hline
XP: & 10 & 75 \\\hline
\end{tabularx}\medskip
\end{flushleft}

Bats have a natural sonar that allows them to operate in total darkness; for game purposes, treat this ability as equivalent to Darkvision.

A group of normal-sized bats has no effective attack (at least in terms of inflicting damage), but can confuse those in the area, flying around apparently randomly. For every ten bats in the area, one creature can be confused; such a creature will suffer a penalty of -2 on all attack and saving throw rolls while the bats remain in the area.

Agiant bat has a wingspan of 15 feet and weighs about 200 pounds. They have the same sensory abilities as normal-sized bats, but being much larger, they are able to attack adventurers; many are carnivorous, making such attacks likely.

\begin{center}
	\includegraphics[width=0.45\textwidth]{Pictures132/10000000000003CF0000033EED4848CB51A37897.png}
\end{center}


\subsection*{Bear}\index{Bear}\label{bear}

Bears attack by rending opponents with their claws, dragging them in and biting them. A successful hit with both paws indicates a hug attack for additional damage (as given for each specific bear type). All bears are very tough to kill, and are able to move and attack for one round after losing all hit points.


\subsection*{Bear, Black}\index{Bear, Black}\label{bear-black}

\begin{tabularx}{0.48\textwidth}{@{}lX@{}}
Armor Class: & 14 \\\hline
Hit Dice: & 4 \\\hline
No. of Attacks: & 2 claws, 1 bite + hug \\\hline
Damage: & 1d4 claw, 1d6 bite, 2d6 hug \\\hline
Movement: & 40' \\\hline
No. Appearing: & 1d4, Wild 1d4, Lair 1d4 \\\hline
Save As: & Fighter: 4 \\\hline
Morale: & 7 \\\hline
Treasure Type: & None \\\hline
XP: & 240 \\\hline
\end{tabularx}\medskip

Black bears are omnivorous, and despite their formidable size and strength are not particularly aggressive, though a female will fight fiercely if her cubs are threatened.

\subsection*{Bear, Cave}\index{Bear, Cave}\label{bear-cave}

\begin{tabularx}{0.48\textwidth}{@{}lX@{}}
Armor Class: & 15 \\\hline
Hit Dice: & 7 \\\hline
No. of Attacks: & 2 claws, 1 bite + hug \\\hline
Damage: & 1d8 claw, 2d6 bite, 2d8 hug \\\hline
Movement: & 40' \\\hline
No. Appearing: & 1d2, Wild 1d2, Lair 1d2 \\\hline
Save As: & Fighter: 7 \\\hline
Morale: & 9 \\\hline
Treasure Type: & None \\\hline
XP: & 670 \\\hline
\end{tabularx}

These monstrous bears are even larger than brown bears, with one weighing up to 1,800 pounds and when on all four feet are up to six feet high at the shoulder. They are ferocious killers, attacking almost anything of equal or smaller size.

\begin{center}
	\includegraphics[width=0.45\textwidth]{Pictures132/10000000000003FC000002607256097F4DB0F160.png}
\end{center}

\subsection*{Bear, Grizzly (or Brown)}\index{Bear, Grizzly (or Brown}\label{bear-grizzly-or-brown}

\begin{tabularx}{0.48\textwidth}{@{}lX@{}}
Armor Class: & 14 \\\hline
Hit Dice: & 5 \\\hline
No. of Attacks: & 2 claws, 1 bite + hug \\\hline
Damage: & 1d4 claw, 1d8 bite, 2d8 hug \\\hline
Movement: & 40' \\\hline
No. Appearing: & 1, Wild 1d4, Lair 1d4 \\\hline
Save As: & Fighter: 5 \\\hline
Morale: & 8 \\\hline
Treasure Type: & None \\\hline
XP: & 360 \\\hline
\end{tabularx}

Brown bears are huge, carnivorous, and aggressive. An adult male weighs 400 to 800 pounds and four feet high at the shoulder; females are slightly smaller, but just as bloodthirsty.

\subsection*{Bear, Polar}\index{Bear, Polar}\label{bear-polar}


\begin{center}
	\includegraphics[width=0.47\textwidth]{Pictures132/10000000000003D70000021B677F34FAEA47FEAD.png}
\end{center}

\begin{tabularx}{0.48\textwidth}{@{}lX@{}}
Armor Class: & 14 \\\hline
Hit Dice: & 6 \\\hline
No. of Attacks: & 2 claws, 1 bite + hug \\\hline
Damage: & 1d6 claw, 1d10 bite, 2d8 hug \\\hline
Movement: & 40' \\\hline
No. Appearing: & 1, Wild 1d2, Lair 1d2 \\\hline
Save As: & Fighter: 6 \\\hline
Morale: & 8 \\\hline
Treasure Type: & None \\\hline
XP: & 500 \\\hline
\end{tabularx}



Polar bears are found in far northern regions. They are larger and more powerful than brown bears, and just as hostile.

\subsection*{Bee, Giant}\index{Bee, Giant}\label{bee-giant}

\begin{tabularx}{0.48\textwidth}{@{}lX@{}}
Armor Class: & 13 \\\hline
Hit Dice: & ½* (1d4 hit points) \\\hline
No. of Attacks: & 1 sting \\\hline
Damage: & 1d4 + poison sting \\\hline
Movement: & 10' Fly 50' \\\hline
No. Appearing: & 1d6, Wild 1d6, Lair 5d6 \\\hline
Save As: & Fighter: 1 \\\hline
Morale: & 9 (12 if queen is threatened) \\\hline
Treasure Type: & Special \\\hline
XP: & 13 \\\hline
\end{tabularx}\medskip

Giant bees live in hives, generally in underground areas. In each such hive will be a queen who has 2 hit dice and inflicts only a bite doing 1d8 points of damage. She is immobile, and if she is threatened all bees in the hive will fight without checking morale. The queen is worth 75 XP if defeated.

Those stung by a giant bee must save vs. Poison or die. A giant bee that successfully stings another creature pulls away, leaving its stinger in the creature; the bee then dies.

Each giant bee hive will contain honeycomb filled with honey, which is entirely safe to eat and is worth 10 GP per gallon if carefully removed. Generally 2d10+10 gallons of honey will be present in any given hive. There is also a 15\% chance that one of the cells in the honeycomb will contain special honey which acts as 1d6+1 \textbf{Potions of Healing} if consumed. This honey can be discovered by chance, or through the use of \textbf{detect magic}.

\begin{center}
	\includegraphics[width=0.47\textwidth]{Pictures132/10000000000003CF0000032B17F232968844E223.png}
\end{center}

\subsection*{Beetle, Giant Bombardier}\index{Beetle, Giant Bombardier}\label{beetle-giant-bombardier}

\begin{tabularx}{0.48\textwidth}{@{}lX@{}}
Armor Class: & 16 \\\hline
Hit Dice: & 2* \\\hline
No. of Attacks: & 1 bite, 1 spray (special, see below) \\\hline
Damage: & 1d6 bite, 2d6 spray \\\hline
Movement: & 40' \\\hline
No. Appearing: & 1d8, Wild 2d6, Lair 2d6 \\\hline
Save As: & Fighter: 2 \\\hline
Morale: & 8 \\\hline
Treasure Type: & None \\\hline
XP: & 100 \\\hline
\end{tabularx}\medskip

Giant bombardier beetles have red head and thorax sections and black abdomens. They are 3 to 4 feet long. In combat, a giant bombardier beetle bites opponents in front of it, and sprays a cone of very hot and noxious gases from a nozzle in the rearmost tip of the abdomen. This toxic blast causes 2d6 points of damage to all within a cone 10' long and 10' wide at the far end (a save vs. Death Ray for half damage is allowed). A giant bombardier beetle can use this spray attack up to five times per day, but no more than once per three rounds. Faced with enemies attacking from one direction, a giant bombardier beetle may choose to turn away and use the spray attack rather than biting. 

Giant bombardier beetles, like most beetles, have about the same visual acuity in all directions, and thus suffer no penalty to Armor Class when attacked from behind.

\subsection*{Beetle, Giant Fire}\index{Beetle, Giant Fire}\label{beetle-giant-fire}

\begin{tabularx}{0.48\textwidth}{@{}lX@{}}
Armor Class: & 16 \\\hline
Hit Dice: & 1+2 \\\hline
No. of Attacks: & 1 bite \\\hline
Damage: & 2d4 bite \\\hline
Movement: & 40' \\\hline
No. Appearing: & 1d8, Wild 2d6, Lair 2d6 \\\hline
Save As: & Fighter: 1 \\\hline
Morale: & 7 \\\hline
Treasure Type: & None \\\hline
XP: & 25 \\\hline
\end{tabularx}\medskip

Giant fire beetles are huge, being 18 to 30 inches long, and have shiny black carapaces. Each has a pair of glowing red organs located just below their eyes which illuminate a radius of 10 feet around the creature. These glands continue to glow for 1d6 days after one is killed, and may be removed and used for illumination by any adventurers not too squeamish to do so.

They are normally timid but will fight if cornered. Like most beetles, they have more or less the same visual acuity in all directions, and thus those who attack them from behind receive no bonus to do so.

\subsection*{Beetle, Giant Oil}\index{Beetle, Giant Oil}\label{beetle-giant-oil}

\begin{tabularx}{0.48\textwidth}{@{}lX@{}}
Armor Class: & 16 \\\hline
Hit Dice: & 2* \\\hline
No. of Attacks: & 1 bite + spray (see below) \\\hline
Damage: & 2d4 bite, special spray (see below) \\\hline
Movement: & 40' \\\hline
No. Appearing: & 1d8, Wild 2d6, Lair 2d6 \\\hline
Save As: & Fighter: 2 \\\hline
Morale: & 8 \\\hline
Treasure Type: & None \\\hline
XP: & 100 \\\hline
\end{tabularx}\medskip

Giant oil beetles are about 3 feet long, and are often found burrowing in soil or roaming dungeon corridors. Their eyes are arranged on the sides of their heads such that they can see perfectly well behind them as well as in front, negating any normal bonus for attacking from behind.

In addition to its bite, a giant oil beetle can attack with a spray of oil from its abdomen; this can only be applied to opponents within 5 feet of the back of the beetle, and an attack roll is needed to hit. Living creatures hit by this spray suffer a penalty of -2 on all attack rolls for 24 hours due to painful blisters inflicted by the irritating oil. A \textbf{cure light wounds} spell may be used to remove this effect, but if so used the spell does not also restore hit points to the victim.

\subsection*{Beetle, Giant Tiger}\index{Beetle, Giant Tiger}\label{beetle-giant-tiger}

\begin{tabularx}{0.48\textwidth}{@{}lX@{}}
Armor Class: & 17 \\\hline
Hit Dice: & 3+1 \\\hline
No. of Attacks: & 1 bite \\\hline
Damage: & 2d6 bite \\\hline
Movement: & 60' (10') \\\hline
No. Appearing: & 1d6, Wild 2d4, Lair 2d4 \\\hline
Save As: & Fighter: 3 \\\hline
Morale: & 9 \\\hline
Treasure Type: & U \\\hline
XP: & 145 \\\hline
\end{tabularx}\medskip

Giant tiger beetles are predatory monsters around 5 feet long. Their carapaces tend to be dark brown with lighter brown striped or spotted patterns, but there are many variations.

They are fast runners, depending on their speed to run down prey, and they willingly prey on any creature of man size or smaller. Like most beetles, they have more or less the same visual acuity in all directions, and thus suffer no penalty to Armor Class when attacked from behind.


\subsection*{Bison}\index{Bison}\label{bison}

See \textbf{Cattle (including Aurochs and Bison)} on page \hyperlink{cattle-including-aurochs-and-bison}{\pageref{cattle-including-aurochs-and-bison}}.

\subsection*{Black Pudding}\index{Black Pudding}\label{black-pudding}

See \textbf{Jelly, Black} on page \hyperlink{jelly-black-black-pudding}{\pageref{jelly-black-black-pudding}}.

\subsection*{Blink Dog (Flicker Beast)}\index{Blink Dog (Flicker Beast)}\label{blink-dog-flicker-beast}

\begin{tabularx}{0.48\textwidth}{@{}lX@{}}
Armor Class: & 15 \\\hline
Hit Dice: & 4* \\\hline
No. of Attacks: & 1 bite \\\hline
Damage: & 1d6 bite \\\hline
Movement: & 40' \\\hline
No. Appearing: & 1d6, Wild 1d6, Lair 1d6 \\\hline
Save As: & Fighter: 4 \\\hline
Morale: & 6 \\\hline
Treasure Type: & C \\\hline
XP: & 280 \\\hline
\end{tabularx}\medskip

Blink dogs, also known as flicker beasts, are strange creatures which resemble wolves or hyenas who can teleport up to 120 feet at will. Teleportation is so easy for them that they can sometimes teleport away before being attacked; specifically, when one knows of the attacker' s presence. The blink dog is allowed a saving throw vs. Death Ray, and on a successful roll teleports 1d6 x 10 feet in a random direction (but never into solid matter, nor into any dangerous area the creature knows about).

These creatures are pack hunters, using their teleportation to confuse their prey until they are able to surround it. In this way, some members of the pack will be able to attack from behind, a trick they are so good at that they receive the same benefits as a thief: +4 to hit and double damage if the hit is successful.

Blink dogs are large canines, typically light brown in color with short bristly hair; some varieties are striped or spotted. They communicate using a language of barks, growls, and yips which is somewhat limited but can convey useful tactical information. They are shy, generally avoiding a fight if possible, but they hate \textbf{deceivers} and will generally attack them on sight.


\begin{center}
	\includegraphics[width=0.47\textwidth]{Pictures132/10000000000003D80000043CE69F2E3E4BD0CB68.png}
\end{center}

\subsection*{Blood Rose}\index{Blood Rose}\label{blood-rose}

\begin{tabularx}{0.48\textwidth}{@{}lX@{}}
	Armor Class: & 13 \\\hline
Hit Dice: & 2* to 4* \\\hline
No. of Attacks: & 1 to 3, each + blood drain \\\hline
Damage: & 1d6 bite, 1d6/round blood drain \\\hline
Movement: & 1' \\\hline
No. Appearing: & Wild 1d8 \\\hline
Save As: & Fighter: 2 \\\hline
Morale: & 12 \\\hline
Treasure Type: & None \\\hline
XP: & 100 -- 280 \\\hline
\end{tabularx}\medskip

Blood roses appear to be normal rose bushes, but are actually animated plants, dimly aware of their surroundings. These plants are always in bloom, bearing beautiful flowers that are normally white (or rarely, yellow) in color.

The fragrance of the flowers is detectable up to 30' from the plant in ideal conditions. Blood roses can move about slowly, and will try to find locations sheltered from the wind in order to achieve those ideal conditions. Living creatures who smell the fragrance must save vs. Poison or become befuddled; such a victim will drop anything carried and try to approach the plant. Each round an unaffected living creature can smell the fragrance it must make this save. Befuddled creatures will not resist the attacks of the blood rose; if affected creatures are removed from the area, the effect of the fragrance will expire 2d4 rounds later. Undead monsters, constructs, etc. are not affected.

Each blood rose plant will have 1, 2 or 3 whiplike canes studded with thorns with which it can attack. When a cane hits, it wraps around the victim and begins to drain blood, doing 1d6 points of damage per round. A blood rose which has recently (within one day) "eaten" in this way will have flowers ranging from pink to deep wine red in color, which will fade slowly back to white or yellow as the plant digests the blood it has consumed.

\subsection*{Boar}\index{Boar}\label{boar}


\begin{center}
	\includegraphics[width=0.47\textwidth]{Pictures132/10000000000005DA0000054997E95DB93EC7E84A.jpg}
\end{center}.

\begin{center}
	\begin{tabularx}{0.48\textwidth}{@{}lX@{}}
Armor Class: & 13 \\\hline
Hit Dice: & 3 \\\hline
No. of Attacks: & 1 tusk \\\hline
Damage: & 2d4 tusk \\\hline
Movement: & 50' (10') \\\hline
No. Appearing: & Wild 1d6 \\\hline
Save As: & Fighter: 3 \\\hline
Morale: & 9 \\\hline
Treasure Type: & None \\\hline
XP: & 145 \\\hline
\end{tabularx}\medskip

\end{center}
Wild
boars are the natural variety of swine. They are surly and aggressive, prone to attacking characters just because they are present. Note that "boar" refers specifically to the male of the species, but females are equally large and fierce, more so if they have offspring. 

Wild boars have stocky and muscular bodies, covered with bristly hair which may be black, brown, or gray in color. Those in colder climates will have an under-layer of warm fur, while those found in warmer areas will not

\subsection*{Bugbear}\index{Bugbear}\label{bugbear}

\begin{center}
\includegraphics[width=0.47\textwidth]{Pictures132/10000000000003D8000004D1119B4C657691018D.png}\medskip
\end{center}

\begin{center}
	\begin{tabularx}{0.48\textwidth}{@{}lX@{}}
Armor Class: & 15 (13) \\\hline
Hit Dice: & 3+1 \\\hline
No. of Attacks: & 1 weapon \\\hline
Damage: & 1d8+1 or by weapon +1 \\\hline
Movement: & 30' Unarmored 40' \\\hline
No. Appearing: & 2d4, Wild 5d4, Lair 5d4 \\\hline
Save As: & Fighter: 3 \\\hline
Morale: & 9 \\\hline
Treasure Type: & Q, R each; B, L, M in lair \\\hline
XP: & 145 \\\hline
\end{tabularx}\medskip
\end{center}

Bugbears look like huge, hairy goblins, standing about 6 feet tall. Their eyes are usually a darkish brown color and they move very quietly. They are wild and cruel, and bully smaller humanoids whenever possible. 

Bugbear attacks are coordinated, and their tactics are sound if not brilliant. They are able to move in nearly complete silence, surprising opponents on 1-3 on 1d6. In order to remain silent, they must wear only leather or hide armor, as indicated in the Armor Class scores above. Bugbears receive a +1 bonus on damage due to their great Strength. As with most goblinoid monsters, they have Darkvision with a 30' range.

One out of every eight bugbears will be a hardened warrior of 4+4 Hit Dice (240 XP), with a +2 bonus to damage. In lairs of 16 or more bugbears, there will be a chieftain of 6+6 Hit Dice (500 XP), with a +3 bonus to damage. Bugbears gain a +1 bonus to their morale if they are led by a hardened warrior or chieftain. In the lair, bugbears never fail a morale check as long as the chieftain is alive. In addition, there is a 2 in 6 chance that a shaman will be present in a lair. A shaman is equal to an ordinary bugbear statistically, but possesses 1d4+1 levels of Clerical abilities.\medskip



\subsection*{Caecilia, Giant}\index{Caecilia, Giant}\label{caecilia-giant}

\begin{tabularx}{0.48\textwidth}{@{}lX@{}}
Armor Class: & 14 \\\hline
Hit Dice: & 6* \\\hline
No. of Attacks: & 1 bite + swallow on natural 19 or 20 \\\hline
Damage: & 1d8 bite + 1d8/round if swallowed \\\hline
Movement: & 20' (10') \\\hline
No. Appearing: & 1d3, Lair 1d3 \\\hline
Save As: & Fighter: 3 \\\hline
Morale: & 9 \\\hline
Treasure Type: & B \\\hline
XP: & 555 \\\hline
\end{tabularx}\medskip

Caecilia are carnivorous, legless amphibians; they strongly resemble earthworms, but they have bony skeletons and sharp teeth. Caecilia live entirely underground. The giant variety can grow up to 30' long and frequently are found in caverns or dungeons. They are nearly blind, but caecilia are very sensitive to sound and vibrations, and are able to find their prey regardless of light or the absence thereof.

A giant caecilia can swallow a single small humanoid (such as a goblin or halfling) whole. On a natural attack roll of 19 or 20, such a victim has been swallowed (assuming that roll does actually hit the victim). A swallowed victim suffers 1d8 damage per round, and may only attack from the inside with a small cutting or stabbing weapon such as a dagger. While the inside of the caecilia is easier for the victim to hit, fighting while swallowed is more difficult, so no modifiers to the attack roll are applied.

Once a caecilia has swallowed an opponent, it will generally attempt to disengage from combat, going to its lair to rest and digest its meal.


\subsection*{Cattle (including Aurochs and Bison)}\index{Cattle (including Aurochs and Bison)}\label{cattle-including-aurochs-and-bison}

\begin{tabularx}{0.48\textwidth}{lXXX}
& Cattle & Aurochs & Bison \\\hline
Armor Class: & 14 & 16 & 16 \\\hline
Hit Dice: & 2+2 & 3 & 4 \\\hline
No. of Attacks: &\multicolumn{3}{c} {1 horn/head butt or 1 trample}\\\hline
&\multicolumn{3}{c} {1d4 horn/head butt / 2d4 trample} \\\hline
Damage: & 1d4 butt 2d4 trample & 1d6 butt 2d4 trample & 2d4 butt 2d6 trample \\\hline
Movement: & -- 50' (10') -- & & \\\hline
No. Appearing: & Special & Wild 10d12 & \\\hline
Save As: & Fighter: 3 & Fighter: 3 & Fighter: 4 \\\hline
Morale: & 5 (8) & 7 (9) & 7 (9) \\\hline
Treasure Type: & None & None & None \\\hline
XP: & 75 & 145 & 240 \\\hline
\end{tabularx}\medskip

Cattle are large mammals with cloven hooves and horned heads. Cattle are raised mostly for their meat (beef), leather, and milk. Cattle eat grass and are fairly gentle unless spooked, in which case they will stampede (run in a group). Anyone caught in the path of the stampede will suffer at least one trampling attack, as determined by the GM. Male cattle are called bulls, females are cows, and young are calves (calf is singular). If attacked, cattle will charge, generally using their horns to attack. Bulls are larger (+1 hit die), less easily frightened (use the second listed morale figure), and are quite aggressive in defense of the herd. A bull will likely attack if he sees quick movements from creatures he might be able to reach with a charge. Meanwhile, if unable to flee cows will usually assume a roughly circular formation with their heads outward, while calves will be kept in the center, though if the opponents are small enough they may instead charge en masse, trampling all creatures in their path.

A typical small farm with cattle will have a bull, 5d4 cows, and 2d10 calves (but not more than the number of cows).

Aurochs are wild cattle; they are shaggy and rough-looking. Bison are the largest species of wild bovines. All types of bovines tend to behave in the same general way, as described above.

An \textbf{ox }is typically a castrated bull used as a draft animal; females may be used, rarely, but males are preferred due to their greater size and strength. Oxen are usually paired as a team to pull a fully-loaded wagon (or the equivalent of 3,000 lb). Oxen require less food and water, being able to eat rough grass better than draft horses, which makes them valuable to merchants with large caravans going over semi-arid prairie.

\subsection*{Cave Locust, Giant}\index{Cave Locust, Giant}\label{cave-locust-giant}

\begin{tabularx}{0.48\textwidth}{@{}lX@{}}
Armor Class: & 16 \\\hline
Hit Dice: & 2** \\\hline
No. of Attacks: & 1 bite or 1 bump or 1 spit \\\hline
Damage: & 1d2 bite or 1d4* bump or special \\\hline
Movement: & 20' Fly 60'
(15') \\\hline
No. Appearing: & 2d10, Wild 1d10 \\\hline
Save As: & Fighter: 2 \\\hline
Morale: & 5 \\\hline
Treasure Type: & None \\\hline
XP: & 125 \\\hline
\end{tabularx}\medskip

Giant cave locusts are pale, cricket-like creatures that live underground. An average giant cave locust is 2 to 4 feet long. They are eyeless, depending on their sound-sensitive antennae, vibration-sensitive feet and a variety of touch-sensitive "hairs" on their legs to sense the environment around them.

These creatures eat subterranean fungus (including shriekers) as well as carrion; they are not predators, but if disturbed they will attack, shrieking loudly, biting, jumping wildly around, or spitting nasty goo.

All giant cave locusts in a group will shriek when disturbed, attracting wandering monsters. The GM should roll a wandering monster check each round that one or more cave locusts are attacking; if wandering monsters are indicated, they will arrive in 1d4 rounds.

Any giant cave locust that is engaged (adjacent to an opponent) will attempt to bite, doing 1d2 damage on a successful hit. This does not interrupt the monster' s shrieking.

A giant cave locust can leap up to 60' horizontally, or up to 30' up. If one of these creatures is not engaged at the beginning of the round, it will leap toward one of the opponent creatures; roll a normal attack roll, and if the attack hits, the target creature takes 1d4 points of non-lethal damage from the impact.

Finally, a giant cave locust can spray a greenish-brown goo (its digestive juices) up to 10' away. Each giant cave locust can perform this attack just once per encounter. This spit attack will usually be reserved until they fail a morale check, in which case all remaining giant cave locusts will spit at their nearest opponent, and then all will attempt to flee in the next round. To spit on an opponent, the giant cave locust rolls an attack against Armor Class 11 (plus Dexterity and magical bonuses, but no normal armor value applies). If the attack hits, the target must save vs. Poison or be unable to do anything for 3d6 rounds due to the horrible smell.

\subsection*{Centaur}\index{Centaur}\label{centaur}

\begin{tabularx}{0.48\textwidth}{@{}lX@{}}
Armor Class: & 15 (13) \\\hline
Hit Dice: & 4 \\\hline
No. of Attacks: & 2 hooves, 1 weapon \\\hline
Damage: & 1d6 hoof, 1d6 or by weapon \\\hline
Movement: & 50' Unarmored 60'
(10') \\\hline
No. Appearing: & Wild 2d10 \\\hline
Save As: & Fighter: 4 \\\hline
Morale: & 8 \\\hline
Treasure Type: & A \\\hline
XP: & 240 \\\hline
\end{tabularx}\medskip


\begin{center}
	\includegraphics[width=0.47\textwidth]{Pictures132/10000000000003CF000003992E2D8DD7315F6915.png}
\end{center}\medskip

Centaurs appear to be half man, half horse, having the torso, arms and head of a man in the position a horse' s head would otherwise occupy. The horse part of a centaur is as large and powerful as a warhorse; males average 7 feet in height and weigh about a ton (2,000 pounds), while females are just a bit shorter and very close to the same weight. Centaurs may charge with a spear or lance just as a man on horseback, with the same bonuses. They typically wear leather armor when prepared for combat.

Centaurs are generally haughty and aloof, but very honorable. Most would rather die than allow any sort of humanoid to ride on their backs.


\subsection*{Centipede, Giant}\index{Centipede, Giant}\label{centipede-giant}

\begin{tabularx}{0.48\textwidth}{@{}lX@{}}
Armor Class: & 11 \\\hline
Hit Dice: & ½* (1d4 hit points) \\\hline
No. of Attacks: & 1 bite \\\hline
Damage: & poison bite \\\hline
Movement: & 40' \\\hline
No. Appearing: & 2d4, Wild 2d4, Lair 2d4 \\\hline
Save As: & Normal Man \\\hline
Morale: & 7 (see below) \\\hline
Treasure Type: & None \\\hline
XP: & 13 \\\hline
\end{tabularx}\medskip

Giant centipedes are larger versions of the normal sort, being 2 to 3 feet long. Centipedes are fast-moving, predatory, venomous arthropods, having long segmented bodies with exoskeletons. They prefer to live in underground areas, shadowy forested areas, and other places out of direct sunlight; however, there are desert-dwelling varieties that hide under the sand waiting for prey to wander by.


\begin{center}
	\includegraphics[width=0.47\textwidth]{Pictures132/10000000000003D8000001FF168277E42C15AA89.png}
\end{center}

These creatures are aggressive and always hungry, attacking any living creature and only checking morale if injured. Giant centipedes attack with a poisonous bite, and those bitten must save vs. Poison or die; however, the poison is somewhat weak and thus grants a bonus of +2 on the saving throw.


\subsection*{Cheetah}\index{Cheetah}\label{cheetah}

\begin{tabularx}{0.48\textwidth}{@{}lX@{}}
Armor Class: & 14 \\\hline
Hit Dice: & 2 \\\hline
No. of Attacks: & 2 claws, 1 bite \\\hline
Damage: & 1d4 claw, 2d4 bite \\\hline
Movement: & 100' \\\hline
No. Appearing: & Wild 1d3, Lair 1d3 \\\hline
Save As: & Fighter: 2 \\\hline
Morale: & 7 \\\hline
Treasure Type: & None \\\hline
XP: & 75 \\\hline
\end{tabularx}\medskip

A Cheetah is one of the fastest land animals; a large (about 100 pounds) cat capable of reaching up to 75 miles per hour when running. It hunts alone or in small groups (usually composed of siblings). It will rarely attack humans unless compelled to do so, but a female will ferociously defend her young.

\subsection*{Chimera}\index{Chimera}\label{chimera}

\begin{tabularx}{0.48\textwidth}{@{}lX@{}}
Armor Class: & 16 \\\hline
Hit Dice: & 9** (+8) \\\hline
No. of Attacks: & 2 claws, 1 lion bite, 1 goat horns, 1~dragon bite or
breath \\\hline
Damage: & 1d4 claw, 2d4 bite (lion or dragon), 1d8 horns (goat), 3d4
dragon breath \\\hline
Movement: & 40' (10') Fly
60' (15') \\\hline
No. Appearing: & 1d2, Wild 1d4, Lair 1d4 \\\hline
Save As: & Fighter: 9 \\\hline
Morale: & 9 \\\hline
Treasure Type: & F \\\hline
XP: & 1,225 \\\hline
\end{tabularx}\medskip

Chimeras are strange creatures having a lion' s body with the heads of a lion, a goat, and a dragon, and the wings of a dragon. The dragon head of a chimera could be any common dragon except cloud (i.e. desert, forest, ice, mountain, plains, or sea), and has the same type of breath weapon as that sort of dragon. Regardless of type, the dragon' s head breathes a 50' long cone with a 10' wide end which inflicts 3d6 points of damage; victims may save vs. Dragon Breath for one-half damage.

Chimeras are cruel and voracious. They can speak Dragon but seldom bother to do so, except when toadying to more powerful creatures.

\subsection*{Cockatrice}\index{Cockatrice}\label{cockatrice}


\begin{tabularx}{0.48\textwidth}{@{}lX@{}}
Armor Class: & 14 \\\hline
Hit Dice: & 5** \\\hline
No. of Attacks: & 1 beak \\\hline
Damage: & 1d6 + petrification beak \\\hline
Movement: & 30' Fly 60'
(10') \\\hline
No. Appearing: & 1d4, Wild 1d8, Lair 1d8 \\\hline
Save As: & Fighter: 5 \\\hline
Morale: & 7 \\\hline
Treasure Type: & D \\\hline
XP: & 450 \\\hline
\end{tabularx}\medskip

A cockatrice is a strange creature, appearing to be a chicken (hen or rooster) with a long serpentine neck and tail; the neck is topped by a more or less normal looking chicken head. Like a common rooster, a male cockatrice has wattles and a full comb, while the much rarer females have neither. An individual cockatrice weighs more than a chicken, averaging about 20 pounds. A cockatrice is no more intelligent than any animal, but they are bad-tempered and prone to attack if disturbed.

Anyone touched by a cockatrice, or who touches one (even if gloved), must save vs. Petrify or be turned to stone.

\begin{center}
	\includegraphics[width=0.47\textwidth]{Pictures132/10000000000003D800000397E2246AED4DCF90DF.png}
\end{center}


\subsection*{Crab, Giant}\index{Crab, Giant}\label{crab-giant}

\begin{tabularx}{0.48\textwidth}{@{}lX@{}}
Armor Class: & 18 \\\hline
Hit Dice: & 3 \\\hline
No. of Attacks: & 2 pincers \\\hline
Damage: & 2d6 pincer \\\hline
Movement: & 20' Swim 20' \\\hline
No. Appearing: & 1d2, Wild 1d6, Lair 1d6 \\\hline
Save As: & Fighter: 3 \\\hline
Morale: & 7 \\\hline
Treasure Type: & None \\\hline
XP: & 145 \\\hline
\end{tabularx}\medskip

Giant crabs naturally resemble the ordinary variety, but are much larger, averaging 5' in diameter (not counting their legs). These creatures are often found in water-filled caves, particularly those connected to a river, lake or sea, and are tolerant of both fresh and salt water. Also, they are able to live in stagnant water, though they prefer a better environment.

Giant crabs carry their eyes on armored stalks, which means that no bonus is awarded for attacking them from behind.

\subsection*{Crocodile}\index{Crocodile}\label{crocodile}

\begin{tabularx}{0.48\textwidth}{lXXX}
& Normal & Large & Giant \\\hline
Armor Class: & 15 & 17 & 19 \\\hline
Hit Dice: & 2 & 6 & 15 (+11) \\\hline
No. of Attacks: & 1 bite & 1 bite & 1 bite \\\hline
Damage: & 1d8 bite & 2d8 bite & 3d8 bite \\\hline
Movement: & \multicolumn{3}{c}{30' (10') Swim 30' (10')} \\\hline
No. Appearing: & Wild 1d8 & Wild 1d4 & Wild 1d3 \\\hline
Save As: & Fighter: 2 & Fighter: 6 & Fighter: 15 \\\hline
Morale: & 7 & 8 & 9 \\\hline
Treasure Type: & None & None & None \\\hline
XP: & 75 & 500 & 2,850 \\\hline
\end{tabularx}\medskip

Crocodiles are large semiaquatic reptiles that live throughout the tropics. They are ambush predators, waiting for fish or land animals to come close, then rushing out to attack. When in their natural element, they surprise on 1-4 on 1d6.

\textbf{Large Crocodiles:} These huge creatures are from 12-20 feet long. Large crocodiles fight and behave like their smaller cousins.

\textbf{Giant Crocodiles:} These gigantic creatures usually live in salt water and are generally more than 20 feet long. Giant crocodiles fight and behave like their smaller cousins.

\subsection*{Deceiver (Panther-Hydra)}\index{Deceiver (Panther-Hydra)}\label{deceiver-panther-hydra}

\begin{tabularx}{0.48\textwidth}{lXX}
& Common & Greater \\\hline
Armor Class: & 16 & 16 \\\hline
Hit Dice: & 6* & 7** \\\hline
No. of Attacks: &\multicolumn{2}{c}{3 bites (see below)} \\\hline
Damage: & 1d6 snake bite, 1d8 panther bite & 1d6 + poison snake bite,2d6 panther bite \\\hline
Movement: & 50' & 40' \\\hline
No. Appearing: & 1d4, Wild 1d4 & \\\hline
Save As: & Fighter: 6 & Fighter: 7 \\\hline
Morale: & 8 & 9 \\\hline
Treasure Type: & D & D \\\hline
XP: & 555 & 800 \\\hline
\end{tabularx}\medskip

Deceivers are greenish-black catlike monsters with thick serpents extending from their shoulders. All three of the monster' s mouths can bite, though the smaller serpent heads do not do as much damage.


\begin{center}
	\includegraphics[width=0.47\textwidth]{Pictures132/10000000000003D80000029625D5C8056964CD34.png}
\end{center}

The real power and danger of the deceiver is its power of \emph{\textbf{deception}}, a mental ability which causes those attacking the monster to believe the creature is about 3 feet from its true location. Any character fighting a deceiver for the first time will miss their first strike regardless of the die roll. Thereafter, all attacks against deceivers will be at a penalty of -2 to the attack roll. This is not cumulative with the penalty for fighting blind. As a mental power, this ability does not affect mindless creatures, constructs such as golems or living statues, or any sort of undead. Living creatures which are not mindless will be affected even if they do not use sight to target the deceiver.

Greater Deceivers are larger and more fierce than common Deceivers, and on top of that, their serpent heads have a deadly venomous bite; victims must save vs. Poison or die.

\subsection*{Deer}\index{Deer}\label{deer}

See \textbf{Antelope} on page \hyperlink{antelope}{\pageref{antelope}}.

\subsection*{Dinosaur, Deinonychus}\index{Dinosaur, Deinonychus}\label{dinosaur-deinonychus}

\begin{tabularx}{0.48\textwidth}{@{}lX@{}}
Armor Class: & 15 \\\hline
Hit Dice: & 3 \\\hline
No. of Attacks: & 1 bite \\\hline
Damage: & 1d8 \\\hline
Movement: & 50' \\\hline
No. Appearing: & 1d3, Wild 2d3, Lair 2d6 \\\hline
Save As: & Fighter: 3 \\\hline
Morale: & 8 \\\hline
Treasure Type: & None \\\hline
XP: & 145 \\\hline
\end{tabularx}\medskip

The Deinonychus (sometimes mistakenly called a "Velociraptor") is a medium-sized feathered dinosaur weighing approximately 150 pounds and reaching about 11 feet of length (tail included). It is an avid predator and a skilled pack-hunter; its warm blood, aerodynamic build and vicious maw allow it to feed on larger but more primitive dinosaurs.

\begin{center}
	\includegraphics[width=0.40\textwidth]{Pictures132/10000000000003D800000556DEC8A404311AD387.png}
\end{center}

\subsection*{Dinosaur, Pterodactyl (and Pteranodon)}\index{Dinosaur, Pterodactyl (and Pteranodon}\label{dinosaur-pterodactyl-and-pteranodon}

\begin{tabularx}{0.48\textwidth}{lXX}
& Pterodactyl & Pteranodon \\\hline
Armor Class: & 12 & 13 \\\hline
Hit Dice: & 1 & 5 \\\hline
No. of Attacks: & 1 bite & 1 bite \\\hline
Damage: & 1d4 bite & 2d6 bite \\\hline
Movement: & Fly 60' (10') & Fly
60' (15') \\\hline
No. Appearing: & Wild 2d4 & Wild 1d4 \\\hline
Save As: & Fighter: 1 & Fighter: 5 \\\hline
Morale: & 7 & 8 \\\hline
Treasure Type: & None & None \\\hline
XP: & 25 & 360 \\\hline
\end{tabularx}\medskip

Pterodactyls are prehistoric winged reptilian creatures, having a wingspan of around 25 to 30 inches. Though they eat mostly fish, they may attack smaller characters or scavenge unguarded packs.

Pteranodons are essentially giant-sized pterodactyls, having wingspans of 25 feet or more. They are predators, and may attack adventuring parties.

\begin{center}
	\includegraphics[width=0.47\textwidth]{Pictures132/10000000000003D8000001EE8D81F5E733BD3985.png}
\end{center}

\subsection*{Dinosaur, Stegosaurus}\index{Dinosaur, Stegosaurus}\label{dinosaur-stegosaurus}

\begin{tabularx}{0.48\textwidth}{@{}lX@{}}
Armor Class: & 17 \\\hline
Hit Dice: & 11 (+9) \\\hline
No. of Attacks: & 1 bite, 1 tail or 1 trample (see below) \\\hline
Damage: & 1d6 bite, 2d8 tail, 2d8 trample \\\hline
Movement: & 20' (15') \\\hline
No. Appearing: & Wild 1d4 \\\hline
Save As: & Fighter: 6 \\\hline
Morale: & 7 \\\hline
Treasure Type: & None \\\hline
XP: & 1,575 \\\hline
\end{tabularx}\medskip

Although fearsome-looking, the stegosaurus is actually a peaceable creature and will only fight in self-defense, either biting, trampling, or using its spiked tail, depending on where the opponent is standing in relation to the dinosaur. A stegosaurus can' t use its tail and bite attacks against the same creature in the same round, and cannot use either bite or tail on any round where it tramples.

\subsection*{Dinosaur, Triceratops}\index{Dinosaur, Triceratops}\label{dinosaur-triceratops}

\begin{tabularx}{0.48\textwidth}{@{}lX@{}}
Armor Class: & 19 \\\hline
Hit Dice: & 11 (+9) \\\hline
No. of Attacks: & 1 gore or 1 trample \\\hline
Damage: & 3d6 gore or 3d6 trample (see below) \\\hline
Movement: & 30' (15') \\\hline
No. Appearing: & Wild 1d4 \\\hline
Save As: & Fighter: 7 \\\hline
Morale: & 8 \\\hline
Treasure Type: & None \\\hline
XP: & 1,575 \\\hline
\end{tabularx}\medskip

A triceratops is a three-horned herbivorous dinosaur. They are aggressive toward interlopers, attacking anyone who might appear to be a threat. Individuals are quite large, weighing 11,000 to 20,000 pounds and ranging from 26 to 30 feet in length.


\begin{center}
	\includegraphics[width=0.47\textwidth]{Pictures132/10000000000003D80000032816EEA26369834B98.png}
\end{center}

When facing opponents of smaller size, a triceratops will usually attempt to trample them, reserving the gore attack for larger opponents. Up to two adjacent man-sized or up to four smaller opponents may be trampled simultaneously; the triceratops rolls a single attack roll which is compared to the Armor Class of each of the potential victims, and then rolls a separate damage roll for each one successfully hit. The gore attack may only be used against a single man-sized or larger creature, but may be used in the same round as the trample if the creature being gored is larger than man sized. Also note that a charging bonus may be applied to the gore attack.


\subsection*{Dinosaur, Tyrannosaurus Rex}\index{Dinosaur, Tyrannosaurus Rex}\label{dinosaur-tyrannosaurus-rex}


\begin{tabularx}{0.48\textwidth}{@{}lX@{}}
Armor Class: & 23 \\\hline
Hit Dice: & 18 (+12) \\\hline
No. of Attacks: & 1 bite \\\hline
Damage: & 6d6 bite \\\hline
Movement: & 40' (10') \\\hline
No. Appearing: & Wild 1d4 \\\hline
Save As: & Fighter: 18 \\\hline
Morale: & 11 \\\hline
Treasure Type: & None \\\hline
XP: & 4,000 \\\hline
\end{tabularx}\medskip

The tyrannosaurus rex is a bipedal carnivorous dinosaur with a massive skull balanced by a long, heavy tail. Relative to its large and powerful hind limbs, its forelimbs are short but unusually powerful for their size, with two clawed digits. Despite this, they are not used to attack, as the tyrannosaur' s powerful bite is its preferred weapon.

Individuals can grow to lengths of over 40 feet and can weigh up to 20,000 pounds, though most are a bit smaller than this, averaging around 35 feet in length and 17,000 pounds in weight.

The statistics above can also be used to represent other large bipedal carnosaurs, such as the allosaurus.

\begin{center}
	\includegraphics[width=0.47\textwidth]{Pictures132/10000000000003D8000003D5A3D2D1CA9BEF7C04.png}
\end{center}

\subsection*{Djinni*}\index{Djinni}\label{djinni}


\begin{tabularx}{0.48\textwidth}{@{}lX@{}}
Armor Class: & 15 (m) \\\hline
Hit Dice: & 7+1** \\\hline
No. of Attacks: & 1 fist or 1 whirlwind \\\hline
Damage: & 2d8 fist, 2d6 whirlwind \\\hline
Movement: & 30' Fly 80' \\\hline
No. Appearing: & 1 \\\hline
Save As: & Fighter: 12 \\\hline
Morale: & 12 (8) \\\hline
Treasure Type: & None \\\hline
XP: & 800 \\\hline
\end{tabularx}\medskip

Djinn (singular djinni) are a race of manlike creatures believed to be from the Elemental Plane of Air. They are large beings, 10 to 11 feet in height and weighing around 1,000 pounds, though their weight is generally immaterial due to their ability to fly by magical means.

The djinni' s morale score of 12 reflects its absolute control over its own fear, but does not indicate that the creature will throw its life away easily. Use the "8" figure to determine whether an outmatched djinni decides to leave a combat.


\begin{center}
	\includegraphics[width=0.47\textwidth]{Pictures132/10000000000003CF000005F12D60B1819E3D651A.png}
\end{center}

Djinn have a number of magical powers, which can be used at will (that is, without needing magic words or gestures): \textbf{create food and drink}, creating tasty and nourishing food for up to 2d6 humans or similar creatures, once per day; become \textbf{invisible}, with unlimited uses per day; \textbf{create normal items}, creating up to 1,000 pounds of soft goods or wooden items of permanent nature or metal goods lasting at most a day, once per day; assume \textbf{gaseous form}, as the potion, up to one hour per day; and \textbf{create illusions}, as the spell \textbf{phantasmal force} but including sound as well as visual elements, three times per day.

Djinn may assume the form of a whirlwind at will, with no limit as to the number of times per day this power may be used; a djinni in whirlwind form fights as if it were an air elemental.

Due to their highly magical nature, djinn cannot be harmed by non-magical weapons. They are immune to normal cold, and suffer only half damage from magical attacks based on either cold or wind.

\subsection*{Dog}\index{Dog}\label{dog}

\begin{tabularx}{0.48\textwidth}{lXX}
& Normal & Riding \\\hline
Armor Class: & 14 & 14 \\\hline
Hit Dice: & 1+1 & 2 \\\hline
No. of Attacks: & 1 bite & 1 bite \\\hline
Damage: & 1d4 + hold & 1d4+1 + hold \\\hline
Movement: & 50' & 50' \\\hline
No. Appearing: & Wild 3d4 & domestic only \\\hline
Save As: & Fighter: 1 & Fighter: 2 \\\hline
Morale: & 9 & 9 \\\hline
Treasure Type: & None & None \\\hline
XP: & 25 & 75 \\\hline
\end{tabularx}\medskip

Normal dogs include most medium and large breeds, including wild dogs. After biting an opponent, a dog can hold on, doing 1d4 damage automatically every round, until killed or until the victim spends an attack breaking free (which requires a save vs. Death Ray, adjusted by the character' s Strength bonus).


\begin{center}
	\includegraphics[width=0.47\textwidth]{Pictures132/10000000000003CF0000032A1EC4A02624CF5C70.png}
\end{center}

Riding dogs are a large breed, used primarily by Halflings for transport. They may be trained for war, and equipped with barding to improve their Armor Class. They can maintain a hold in the same way that normal dogs do. A light load for a riding dog is up to 150 pounds; a heavy load, up to 300 pounds.

\subsection*{Doppleganger}\index{Doppleganger}\label{doppleganger}


\begin{tabularx}{0.48\textwidth}{@{}lX@{}}
Armor Class: & 15 \\\hline
Hit Dice: & 4* \\\hline
No. of Attacks: & 1 fist or by weapon \\\hline
Damage: & 1d12 fist, 1d6 or by weapon \\\hline
Movement: & 30' \\\hline
No. Appearing: & 1d6, Wild 1d6, Lair 1d6 \\\hline
Save As: & Fighter: 4 \\\hline
Morale: & 10 \\\hline
Treasure Type: & E \\\hline
XP: & 280 \\\hline
\end{tabularx}\medskip

Dopplegangers are weird humanoid creatures who are able to take on the appearance of nearly any other humanoid ranging from 3 feet up to 7 feet in height. They can also read minds (as the spell, \textbf{mind reading}, but with unlimited duration), an ability that can even be used to speak any language known to the creature whose mind is being read. In their natural form (which few creatures ever see) they are pale and pasty looking, slim, around 5½ feet tall and weighing about 150 pounds. Their features look unformed and incomplete.

Dopplegangers wish to live a life of ease, and in pursuit of that goal they will seek to take the place of any character they meet who they believe can help them get that kind of life. This usually means separating the victim from any allies and quietly killing them. While a doppleganger can duplicate the appearance of the clothing and equipment worn by a creature, such items are part of the creature and cannot, for example, be laid down or handed to someone. Taking the clothing and equipment of a victim is thus the preferred tactic.

A doppleganger can mimic the sound of the voice of any living creature as well as the appearance of any humanoid. Their \textbf{mind reading} ability will be used to help them behave like the person they have replaced.


\begin{center}
	\includegraphics[width=0.47\textwidth]{Pictures132/10000000000003CF000002FAECD6ACA132EBA1B2.png}
\end{center}

\subsection*{Dragon}\index{Dragon}\label{dragon}

Dragons are large (sometimes very large) winged reptilian monsters. Unlike wyverns, dragons have four legs as well as two wings; this is how experts distinguish "true" dragons from other large reptilian monsters. All dragons are long-lived, and they grow slowly for as long as they live. For this reason, they are described as having seven "age categories," ranging from 3 less to 3 more hit dice than the average. For convenience, a table is provided following the description of each dragon type; this table shows the variation in hit dice, damage from their various attacks, and other features peculiar to dragons. 

If one dragon is encountered, it is equally likely to be a male or female ranging from -2 to +3 hit dice (1d6-3); two are a mated pair ranging from -1 to +2 hit dice (1d4-2). If three or four are encountered, they consist of a mated pair plus one or two young of -3 hit dice in size. If this is the case, the parents receive a Morale of 12 in combat since they are protecting their young.


\begin{center}
	\includegraphics[width=0.47\textwidth]{Pictures132/10000000000003CF00000574C2C0236BF69CACAE.png}
\end{center}

In combat dragons use a powerful bite, slashing claws, a long, whiplike tail, and most famously some form of breath weapon. Tactically, dragons prefer to fight while flying, making passes over ground-based enemies to strike or breathe on them. Smarter and older dragons will look for the toughest or most dangerous foe to strike down first, particularly preferring to eliminate magic-users as early in a fight as possible.

Each dragon can use its breath weapon as many times per day as it has hit dice, except that dragons of the lowest age category do not yet have a breath weapon. The breath may be used no more often than every other round, and the dragon may use its claws and tail in the same round. The tail swipe attack may only be used if there are opponents behind the dragon, while the claws may be used only on those opponents in front of the creature. Due to their serpentine necks, dragons may bite in any direction, even behind them.

The breath weapon of a dragon does 1d8 points of damage per hit die (so, a 7 hit die dragon does 7d8 points of damage with its breath). Victims may make a save vs. Dragon Breath for half damage. The breath weapon may be projected in any direction around the dragon, even behind, for the same reason that the dragon can bite those behind it.

There are three shapes (or areas of effect) which a dragon' s breath weapon can cover. Each variety has a "normal" shape, which that type of dragon can use from the second age category (-2 hit dice) onward. Upon reaching the sixth age category (+2 hit dice), a dragon learns to shape its breath weapon into one of the other shapes (GM' s option); at the seventh age category (+3 hit dice), the dragon can produce all three shapes.


\begin{center}
	\includegraphics[width=0.47\textwidth]{Pictures132/10000000000003D8000004FAAB1FCAD9C0BB72C2.png}
\end{center}\medskip

The shapes are:

\textbf{Cone Shaped:} The breath weapon begins at the dragon' s mouth, and is about 2' wide at that point; it extends up to the maximum length (based on the dragon type and age) and is the maximum width at that point (again, as given for the dragon' s type and age).

\textbf{Line Shaped: } The breath weapon is 5' wide (regardless of the given width figure) and extends the given length in a straight line.

\textbf{Cloud Shaped:} The breath weapon covers an area up to the maximum given width (based on the dragon type and age) in both length and width (that is, any length figure given for the dragon type and age is ignored). A cloud-shaped breath weapon is, at most, 20' deep or high.

All dragons save for those of the lowest age category are able to speak Dragon. Each type has a given chance of "talking;" this is the chance that the dragon will know Common or some other humanoid language. Many who talk choose to learn Elvish. If the first roll for "talking" is successful, the GM may roll again, with each additional roll adding another language which the dragon may speak.

Some dragons learn to cast spells; the odds that a dragon can cast spells are the same as the odds that a dragon will learn to speak to lesser creatures, but each is rolled for separately.

All dragons crave wealth, hoarding as many coins, gems, items of jewelry, and best of all magic items in their lair as possible. The greater a dragon' s hoard, the less it will seek to leave its lair, though of course it must hunt for food from time to time. Dragons rarely trust any other creature to defend their hoard, though some will entice aggressive but unintelligent creatures to live nearby in the hopes that they will kill and eat any thieves who might come nosing around.

Unlike other monsters, multiple dragons encountered in a lair together do not share treasure; generate a full treasure for each dragon individually. Treasure for dragons is special, as noted on the table on page \hyperlink{treasure-types}{\pageref{treasure-types}}. The odds of monetary treasure increase with the monster' s age category, while the chances of gems, jewelry, and magic items are based upon its hit dice.

Young dragons typically have brightly colored, glossy skin; as a dragon ages, both the color and the sheen slowly become dull.

\textbf{Experience Points (XP)} given for dragons on the following pages are for non-spell-casting dragons of age category 4. The GM must calculate the XP award for any specific dragon as given on page \hyperlink{experience-points-xp}{\pageref{experience-points-xp}}, adding a special ability bonus (i.e. a "star") if the dragon casts spells.

\subsection*{Dragon, Cloud}\index{Dragon, Cloud}\label{dragon-cloud}

\begin{tabularx}{0.48\textwidth}{@{}lX@{}}
Armor Class: & 22 \\\hline
Hit Dice: & 11** (+9) \\\hline
No. of Attacks: & 2 claws, 1 bite or breath, 1 tail \\\hline
Damage: & 2d4 claw, 6d6 bite or breath, 2d4 tail \\\hline
Movement: & 30' Fly 80'
(20') \\\hline
No. Appearing: & 1, Wild 1, Lair 1d4 \\\hline
Save As: & Fighter: 11 (as Hit Dice) \\\hline
Morale: & 10 \\\hline
Treasure Type: & H \\\hline
XP: & 1,765 \\\hline
\end{tabularx}\medskip

Cloud dragons have the most varied appearance of all the true dragons. In an indoor environment or when underground a cloud dragon appears to be a bright metallic color, while outdoors their coloration is brighter and less metallic, and may take on a reddish or bluish cast reminiscent of a sunrise or sunset. Hatchlings have a coppery skin tone, brightening to silver at the second age category, then to gold at the fourth before fading to a platinum tone by age category 6. While most true dragons become duller in color and sheen as they age, cloud dragons do not.

Cloud dragons are not cruel and do not seek to kill for pleasure. Many tales are told of cloud dragons offering assistance to adventurers, though they are every bit as avaricious as any dragon; adventurers in need of gold need not bother asking for a loan.

Another way in which cloud dragons differ from other types is that they do not have fixed breath weapons. Upon reaching the second age category, a cloud dragon acquires the breath weapon of a randomly-chosen (or GM assigned) dragon type; upon reaching the fourth age category, they acquire a second breath weapon type. Cloud dragons possess the same immunities as the dragons whose breath weapons they reproduce.

All cloud dragons have the power to assume the form of any type of humanoid (as described in the spell \textbf{charm person}) at will in a manner otherwise equivalent to the spell \textbf{polymorph self}.

\textbf{Cloud Dragon Age Table}
	
\begin{flushleft}
\begin{tabularx}{0.47\textwidth}{@{}lXXXXXXX@{}}
	Age Category & 1 & 2 & 3 & 4 & 5 & 6 & 7 \\\hline
	Hit Dice & 8 & 9 & 10 & 11 & 12 & 13 & 14 \\\hline
	Attack Bonus & +8 & +8 & +9 & +9 & +10 & +11 & +11 \\\hline
	Breath Weapon &  \multicolumn{6}{c}{Special (see above)}\\\hline
	Length & - & 70' & 80' & 	90' & 95' & 100' &	110' \\\hline
	Width & - & 30' & 35' & 	45' & 50' & 55' &	60' \\\hline
	Chance/Talk. & 0\% & 35\% & 70\% & 85\% & 90\% & 95\% & 95\% \\\hline
	Spells by Level & & & & & & & \\\hline
	Level 1 & - & 1 & 2 & 3 & 4 & 5 & 6 \\\hline
	Level 2 & - & - & 1 & 2 & 3 & 4 & 5 \\\hline
	Level 3 & - & - & - & 1 & 2 & 3 & 4 \\\hline
	Level 4 & - & - & - & - & 1 & 2 & 3 \\\hline
	Level 5 & - & - & - & - & - & 1 & 2 \\\hline
	Level 6 & - & - & - & - & - & - & 1 \\\hline
	Claw & 1d6 & 1d6 & 1d6 & 2d4 & 2d4 & 2d6 & 2d8 \\\hline
	Bite & 3d6 & 4d6 & 5d6 & 6d6 & 6d6 & 7d6 & 7d6 \\\hline
	Tail & 1d4 & 1d6 & 1d6 & 2d4 & 2d6 & 2d6 & 2d8 \\\hline
\end{tabularx}
\end{flushleft}

\vfill

\begin{center}
	\includegraphics[width=0.47\textwidth]{Pictures132/10000000000007E90000053F8D91B317BF79F6BE.png}
\end{center}

\pagebreak


\subsection*{Dragon, Desert (Blue Dragon)}\index{Dragon, Desert (Blue Dragon)}\label{dragon-desert-blue-dragon}

\begin{tabularx}{0.48\textwidth}{@{}lX@{}}
Armor Class: & 20 \\\hline
Hit Dice: & 9** (+8) \\\hline
No. of Attacks: & 2 claws, 1 bite or breath, 1 tail \\\hline
Damage: & 1d8 claw, 3d8 bite or breath, 1d8 tail \\\hline
Movement: & 30' Fly 80'
(15') \\\hline
No. Appearing: & 1, Wild 1, Lair 1d4 \\\hline
Save As: & Fighter: 9 (as Hit Dice) \\\hline
Morale: & 9 \\\hline
Treasure Type: & H \\\hline
XP: & 1,225 \\\hline
\end{tabularx}\medskip

Desert dragons have rough, gritty-feeling hide which is a dark steel blue color with a smoother, streaky brown underbelly. Their bodies are wiry and serpentine.

They hunt by day in the heat of the sun, sometimes flying high overhead looking for prey, or sometimes choosing to bury themselves in the sand and lie in wait with only eyes and nostrils exposed. One will wait in this fashion until victims come within 100 feet, then spring out and attack (surprising on a roll of 1-4 on 1d6 in this case).

A desert dragon will usually choose to lair in an underground cavern, or perhaps in a ruined castle or desert outpost. They are evil monsters, though not so fierce as mountain dragons. They particularly enjoy tricking intelligent prey into entering their lairs or passing by their hiding places to be ambushed and killed; usually one member of a party attacked by a desert dragon will be left alive for a while, and the dragon will play with that person as a cat plays with a mouse.

Desert dragons are immune to normal lightning, and suffer only half damage from magical lightning.


\begin{center}
	\textbf{Desert Dragon Age Table}\\
	
\begin{tabularx}{0.47\textwidth}{@{}lXXXXXXX@{}}
Age & 1 & 2 & 3 & 4 & 5 & 6 & 7 \\\hline
Hit Dice & 6 & 7 & 8 & 9 & 10 & 11 & 12 \\\hline
Attack Bonus & +6 & +7 & +8 & +8 & +9 & +9 & +10 \\\hline
Breath Weapon&  \multicolumn{7}{c}{Lightning (Line)}\\\hline
Length & - & 80' & 90' & 100' & 100' & 110' & 120' \\\hline
Width & - & - & - & - & - & 55' & 60' \\\hline
Chance/Talk. & 0\% & 15\% & 20\% & 40\% & 50\% & 60\% & 70\% \\\hline
Spells by Level& &&&&&&\\\hline
Level 1 & - & 1 & 2 & 4 & 4 & 4 & 5 \\\hline
Level 2 & - & - & 1 & 2 & 3 & 4 & 4 \\\hline
Level 3 & - & - & - & - & 1 & 2 & 2 \\\hline
Level 4 & - & - & - & - & - & - & 1 \\\hline
Claw & 1d4 & 1d4 & 1d6 & 1d8 & 1d8 & 1d8 & 1d10 \\\hline
Bite & 2d6 & 3d6 & 3d8 & 3d8 & 3d8 & 3d8 & 3d10 \\\hline
Tail & 1d4 & 1d6 & 1d6 & 1d8 & 1d8 & 1d8 & 1d8 \\\hline
\end{tabularx}

\end{center}

\subsection*{Dragon, Forest (Green Dragon)}\index{Dragon, Forest (Green Dragon}\label{dragon-forest-green-dragon}

\begin{tabularx}{0.48\textwidth}{@{}lX@{}}
Armor Class: & 19 \\\hline
Hit Dice: & 8** \\\hline
No. of Attacks: & 2 claws, 1 bite or breath, 1 tail \\\hline
Damage: & 1d6 claw, 3d8 bite or breath, 1d6 tail \\\hline
Movement: & 30' Fly 80'
(15') \\\hline
No. Appearing: & 1, Wild 1, Lair 1d4 \\\hline
Save As: & Fighter: 8 (as Hit Dice) \\\hline
Morale: & 8 \\\hline
Treasure Type: & H \\\hline
XP: & 1,015 \\\hline
\end{tabularx}\medskip

Forest dragons are bright leaf green in color, with a tan underbelly. They have long sinuous bodies and move with catlike grace. They are cruel monsters, but they are renowned for their curiosity. They especially like to question adventurers to learn more about their society and abilities, what is going on in the countryside, and if there is treasure nearby. Adventurers may be allowed to live so long as they remain interesting... but woe to them when the dragon becomes bored.

Forest dragons are immune to all poisons. Note that, despite their breath weapon being described as "poison gas," damage done by it is exactly the same as with other dragons. More specifically, those in the area of effect do not have to "save or die" as with ordinary poison, but rather save vs. Dragon Breath for half damage.\\

\begin{center}
	\textbf{Forest Dragon Age Table}\\
	
\begin{tabularx}{0.47\textwidth}{@{}lXXXXXXX@{}}
Age & 1 & 2 & 3 & 4 & 5 & 6 & 7 \\\hline
Hit Dice & 5 & 6 & 7 & 8 & 9 & 10 & 11 \\\hline
Attack Bonus & +5 & +6 & +7 & +8 & +8 & +9 & +9 \\\hline
Breath Weapon&  \multicolumn{7}{c}{Poison Gas (Cloud)}\\\hline
Length & - & 70' & 80' & 90' & 95' & 100' & 100' \\\hline
Width & - & 25' & 30' & 40' & 45' & 50' & 55' \\\hline
Chance/Talk.  & 0\% & 15\% & 20\% & 30\% & 45\% & 55\% & 65\% \\\hline
Spells by Level &&&&&&&\\\hline
Level 1 & - & 1 & 2 & 3 & 3 & 4 & 4 \\\hline
Level 2 & - & - & 1 & 2 & 3 & 3 & 4 \\\hline
Level 3 & - & - & - & - & 1 & 2 & 3 \\\hline
Level 4 & - & - & - & - & - & - & 1 \\\hline
Claw & 1d4 & 1d6 & 1d6 & 1d6 & 1d6 & 1d8 & 1d10 \\\hline
Bite & 2d4 & 3d4 & 3d6 & 3d8 & 3d8 & 3d8 & 3d10 \\\hline
Tail & 1d4 & 1d4 & 1d6 & 1d6 & 1d6 & 1d8 & 1d8 \\\hline
\end{tabularx}
\end{center}

\subsection*{Dragon, Ice (White Dragon)}\index{Dragon, Ice (White Dragon}\label{dragon-ice-white-dragon}

\begin{tabularx}{0.48\textwidth}{@{}lX@{}}
Armor Class: & 17 \\\hline
Hit Dice: & 6**  \\\hline
No. of Attacks: & 2 claws, 1 bite or breath, 1 tail \\\hline
Damage: & 1d4 claw, 2d8 bite or breath, 1d4 tail \\\hline
Movement: & 30' Fly 80'
(10') \\\hline
No. Appearing: & 1, Wild 1, Lair 1d4 \\\hline
Save As: & Fighter: 6 (as Hit Dice) \\\hline
Morale: & 8 \\\hline
Treasure Type: & H \\\hline
XP: & 610 \\\hline
\end{tabularx}\medskip

Ice dragons have pale blue-white skin, ranging from sky blue for a hatchling to the stark pure white of an ancient individual. They are the same color all over, having no contrasting underbelly color.

They prefer to live in cold regions, whether in the highest mountains or in the cold northern lands. They are the least intelligent of dragons, though this does not mean that they are stupid by any stretch of the imagination. They are motivated completely by a drive to live, to reproduce, and (of course) to accumulate treasure; they kill to live, not for pleasure.

In a fashion similar to swamp and desert dragons, an ice dragon will sometimes choose to bury itself in snow and wait, with only its eyes and nostrils exposed, in a place where prey is likely to pass by. The ice dragon will then burst out when likely prey approaches within 100', surprising on a roll of 1-4 on 1d6.

Ice dragons are immune to normal cold, and take only half damage from magical cold or ice.\\
\begin{center}
	\textbf{Ice Dragon Age Table}\\
\begin{tabularx}{0.47\textwidth}{@{}lXXXXXXX@{}}
Age Category & 1 & 2 & 3 & 4 & 5 & 6 & 7 \\\hline
Hit Dice & 3 & 4 & 5 & 6 & 7 & 8 & 9 \\\hline
Attack Bonus & +3 & +4 & +5 & +6 & +7 & +8 & +8 \\\hline
Breath Weapon & \multicolumn{7}{c}{Cold (Cone)}\\\hline
Length & - & 60' & 70' & 80' & 85' & 90' & 95' \\\hline
Width & - & 25' & 30' & 30' & 35' & 40' & 45' \\\hline
Chance/Talk. & 0\% & 10\% & 15\% & 20\% & 30\% & 40\% & 50\% \\\hline
Spells by Level & & & & & & & \\\hline
Level 1 & - & 1 & 2 & 3 & 3 & 3 & 3 \\\hline
Level 2 & - & - & - & - & 1 & 2 & 3 \\\hline
Level 3 & - & - & - & - & - & - & 1 \\\hline
Claw & 1d4 & 1d4 & 1d4 & 1d4 & 1d4 & 1d6 & 1d8 \\\hline
Bite & 2d4 & 2d6 & 2d6 & 2d8 & 2d8 & 2d10 & 2d10 \\\hline
Tail & 1d4 & 1d4 & 1d4 & 1d4 & 1d4 & 1d6 & 1d6 \\\hline
\end{tabularx}
\end{center}


\subsection*{Dragon, Mountain (Red Dragon)}\index{Dragon, Mountain (Red Dragon)}\label{dragon-mountain-red-dragon}

\begin{tabularx}{0.48\textwidth}{@{}lX@{}}
Armor Class: & 21 \\\hline
Hit Dice: & 10** (+9) \\\hline
No. of Attacks: & 2 claws, 1 bite or breath, 1 tail \\\hline
Damage: & 1d8 claw, 4d8 bite or breath, 1d8 tail \\\hline
Movement: & 30' Fly 80'(20') \\\hline
No. Appearing: & 1, Wild 1, Lair 1d4 \\\hline
Save As: & Fighter: 10 (as Hit Dice) \\\hline
Morale: & 8 \\\hline
Treasure Type: & H \\\hline
XP: & 1,480 \\\hline
\end{tabularx}\medskip

Mountain dragons are red in color, ranging from the brilliant blood red of a hatchling to the dull terracotta color of an ancient individual. These dragons are powerfully built, with heavy-jawed heads and thick muscular bodies, yet their necks are still long enough to give them the legendary flexibility of a true dragon.

They are cruel monsters, actively seeking to hunt, torment, kill and consume intelligent creatures. They are often said to prefer women and elves, but in truth a mountain dragon will attack almost any creature less powerful than itself.

They are intelligent and self-assured, but also impatient and overconfident. One will often plan strategies in advance and then choose one at random when facing unknown opponents, without regard to whether or not the strategy is likely to work. The sheer power of a mountain dragon is often the only reason one is still alive.

Mountain dragons are immune to normal fire, and suffer only half damage from magical fire.\\

\begin{center}
	\textbf{Mountain Dragon Age Table}\\

\begin{tabularx}{0.47\textwidth}{@{}lXXXXXXX@{}}
Age Category & 1 & 2 & 3 & 4 & 5 & 6 & 7 \\\hline
Hit Dice & 7 & 8 & 9 & 10 & 11 & 12 & 13 \\\hline
Attack Bonus & +7 & +8 & +8 & +9 & +9 & +10 & +11 \\\hline
Breath W. & \multicolumn{7}{c}{Fire (Cone)}\\\hline
Length & - & 70' & 80' & 90' & 95' & 100' & 110' \\\hline
Width & - & 30' & 35' & 45' & 50' & 55' & 60' \\\hline
Chance/Talk. & 0\% & 15\% & 30\% & 50\% & 60\% & 70\% & 85\% \\\hline
Spells &\multicolumn{7}{c}{by Level}\\\hline
Level 1 & - & 1 & 2 & 3 & 4 & 5 & 5 \\\hline
Level 2 & - & - & 1 & 2 & 3 & 4 & 5 \\\hline
Level 3 & - & - & - & 1 & 2 & 2 & 3 \\\hline
Level 4 & - & - & - & - & 1 & 2 & 2 \\\hline
Level 5 & - & - & - & - & - & 1 & 2 \\\hline
Claw & 1d4 & 1d6 & 1d8 & 1d8 & 1d8 & 1d10 & 1d10 \\\hline
Bite & 2d6 & 3d6 & 4d6 & 4d8 & 5d8 & 5d8 & 6d8 \\\hline
Tail & 1d4 & 1d6 & 1d6 & 1d8 & 1d8 & 1d8 & 1d10 \\\hline
\end{tabularx}

\end{center}
	
\subsection*{Dragon, Plains (Yellow Dragon)}\index{Dragon, Plains (Yellow Dragon)}\label{dragon-plains-yellow-dragon}

\begin{tabularx}{0.48\textwidth}{@{}lX@{}}
Armor Class: & 16 \\\hline
Hit Dice: & 5**  \\\hline
No. of Attacks: & 2 claws, 1 bite or breath, 1 tail \\\hline
Damage: & 1d6 claw, 2d10 or breath, 1d8 tail \\\hline
Movement: & 50' Fly 80'
(10') \\\hline
No. Appearing: & 1, Wild 1, Lair 1d6 \\\hline
Save As: & Fighter: 5 (as Hit Dice) \\\hline
Morale: & 8 \\\hline
Treasure Type: & H \\\hline
XP: & 450 \\\hline
\end{tabularx}\medskip

Plains dragons are the smallest of the true dragons. They have yellow skin dappled with light green patches, spots, or sometimes thin stripes. The color dulls and darkens with age until it reaches an almost uniform tan color at the oldest age category. They have long sinuous bodies and unusually long legs, giving them the fastest land movement rate of any dragon, but they are also accomplished fliers, as fast and maneuverable as any dragon.

Plains dragons may hunt on the wing, attacking suddenly from above and surprising on 1-3 on 1d6; or, they may lie in wait in tall grass or a copse of trees, using their coloration as camouflage. As one ages and its colors dull, this ability improves; when lying in wait, they gain surprise on 1-3 on 1d6 at age categories 2 and 3, on 1-4 in age categories 4 through 6, and on 1-5 at age category 7.

The breath weapon of the plains dragon is a scorching, shimmering cone of heat, barely visible to the naked eye; at night, however, a glow like steel being forged can be seen streaming from the dragon' s open mouth.\\

\begin{center}
	\textbf{Plains Dragon Age Table}\\

\begin{tabularx}{0.47\textwidth}{@{}lXXXXXXX@{}}
Age Category & 1 & 2 & 3 & 4 & 5 & 6 & 7 \\\hline
Hit Dice & 2 & 3 & 4 & 5 & 6 & 7 & 8 \\\hline
Attack Bonus & +2 & +3 & +4 & +5 & +6 & +7 & +8 \\\hline
Breath Weapon & \multicolumn{7}{c}{Heat (Cone)}\\\hline
Length & - & 50' & 60' & 70' & 80' & 85' & 90' \\\hline
Width & - & 25' & 30' & 30' & 35' & 40' & 45' \\\hline
Chance/Talk. & 0\% & 10\% & 15\% & 20\% & 30\% & 40\% & 50\% \\\hline
Spells by Level & & & & & & & \\\hline
Level 1 & - & 1 & 2 & 3 & 3 & 3 & 3 \\\hline
Level 2 & - & - & - & - & 1 & 2 & 3 \\\hline
Level 3 & - & - & - & - & - & - & 1 \\\hline
Claw & 1d4 & 1d4 & 1d6 & 1d6 & 1d6 & 1d8 & 1d8 \\\hline
Bite & 2d4 & 2d6 & 2d8 & 2d10 & 2d10 & 2d10 & 2d12 \\\hline
Tail & 1d4 & 1d6 & 1d6 & 1d8 & 1d8 & 1d8 & 1d10 \\\hline
\end{tabularx}
\end{center}
	
\subsection*{Dragon, Sea (Gray Dragon)}\index{Dragon, Sea (Gray Dragon)}\label{dragon-sea-gray-dragon}

\begin{tabularx}{0.48\textwidth}{@{}lX@{}}
Armor Class: & 19 \\\hline
Hit Dice: & 8** \\\hline
No. of Attacks: & 2 claws, 1 bite or breath \\\hline
Damage: & 1d6 claw, 3d8 bite or breath \\\hline
Movement: & 10' Fly 60'
(20') Swim 60' (15') \\\hline
No. Appearing: & 1, Wild 1, Lair 1d4 \\\hline
Save As: & Fighter: 8 (as Hit Dice) \\\hline
Morale: & 8 \\\hline
Treasure Type: & H \\\hline
XP: & 1,015 \\\hline
\end{tabularx}\medskip

Young sea dragons are light bluish-gray in color (similar to dolphins), darkening to a deep slate color in older individuals. Their skin is smooth and sleek, and their bodies are more compact than most dragons, though their long neck gives them the same flexibility.

Though they live in the water and are somewhat adapted to it, sea dragons still must breathe air, similar to dolphins or whales. A sea dragon may hold its breath up to three turns while swimming or performing other moderate activity.

These dragons have much the same physical structure as other dragons, but their feet are webbed and their tails are short, flat and broad; these adaptations help the sea dragon swim efficiently, but severely limit their ability to walk on dry land. Unlike other dragons, sea dragons do not have a tail attack. The breath weapon of a sea dragon is a cloud of steam; they are immune to damage from non-magical steam (including the breath weapon of another sea dragon), and suffer only half damage from magical steam attacks.

Sea dragons are neutral in outlook, in much the same way as ice dragons. They often maintain lairs in air-filled undersea caverns.\\

\begin{center}
	\textbf{Sea Dragon Age Table}\medskip

\begin{tabularx}{0.47\textwidth}{@{}lXXXXXXX@{}}
Age Category & 1 & 2 & 3 & 4 & 5 & 6 & 7 \\\hline
Hit Dice & 5 & 6 & 7 & 8 & 9 & 10 & 11 \\\hline
Attack Bonus & +5 & +6 & +7 & +8 & +8 & +9 & +9 \\\hline
Breath Weapon &  \multicolumn{7}{c}{Steam (Cloud)}\\\hline
Length & - & 70' & 80' & 90' & 95' & 100' & 100' \\\hline
Width & - & 25' & 30' & 40' & 45' & 50' & 55' \\\hline
Chance/Talk. & 0\% & 15\% & 20\% & 30\% & 45\% & 55\% & 65\% \\\hline
Spells by Level & & & & & & & \\\hline
Level 1 & - & 1 & 2 & 3 & 3 & 4 & 4 \\\hline
Level 2 & - & - & 1 & 2 & 3 & 3 & 4 \\\hline
Level 3 & - & - & - & - & - & 1 & 2 \\\hline
Claw & 1d4 & 1d6 & 1d6 & 1d6 & 1d6 & 1d8 & 1d10 \\\hline
Bite & 2d4 & 3d4 & 3d6 & 3d8 & 3d8 & 3d8 & 3d10 \\\hline
\end{tabularx}
\end{center}

\subsection*{Dragon, Swamp (Black Dragon)}\index{Dragon, Swamp (Black Dragon)}\label{dragon-swamp-black-dragon}

\begin{tabularx}{0.48\textwidth}{@{}lX@{}}
Armor Class: & 18 \\\hline
Hit Dice: & 7** \\\hline
No. of Attacks: & 2 claws, 1 bite or breath, 1 tail \\\hline
Damage: & 1d6 claw, 2d10 bite or breath, 1d6 tail \\\hline
Movement: & 30' Fly 80'
(15') \\\hline
No. Appearing: & 1, Wild 1, Lair 1d4 \\\hline
Save As: & Fighter: 7 (as Hit Dice) \\\hline
Morale: & 8 \\\hline
Treasure Type: & H \\\hline
XP: & 800 \\\hline
\end{tabularx}\medskip

Swamp dragons have green skin so dark as to appear to be black, especially in uncertain light or while wet. They often choose to hide underwater, leaving only part of the head above the waterline, and leap up suddenly when prey comes within 100' (surprising on a roll of 1-4 on 1d6 in this case).

Though swamp dragons are more cruel than ice dragons, they are still motivated mostly by the urge to live, breed and collect valuable items.

Swamp dragons are immune to all forms of acid. A swamp dragon may hold its breath up to three turns while lying in wait underwater.\\

\begin{center}
	\textbf{Swamp Dragon Age Table}\medskip

\begin{tabularx}{0.47\textwidth}{@{}lXXXXXXX@{}}
Age Category & 1 & 2 & 3 & 4 & 5 & 6 & 7 \\\hline
Hit Dice & 4 & 5 & 6 & 7 & 8 & 9 & 10 \\\hline
Attack Bonus & +4 & +5 & +6 & +7 & +8 & +8 & +9 \\\hline
Breath Weapon &  \multicolumn{7}{c}{Acid (Line)}\\\hline
Length & - & 70' & 80' & 90' & 95' & 100' & 100' \\\hline
Width & - & - & - & - & - & 40' & 45' \\\hline
Chance/Talk. & 0\% & 15\% & 20\% & 25\% & 35\% & 50\% & 60\% \\\hline
Spells by Level & & & & & & & \\\hline
Level 1 & - & 1 & 2 & 4 & 4 & 4 & 4 \\\hline
Level 2 & - & - & - & - & 1 & 2 & 3 \\\hline
Level 3 & - & - & - & - & - & 1 & 2 \\\hline
Claw & 1d4 & 1d4 & 1d6 & 1d6 & 1d6 & 1d8 & 1d8 \\\hline
Bite & 2d4 & 2d6 & 2d8 & 2d10 & 2d10 & 2d10 & 2d12 \\\hline
Tail & 1d4 & 1d4 & 1d4 & 1d6 & 1d6 & 1d8 & 1d8 \\\hline
\end{tabularx}
\end{center}

\subsection*{Dragon Turtle}\index{Dragon Turtle}\label{dragon-turtle}

\begin{tabularx}{0.48\textwidth}{@{}lX@{}}
Armor Class: & 22 \\\hline
Hit Dice: & 30** (AB +15) \\\hline
No. of Attacks: & 2 claws, 1 bite or breath \\\hline
Damage: & 2d8 claw, 10d6 bite or 30d8 breath \\\hline
Movement: & 10' (10') Swim
30' (15') \\\hline
No. Appearing: & Wild 1 \\\hline
Save As: & Fighter: 20 at +5 \\\hline
Morale: & 10 \\\hline
Treasure Type: & H (calculated at one-quarter hit dice) \\\hline
XP: & 13,650 \\\hline
\end{tabularx}\medskip

Dragon turtles are so large, up to 200 feet long, that they are occasionally mistaken for rocky outcroppings or even small islands. Though they are not true dragons, they do advance through the same sort of age categories as the true dragons do; however, each age category changes the dragon turtle' s Hit Dice by 5.

Due to their massive size, dragon turtles are immune to virtually all poisons.


\begin{center}
\textbf{Dragon Turtle Age Table}\\

\begin{tabularx}{0.47\textwidth}{@{}lXXXXXXX@{}}
Age Category & 1 & 2 & 3 & 4 & 5 & 6 & 7 \\\hline
Hit Dice & 15 & 20 & 25 & 30 & 35 & 40 & 45 \\\hline
Attack Bonus & +11 & +13 & +14 & +15 & +16 & +16 & +16 \\\hline
Breath Weapon & \multicolumn{7}{c}{Steam (Cloud)}\\\hline
Length & - & 50' & 75' & 100' & 125' & 150' & 175' \\\hline
Width & - & 25' & 50' & 75' & 100' & 125' & 150' \\\hline 
Chance/Talk. & 0\% & 15\% & 20\% & 30\% & 45\% & 55\% & 65\% \\\hline
Spells by Level & & & & & & & \\\hline
Level 1 & - & - & 1 & 2 & 2 & 3 & 3 \\\hline
Level 2 & - & - & - & 1 & 2 & 2 & 3 \\\hline
Claw & 1d6 & 2d4 & 2d6 & 2d8 & 2d10 & 2d12 & 3d10 \\\hline
Bite & 4d6 & 6d6 & 8d6 & 10d6 & 12d6 & 14d6 & 16d6 \\\hline
\end{tabularx}
\end{center}


\begin{center}
	\includegraphics[width=0.47\textwidth]{Pictures132/10000000000003CF00000350DEF722D395F7CC8D.png}	
\end{center}

\columnbreak

\subsection*{Dryad}\index{Dryad}\label{dryad}

\begin{tabularx}{0.48\textwidth}{@{}lX@{}}
Armor Class: & 15 \\\hline
Hit Dice: & 2* \\\hline
No. of Attacks: & 1 dagger or 1 fist \\\hline
Damage: & 1d4 dagger or 1d4 fist \\\hline
Movement: & 40' \\\hline
No. Appearing: & Lair 1d6 \\\hline
Save As: & Magic-User: 4 \\\hline
Morale: & 6 \\\hline
Treasure Type: & D \\\hline
XP: & 100 \\\hline
\end{tabularx}\medskip

Dryads are female nature spirits; each is mystically bound to a single, enormous oak tree and must never stray more than 300 yards from it. Any who do become ill and die within 4d6 hours. A dryad's oak does not radiate magic. A dryad lives as long as her tree, and dies when the tree dies; likewise, if the dryad is killed, her tree dies also.

\begin{center}
	\includegraphics[width=0.47\textwidth]{Pictures132/10000001000003CF0000073A1A6F1165DBF645E5.png}\medskip
\end{center}


A dryad resembles an elf woman, with skin that resembles fine polished wood or smooth bark and hair like leaves; the hair color of a dryad usually changes with the seasons, being brown in the winter, pale green in the spring, darker green in the summer, and yellow, orange, or red in the fall.

Though they are usually content to live alone, dryads are often friends with creatures such as treant or other nature spirits who might live nearby. Once per day a dryad can cast a charm similar to the spell \textbf{charm person}, and one may choose to use this ability to compel an interesting human or elf to remain with her as a companion for a period of up to a year. The companion can be taken into and brought out of the dryad' s tree so long as the charm is still in effect.\\


\subsection*{Eagle}\index{Eagle}\label{eagle}

\begin{tabularx}{0.48\textwidth}{@{}lX@{}}
Armor Class: & 13 \\\hline
Hit Dice: & 2 \\\hline
No. of Attacks: & 2 talons, 1 beak \\\hline
Damage: & 1d6 talon, 1d4 beak \\\hline
Movement: & 10' Fly 160' (10') \\\hline
No. Appearing: & 1, Wild 1d4 \\\hline
Save As: & Fighter: 2 \\\hline
Morale: & 8 \\\hline
Treasure Type: & None \\\hline
XP: & 145 \\\hline
\end{tabularx}\medskip

Eagles
are large birds with wingspans 6½ to 7½ feet in width. They stand about 2½ to 3 feet tall and weigh about 13 lbs. These birds have heavy talons and large beaks with a sharp hooked end. Eagles are known to carry prey up to 15 lbs. An eagle will also kill and start to eat animals that are 5 times the eagles size, up to 80 lbs. The largest eagles tend to be fish-eating or sea eagles, while other types of eagles will eat snakes or small animals. Coloring is dependent on the species of eagle from dark browns with a white head, to fully brown, to various grays ranging from black to white. The beaks can be bright yellowish orange to black. Eagles will fly at a prey and attack with both sets of talons, then attack with the beak.

\begin{center}
	\includegraphics[width=0.47\textwidth]{Pictures132/10000000000003D8000003444D4B3C5E38611D2B.png}
\end{center}

\subsection*{Eagle, Giant}\index{Eagle, Giant}\label{eagle-giant}

\begin{tabularx}{0.48\textwidth}{@{}lX@{}}
Armor Class: & 15 \\\hline
Hit Dice: & 4 \\\hline
No. of Attacks: & 2 claws, 1 bite \\\hline
Damage: & 1d6 claw, 1d8 bite \\\hline
Movement: & 10' fly 90' \\\hline
No. Appearing: & 2d6 \\\hline
Save As: & Fighter: 4 \\\hline
Morale: & 7 (12 if defending a nest) \\\hline
Treasure Type: & None \\\hline
XP: & 240 \\\hline
\end{tabularx}\medskip

An average giant eagle has a wingspan of 15 to 20 feet and stands 8 to 12 feet tall. They are intelligent creatures, and many speak Common or another language common in their home territory.

Individual giant eagles are rarely encountered alone, as they prefer to live in loose communities. However, when they hunt they do so in a solitary fashion, with each eagle choosing a single creature as prey. As they can communicate with each other, it is extremely rare for two of them to make the mistake of attacking the same prey; in fact, it is not uncommon for one to attack alone to scatter a group of prey so they can each more easily choose a victim.

Giant eagles mate for life. If a nest with eggs or hatchlings is threatened, both parents will fight without checking morale, and other giant eagles in their community may come to their defense but will still seek to scatter any opponents so as to attack them one on one.

\begin{center}
	\includegraphics[width=0.47\textwidth]{Pictures132/10000000000003CF000005304E93D99F01716F50.png}
\end{center}

\subsection*{Efreeti*}\index{Efreeti}\label{efreeti}

\begin{tabularx}{0.48\textwidth}{@{}lX@{}}
Armor Class: & 21 (m) \\\hline
Hit Dice: & 10* (+9) \\\hline
No. of Attacks: & 1 huge weapon \\\hline
Damage: & 2d8 huge weapon or special \\\hline
Movement: & 30' Fly 80'
(10') \\\hline
No. Appearing: & 1 \\\hline
Save As: & Fighter: 15 \\\hline
Morale: & 12 (9) \\\hline
Treasure Type: & None \\\hline
XP: & 1,390 \\\hline
\end{tabularx}\medskip

Efreet (singular efreeti) are a race of manlike creatures believed to be from the Elemental Plane of Fire. They are large beings, 11 to 12 feet in height and weighing around 2,000 pounds, though their weight is generally immaterial due to their ability to fly by magical means.

Note that the 12 morale reflects an efreeti' s absolute control over its own fear, but does not indicate that the creature will throw its life away easily. Use the "9" figure to determine whether an outmatched efreeti decides to leave a combat.


\begin{center}
	\includegraphics[width=0.47\textwidth]{Pictures132/10000000000003CF000006C1CC63139DEF604291.png}
\end{center}

Efreet have a number of magical powers, which can be used at will (that is, without needing magic words or gestures): become \textbf{invisible}, with unlimited uses per day; assume \textbf{gaseous form}, as the potion, up to one hour per day; \textbf{create illusions}, as the spell \textbf{phantasmal force} but including sound as well as visual elements, three times per day; \textbf{create flame}, with unlimited uses; and create a \textbf{wall of fire }(as the spell), once per day. Create flame allows the efreet to cause a flame to appear in its hand or otherwise on its person at will; it behaves as desired by the efreet, becoming as large as a torch flame or as small as a candle, and ignites flammable material just as any ordinary flame does. The flame can be thrown as a weapon with a range of up to 60', causing 1d8 points of damage on a successful hit. The efreet can create another flame, and throw it as well if desired, once per round.

Efreet may assume the form of a column of fire at will, with no limit as to the number of times per day this power may be used; an efreeti in flame-form fights as if it were a fire elemental.

Due to their highly magical nature, efreet cannot be harmed by non-magical weapons. They are immune to normal fire, and suffer only half damage from magical fire attacks.


\subsection*{Elemental*}\index{Elemental}\label{elemental}

An elemental is a being formed from one of the foundational elements of reality. In Western traditions, the classical elements are air, earth, fire, and water; Asian traditions include a different group: fire, earth, metal, water, and wood. This book presents the full range needed for either tradition, and to those types are added cold and lightning elementals for those who wish to be less traditional. As always, the Game Master decides what sort of monsters appear in their world, and sonot all of the following creatures may be encountered.

Each type of elemental may be summoned to the material plane by means of one of three different methods:

\textbf{Conjured} by the 5\textsuperscript{th} level Magic-User spell
\textbf{conjure elemental}; or, Summoned by means of a magical \textbf{staff}; or,

Summoned by a \textbf{device} (as given in the \textbf{Miscellaneous Magic} subsection of the \textbf{Treasure} section of this book.

These three types of elementals are quite reasonably called \textbf{staff}, \textbf{device}, and \textbf{conjured} elementals. The hit dice of an elemental depends on the type, as follows:\medskip

\begin{tabular*}{0.93\linewidth}{@{\extracolsep{\fill}}ll}
\textbf{Type} & \textbf{Hit Dice} \\\hline
Staff & 8 \\\hline
Device & 12 \\\hline
Conjured & 16 \\\hline
\end{tabular*}\medskip

The summoner of an elemental must concentrate on it to control it, and may take no other action, including attacking, being attacked, or moving, or control will be lost. Once control is lost it cannot be regained, and the uncontrolled elemental will move directly toward the summoner and attack.

Elementals must be summoned from a large quantity of the appropriate natural material. For example, air elementals require a large quantity of air (so small underground spaces will not support the summoning of one); earth elementals require access to natural earth or stone (and worked stone such as the stone walls of a castle will not work); fire elementals require a large fire such as a bonfire; and water elementals require access to a substantial body of water, at the very least a river or lake (small streams and artificial pools will not work). Finally, when an elemental is summoned, no other elemental of the same type may be summoned in the same day within a radius of 100 miles of the location.

Non-magical weapons cannot harm an elemental. Attacks made by an elemental should be considered magical for purposes of determining how much damage creatures resistant to the elemental' s attack form should suffer.

Generally, elementals are immune to both normal and magical forms of their own attack form. Most are more susceptible to attacks from one or two specific other types of elemental; this is noted in the text for each type.

\subsection*{Elemental, Air*}\index{Elemental, Air}\label{elemental-air}

\begin{center}
	\includegraphics[width=0.47\textwidth]{Pictures132/10000000000003CF000004E30140D999861E9264.jpg}
\end{center}

\begin{tabularx}{0.48\textwidth}[]{@{}lXXX@{}}
& Staff & Device & Spell \\\hline
Armor Class: & 18 (m) & 20 (m) & 22 (m) \\\hline
Hit Dice: & 8* & 12* (+10) & 16* (+12) \\\hline
No. of Attacks: & \multicolumn{3}{c}{special, see below} \\\hline
Damage: & 1d12 & 2d8 & 3d6 \\\hline
Movement: & \multicolumn{3}{c}{Fly 120'} \\\hline
No. Appearing: &  \multicolumn{3}{c}{special} \\\hline
Save As: & Fighter 8 & Fighter 12 & Fighter 16 \\\hline
Morale: & \multicolumn{3}{c}{10} \\\hline
Treasure Type: & \multicolumn{3}{c}{none} \\\hline
XP: & 945 & 1,975 & 3,385 \\\hline
\end{tabularx}\medskip



Air elementals resemble "dust devils," that is, small whirlwinds, but they are much more powerful. Air elementals take double damage when attacked by earth-based attacks (including by earth elementals). An air elemental may choose either to attack a single opponent, thus receiving one attack per round at the listed damage, or may choose to knock all opponents in a 5' radius to the ground; if the latter attack is used, all creatures of 2 hit dice or less must save vs. Death Ray or fall prone. 

Creatures of 3 or more levels or hit dice are not so affected. Air elementals do an additional 1d8 points of damage against creatures or vehicles which are airborne.

\subsection*{Elemental, Cold*}\index{Elemental, Cold}\label{elemental-cold}

\begin{tabularx}{0.48\textwidth}{@{}lllX@{}}
& Staff & Device & Spell \\\hline
Armor Class: & 18 (m) & 20 (m) & 22 (m) \\\hline
Hit Dice: & 8* & 12* (+10) & 16* (+12) \\\hline
No. of Attacks: & 1 & 1 & 1 \\\hline
Damage: & 1d12 & 2d8 & 3d6 \\\hline
Movement: & \multicolumn{3}{c}{40'} \\\hline
No. Appearing: &\multicolumn{3}{c}{special} \\\hline
Save As: & Fighter 8 & Fighter 12 & Fighter 16 \\\hline
Morale: & \multicolumn{3}{c}{10} \\\hline
Treasure Type: & \multicolumn{3}{c}{none} \\\hline
XP: & 945 & 1,975 & 3,385 \\\hline
\end{tabularx}\medskip

A cold elemental resembles a crude, headless ice statue with long sharp icicles in place of hands. A cold elemental suffers double damage from fire attacks, including the attacks of fire elementals. It deals an additional 1d8 points of damage against creatures that are hot or flaming in nature, as well as creatures made of liquids or jelly. A cold elemental' s body is so bitterly cold that creatures

\begin{center}
	\includegraphics[width=0.47\textwidth]{Pictures132/10000000000003C8000004501C03DB36ECFA34F5.png}
\end{center}

within 5 feet take 1d6 points of damage automatically, unless they are immune to the effects of cold. Any liquids the cold elemental touches immediately freeze solid. A cold elemental cannot enter places where the temperature is above 50 degrees Fahrenheit, and if forced to do so will suffer 1d6 points of damage each round.


\subsection*{Elemental, Earth*}\index{Elemental, Earth}\label{elemental-earth}

\begin{tabularx}{0.48\textwidth}{@{}lXXX@{}}
& Staff & Device & Spell \\\hline
Armor Class: & 18 (m) & 20 (m) & 22 (m) \\\hline
Hit Dice: & 8* & 12* (+10) & 16* (+12) \\\hline
No. of Attacks: & 1 & 1 & 1 \\\hline
Damage: & 1d12 & 2d8 & 3d6 \\\hline
Movement: & \multicolumn{3}{c}{20' (10')} \\\hline
No. Appearing: &\multicolumn{3}{c}{special} \\\hline
Save As: & Fighter 8 & Fighter 12 & Fighter 16 \\\hline
Morale: & \multicolumn{3}{c}{10} \\\hline
Treasure Type: & \multicolumn{3}{c}{none} \\\hline
XP: & 945 & 1,975 & 3,385 \\\hline
\end{tabularx}\medskip

Earth elementals resemble crude, headless humanoid statues, with clublike hands and feet. They cannot cross a body of water wider than their own height. Earth elementals take double damage when attacked by fire (including fire elementals). They do an additional 1d8 points of damage against creatures, vehicles, or structures which rest on the ground.

\begin{center}
	\includegraphics[width=0.47\textwidth]{Pictures132/10000000000003CF00000401881253A95C0467E9.jpg}
\end{center}

\columnbreak


\subsection*{Elemental, Fire*}\index{Elemental, Fire}\label{elemental-fire}
\begin{tabularx}{0.48\textwidth}{@{}lllX@{}}
& Staff & Device & Spell \\\hline
Armor Class: & 18 (m) & 20 (m) & 22 (m) \\\hline
Hit Dice: & 8* & 12* (+10) & 16* (+12) \\\hline
No. of Attacks: & 1 & 1 & 1 \\\hline
Damage: & 1d12 & 2d8 & 3d6 \\\hline
Movement:  & \multicolumn{3}{c}{40' Fly 30'}\\\hline
No. Appearing: &\multicolumn{3}{c}{special} \\\hline
Save As: & Fighter 8 & Fighter 12 & Fighter 16 \\\hline
Morale: & \multicolumn{3}{c}{10} \\\hline
Treasure Type: & \multicolumn{3}{c}{none} \\\hline
XP: & 945 & 1,975 & 3,385 \\\hline
\end{tabularx}\medskip

Fire elementals are simply flames, which may appear generally humanoid for brief moments when they attack. Fire elementals take double damage when attacked by water (including water elementals). They cannot cross a body of water wider than their own diameter. They do an additional 1d8 points of damage against creatures which are cold or icy in nature.

Remember that a fire elemental is constantly burning; such a creature may easily start fires if it moves into an area containing items which burn easily, such as dry wood, paper, or oil. No specific rules are given for such fires, but the GM is directed to the rules for burning oil for an example of fire damage.

\begin{center}
	\includegraphics[width=0.47\textwidth]{Pictures132/10000000000003CF0000058482BBE4A57396BEF9.jpg}
\end{center}

\subsection*{Elemental, Lightning*}\index{Elemental, Lightning}\label{elemental-lightning}

\begin{tabularx}{0.48\textwidth}{@{}lllX@{}}
& Staff & Device & Spell \\\hline
Armor Class: & 19 (m) & 21 (m) & 23 (m) \\\hline
Hit Dice: & 8* & 12* (+10) & 16* (+12) \\\hline
No. of Attacks: & -- special -- & & \\\hline
Damage: & 1d12 & 2d8 & 3d6 \\\hline
Movement:  & \multicolumn{3}{c}{40' Fly 80'}\\\hline 
No. Appearing: &\multicolumn{3}{c}{special} \\\hline
Save As: & Fighter 8 & Fighter 12 & Fighter 16 \\\hline
Morale: & \multicolumn{3}{c}{10} \\\hline
Treasure Type: & \multicolumn{3}{c}{none} \\\hline
XP: & 945 & 1,975 & 3,385 \\\hline
\end{tabularx}\medskip

A lightning elemental resembles dark clouds lit from within by flashes of lightning. One can magnetically draw metal items towards itself as if using \textbf{telekinesis}. It deals 1d8 extra points of damage to creatures that are in contact with water or metal but not touching solid ground. A lightning elemental takes double damage when attacked by air or wind attacks (including air elementals), and from the attacks of wood elementals as well. A lightning elemental can choose either to strike a single creature or create a mighty thunderclap. If the latter attack is used, all creatures within a 30 foot radius must save vs. Paralysis or be deafened for 1d8 turns.


\begin{center}
	\includegraphics[width=0.47\textwidth]{Pictures132/10000000000003D8000003D15212159521A19CD6.png}
\end{center}


\subsection*{Elemental, Metal*}\index{Elemental, Metal}\label{elemental-metal}

\begin{tabularx}{0.48\textwidth}{@{}lllX@{}}
& Staff & Device & Spell \\\hline
Armor Class: & 19 (m) & 21 (m) & 23 (m) \\\hline
Hit Dice: & 8* & 12* (+10) & 16* (+12) \\\hline
No. of Attacks: & 1 & 1 & 1 \\\hline
Damage: & 1d12 & 2d8 & 3d6 \\\hline
Movement:  & \multicolumn{3}{c}{20' (10')}\\\hline
No. Appearing: &\multicolumn{3}{c}{special} \\\hline
Save As: & Fighter 8 & Fighter 12 & Fighter 16 \\\hline
Morale: & \multicolumn{3}{c}{10} \\\hline
Treasure Type: & \multicolumn{3}{c}{none} \\\hline
XP: & 945 & 1,975 & 3,385 \\\hline
\end{tabularx}\medskip

Metal elementals appear to be somewhat abstract humanoid figures formed from metal. They are able to move as if liquid, though they are cool and hard to the touch. Their semi-liquid form permits them to form their extremities into wickedly sharp blades, which is their preferred means of attack. Those wearing metal armor receive no protection against a metal elemental (except for magical bonuses, if any); indeed, on a successful hit one deals an additional 1d8 points of damage to creatures, vehicles, or structures that are made of or in direct contact with some form of metal. Lightning attacks deal double damage to a metal elemental. Like an earth elemental, a metal elemental cannot cross a body of water greater than its own height.

\begin{center}
	\includegraphics[width=0.47\textwidth]{Pictures132/10000000000003D8000004C3640B40AED9C69E7B.png}
\end{center}

\subsection*{Elemental, Water*}\index{Elemental, Water}\label{elemental-water}

\begin{tabularx}{0.48\textwidth}{@{}lllX@{}}
& Staff & Device & Spell \\\hline
Armor Class: & 18 (m) & 20 (m) & 22 (m) \\\hline
Hit Dice: & 8* & 12* (+10) & 16* (+12) \\\hline
No. of Attacks: & 1 & 1 & 1 \\\hline
Damage: & 1d12 & 2d8 & 3d6 \\\hline
Movement:  & \multicolumn{3}{c}{20' (15') Swim 60')}\\\hline
No. Appearing: &\multicolumn{3}{c}{special} \\\hline
Save As: & Fighter 8 & Fighter 12 & Fighter 16 \\\hline
Morale: & \multicolumn{3}{c}{10} \\\hline
Treasure Type: & \multicolumn{3}{c}{none} \\\hline
XP: & 945 & 1,975 & 3,385 \\\hline
\end{tabularx}\medskip

Water elementals resemble roiling waves of water, which seem to fall upon any creature attacked, only to reform the next round. They take double damage when attacked with air or wind attacks (including air elementals). A water elemental cannot move more than 60' from a body of water. They do an extra 1d8 points of damage against creatures, vehicles, or structures which are in the water.


\begin{center}
	\includegraphics[width=0.47\textwidth]{Pictures132/10000000000003CF0000036CB8F100A5C44C2CCC.jpg}
\end{center}

\columnbreak

\subsection*{Elemental, Wood*}\index{Elemental, Wood}\label{elemental-wood}

\begin{tabularx}{0.48\textwidth}{@{}lllX@{}}
& Staff & Device & Spell \\\hline
Armor Class: & 17 (m) & 19 (m) & 21 (m) \\\hline
Hit Dice: & 8* & 12* (+10) & 16* (+12) \\\hline
No. of Attacks: & 1 & 1 & 1 \\\hline
Damage: & 1d12 & 2d8 & 3d6 \\\hline
Movement:  & \multicolumn{3}{c}{40'}\\\hline
No. Appearing: &\multicolumn{3}{c}{special} \\\hline
Save As: & Fighter 8 & Fighter 12 & Fighter 16 \\\hline
Morale: & \multicolumn{3}{c}{10} \\\hline
Treasure Type: & \multicolumn{3}{c}{none} \\\hline
XP: & 945 & 1,975 & 3,385 \\\hline
\end{tabularx}\medskip

A wood elemental appears to be a large, leafless tree. They deal 1d8 points of extra damage to creatures in contact with any woody materials, living or dead (including weapons or shields made mainly of wood). On the other hand, they suffer double damage from fire or lightning attacks of any kind, including the attacks of fire or lightning elementals.\\

\begin{center}
	\includegraphics[width=0.47\textwidth]{Pictures132/10000000000003D80000045C562EE3AE621B9A2B.png}

\end{center}


\subsection*{Elephant}\index{Elephant}\label{elephant}

\begin{tabularx}{0.48\textwidth}{@{}lXX@{}}
& Asiatic & African \\\hline
Armor Class: & 16 & 18 \\\hline
Hit Dice: & 9 (+8) & 10 (+9) \\\hline
No. of Attacks: & -- 2 tusks, 1 trunk grab, 2 tramples -- & \\\hline
Damage: & 2d4 tusk, 2d6 grab, 2d8 trample & 2d6 tusk, 2d6 grab, 2d8 trample \\\hline
Movement: & 40' (10') & 50' (15') \\\hline
No. Appearing: & Wild 1d20 & Wild 1d12 \\\hline
Save As: & Fighter: 9 & Fighter: 10 \\\hline
Morale: & 8 & 8 \\\hline
Treasure Type: & special & special \\\hline
XP: & 1,075 & 1,300 \\\hline
\end{tabularx}\medskip

A light load for an African elephant is 7,500 pounds; a heavy load, up to 15,000 pounds. For an Asiatic elephant, a light load is up to 7,000 pounds, and a heavy load up to 14,000 pounds.

Though elephants have five distinct attack modes (two tusks, a trunk grab, and two tramples with the front feet), a single elephant can apply no more than two of these attacks to any single opponent of small or medium size; large opponents may be targeted by three of these attacks in a round. However, an elephant can attack multiple opponents in its immediate area at the same time.

An elephant has no treasure as such, but the tusks of an elephant are worth 1d8 x 100 gp each.\medskip

\begin{center}
	\includegraphics[width=0.47\textwidth]{Pictures132/10000000000003CF000002E4CEC4747FCD15E247.png}
\end{center}

\subsection*{Elk}\index{Elk}\label{elk}

See \textbf{Antelope} on page \hyperlink{antelope}{\pageref{antelope}}.

\subsection*{Falcon}\index{Falcon}\label{falcon}

\begin{tabularx}{0.48\textwidth}{@{}lX@{}}
Armor Class: & 11 \\\hline
Hit Dice: & ½ (1d4 hit points) \\\hline
No. of Attacks: & 2 talons, 1 beak \\\hline
Damage: & 1d4 talon, 1d4 beak \\\hline
Movement: & 10' Fly 160'(10') \\\hline
No. Appearing: & 1, Wild 1d4 \\\hline
Save As: & Fighter: 1 \\\hline
Morale: & 8 \\\hline
Treasure Type: & None \\\hline
XP: & 10 \\\hline
\end{tabularx}\medskip


\begin{center}
	\includegraphics[width=0.47\textwidth]{Pictures132/10000000000003D8000004846D5C00269F109365.png}
\end{center}

Falcons
are birds with wingspans of 16 to 20 inches; they stand about 1.5 to 2 feet tall and weigh about 3 pounds. These are the most popular of the hunting birds used by royalty. Falcons can only carry prey up to 1 pound. They will hunt snakes, small rodents, and even other birds such as wild ducks. Falcons appear in a variety of colors, typically ranging from dark gray on top to white with dark stripes on the bottom; kestrels, a variety of falcon, have reddish brown to dark brown feathers with dark stripes. A falcon typically attacks prey by diving, striking first with the talons before making the kill with its sharp beak.


\subsection*{Fish, Giant Barracuda}\index{Fish, Giant Barracudaa}\label{fish-giant-barracuda}

\begin{tabularx}{0.48\textwidth}{@{}llX@{}}
& Huge & Giant \\\hline
Armor Class: & 16 & 15 \\\hline
Hit Dice: & 5 & 9 (+8) \\\hline
No. of Attacks: & 1 bite & 1 bite \\\hline
Damage: & 2d6 bite & 2d8+1 bite \\\hline
Movement: & Swim 60' & Swim 60'
(10') \\\hline
No. Appearing: & Wild 2d4 & Wild 1 \\\hline
Save As: & Fighter: 5 & Fighter: 9 \\\hline
Morale: & 8 & 10 \\\hline
Treasure Type: & None & None \\\hline
XP: & 360 & 1,075 \\\hline
\end{tabularx}\medskip

Barracuda are predatory fish found in salt water. Huge barracudas are about 12' long, while giant specimens can exceed 20'. They have elongated bodies, pointed heads and prominent jaws. Their bodies are covered with smooth scales, typically blue, gray or silver in color. They have extremely keen eyesight and are surprised only on a 1 on 1d6. Due to the quickness of their attack, barracuda are capable of surprising on 1-3 on 1d6 and gain a +2 bonus to Initiative.

Giant barracuda always appear singly and are 50\% likely to break off the attack after 1d4 rounds if they haven' t killed their prey. Both kinds are attracted to shiny objects.

\subsection*{Fish, Giant Bass}\index{Fish, Giant Bass}\label{fish-giant-bass}

\begin{tabularx}{0.48\textwidth}{@{}lX@{}}
Armor Class: & 13 \\\hline
Hit Dice: & 2 \\\hline
No. of Attacks: & 1 bite \\\hline
Damage: & 1d6 \\\hline
Movement: & Swim 40' (10') \\\hline
No. Appearing: & Wild 1d6 \\\hline
Save As: & Fighter: 2 \\\hline
Morale: & 8 \\\hline
Treasure Type: & None \\\hline
XP: & 75 \\\hline
\end{tabularx}\medskip

Giant bass are generally between 10' and 25' long. Most are greenish-grey marked with dark lateral stripes, though some are almost completely black. They are generally found in lakes or rivers, as they are not adapted for salt water.

Giant bass are predatory, and on a natural attack roll of 20 a giant bass will swallow whole a dwarf-sized or smaller creature, which then takes 2d4 damage per round until it is dead. Swallowed characters can attack only with daggers or similar short weapons. Note that each giant bass can swallow at most one character, and a giant bass which has swallowed a character will attempt to retreat (having achieved its goal).

\subsection*{Fish, Giant Catfish}\index{Fish, Giant Catfish}\label{fish-giant-catfish}

\begin{tabularx}{0.48\textwidth}{@{}lX@{}}
Armor Class: & 16 \\\hline
Hit Dice: & 8 \\\hline
No. of Attacks: & 1 bite, 2 fins \\\hline
Damage: & 2d8 bite, 1d4+poison fin \\\hline
Movement: & Swim 30' (10') \\\hline
No. Appearing: & Wild 1d2 \\\hline
Save As: & Fighter: 8 \\\hline
Morale: & 8 \\\hline
Treasure Type: & None \\\hline
XP: & 875 \\\hline
\end{tabularx}\medskip

Giant catfish fins are edged with a natural poison that causes a painful burning sensation for 3d10 rounds if a save vs. Poison is failed. The pain causes the affected character or creature to suffer a -1 penalty on all attack rolls and saving throws; further poisonings will increase this penalty by -1 each, down to a maximum penalty of -5 as well as adding 6 rounds to the duration of the poison effect.

Because of its large size (15 to 20 feet long) and body design, a giant catfish cannot target more than one of its attacks on any single creature; that is, it cannot bite and fin the same opponent, nor use both fins on one victim.

\subsection*{Fish, Giant Piranha}\index{Fish, Giant Piranha}\label{fish-giant-piranha}

\begin{tabularx}{0.48\textwidth}{@{}lX@{}}
Armor Class: & 15 \\\hline
Hit Dice: & 4 \\\hline
No. of Attacks: & 1 bite \\\hline
Damage: & 1d8 \\\hline
Movement: & Swim 50' \\\hline
No. Appearing: & Wild 2d4 \\\hline
Save As: & Fighter: 4 \\\hline
Morale: & 7 (11) \\\hline
Treasure Type: & None \\\hline
XP: & 240 \\\hline
\end{tabularx}\medskip


Giant piranha average 5' in length at adulthood, and are aggressive carnivores. They are able to sense blood in the water just as sharks do, and once they smell or taste blood in the water, their morale rises to the parenthesized figure.

\begin{center}
	\includegraphics[width=0.47\textwidth]{Pictures132/10000000000003D8000001B89E0B76FE78D9A4B2.png}
\end{center}

\subsection*{Fly, Giant}\index{Fly, Giant}\label{fly-giant}

\begin{tabularx}{0.48\textwidth}{@{}lX@{}}
Armor Class: & 14 \\\hline
Hit Dice: & 2 \\\hline
No. of Attacks: & 1 bite \\\hline
Damage: & 1d8 \\\hline
Movement: & 30' Fly 60' \\\hline
No. Appearing: & 1d6, Wild 2d6 \\\hline
Save As: & Fighter: 2 \\\hline
Morale: & 8 \\\hline
Treasure Type: & None \\\hline
XP: & 75 \\\hline
\end{tabularx}\medskip

Giant flies look much like ordinary houseflies, but are about 3' long. Some are banded yellow and black, and are thus mistaken for giant bees. Giant flies are predators; after killing prey, they will sometimes lay eggs in the remains.

\subsection*{Frog, Giant (and Toad, Giant)}\index{Frog, Giant (and Toad, Giant)}
\begin{tabularx}{0.48\textwidth}{@{}lX@{}}
Armor Class: & 13 \\\hline
Hit Dice: & 2 \\\hline
No. of Attacks: & 1 tongue or 1 bite \\\hline
Damage: & tongue grab or 1d4+1 bite \\\hline
Movement: & 30' Swim 30' \\\hline
No. Appearing: & 1d4, Wild 1d4 \\\hline
Save As: & Fighter: 2 \\\hline
Morale: & 6 \\\hline
Treasure Type: & None \\\hline
XP: & 75 \\\hline
\end{tabularx}\medskip

Giant frogs are enlarged versions of the common frog; most resemble
bullfrogs in appearance, but an adult giant frog is 3'
long and weighs about 250 pounds. They are predators, but will normally
only attack creatures smaller than themselves. Giant toads are
statistically just like giant frogs; however, they are often found in
"drier" areas as they do not have to maintain a wet skin surface.

A giant frog can stretch its tongue out up to 15' and
drag up to dwarf-sized prey to its mouth; on every subsequent round, the
victim is hit automatically. On a natural 20 attack roll, the victim is
swallowed whole, taking 1d6 points of damage per round thereafter. Each
giant frog can swallow only one such victim.

\end{multicols}

\begin{center}
	\includegraphics[width=1\textwidth]{Pictures132/10000000000007E90000034C3F757160BC344334.png}
\end{center}

\begin{multicols}{2}
	

\subsection*{Gargoyle*}\index{Gargoyle}\label{gargoyle}

\begin{tabularx}{0.48\textwidth}{@{}lX@{}}
Armor Class: & 15 (m) \\\hline
Hit Dice: & 4** \\\hline
No. of Attacks: & 2 claws, 1 bite, 1 horn \\\hline
Damage: & 1d4 claw, 1d6 bite, 1d4 horn \\\hline
Movement: & 30' Fly 50'
(15') \\\hline
No. Appearing: & 1d6, Wild 2d4, Lair 2d4 \\\hline
Save As: & Fighter: 6 \\\hline
Morale: & 11 \\\hline
Treasure Type: & C \\\hline
XP: & 320 \\\hline
\end{tabularx}\medskip



Gargoyles are demonic-looking winged humanoid monsters with gray stone-like skin. They can remain still for an extended period, and are thus often mistaken for stone statues. Gargoyles use this disguise to ambush their foes, surprising on 1-4 on 1d6 if their foes do not otherwise suspect them. They are cruel monsters, inflicting pain on other creatures for the sole purpose of enjoyment.

Gargoyles can go for very long periods of time, on the order of decades, without air or sustenance. Due to their highly magical nature, they can only be harmed by magical weapons.

\end{multicols}

\vfill

\begin{center}
	\includegraphics[width=0.47\textwidth]{Pictures132/10000000000003D8000004A19C605A291AB7A1D8.png}
\end{center}

\pagebreak

\begin{multicols}{2}



\subsection*{Gelatinous Cube}\index{Gelatinous Cube}\label{gelatinous-cube}

See \textbf{Jelly, Glass }on page \hyperlink{jelly-glass-gelatinous-cube}{\pageref{jelly-glass-gelatinous-cube}}.

\subsection*{Ghast}\index{Ghast}\label{ghast}

See \textbf{Ghoul (and Ghast)} on page
\hyperlink{ghoul-and-ghast}{\pageref{ghoul-and-ghast}}.

\subsection*{Ghost*}\index{Ghost}\label{ghost}

\begin{tabularx}{0.48\textwidth}{@{}lX@{}}
Armor Class: & 20 (m) \\\hline
Hit Dice: & 10* (+9) \\\hline
No. of Attacks: & 1 touch or possession + fear \\\hline
Damage: & 1d8 + special touch and see below \\\hline
Movement: & 30' \\\hline
No. Appearing: & 1 \\\hline
Save As: & Fighter: 10 \\\hline
Morale: & 10 \\\hline
Treasure Type: & E, N, O \\\hline
XP: & 1,390 \\\hline
\end{tabularx}\medskip

A ghost is the soul or spirit of a deceased sentient creature that has for some reason remained on the material plane. They usually appear as they did in life, but sometimes the appearance of a ghost is altered by its original personality; for instance, the ghost of an angry person might have a threatening or even demonic visage. Ghosts are \textbf{undead}, and as such are immune to \textbf{sleep}, \textbf{charm}, and \textbf{hold }magic.

Seeing a ghost is so terrible that any living creature who does so must save vs. Spells or flee for 2d6 rounds. Anyone who successfully makes this save may not be so affected by that ghost again.

A ghost that hits a living target with its touch attack does 1d8 points of damage, and at the same time regenerates the same number of hit points. In addition, the victim loses 1 Constitution point. Elves and dwarves (and other long-lived creatures such as dragons) are allowed a saving throw vs. Death Ray to resist this effect, which must be rolled on each hit. Characters who lose Constitution appear to have aged. If a ghost is fighting a living creature which does not have a Constitution score, the GM should assign whatever score they see fit. 

Like most incorporeal creatures, ghosts may normally be hit only by magical weapons. However, if a ghost makes use of its touch attack, it becomes vulnerable to non-magical weapons until the beginning of the next round of combat.

Lost Constitution can be regained at a rate of one point per casting of \textbf{restoration}; nothing else (except a \textbf{wish}) can restore Constitution lost to a ghost. If a character' s Constitution falls to 0, they die permanently and cannot be \textbf{raised} (but still may be \textbf{reincarnated}).

Once per turn, a ghost can use \textbf{telekinesis} (as the spell) as if it were a 10\textsuperscript{th} level Magic-User.

Instead of attacking, a ghost may attempt to possess a living creature. This ability is similar to a \textbf{magic jar} spell (as if cast by a 10th level Magic-User), except that it does not require a receptacle. To use this ability, the ghost must be able to move into the target (so it is possible to avoid this attack by outrunning the ghost). The target can resist the attack with a successful save vs. Spells. A creature that successfully saves is immune to being possessed by that ghost for 24 hours. If the save fails, the ghost enters the target' s body and controls it; control may be maintained until the ghost chooses to leave the victim' s body, or until it is driven out by means of a \textbf{remove curse} or \textbf{dispel evil} spell. While it is possessing a living creature, a ghost may not use any of its special abilities.

\begin{center}
	\includegraphics[width=0.47\textwidth]{Pictures132/10000000000003230000051775E03AE418BE5E6D.png}
\end{center}



\subsection*{Ghoul (and Ghast)}\index{Ghoul (and Ghast)}\label{ghoul-and-ghast}

\begin{tabularx}{0.48\textwidth}{@{}lXX@{}}
& Ghoul & Ghast \\\hline
Armor Class: & 14 & 15 \\\hline
Hit Dice: & 2* & 2** \\\hline
No. of Attacks: & -- 2 claws, 1 bite -- & \\\hline
Damage: & 1d4 claw, 1d4 bite, all plus paralysis & 1d4 claw, 1d4 bite, +
paralysis + stench \\\hline
Movement: & 30' & 30' \\\hline
No. Appearing: & 1d6, Wild 2d8, Lair 2d8 & 1d4, Wild 1d8, Lair 1d8 \\\hline
Save As: & Fighter: 2 & Fighter: 2 \\\hline
Morale: & 9 & 9 \\\hline
Treasure Type: & B & B \\\hline
XP: & 100 & 125 \\\hline
\end{tabularx}\medskip

\textbf{Ghouls} are \textbf{undead }monsters which eat the flesh of dead humanoids to survive. They are vile, disgusting carrion-eaters, but are more than willing to kill for food. Those slain by ghouls will generally be stored until they begin to rot before the ghouls will actually eat them.

Living creatures hit by a ghoul's bite or claw attack must save vs. Paralysis or be paralyzed for 2d8 turns. Elves are immune to this paralysis. Ghouls try to attack with surprise whenever possible, striking from behind tombstones or bursting from shallow graves; when they attack in this way, they are able to surprise opponents on 1-3 on 1d6. Like all undead, they may be Turned by Clerics and are immune to \textbf{sleep}, \textbf{charm}, and \textbf{hold} magics.

Humanoids bitten by ghouls may be infected with ghoul fever. Each time a humanoid is bitten, there is a 5\% chance of the infection being passed. The afflicted humanoid is allowed to save vs. Death Ray; if the save is failed, the humanoid dies within a day.

An afflicted humanoid who dies of ghoul fever rises as a ghoul at the next midnight. A humanoid who becomes a ghoul in this way retains none of the knowledge or abilities they possessed in life. The newly-risen ghoul is not under the control of any other ghouls, but hungers for the flesh of the dead and behaves like any other ghoul in all respects.

\textbf{Ghasts} look and fight almost exactly like ghouls, but they are smarter and just a bit more powerful. Refer to the previous paragraphs for information about their claw attacks and other abilities.

The stink of death and corruption surrounding these creatures is overwhelming. Living creatures within 10 feet must succeed on a save vs. Poison or be sickened for 2d6 rounds (-2 to attack rolls). A creature that successfully saves cannot be affected again by the same ghast's stench for 24 hours. A \textbf{neutralize poison} spell removes the effect from a sickened creature.

They may be Turned by Clerics using the same column as the ghoul, but as they are superior to ghouls, in a mixed group of ghasts and ghouls the GM should apply Turning effects to the ordinary ghouls first.

Humanoids bitten by ghasts may be infected with ghast fever. Each time a humanoid is bitten, there is a 10\% chance of the infection being passed. The afflicted humanoid is allowed to save vs. Death Ray; if the save is failed, the humanoid dies within a day.

An afflicted humanoid who dies of ghast fever rises as a ghast at the next midnight, in a similar fashion to the ghoul. However, a humanoid who becomes a ghast in this way retains all of the knowledge and abilities they possessed in life, unless those abilities are directly incompatible with the creature' s new form (as decided by the GM). For such ghasts, the GM should also adjust XP values to include any such abilities. The newly-risen ghast is not under the control of any other ghasts, but hungers for the flesh of the living and behaves like any other ghast in all respects.

\end{multicols}

\begin{center}
	\includegraphics[width=0.47\textwidth]{Pictures132/10000000000003CF000003D03F9FBED38E8ED285.jpg}
\end{center}

\begin{multicols}{2}
	



\subsection*{Giant, Cloud}\index{Giant, Cloud}\label{giant-cloud}

\begin{tabularx}{0.48\textwidth}{@{}lX@{}}
Armor Class: & 19 (13) \\\hline
Hit Dice: & 12+3* (+10) \\\hline
No. of Attacks: & 1 giant weapon or 1 thrown rock \\\hline
Damage: & 6d6 giant weapon or 3d6 rock \\\hline
Movement: & 20' Unarmored 40'
(10') \\\hline
No. Appearing: & 1d2, Wild 1d3, Lair 1d3 \\\hline
Save As: & Fighter: 12 \\\hline
Morale: & 10 \\\hline
Treasure Type: & E + 1d12x1,000 gp \\\hline
XP: & 1,975 \\\hline
\end{tabularx}\medskip


\begin{center}
	\includegraphics[width=0.47\textwidth]{Pictures132/100000000000027C000003A0D507F881E24E0188.png}
\end{center}

Cloud giants are huge, with males averaging 18 feet tall and females 17 feet tall; an adult cloud giant will normally weigh more than 4,750 pounds. They have very pale skin, sometimes with a slight bluish tint, and their hair is bright metallic colors such as silver, brass, or even gold. They have long lives, with most surviving to 400 years barring disease or misadventure.

Cloud giants are often vain, convinced they are the highest form of creation. They wear the finest clothing, make and eat the best food and drink, and live in castles built on high mountains or even in the clouds themselves. Like most giants, they are suspicious of the smaller races, but cloud giants do not usually prey upon them, preferring instead to demand tribute from smaller humanoids living nearby.

Despite being self-centered, they fight in highly trained units that use good tactics; their belief in their own superiority allows no less. Attacking from above is preferred, and ranged attacks are always considered better than melee. To that end, cloud giants can throw large stones up to 200' for 3d6 points of damage each. Also, 5\% of cloud giants have the abilities of a Magic-User of level 2 to 8 (2d4). A favorite tactic is to encircle enemies, barraging them with rocks while the giants with magical abilities confound them with spells. In battle, cloud giants wear finely crafted, intricately engraved plate mail.


\subsection*{Giant, Cyclops}\index{Giant, Cyclops}\label{giant-cyclops}

\begin{tabularx}{0.48\textwidth}{@{}lX@{}}
Armor Class: & 15 (13) \\\hline
Hit Dice: & 13* (+10) \\\hline
No. of Attacks: & 1 giant club or 1 rock (thrown) \\\hline
Damage: & 3d10 giant club or 3d6 rock \\\hline
Movement: & 20' Unarmored 30' \\\hline
No. Appearing: & 1, Wild 1d4, Lair 1d4 \\\hline
Save As: & Fighter: 13 \\\hline
Morale: & 9 \\\hline
Treasure Type: & E + 1d8x1,000 gp \\\hline
XP: & 2,285 \\\hline
\end{tabularx}\medskip

A cyclops is a one-eyed giant. Huge and brutish, they most resemble hill giants, and even dress in the same "style," layers of crudely prepared hides with the fur left on, unwashed and unrepaired. They are reclusive and unfriendly to almost all of the smaller races.

\begin{center}
	\includegraphics[width=0.47\textwidth]{Pictures132/10000000000003CF000003031F26F288C88C10BE.png}
\end{center}


A cyclops can throw a large rock up to 200' for 3d6 points of damage, but they aim poorly and thus suffer an attack penalty of -2. Once per year, a cyclops can cast the spell \textbf{bestow curse} (the reverse of the spell \textbf{remove curse}).


\subsection*{Giant, Fire}\index{Giant, Fire}\label{giant-fire}


\includegraphics[width=0.47\textwidth]{Pictures132/1000000000000384000005411209E68958BC99C6.png}


\begin{center}
	\begin{tabularx}{0.48\textwidth}{@{}lX@{}}
Armor Class: & 17 (13) \\\hline
Hit Dice: & 11+2* (+9) \\\hline
No. of Attacks: & 1 giant weapon or 1 thrown rock \\\hline
Damage: & 5d6 giant weapon or 3d6 rock \\\hline
Movement: & 20' Unarmored 40'
(10') \\\hline
No. Appearing: & 1d2, Wild 1d3, Lair 1d3 \\\hline
Save As: & Fighter: 11 \\\hline
Morale: & 9 \\\hline
Treasure Type: & E + 1d10x1,000 gp \\\hline
XP: & 1,670 \\\hline
\end{tabularx}\medskip

\end{center}

Despite their great size, fire giants have a Dwarf-like appearance, being barrel-chested with thick arms and legs; average males stand 14 feet tall and weigh around 3,200 pounds, while females average 13 feet tall and around 3,000 pounds.  Their skin is ruddy, their hair is black, and their eyes are a very dark red that is almost black (and looks that color in poor lighting). Fire giants are unfriendly to almost all other humanoid races, though they sometimes subjugate those living nearby to act as their servants. 

In combat they favor heavy plate steel armor (the first AC given above) made of steel blackened by quenching in oil. They arm themselves with massive weapons made of the same material. A fire giant can throw large stones up to 200' for 3d6 points of damage. Fire giants are immune to all fire-based attacks.



\subsection*{Giant, Frost}\index{Giant, Frost}\label{giant-frost}

\begin{tabularx}{0.48\textwidth}{@{}lX@{}}
Armor Class: & 17 (13) \\\hline
Hit Dice: & 10+1* (+9) \\\hline
No. of Attacks: & 1 giant weapon or 1 thrown rock \\\hline
Damage: & 4d6 giant weapon or 3d6 rock \\\hline
Movement: & 20' Unarmored 40'
(10') \\\hline
No. Appearing: & 1d2, Wild 1d4, Lair 1d4 \\\hline
Save As: & Fighter: 10 \\\hline
Morale: & 9 \\\hline
Treasure Type: & E + 1d10x1,000 gp \\\hline
XP: & 1,390 \\\hline
\end{tabularx}\medskip

Frost giants have pale, almost white skin, blonde or pale blue hair, and bright blue eyes. Average males stand 15 feet tall and weigh around 2,800 pounds, while females average 14 feet tall and 2,500 pounds.


\begin{center}
	\includegraphics[width=0.47\textwidth]{Pictures132/10000000000003CF0000055739238D77286C1F5B.png}
\end{center}

Frost giants are, first and foremost, cunning. They dislike the smaller races as much as any giant, but rather than attacking outright they will try to use their advantages to convince those weaker than them to submit. If faced with a stronger force, frost giants will parley or withdraw if possible, attacking only if victory seems assured.

In combat frost giants prefer brightly-polished steel chainmail worn over their customary clothing of leather and fur (the first AC given above), and weapons of the same material. A frost giant can throw large stones up to 200' for 3d6 points of damage. Frost giants are immune to all ice or cold-based attacks.


\subsection*{Giant, Hill}\index{Giant, Hill}\label{giant-hill}

\begin{tabularx}{0.48\textwidth}{@{}lX@{}}
Armor Class: & 15 (13) \\\hline
Hit Dice: & 8 \\\hline
No. of Attacks: & 1 giant weapon (club) \\\hline
Damage: & 2d8 giant weapon \\\hline
Movement: & 30' Unarmored 40' \\\hline
No. Appearing: & 1d4, Wild 2d4, Lair 2d4 \\\hline
Save As: & Fighter: 8 \\\hline
Morale: & 8 \\\hline
Treasure Type: & E + 1d8x1,000 gp \\\hline
XP: & 875 \\\hline
\end{tabularx}\medskip

The
smallest of giants, adult hill giants stand between 10 and 12 feet in height and weigh about 1,100 pounds. They have medium brown skin, though they are often so dirty as to hide their true skin color; their hair is dark, lank, and greasy, and their eyes are dark as well. They wear crude clothing made of leather; the lack of livestock or game of great enough size often results in whole pelts being stitched together.

Whether attacking with a weapon or fist, hill giants deal 2d8 damage. Hill giants are brutish and aggressive. They are sometimes found leading groups of ogres or bugbears. Hill giants often keep \textbf{dire wolves} as pets.

\begin{center}
	\includegraphics[width=0.47\textwidth]{Pictures132/10000000000003B10000065CC25A193BD4F5FC7B.png}
\end{center}

\subsection*{Giant, Mountain}\index{Giant, Mountain}\label{giant-mountain}

\begin{tabularx}{0.48\textwidth}{@{}lX@{}}
Armor Class: & 15 (13) \\\hline
Hit Dice: & 16 (+12) \\\hline
No. of Attacks: & 1 giant weapon or 1 thrown rock \\\hline
Damage: & 7d6 (8d6) giant weapon, 4d6 rock \\\hline
Movement: & 40' Unarmored 50'
(10') \\\hline
No. Appearing: & 1d4, Wild 1d4, Lair 1d4+1 \\\hline
Save As: & Fighter: 16 \\\hline
Morale: & 10 \\\hline
Treasure Type: & E + 1d12x1,000 gp \\\hline
XP: & 3,250 \\\hline
\end{tabularx}\medskip

Mountain giants are the largest of all giants, with adult males averaging 24 feet in height and weighing around 16,000 pounds. Females are slightly smaller, standing 22 feet tall on the average and weighing around 14,000 pounds. They look very much like enormous, thickly built hill giants, and among humans they are often seen as little more than that. This is underselling the mountain giants, however, for though their crafting is primitive they have an intricately detailed social structure.

Mountain giants live in clans, led by the oldest male and female (or by their oldest offspring if the elders choose to delegate leadership due to advanced age). Each clan rules over and controls a group of 1d6 mountain peaks, generally spanning an area with a 5-6 mile diameter. They prefer peaks connected by high ridges so that they do not have to descend too far when walking their territory. Clan members who are mated establish households within the territory; unmarried members typically live with their parents until they are wed. The "lair" encounter figures above are for a single household, but if one household is attacked and the attackers subsequently retreat, upon their return they will usually find much of the rest of the clan waiting for them in ambush.

They wear armor made of wood held together with heavy rope, and wield giant clubs or staves carved from fully grown trees harvested from the lower reaches of their territory. If they live near storm giants, they may trade for better weapons (the second damage rating given above) but usually cannot afford storm giant armor.

A mountain giant can throw large stones up to 240' for 4d6 points of damage each.

\columnbreak

\subsection*{Giant, Stone}\index{Giant, Stone}\label{giant-stone}

\begin{tabularx}{0.48\textwidth}{@{}lX@{}}
Armor Class: & 17 (15) \\\hline
Hit Dice: & 9 (+8) \\\hline
No. of Attacks: & 1 stone club or 1 thrown rock \\\hline
Damage: & 3d6 stone club or 3d6 rock \\\hline
Movement: & 30' Unarmored 40' \\\hline
No. Appearing: & 1d2, Wild 1d6, Lair 1d6 \\\hline
Save As: & Fighter: 9 \\\hline
Morale: & 9 \\\hline
Treasure Type: & E + 1d8x1,000 gp \\\hline
XP: & 1,075 \\\hline
\end{tabularx}\medskip

Stone giants are not the largest of giants, but with an average adult standing 12 feet tall and weighing roughly 1,500 pounds they are still formidable. There is no substantial difference in height between males and females. They usually dress in heavy leather clothing with sections having been boiled to stiffen them; these outfits serve as armor and give them the first AC above. Stone giants are reclusive, but they will defend their territory (typically in rocky mountainous terrain) against any who trespass therein.

A stone giant can throw large stones up to 300' for 3d6 points of damage. They will fight in groups to defend their territory but use only simple, basic tactics and strategy. 

\begin{center}
	\includegraphics[width=0.47\textwidth]{Pictures132/10000000000003CF00000549D0410030BAEA0112.png}
\end{center}

\subsection*{Giant, Storm}\index{Giant, Storm}\label{giant-storm}

\begin{tabularx}{0.48\textwidth}{@{}lX@{}}
Armor Class: & 19 (13) \\\hline
Hit Dice: & 15** (+11) \\\hline
No. of Attacks: & 1 giant weapon or 1 lightning bolt \\\hline
Damage: & 8d6 giant weapon or 15d6 lightning \\\hline
Movement: & 30' Unarmored 50'
(10') \\\hline
No. Appearing: & 1, Wild 1d3, Lair 1d3 \\\hline
Save As: & Fighter: 15 \\\hline
Morale: & 10 \\\hline
Treasure Type: & E + 1d20x1,000 gp \\\hline
XP: & 3,100 \\\hline
\end{tabularx}\medskip

Storm giants are nearly the largest of the giants, with adult males standing 21 feet tall and weighing around 12,000 pounds; adult females average 20 feet and typically weigh around 11,000 pounds. Most storm giants have pale skin and dark hair, but some individuals have skin of a lavender color, and some have pale white or silver hair. Their eyes range from bright blue to deep gray in color.

\begin{center}
	\includegraphics[width=0.47\textwidth]{Pictures132/10000000000003CF00000465FAA60F2D9E4C2C69.png}
\end{center}

They prefer to dress in light clothing, such as tunics, sandals, and so on, but in battle they wear finely-crafted plate mail armor of bright metal and wield weapons of the same sort. Their equipment usually appears to be silver but those of the highest class or rank wear armor and bear arms of a bright and shining golden color.

Unlike most other giants, storm giants have been known to befriend humans, elves, or dwarves.

Storm giants have the ability to throw \textbf{lightning bolts} as if they were spears (which work just as the spell does, and can be used once per five rounds; a save vs. Spells reduces damage to half). They prefer to attack first with lightning before moving on to other attack forms. Not surprisingly, storm giants are resistant to all forms of lightning or electrical attack, suffering only half damage normal when so attacked.

Also note that 10\% of storm giants have the abilities of a Magic-User of level 2 to 12 (2d6).


\subsection*{Gnoll}\index{Gnoll}\label{gnoll}

\begin{tabularx}{0.48\textwidth}{@{}lX@{}}
Armor Class: & 15 (13) \\\hline
Hit Dice: & 2 \\\hline
No. of Attacks: & 1 weapon \\\hline
Damage: & 2d4 or by weapon +1 \\\hline
Movement: & 30' Unarmored 40' \\\hline
No. Appearing: & 1d6, Wild 3d6, Lair 3d6 \\\hline
Save As: & Fighter: 2 \\\hline
Morale: & 8 \\\hline
Treasure Type: & Q, S each; D, K in lair \\\hline
XP: & 75 \\\hline
\end{tabularx}\medskip

Gnolls are large fur-covered humanoids, averaging 7½ feet in height and weighing about 300 pounds. They are best recognized by their heads, which resemble those of hyenas or perhaps wolves but with shorter muzzles than either.

Gnolls are nocturnal and have Darkvision with a 30' range. They are cruel carnivores, preferring intelligent creatures for food because they scream more. They show little discipline when fighting unless they have a strong leader.


\begin{center}
	\includegraphics[width=0.47\textwidth]{Pictures132/10000000000003D800000498C3DF5F8BCE543C00.png}
\end{center}


One out of every six gnolls will be a hardened warrior of 4 Hit Dice (240 XP) having a +1 bonus to damage due to strength. Gnolls gain a +1 bonus to their morale if they are led by such a warrior. In lairs of 12 or greater, there will be a pack leader of 6 Hit Dice (500 XP) having a +2 bonus to damage. In the lair, gnolls never fail a morale check as long as the pack leader is alive. In addition, a lair has a chance equal to 1-2 on 1d6 of a shaman being present, and 1 on 1d6 of a witch or warlock. A shaman is equivalent to a hardened warrior statistically, and in addition has Clerical abilities at level 1d4+1. A witch or warlock is equivalent to a regular gnoll, and has Magic-User abilities of level 1d4.

\subsection*{Gnome}\index{Gnome}\label{gnome}

\begin{tabularx}{0.48\textwidth}{@{}lX@{}}
Armor Class: & 15 (11) \\\hline
Hit Dice: & 1 \\\hline
No. of Attacks: & 1 weapon \\\hline
Damage: & 1d6 or by weapon \\\hline
Movement: & 20' Unarmored 40' \\\hline
No. Appearing: & 1d8, Wild 5d8, Lair 5d8 \\\hline
Save As: & Fighter: 1 (with Dwarf bonuses) \\\hline
Morale: & 8 \\\hline
Treasure Type: & D \\\hline
XP: & 25 \\\hline
\end{tabularx}\medskip

Gnomes are humanoids distantly related to dwarves; they have a similar appearance, but are smaller (averaging 3 to 3½ feet tall and weighing 40 to 45 pounds) and less stocky. Their ears are pointed, not in as pronounced a fashion as elves but certainly noticeable, and their noses likewise are pointed a bit more than those of most other kinds of humanoids. Their preferred habitat is the forest, preferably in a hilly temperate region.


\begin{center}
	\includegraphics[width=0.47\textwidth]{Pictures132/10000000000003CF000005678C607A2F5E07D7F1.png}
\end{center}

They have Darkvision with a 30' range. When attacked in melee by creatures larger than man-sized, gnomes gain a +1 bonus to their Armor Class. Outdoors in their preferred forest terrain they are able to hide very effectively; so long as they remain still there is only a 20\% chance they will be detected. If one or more gnomes who are successfully hiding attack from ambush, they surprise their foes on 1-4 on 1d6.

Gnomes have their own language, and many also know the language of the dwarves. Some gnomes make their living as traders, often acting as go-betweens for dwarves and humans, and those gnomes naturally tend to learn Common. Those who are engaged in defending their forest settlements from humanoid incursions often choose to learn Goblin or Orc.

\begin{center}
	\includegraphics[width=0.47\textwidth]{Pictures132/10000000000003CF000005678C607A2F5E07D7F1.png}
\end{center}

Gnomes encountered in the wilderness (who are not traders or merchants) are likely to be unfriendly, but not hostile. They tolerate dwarves but dislike most other humanoid races. When forced to interact with other races, a gnome will generally be recalcitrant, unless offered a significant amount of treasure.

The statistics given above are for warriors. In a settlement or lair, for every warrior there will be an average of three civilians having 1-1 Hit Dice and Armor Class 11; such gnomes have Morale of 7. One out of every eight gnome warriors will be a sergeant having 3 Hit Dice (145 XP). Gnomes gain a +1 bonus to their morale if they are led by a sergeant. Both warriors and sergeants commonly wear chainmail. In gnomish communities, one out of every sixteen warriors will be a captain of 5 Hit Dice (360 XP) with an Armor Class of 16 (11), adding a shield. In addition, in communities of 35 or greater, there will be a king of 7 Hit Dice (670 XP), with an Armor Class of 18 (11), in plate mail and carrying a shield, having a +1 bonus damage due to strength. In their community, gnomes never fail a morale check as long as the king is alive. There is a chance equal to 1-4 on 1d6 that a community will have a Cleric of level 1d6+1, and 1-2 on 1d6 of a Magic-User of level 1d6. Gnomish Clerics use 1d6 hit dice and Magic-Users use 1d4 hit dice, and in all other ways behave as if they were normal characters.

\subsection*{Goblin}\index{Goblin}\label{goblin}

\begin{tabularx}{0.48\textwidth}{@{}lX@{}}
Armor Class: & 14 (11) \\\hline
Hit Dice: & 1-1 \\\hline
No. of Attacks: & 1 weapon \\\hline
Damage: & 1d6 or by weapon \\\hline
Movement: & 20' Unarmored 30' \\\hline
No. Appearing: & 2d4,Wild 6d10, Lair 6d10 \\\hline
Save As: & Fighter: 1 \\\hline
Morale: & 7 or see below \\\hline
Treasure Type: & R each; C in lair \\\hline
XP: & 10 \\\hline
\end{tabularx}\medskip


Goglins are small, reputedly wicked humanoids. They are cunning and vicious, and very sneaky. Adult goblins are 3 to 3½ feet tall and weigh 40 to 45 pounds, with very little difference between males and females. Their skin color ranges from gray to green, and their eyes are usually bright and crafty-looking, varying in color from red to yellow.

All goblins have Darkvision with a 30' range.

The statistics given above are for a standard Goblin in leather armor with a shield; they have a natural Movement rate of 30' and a natural Armor Class of 11.

\begin{center}
	\includegraphics[width=0.47\textwidth]{Pictures132/10000000000003CF000003A91083F022EE90AC11.png}
\end{center}

Some goblins ride \textbf{dire wolves} into combat, and large groups of goblins will often employ them to track and attack their foes.

One out of every eight goblins will be a warrior of 3-3 Hit Dice (145 XP). Goblins gain a +1 bonus to their morale if they are led by a warrior. In a lair or other settlement, one out of every fifteen will be a chieftain of 5-5 Hit Dice (360 XP) in chainmail with an Armor Class of 15 (11) and movement of 10' that gains a +1 bonus to damage due to strength. 


In lairs or settlements of 30 or more goblins, there will be a goblin king of 7-7 Hit Dice (670 XP), with an Armor Class of 16 (11), wearing chainmail and carrying a shield, with a movement of 10', and having a +1 bonus to damage. Goblins have a +2 bonus to morale while their king is present (this is not cumulative with the bonus given by a warrior leader). In addition, a lair has a chance equal to 1 on 1d6 of a shaman being present (or 1-2 on 1d6 if a goblin king is present). A shaman is equivalent to a regular goblin statistically, but has Clerical abilities at level 1d4+1.


\begin{center}
	\includegraphics[width=0.47\textwidth]{Pictures132/10000001000003D8000004719502E2C8AA9952D3.png}
\end{center}

\columnbreak


\subsection*{Golem*}\index{Golem}\label{golem}

Golems are a kind of construct, a creature created from non-living matter and animated by application of magic. The powers required to animate a golem are prodigious, and involve summoning, capturing, and binding an elemental spirit to the constructed body. This process also binds the golem to the will of its creator.

They are mindless, and thus immune to magics affecting the mind such as \textbf{sleep}, \textbf{charm}, \textbf{hold}, and any form of \textbf{mind reading} or telepathy. They must be given explicit, detailed instructions verbally, and the controller must usually be within 60 feet of the golem to do so. If not actively being commanded, a golem will follow the last instructions given to it until the controller returns. If such a golem is attacked, it will fight in its own defense but will usually not pursue the attackers if they flee. The controller can order the golem to follow the commands of another, but can always resume control if desired (i.e. the controller' s commands always take precedence).

Employing a golem in combat is tricky, for once one attacks an opponent there is a cumulative 1\% chance each round (so 1\% the first round, 2\% the second, 3\% the third, and so on) that the golem will stop following commands and become berserk. Once this happens the golem will attack any creature in range, choosing targets randomly when there are more than one. If all targets are killed or driven away the golem will move on, looking for more creatures to kill and breaking down any barrier that stands in its way if it is at all possible. 

The berserk chance for a golem that is still under control is reset to 0\% only when the golem is inactive, neither attacking nor being attacked, for one full round.

The creator of the golem (but not any other person who might have been delegated control) may try to calm the golem, speaking firmly to it to convince it to stop. The creator needs to succeed at a saving throw vs. Spells to do this, after spending a round talking to the golem. If this roll fails the golem turns its attention to the creator and pursues them with single-minded hatred.

If a berserk golem is unable to attack anyone for 5 rounds it resumes its inactive state, and the controller can again give it commands. If it begins to pursue its creator, though, it will never stop no matter how long it takes, and must normally be trapped or destroyed to stop it. It has no special way to find the creator, however, and will become inactive if it loses sight of the creator for a minimum of 1 day. If the golem is successfully calmed, it can be given commands again on the very next round of combat.

As their bodies are made of non-living matter, golems can only be hit by magical weapons. Conversely, they are less resistant to various effects due to the fact that they are not living creatures; in general, golems save as if they were Fighters of ½ their hit dice in levels. For example, a Bone Golem has 8 hit dice, but saves as a Fighter of 4\textsuperscript{th} level.

\subsection*{Golem, Amber*}\index{Golem, Amber}\label{golem-amber}

\begin{tabularx}{0.48\textwidth}{@{}lX@{}}
Armor Class: & 21 (m) \\\hline
Hit Dice: & 10* (+9) \\\hline
No. of Attacks: & 2 claws, 1 bite \\\hline
Damage: & 2d6 claw, 2d10 bite \\\hline
Movement: & 60' \\\hline
No. Appearing: & 1 \\\hline
Save As: & Fighter: 5 \\\hline
Morale: & 12 \\\hline
Treasure Type: & None \\\hline
XP: & 1,390 \\\hline
\end{tabularx}\medskip



Amber golems are generally built to resemble lions or other great cats. They are able to detect invisible creatures or objects within 60', and can track with 95\% accuracy through any terrain type.

A magical attack that deals electricity damage heals 1 point of damage for every 3 full points of damage the attack would otherwise deal. For example, an amber golem hit by a \textbf{lightning bolt} for 20 points of damage is instead healed up to 6 points. If the amount of healing would cause the golem to exceed its full normal hit points, the excess is ignored.


\begin{center}
	\includegraphics[width=0.47\textwidth]{Pictures132/10000000000003CF000002593EED082E4CF80F09.jpg}
\end{center}

\columnbreak


\subsection*{Golem, Bone*}\index{Golem, Bone}\label{golem-bone}


\begin{center}
	\includegraphics[width=0.47\textwidth]{Pictures132/10000000000003CF0000056B8F858CDDE47CBDB4.jpg}
\end{center}

\begin{center}
	\begin{tabularx}{0.48\textwidth}{@{}lX@{}}
Armor Class: & 19 (m) \\\hline
Hit Dice: & 8* \\\hline
No. of Attacks: & 4 weapons \\\hline
Damage: & 1d6 or by weapon (each) \\\hline
Movement: & 40' (10') \\\hline
No. Appearing: & 1 \\\hline
Save As: & Fighter: 4 \\\hline
Morale: & 12 \\\hline
Treasure Type: & None \\\hline
XP: & 945 \\\hline
\end{tabularx}\medskip
\end{center}

Bone golems are huge four-armed monsters created from the skeletons of at least two dead humanoids. Though made of bone, they are not undead and cannot be turned.

Instead of four one-handed weapons, a bone golem can be armed with two two-handed weapons, giving 2 attacks per round and a damage figure of 1d10 or by weapon (each).



\subsection*{Golem, Bronze*}\index{Golem, Bronze}\label{golem-bronze}

\begin{tabularx}{0.48\textwidth}{@{}lX@{}}
Armor Class: & 20 (m) \\\hline
Hit Dice: & 20** (+13) \\\hline
No. of Attacks: & 1 fist + special \\\hline
Damage: & 3d10 fist + special \\\hline
Movement: & 80' (10') \\\hline
No. Appearing: & 1 \\\hline
Save As: & Fighter:10 \\\hline
Morale: & 12 \\\hline
Treasure Type: & None \\\hline
XP: & 5,650 \\\hline
\end{tabularx}\medskip

These golems resemble statues made of bronze; unlike natural bronze statues, they never turn green from verdigris. A bronze golem is 10 feet tall and weighs about 4,500 pounds. A bronze golem cannot speak or make any vocal noise, nor does it have any distinguishable odor. It moves with a ponderous but smooth gait. Each step causes the floor to tremble unless it is on a thick, solid foundation.

The interior of a bronze golem is molten metal. Creatures hit by one in combat suffer an additional 1d10 points of damage from the heat (unless resistant to heat or fire). If one is hit in combat, molten metal spurts out, spraying the attacker for 2d6 damage. A save vs. Death Ray is allowed to avoid the metal spray.

\begin{center}
	\includegraphics[width=0.35\textwidth]{Pictures132/10000000000003CF000007CB3D695DDBF84A428D.jpg}
\end{center}

\subsection*{Golem, Clay*}\index{Golem, Clay}\label{golem-clay}

\begin{tabularx}{0.48\textwidth}{@{}lX@{}}
Armor Class: & 22 (m) \\\hline
Hit Dice: & 11** (+9) \\\hline
No. of Attacks: & 1 fist \\\hline
Damage: & 3d10 fist \\\hline
Movement: & 20' \\\hline
No. Appearing: & 1 \\\hline
Save As: & Fighter: 6 \\\hline
Morale: & 12 \\\hline
Treasure Type: & None \\\hline
XP: & 1,765 \\\hline
\end{tabularx}\medskip

Clay golems are made of clay, naturally, and thus may be any natural clay color; generally, one will be grayish in color, but common clay containing iron oxide may be used which results in a red, brown, or even orange clay golem. They are usually unclad, but some golem-makers choose to put a leather belt, girdle, or apron on their creation. A clay golem weighs about 600 pounds.

Wounds inflicted by a clay golem do not heal normally; worse, magical healing cures only 1 point per die rolled (but add all bonuses normally). Thus, a \textbf{cure light wounds }spell heals just 2 points.


\begin{center}
	\includegraphics[width=0.47\textwidth]{Pictures132/10000000000003CF0000061D90C559CD40418816.jpg}
\end{center}

\subsection*{Golem, Flesh*}\index{Golem, Flesh}\label{golem-flesh}

\begin{tabularx}{0.48\textwidth}{@{}lX@{}}
Armor Class: & 20 (m) \\\hline
Hit Dice: & 9** (+8) \\\hline
No. of Attacks: & 2 fists \\\hline
Damage: & 2d8 fist \\\hline
Movement: & 30' \\\hline
No. Appearing: & 1 \\\hline
Save As: & Fighter: 5 \\\hline
Morale: & 12 \\\hline
Treasure Type: & None \\\hline
XP: & 1,225 \\\hline
\end{tabularx}\medskip

Flesh golems are horrible creations made of body parts from deceased humanoids (including all character races as well as humanoid monsters), crudely stitched together and animated by magic. A flesh golem is 8 feet tall and weighs about 450 pounds.


\begin{center}
	\includegraphics[width=0.47\textwidth]{Pictures132/10000000000003CF000006C4FE82B47FFFDCC3F3.jpg}
\end{center}

A magical attack that deals cold or fire damage slows a flesh golem (as the \textbf{slow }spell) for 2d6 rounds, with no saving throw. Attacks using lightning or electricity heal 1 point of damage per every 3 points the attack would normally inflict, rounded down; further, such an attack breaks any ongoing slow effect on the golem. As usual, healing will not increase the monster above its normal hit points. For example, a flesh golem hit by a \textbf{lightning bolt} which should deal 14 points of damage instead receives up to 4 points of healing.


\subsection*{Golem, Iron*}\index{Golem, Iron}\label{golem-iron}

\begin{tabularx}{0.48\textwidth}{@{}lX@{}}
Armor Class: & 25 (m) \\\hline
Hit Dice: & 17** (+12) \\\hline
No. of Attacks: & 1 strike + special \\\hline
Damage: & 4d10 strike + special \\\hline
Movement: & 20' (10') \\\hline
No. Appearing: & 1 \\\hline
Save As: & Fighter: 9 \\\hline
Morale: & 12 \\\hline
Treasure Type: & None \\\hline
XP: & 3,890 \\\hline
\end{tabularx}

Iron
golems are huge, generally 11 to 12 feet in height with a weight of 4,500 to 5,500 pounds. Such golems are usually fashioned like statues, and may appear to be wearing armor and armed with a short sword or mace; the sculpting will often be simpler and cruder than that done by a real sculptor. They have no voices, but are hardly silent when in motion, as its gentlest footstep on any hard surface shakes the floor and walls. Note that, while an iron golem has little if any odor normally, when wet they have a strong metallic smell detectable up to 60 feet away.

\begin{center}
	\includegraphics[width=0.47\textwidth]{Pictures132/10000000000003D8000004D94D62F9F182E4AF45.png}
\end{center}

Iron golems can exhale a cloud of poisonous gas which fills a 10-foot cube and persists for 1 round. Those within the area of effect must save vs. Dragon Breath or die. This ability can be used up to 3 times per day.


A magical attack that deals lightning or electrical damage slows an iron golem (as the \textbf{slow }spell) for 1d6 rounds, with no saving throw. Attacks using fire heal 1 point of damage per every 3 points the attack would normally inflict, rounded down; further, such an attack breaks any ongoing slow effect on the golem. As usual, healing will not increase the monster above its normal hit points. For example, an iron golem hit by a fireball which should deal 17 points of damage instead receives up to 5 points of healing. An iron golem is affected by rust attacks such as that of a rust monster, suffering 2d6 points of damage for each hit (with no saving throw normally allowed). 

\subsection*{Golem, Stone*}\index{Golem, Stone}\label{golem-stone}

\begin{tabularx}{0.48\textwidth}{@{}lX@{}}
Armor Class: & 25 (m) \\\hline
Hit Dice: & 14** (+11)  \\\hline
No. of Attacks: & 1 strike + special \\\hline
Damage: & 3d8 strike + special \\\hline
Movement: & 20' (10') \\\hline
No. Appearing: & 1 \\\hline
Save As: & Fighter: 7 \\\hline
Morale: & 12 \\\hline
Treasure Type: & None \\\hline
XP: & 2,730 \\\hline
\end{tabularx}\medskip

Stone golems are quite large, normally being about 8 to 10 feet in height with a weight of around 2,000 pounds. These golems are usually fashioned like statues, and may appear to be wearing armor and armed with a short sword or mace; the sculpting will often be simpler and cruder than that done by a real sculptor.

Once every other round a stone golem has the ability to cast a \textbf{slow }effect, as the spell; a save vs. Spells is allowed to resist. This effect lasts for 2d6 rounds and has an effective range of just 10 feet.

A \textbf{stone to flesh }spell may be used to weaken the monster. The spell does not actually change the golem's structure, but for one full round after being affected, the golem is vulnerable to normal weapons. The stone golem is allowed a save vs. Spells to resist this effect.

\subsection*{Golem, Wood*}\index{Golem, Wood}\label{golem-wood}

\begin{tabularx}{0.48\textwidth}{@{}lX@{}}
Armor Class: & 13 (m) \\\hline
Hit Dice: & 2+2* \\\hline
No. of Attacks: & 1 fist \\\hline
Damage: & 1d8 fist \\\hline
Movement: & 40' \\\hline
No. Appearing: & 1 \\\hline
Save As: & Fighter: 1 \\\hline
Morale: & 12 \\\hline
Treasure Type: & None \\\hline
XP: & 100 \\\hline
\end{tabularx}\medskip

Wood golems are small constructs, not more than 4' in height, and are crudely made. Being made of wood makes them vulnerable to fire-based attacks; thus, wood golems suffer one extra point of damage per die from fire; any saving throws against such effects are at a penalty of -2. They move stiffly, suffering a -1 penalty to Initiative.

\begin{center}
	\includegraphics[width=0.47\textwidth]{Pictures132/10000000000003D8000004B75D73528FC4A104B8.png}
\end{center}

\subsection*{Gorgon}\index{Gorgon}\label{gorgon}

\begin{tabularx}{0.48\textwidth}{@{}lX@{}}
Armor Class: & 19 \\\hline
Hit Dice: & 8* \\\hline
No. of Attacks: & 1 gore or 1 breath \\\hline
Damage: & 2d6 gore, petrification breath \\\hline
Movement: & 40' (10') \\\hline
No. Appearing: & Wild 1d4 \\\hline
Save As: & Fighter: 8 \\\hline
Morale: & 8 \\\hline
Treasure Type: & None \\\hline
XP: & 945 \\\hline
\end{tabularx}\medskip

Gorgons are magical monsters resembling cattle made of iron. Their breath can turn living creatures to stone; it covers an area 60' long by 10' wide, and can be used as many times per day as the monster has hit dice, but no more often than every other round. A save vs. Petrify is allowed to resist. 

An adult male gorgon weighs up to 4,500 pounds and can be as much as 7 feet tall at the shoulder and up to 9 feet long. Females (cows) will be a bit smaller, perhaps no more than 6 feet at the shoulder and 8 feet long with a weight of around 4,000 pounds. However, their combat statistics are much the same as the males. Any group of more than one gorgon will consist of one bull with the rest being cows.

\subsection*{Gray Ooze}\index{Gray Ooze}\label{gray-ooze}

See \textbf{Jelly, Gray }on page \hyperlink{jelly-gray-gray-ooze}{\pageref{jelly-gray-gray-ooze}}.

\subsection*{Green Slime}\index{Green Slime}\label{green-slime}

See \textbf{Jelly, Green} on page \hyperlink{jelly-green-green-slime}{\pageref{jelly-green-green-slime}}.

\subsection*{Griffon}\index{Griffon}\label{griffon}

\begin{tabularx}{0.48\textwidth}{@{}lX@{}}
Armor Class: & 18 \\\hline
Hit Dice: & 7 \\\hline
No. of Attacks: & 2 claws, 1 bite \\\hline
Damage: & 1d4 claw, 2d8 bite \\\hline
Movement: & 40' (10') Fly
120' (10') \\\hline
No. Appearing: & Wild 2d8, Lair 2d8 \\\hline
Save As: & Fighter: 7 \\\hline
Morale: & 8 \\\hline
Treasure Type: & E \\\hline
XP: & 670 \\\hline
\end{tabularx}\medskip

Griffons are large carnivorous creatures resembling lions with the head, foreclaws and wings of eagles. Average adults (male or female) have a wingspan of around 22 feet and weigh around 500 pounds.

\begin{center}
	\includegraphics[width=0.47\textwidth]{Pictures132/10000000000003D8000003A3EC808DAE43217028.png}
\end{center}

Griffons nest on high mountaintops, soaring down to feed on horses, the beast's preferred prey. Indeed, a griffon will attack a horse over anything else. They hunt and travel in flocks, diving low when attacking to swipe with their claws. They are not above retreating and then returning when they may be unexpected.

Griffons can be trained as mounts if raised in captivity, but even in this case they may try to attack horses if any come within about 120'. Roll a morale check in this case; if the check is failed, the griffon will try to attack immediately. A light load for a griffon is up to 400 pounds; a heavy load, up to 900 pounds.


\subsection*{Hangman Tree}\index{Hangman Tree}\label{hangman-tree}

\begin{tabularx}{0.48\textwidth}{@{}lX@{}}
Armor Class: & 16 \\\hline
Hit Dice: & 5 \\\hline
No. of Attacks: & 4 limbs \\\hline
Damage: & 1d6 limb +1d6/round strangle each \\\hline
Movement: & 0 \\\hline
No. Appearing: & Wild 1 \\\hline
Save As: & Fighter: 4 \\\hline
Morale: & 12 \\\hline
Treasure Type: & None \\\hline
XP: & 360 \\\hline
\end{tabularx}\medskip

Hangman trees are horrible, semi-animate creatures that fertilize themselves with dead bodies. A hangman tree has four animated limbs that can wrap around the necks of living creatures that pass beneath, strangling for 1d6 points of damage per round. These limbs are arranged evenly around the tree in most cases, and generally no more than one limb can attack any single creature at a time.

The roots of this tree are also animated; they do not attack, but they do pull dead bodies below the surface of the ground for "digestion."

\begin{center}
	\includegraphics[width=0.47\textwidth]{Pictures132/10000000000003D8000004B6DC70701BF2DF48F6.png}
\end{center}

\subsection*{Harpy}\index{Harpy}\label{harpy}

\begin{tabularx}{0.48\textwidth}{@{}lX@{}}
Armor Class: & 13 \\\hline
Hit Dice: & 2* \\\hline
No. of Attacks: & 2 claws, 1 weapon + special \\\hline
Damage: & 1d4 claw, 1d6 or by weapon + special \\\hline
Movement: & 20' Fly 50'
(10') \\\hline
No. Appearing: & 1d6, Wild 2d4, Lair 2d4 \\\hline
Save As: & Fighter: 2 \\\hline
Morale: & 7 \\\hline
Treasure Type: & C \\\hline
XP: & 100 \\\hline
\end{tabularx}\medskip

A harpy looks like a giant vulture bearing the torso and face of a human female. They are able to attack with their claws as well as with a normal weapon, but they are most feared for the power of their song by which they are able to charm living creatures, and, having charmed them, tear them to pieces at their leisure.

All living creatures within a 300' range of one or more singing harpies must make a save vs. Spells or become \textbf{charmed}. The same harpy's song cannot affect a creature that successfully saves again for 24 hours. This charm is very powerful, such that a victim will approach the harpy or harpies without fear with a dazed expression on its face. If the victim is led toward some danger, such as a ravine, fire, or the like, a second saving throw is allowed immediately; however, if this save fails the victim will proceed directly into danger.

Once in reach of a harpy, a charmed victim will surrender completely, even allowing the harpy to attack and kill it without putting up any sort of resistance. The charm effect lasts one full round after all harpies have ceased singing.

\begin{center}
	\includegraphics[width=0.47\textwidth]{Pictures132/10000000000003D80000042B0EDB1290AA1FEEE7.png}
\end{center}

\subsection*{Hawk}\index{Hawk}\label{hawk}

\begin{tabularx}{0.48\textwidth}{@{}llX@{}}
& Normal & Giant \\\hline
Armor Class: & 12 & 14 \\\hline
Hit Dice: & ½ (1d4 hit points) & 4 \\\hline
No. of Attacks: & 1 claw or bite & 1 claw or bite \\\hline
Damage: & 1d2 claw or bite & 1d6 claw or bite \\\hline
Movement: & Fly 160' & Fly 150'
(10') \\\hline
No. Appearing: & Wild 1d6, Lair 1d6 & Wild 1d3, Lair 1d3 \\\hline
Save As: & Fighter: 1 & Fighter: 4 \\\hline
Morale: & 7 & 8 \\\hline
Treasure Type: & None & None \\\hline
XP: & 10 & 240 \\\hline
\end{tabularx}\medskip

Normal hawks (or falcons) are raptors, predatory birds that typically subsist on small snakes and other vermin. Most have wingspans of less than 5 feet and a body length of no more than 2 feet.

Giant hawks are 4 to 6 feet long, with wingspans of 12 feet or more; they can carry off creatures of Halfling size or smaller.

All hawks will shy away from combat with any creature of equal or greater size, unless forced or cornered or their eggs or offspring are threatened.

\subsection*{Hellhound}\index{Hellhound}\label{hellhound}

\begin{tabularx}{0.48\textwidth}{@{}lX@{}}
Armor Class: & 14 to 18 \\\hline
Hit Dice: & 3** to 7** \\\hline
No. of Attacks: & 1 bite or 1 breath \\\hline
Damage: & 1d6 bite, 1d6 per Hit Die breath \\\hline
Movement: & 40' \\\hline
No. Appearing: & 2d4, Wild 2d4, Lair 2d4 \\\hline
Save As: & Fighter: 3 to 7 (same as Hit Dice) \\\hline
Morale: & 9 \\\hline
Treasure Type: & C \\\hline
XP: & 3 HD 205, 4 HD 320, 5 HD 450, 6 HD 610, 7 HD 800 \\\hline
\end{tabularx}\medskip

Hellhounds are large canine creatures sheathed in hellish flame; they range in size from 3 to 5 feet at the shoulder (3 feet for a 3 hit die monster, plus ½ foot for each additional hit die) with a weight ranging from 100 to 200 pounds (100 pounds at 3 hit dice plus 25 pounds per additional hit die). These monsters are native to another plane where they hunt in packs; sometimes powerful wizards or evil priests summon them for use as watchdogs. In addition to biting, each hellhound may breathe fire a number of times per day equal to its hit dice. In combat, one-third of the time (1-2 on 1d6) a hellhound will choose to breathe fire; otherwise it will attempt to bite. Roll each round to determine which attack form will be used.

A hellhound' s breath weapon is a cone of flame 10' wide at the far end which is 10' long for those with 3 or 4 hit dice, 20' long for those with 5 or 6 hit dice, and 30' long for the largest hellhounds. This breath weapon does 1d6 points of damage per each hit die of the hellhound to all within the area of effect; a successful saving throw vs. Dragon Breath reduces damage to half normal. 

\textbf{Note} that hellhounds vary with regard to the number of hit dice each has. If generating a group randomly, roll 1d6+1 for the hit dice of each, reading a total of 2 as 3. A hellhound has an Armor Class equal to 11 plus its hit dice.


\begin{center}
	\includegraphics[width=0.47\textwidth]{Pictures132/10000000000003CF000003ACD3EB832E150303D9.png}
\end{center}

\subsection*{Hippogriff}\index{Hippogriff}\label{hippogriff}

\begin{tabularx}{0.48\textwidth}{@{}lX@{}}
Armor Class: & 15 \\\hline
Hit Dice: & 3 \\\hline
No. of Attacks: & 2 claws, 1 bite \\\hline
Damage: & 1d6 claw, 1d10 bite \\\hline
Movement: & 60' (10') Fly
120' (10') \\\hline
No. Appearing: & Wild 2d8 \\\hline
Save As: & Fighter: 3 \\\hline
Morale: & 8 \\\hline
Treasure Type: & None \\\hline
XP: & 145 \\\hline
\end{tabularx}\medskip

Hippogriffs resemble large flying horses with the forefront of a bird of prey; they have wingspans of around 20 feet and an overall body length up to 9 feet, and weigh 900 to 1,200 pounds. They are found in a variety of colors and patterns, including white, black, tan, brown, and reddish brown with markings such as blazes, stripes, stars, "bald" faces, and so on.

A hippogriff avoids the territories and civilizations of other creatures, dwelling in extreme altitudes. \textbf{Griffons} sometimes prey upon them, and hippogriffs will generally attack griffons on sight if they have a numerical advantage.

Hippogriffs are omnivorous, entering combat only as defense, save for those times a griffon is met. They are prized as flying mounts since, unlike griffons, they are relatively safe around horses; note that it is still necessary to raise one in captivity in order to use it as a mount. A light load for a hippogriff is up to 400 pounds; a heavy load, up to 900 pounds.

\subsection*{Hobgoblin}\index{Hobgob}\label{hobgoblin}

\begin{tabularx}{0.50\textwidth}{@{}lX@{}}
Armor Class: & 14 (11) \\\hline
Hit Dice: & 1 \\\hline
No. of Attacks: & 1 weapon \\\hline
Damage: & 1d8 or by weapon \\\hline
Movement: & 30' Unarmored 40' \\\hline
No. Appearing: & 1d6, Wild 2d4, Lair 4d8 \\\hline
Save As: & Fighter: 1 \\\hline
Morale: & 8 \\\hline
Treasure Type: & Q, R each; D, K in lair \\\hline
XP: & 25 \\\hline
\end{tabularx}

Hobgoblins are, basically, man-sized goblins. They are better organized than their smaller kin, and are also better adapted to life in the sun. Their skin tone ranges from a greenish tan to a grayish brown, and their hair and eyes are usually dark. Most wear toughened hides and carry wooden shields for armor. As with most goblinoids, they have Darkvision with a 30' range.

\begin{wrapfigure}{r}{0.25\textwidth}
	\includegraphics[width=0.25\textwidth]{Pictures132/100000000000036C00000854C489D4D92DD04B97.png}
\end{wrapfigure}

Tribes of hobgoblins prefer to live in walled villages, or preferably in castles, and are quite willing to overrun villages or castles built by other races rather than build their own. Some tribes do choose to live underground. Hobgoblins have a well-known hatred of elves, and will attack them whenever they think they can win.

One out of every six hobgoblins will be a warrior of 3 Hit Dice (145 XP). Regular hobgoblins gain a +1 bonus to their morale if they are led by a warrior. In hobgoblin lairs, one out of every twelve will be a chieftain of 5 Hit Dice (360 XP) in chainmail with an Armor Class of 15 (11) and a movement of 20', having a +1 bonus to damage due to strength. In lairs of 30 or more there will be a hobgoblin king of 7 Hit Dice (670 XP), with a shield for an Armor Class of 16 (11) having a +2 bonus to damage. In the lair, hobgoblins never fail a morale check as long as the king is alive. In addition, a lair has a chance qual to 1-2 on 1d6 of a shaman being present (or 1-3 on 1d6 if a hobgoblin king is present), and 1 on 1d6 of a witch or warlock. A shaman is equivalent to a hobgoblin warrior statistically, but has Clerical abilities at level 1d6+1. A witch or warlock is equivalent to a regular hobgoblin, but has Magic-User abilities of level 1d6.


\subsection*{Hydra}\index{Hydra}\label{hydra}
	
\begin{tabularx}{0.50\textwidth}{@{}lX@{}}
Armor Class: & 16 to 23 \\\hline
Hit Dice: & 5 to 12 (+10) \\\hline
No. of Attacks: & 5 to 12 bites \\\hline
Damage: & 1d10 per bite \\\hline
Movement: & 40' (10') \\\hline
No. Appearing: & 1, Wild 1, Lair 1 \\\hline
Save As: & Fighter: 5 to 12 \\\hline
Morale: & 9 \\\hline
Treasure Type: & B \\\hline
XP: & 5 HD 360, 6 HD 500, 7 HD 670, 8 HD 875, 9 HD 1,075, 10 HD 1,300,
11 HD 1,575, 12 HD 1,875 \\\hline
\end{tabularx}\medskip

Hydras are reptilian creatures with multiple heads. They are bad-tempered and territorial, but not particularly cunning. 

The Armor Class and Hit Dice of a hydra are keyed to the number of heads; specifically, a hydra has a number of Hit Dice exactly equal to the number of heads, and an Armor Class equal to the number of heads plus 11.

A hydra may be slain by damage in the normal fashion; however, most who fight them choose to strike at their heads. If a character using a melee weapon chooses to strike at a particular head and succeeds in doing 8 points of damage, that head is disabled (severed or severely damaged) and will not be able to attack anymore. Such damage also applies to the monster' s total hit points, of course.

\begin{center}
	\includegraphics[width=0.47\textwidth]{Pictures132/10000000000003D80000050EECC677C2B6BC9FAA.png}
\end{center}

Some hydras live in the ocean; use the given movement as a swimming rate rather than walking in this case. A very few hydras can breathe fire; those that have this ability can emit a flame 10' wide and 20' long one time per head per day. This attack will be used about one time in three (1-2 on 1d6) if it is available; roll for each head which is attacking. Each such attack does 3d6 damage, with a save vs. Dragon Breath reducing the amount by half. Note that the XP value of a fire-breathing hydra should be higher; treat them as if they have a single asterisk (i.e. add the special ability bonus to the base XP for the monster).


\subsection*{Hyena (and Hyenodon)}\index{Hyena (and Hyenodon)}\label{hyena-and-hyenodon}

\begin{tabularx}{0.50\textwidth}{@{}llX@{}}
& Hyena & Hyenodon \\\hline
Armor Class: & 13 & 13 \\\hline
Hit Dice: & 2+1 & 3+1 \\\hline
No. of Attacks: & 1 bite & 1 bite \\\hline
Damage: & 1d6 bite & 1d8 bite \\\hline
Movement: & 60' & 40' \\\hline
No. Appearing: & 1d8 & 1d6 Wild/Lair 1d8 \\\hline
Save As: & Fighter: 2 & Fighter: 3 \\\hline
Morale: & 8 & 8 \\\hline
Treasure Type: & None & None \\\hline
XP: & 75 & 145 \\\hline
\end{tabularx}\medskip

Hyenas are doglike carnivores who exhibit some of the behaviors of canines but are not related. They not only hunt but also scavenge and steal meals. A hungry hyena will chew on anything that is even remotely tainted by blood, meat or other food traces. They will mostly be found in the same savanna-like environments where lions and zebras may be found. They can live in clans of up to a hundred individuals, though smaller groups are more common. They are among the favorite pets of gnolls, who may take them into regions where they are not normally found.

\textbf{Hyenodon }refers to ancient four-legged predators whose name means "hyena tooth." While they are not technically hyenas, the statistics given may be used for the giant prehistoric hyena varieties as well; likewise, some varieties of hyenodon were smaller, and the statistics for standard hyenas may be used for them.

\subsection*{Insect Swarm}\index{Insect Swarm}\label{insect-swarm}

\begin{tabularx}{0.50\textwidth}{@{}lX@{}}
Armor Class: & Immune to normal weapons, including most magical types \\\hline
Hit Dice: & 2* to 4* \\\hline
No. of Attacks: & 1 swarm \\\hline
Damage: & 1d3 (double against no armor) \\\hline
Movement: & 10' Fly 20' \\\hline
No. Appearing: & 1 swarm, Wild 1d3 swarms \\\hline
Save As: & N/A \\\hline
Morale: & 11 \\\hline
Treasure Type: & None \\\hline
XP: & 2 HD 100, 3 HD 175, 4 HD 280 \\\hline
\end{tabularx}\medskip

An insect swarm is not a single creature; rather, it is a large group of ordinary flying or crawling insects moving as a unit. In general, a swarm fills a volume equal to three 10' cubes, though it is possible for a swarm to become more compact in order to move through a small doorway or narrow corridor. If the swarm consists of crawling insects, it covers three 10' squares and the flying movement above is ignored.

Any living creature within the volume or area of the swarm suffers 1d3 points of damage each round. Damage rolls are doubled if the victim is unarmored (for creatures which do not wear armor, any creature having less than AC 15 is considered unarmored). 

Damage is reduced to a single point per round for three rounds if the character manages to exit the swarm. It is possible to "ward off" the insects by swinging a weapon, shield, or other similar-sized object around, and in this case damage is likewise reduced to 1 point per round. If a lit torch is used in this way, the swarm takes 1d4 points of damage per round. Weapons, even magic weapons, do not harm an insect swarm. An entire swarm can be affected by a \textbf{sleep} spell. Smoke can be used to drive a swarm away (if the swarm moves away from any victims due to smoke, the damage stops immediately). Finally, a victim who dives into water will take damage for only one more round.

\subsection*{Invisible Stalker}\index{Invisible Stalker}\label{invisible-stalker}

\begin{tabularx}{0.50\textwidth}{@{}lX@{}}
Armor Class: & 19 \\\hline
Hit Dice: & 8* \\\hline
No. of Attacks: & 1 \\\hline
Damage: & 4d4 \\\hline
Movement: & 40' \\\hline
No. Appearing: & 1 (special) \\\hline
Save As: & Fighter: 8 \\\hline
Morale: & 12 \\\hline
Treasure Type: & None \\\hline
XP: & 945 \\\hline
\end{tabularx}

Invisible stalkers are monsters from another plane of existence who may be summoned to slay the enemies of a wizard or to perform some other simple task. They are naturally invisible, and remain so even after attacking. 

A summoned invisible stalker does whatever the summoner commands, even if the task takes days or weeks to perform. The creature is compelled to complete the task regardless of time required. Invisible stalkers don' t like tasks that take too long or are too complicated, and if assigned an unwanted task will try to find a loophole or otherwise subvert the summoner' s command.

An invisible stalker' s form is amorphous, such that a \textbf{detect invisible} spell shows only an undulating blob-shaped outline. Don' t forget to apply the standard penalty of -4 on the attack die when an invisible stalker is attacked by a creature which is unable to see it.


\begin{center}
	\includegraphics[width=0.47\textwidth]{Pictures132/10000000000003D8000005C4EB7612AD06AEF697.png}
\end{center}

\subsection*{Ironbane*}\index{Ironbane}\label{ironbane}

\begin{tabularx}{0.50\textwidth}{@{}lX@{}}
Armor Class: & 15 \\\hline
Hit Dice: & 3* \\\hline
No. of Attacks: & 1 touch \\\hline
Damage: & special \\\hline
Movement: & 50' \\\hline
No. Appearing: & 1d4 \\\hline
Save As: & Fighter: 4 \\\hline
Morale: & 8 \\\hline
Treasure Type: & None \\\hline
XP: & 175 \\\hline
\end{tabularx}\medskip

An ironbane resembles a large armadillo in its overall body plan, but has an anteater-like snout with a long flicking tongue, and long, strangely hare-like back legs which allow it to hop from place to place. When attacking or pursuing, the ironbane stands up on its hind legs, but when resting or moving slowly it folds them and walks on all four feet.

Like the more common rust monster (as found on page \hyperlink{rust-monster}{\pageref{rust-monster}}), the touch of any part of an ironbane' s body transforms metal objects into rust (or verdigris, or other oxides as appropriate). Non-magical metal attacked by an ironbane, or that touches the monster (such as a sword used to attack it), is instantly ruined. A non-magical metal weapon used to attack the monster does half damage before being destroyed. Magic weapons or armor lose one "plus" each time they make contact with the ironbane; this loss is permanent.

The metal oxides created by this monster are its food; thus, a substantial amount of metal dropped in its path may cause it to cease pursuit of metal-armored characters. Use a morale check to determine this.  Metals that do not normally oxidize, such as gold, are of no interest to an ironbane and will be ignored. Silver and copper on the other hand are candy for this creature and one will pursue the tastiest-smelling adventurer in any party.


\begin{center}
	\includegraphics[width=0.47\textwidth]{Pictures132/10000000000003D80000031EAB555E33C43B2F3A.png}
\end{center}


\subsection*{Jaguar}\index{Jaguar}\label{jaguar}

\begin{tabularx}{0.50\textwidth}{@{}lX@{}}
Armor Class: & 16 \\\hline
Hit Dice: & 4 \\\hline
No. of Attacks: & 2 claws, 1 bite \\\hline
Damage: & 1d4 claw, 2d4 bite \\\hline
Movement: & 70' Swim 30' \\\hline
No. Appearing: & 1d2, Wild 1d6 \\\hline
Save As: & Fighter: 4 \\\hline
Morale: & 8 \\\hline
Treasure Type: & None \\\hline
XP: & 240 \\\hline
\end{tabularx}\medskip

These great cats are about 8 to 9 feet long (from nose to tail-tip) and weigh about 165 pounds. Unlike other great cats, they enjoy swimming and often hunt near rivers or lakes. Jaguars kill with their powerful bite, preferring to deliver a fatal wound to the skull of their prey. 

\subsection*{Jelly}\index{Jelly}\label{jelly}

Jellies are strange creatures made of amorphous protoplasm. They are similar to tiny single-celled creatures such as a few wizards may have studied using magic, but far larger. Jellies are always completely non-intelligent, and are thus immune to \textbf{sleep }or \textbf{charm }magic as well as any form or \textbf{mind reading }or telepathy. Generally they also do not check morale, but simply move toward any potential meal and attack automatically.

\subsection*{Jelly, Black* (Black Pudding)}\index{Jelly, Black* (Black Pudding)}\label{jelly-black-black-pudding}

\begin{tabularx}{0.50\textwidth}{@{}lX@{}}
Armor Class: & 14 \\\hline
Hit Dice: & 10* (+9) \\\hline
No. of Attacks: & 1 pseudopod \\\hline
Damage: & 3d8 \\\hline
Movement: & 20' \\\hline
No. Appearing: & 1 \\\hline
Save As: & Fighter: 10 \\\hline
Morale: & 12 \\\hline
Treasure Type: & None \\\hline
XP: & 1,390 \\\hline
\end{tabularx}\medskip

Black jellies are amorphous creatures that live only to eat. They inhabit underground areas throughout the world, scouring caverns, ruins, and dungeons in search of organic matter, living or dead. They attack any creatures they encounter, lashing out with pseudopods or simply engulfing opponents with their bodies, which secrete acids that help them catch and digest their prey.

If attacked with normal or magical weapons, or with lightning or electricity, a black jelly suffers no injury, but will be split into two jellies; the GM should divide the original creature' s hit dice between the two however they see fit, with the limitation that neither pudding may have less than two hit dice. A two hit die black jelly is simply unharmed by such attacks, but cannot be split further.

Cold or ice based attacks do not harm a black jelly, but such an attack will paralyze the jelly for one round per die of damage the attack would normally cause. Other attack forms will affect a black jelly normally; the preferred method of killing one usually involves fire. 

The typical black jelly measures 10 feet across and 2 feet thick, and weighs about 10,000 pounds. Black jellies of smaller sizes may be encountered, possibly as a result of the splitting described above. 

\begin{center} \includegraphics[width=0.47\textwidth]{Pictures132/10000000000003CF00000401E9F9820F423BADB5.png} \end{center}

\subsection*{Jelly, Glass (Gelatinous Cube)}\index{Jelly, Glass (Gelatinous Cube)}\label{jelly-glass-gelatinous-cube}

\begin{tabularx}{0.50\textwidth}{@{}lX@{}}
Armor Class: & 12 \\\hline
Hit Dice: & 4* \\\hline
No. of Attacks: & 1 \\\hline
Damage: & 2d4 + paralysis \\\hline
Movement: & 20' \\\hline
No. Appearing: & 1 \\\hline
Save As: & Fighter: 2 \\\hline
Morale: & 12 \\\hline
Treasure Type: & V \\\hline
XP: & 280 \\\hline
\end{tabularx}

The
glass jelly travels slowly along dungeon corridors and cave floors, absorbing carrion, creatures, and trash. Inorganic material remains trapped and visible inside the jelly's body. Glass jellies are huge, averaging 1,000 cubic feet and weighing as much as 20,000 pounds. The body of a glass jelly is more viscous than that of other jellies, and it tends to take on the shape of its surroundings such that one in a dungeon with 10' wide corridors might have a shape similar to that of a cube, while one living in round sewer drains under a city would have a cylindrical or half-cylindrical shape.

If a glass jelly encounters a narrow place (a doorway, for example) it will need 2d4 rounds to push through the constriction and will have assumed a new shape after doing so; such a new shape will persist for twice as long as it took to pass the constriction before the monster begins to flow into a shape that fills the available space again. 

Glass jellies move quietly, making a faint sucking or slurping sound if anyone thinks to listen. Combined with their transparency, they are able to surprise prey on a roll of 1-3 on 1d6. Worse, any living creature hit by a glass jelly must save vs. Paralysis or be paralyzed for 2d4 turns in addition to suffering damage from its acid secretions.

Any treasure indicated will be visible inside the creature, which must be slain if the treasure is to be recovered.

\begin{center} \includegraphics[width=0.47\textwidth]{Pictures132/10000000000003CF000003346F31675C7341E9F9.jpg} \end{center}


\subsection*{Jelly, Gray (Gray Ooze)}\index{Jelly, Gray (Gray Ooze)}\label{jelly-gray-gray-ooze}

\begin{tabularx}{0.50\textwidth}{@{}lX@{}}
Armor Class: & 12 \\\hline
Hit Dice: & 3*  \\\hline
No. of Attacks: & 1 pseudopod \\\hline
Damage: & 2d8 \\\hline
Movement: & 1' \\\hline
No. Appearing: & 1 \\\hline
Save As: & Fighter: 3 \\\hline
Morale: & 12 \\\hline
Treasure Type: & None \\\hline
XP: & 175 \\\hline
\end{tabularx}\medskip

Gray jellies are amorphous creatures that live only to eat. They inhabit underground areas, scouring caverns, ruins, and dungeons in search of organic matter, living or dead. Average individuals will be up to 10 feet in diameter, about 6 inches thick (high), and weigh up to 2,500 pounds.

The acid secretions of the gray jelly can dissolve most organic matter and most metals; stone and glass are not affected, however. After a successful hit, the jelly will stick to the creature attacked, dealing 2d8 damage per round automatically. Normal (non-magical) armor or clothing dissolves and becomes useless immediately. Any non-magical weapon made of metal or wood which hits a gray jelly will be similarly destroyed. Magical weapons, armor, and clothing are allowed a saving throw (use the wearer' s save vs. Death Ray, adding any magical "plus" value to the roll if applicable).

\begin{center} \includegraphics[width=0.47\textwidth]{Pictures132/10000000000003BF000001BFDBCA924AD68362B7.png} \end{center}


\subsection*{Jelly, Green* (Green Slime)}\index{Jelly, Green* (Green Slime)}\label{jelly-green-green-slime}

\begin{tabularx}{0.50\textwidth}{@{}lX@{}}
Armor Class: & 12 (only hit by fire or cold) \\\hline
Hit Dice: & 2** \\\hline
No. of Attacks: & 1 special \\\hline
Damage: & special \\\hline
Movement: & 1' \\\hline
No. Appearing: & 1 \\\hline
Save As: & Fighter: 2 \\\hline
Morale: & 12 \\\hline
Treasure Type: & None \\\hline
XP: & 125 \\\hline
\end{tabularx}\medskip

Green jelly devours flesh and organic materials on contact and is even capable of dissolving metal given enough time. Bright green, wet, and sticky, it clings to walls, floors, and ceilings in patches, reproducing as it consumes organic matter. It drops from walls and ceilings when it detects movement (and possible food) below. Green jelly cannot grow in sunlight; even the indirect sunlight of a dense forest will stunt it and prevent it from spreading, and direct sunlight will kill it outright within a turn.

On the first round of contact, the jelly can be scraped off a creature (most likely destroying the scraping device), but after that it must be frozen, burned, or cut away (dealing the same damage to both the victim and the jelly). A \textbf{cure disease} spell will destroy a patch of green jelly. It does not harm stone or enchanted metal, but can dissolve normal metal or enchanted wood in a turn and normal wood in 2d4 rounds.

If not destroyed or scraped off within 6+1d4 rounds, the victim will be completely transformed into more green jelly; such a character or creature cannot be retrieved by any magic short of a \textbf{wish}.

\begin{center} \includegraphics[width=0.47\textwidth]{Pictures132/10000000000004A80000042D4673A24196038FF7.jpg} \end{center}


\subsection*{Jelly, Ochre*}\index{Jelly, Ochre}\label{jelly-ochre}

\begin{tabularx}{0.50\textwidth}{@{}lX@{}}
Armor Class: & 12 (only hit by fire or cold) \\\hline
Hit Dice: & 5* \\\hline
No. of Attacks: & 1 pseudopod \\\hline
Damage: & 2d6 \\\hline
Movement: & 10' \\\hline
No. Appearing: & 1 \\\hline
Save As: & Fighter: 5 \\\hline
Morale: & 12 \\\hline
Treasure Type: & None \\\hline
XP: & 405 \\\hline
\end{tabularx}\medskip

Ochre jellies are yellowish-brown amorphous monsters. Average individuals will be up to 10 feet in diameter, about 6 inches thick (high), and weigh up to 2,500 pounds.

Ochre jellies can only be hit (damaged) by fire or cold. Attacks with weapons or electricity/lightning cause the creature to divide into 1d4+1 smaller jellies of 2 hit dice apiece. If divided, the resulting smaller jellies do 1d6 points of damage with each hit. Other attack forms simply have no effect on the monster.

\subsection*{Kobold}\index{Kobold}\label{kobold}

\begin{tabularx}{0.50\textwidth}{@{}lX@{}}
Armor Class: & 13 (11) \\\hline
Hit Dice: & ½ (1d4 hit points) \\\hline
No. of Attacks: & 1 weapon \\\hline
Damage: & 1d4 or by weapon \\\hline
Movement: & 20' Unarmored 30' \\\hline
No. Appearing: & 4d4, Wild 6d10, Lair 6d10 \\\hline
Save As: & Normal Man \\\hline
Morale: & 6 \\\hline
Treasure Type: & P, Q each; C in lair \\\hline
XP: & 10 \\\hline
\end{tabularx}\medskip

Kobolds are diminutive, dog-faced humanoids with reptilian skin. They stand between 2½ and 3½ feet tall and typically weigh around 40 pounds. They are cowardly by nature, and in combat they prefer ambushing their foes and constructing traps both clever and deadly; to simply stand and fight is considered the worst possible strategy, and kobolds may fake withdrawing in fear in order to draw enemies into a trap.

Kobolds prefer to rely on ranged combat whenever possible. They tend to favor daggers and short spears since both can be used in close quarters or thrown at foes at range. Some may be armed with diminutive bows or crossbows which inflict 1d4 points of damage on a hit. 

Kobolds normally dwell underground, while those who choose to dwell above ground may adopt a nocturnal lifestyle. All possess Darkvision with a range of 60 feet, and generally avoid exposure to daylight. When they are exposed to sunlight or the effect of \textbf{continual light }they suffer a penalty of -1 to all attack rolls. Most kobolds wear leather armor in combat.\\


\begin{center}
	 \includegraphics[width=0.47\textwidth]{Pictures132/10000001000003D80000039F8722EF55EC7BB9BB.png}
\end{center} \medskip

One out of every six kobolds will be a warrior with 1 hit die (25 XP). Kobolds gain a +1 bonus to their morale if they are led by a warrior. In kobold lairs, one out of every twelve will be a chieftain of 2 hit dice (75 XP) with an armor class of 14 (11) and having a +1 bonus to damage due to strength. In lairs of 30 or greater, there will be a kobold king of 3 hit dice (145 XP) who wears chain mail with an armor class of 15 (11) and has a movement rate of 10' and a +1 bonus to damage. In the lair, kobolds never fail a morale check as long as the kobold king is alive. In addition, a lair has a chance equal to 1 on 1d6 of a shaman being present (or 1-2 on 1d6 if a kobold king is present). A shaman is equivalent to a 1 hit die warrior kobold statistically, but has Clerical abilities at level 1d4+1. For XP purposes, treat the shaman as if it has a number of hit dice equal to its clerical level -1, and assign one special ability bonus asterisk.

Kobolds are sometimes confused with barklings, for whom they have a particular hatred; calling a kobold a barkling or suggesting that the two species are related is considered a terrible insult.


\subsection*{Leech, Giant}\index{Leech, Giant}\label{leech-giant}


\begin{center} \includegraphics[width=0.47\textwidth]{Pictures132/10000000000003D800000558F9D08DA9D7AA56C3.png} \end{center}

\begin{flushleft}
	\begin{tabularx}{0.50\textwidth}{@{}lX@{}}
Armor Class: & 17 \\\hline
Hit Dice: & 6 \\\hline
No. of Attacks: & 1 bite + hold \\\hline
Damage: & 1d6 bite + 1d6/round hold \\\hline
Movement: & 30' \\\hline
No. Appearing: & Wild 1d4 \\\hline
Save As: & Fighter: 6 \\\hline
Morale: & 10 \\\hline
Treasure Type: & None \\\hline
XP: & 500 \\\hline
\end{tabularx}\medskip
\end{flushleft}

Giant leeches are slimy, segmented wormlike creatures which live in water. Salt or fresh, clean or stagnant, there are giant leech varieties for all wet environments. However, only a true leech expert can tell thevarious types apart. An average giant leech will be 4 to 6 feet long.

Once a giant leech hits in combat, it attaches to the victim and sucks blood, causing an additional 1d6 points of damage each round until the victim or the leech is dead. There is no way to remove the leech other than to kill it.


\subsection*{Leopard (Panther)}\index{Leopard (Panther)}\label{leopard-panther}

\begin{tabularx}{0.50\textwidth}{@{}lX@{}}
Armor Class: & 16 \\\hline
Hit Dice: & 4 \\\hline
No. of Attacks: & 2 claws, 1 bite \\\hline
Damage: & 1d4 claw, 2d4 bite \\\hline
Movement: & 60' (10') \\\hline
No. Appearing: & 1, Wild 1d4 \\\hline
Save As: & Fighter: 4 \\\hline
Morale: & 8 \\\hline
Treasure Type: & None \\\hline
XP: & 240 \\\hline
\end{tabularx}\medskip

Leopards are large cats that are 7 to 8 feet long including tail, and weigh about 175 lb. These cats have four black spots that form a large circle (rosette). The center of the circle and around the rosette spots range from yellow tan to white on the underside. An all-black leopard is called a Panther. The tail has a white tip. They don't like to swim as much as jaguars, preferring instead to climb trees. A leopard will drag prey into a tree.

\subsection*{Lion}\index{Lion}\label{lion}

\begin{tabularx}{0.50\textwidth}{@{}lX@{}}
Armor Class: & 14 \\\hline
Hit Dice: & 5 \\\hline
No. of Attacks: & 2 claws, 1 bite \\\hline
Damage: & 1d6 claw, 1d10 bite \\\hline
Movement: & 50' \\\hline
No. Appearing: & Wild 1d8 \\\hline
Save As: & Fighter: 5 \\\hline
Morale: & 9 \\\hline
Treasure Type: & None \\\hline
XP: & 360 \\\hline
\end{tabularx}\medskip

The lion is a large cat found in grasslands and savannas. They have muscular, broad-chested bodies, short, rounded heads, round ears, and a hairy tuft at the end of its tail. Adult male lions are larger than females, and in most varieties have a prominent mane. They live in social groups called prides consisting of a few adult males, several related females, and their cubs.

Groups of female lions usually hunt together, preying mostly on large ungulates; any hunting group of 1 or 2 is 80\% likely to be females, and any group of 3 or more almost certainly are. Sometimes male lions are rogues, living outside a pride, and in those cases they may hunt as individuals or in pairs.

Lions typically do not actively seek out and prey on humans. Injured or obviously vulnerable humans may be too much of a temptation though.

\subsection*{Living Statue}\index{Living Statue}\label{living-statue}

Living statues are magically animated. They are true automatons, unlike golems, which are animated by elemental spirits. While this means that living statues have no chance of going "berserk," it also means that they may only perform simple programmed activities. They may not be commanded in any meaningful fashion. They make very effective guards for tombs, treasure rooms, and similar places.

Living statues can be crafted to resemble any sort of living creature, but most commonly are made to look like the type of humanoid which crafted them.

\vfill

\begin{center} \includegraphics[width=0.47\textwidth]{Pictures132/10000000000003D80000063ED4C6400067714C16.png} \end{center}

\columnbreak


\subsection*{Living Statue, Crystal}\index{Living Statue, Crystal}\label{living-statue-crystal}

\begin{tabularx}{0.50\textwidth}{@{}lX@{}}
Armor Class: & 16 \\\hline
Hit Dice: & 3 \\\hline
No. of Attacks: & 2 fists \\\hline
Damage: & 1d6 fist \\\hline
Movement: & 30' \\\hline
No. Appearing: & 1d6 \\\hline
Save As: & Fighter: 3 \\\hline
Morale: & 12 \\\hline
Treasure Type: & None \\\hline
XP: & 145 \\\hline
\end{tabularx}\medskip

Crystal living statues have no particular special powers, unlike those made of iron or stone.

\subsection*{Living Statue, Iron}\index{Living Statue, Iron}\label{living-statue-iron}

\begin{tabularx}{0.50\textwidth}{@{}lX@{}}
Armor Class: & 18 \\\hline
Hit Dice: & 4* \\\hline
No. of Attacks: & 2 fists \\\hline
Damage: & 1d8 fist + special \\\hline
Movement: & 10' \\\hline
No. Appearing: & 1d4 \\\hline
Save As: & Fighter: 4 \\\hline
Morale: & 12 \\\hline
Treasure Type: & None \\\hline
XP: & 280 \\\hline
\end{tabularx}\medskip

If struck by a non-magical metal (even partially metal) weapon, the wielder must make a successful save vs. Spells or the weapon will become stuck in the monster. If this happens, it cannot be removed until the statue is "killed."

\subsection*{Living Statue, Stone}\index{Living Statue, Stone}\label{living-statue-stone}

\begin{tabularx}{0.50\textwidth}{@{}lX@{}}
Armor Class: & 16 \\\hline
Hit Dice: & 5* \\\hline
No. of Attacks: & 2 lava sprays \\\hline
Damage: & 2d6 lava spray \\\hline
Movement: & 20' \\\hline
No. Appearing: & 1d3 \\\hline
Save As: & Fighter: 5 \\\hline
Morale: & 12 \\\hline
Treasure Type: & None \\\hline
XP: & 405 \\\hline
\end{tabularx}\medskip

A stone living statue attacks by spraying molten rock from its fingertips. The range of the spray is 5'.

\subsection*{Lizard, Giant Draco}\index{Lizard, Giant Draco}\label{lizard-giant-draco}

\begin{tabularx}{0.50\textwidth}{@{}lX@{}}
Armor Class: & 15 \\\hline
Hit Dice: & 4+2 \\\hline
No. of Attacks: & 1 bite \\\hline
Damage: & 1d10 \\\hline
Movement: & 40' Fly 70'
(20', and see below) \\\hline
No. Appearing: & 1d4, Wild 1d8 \\\hline
Save As: & Fighter: 3 \\\hline
Morale: & 7 \\\hline
Treasure Type: & None \\\hline
XP: & 240 \\\hline
\end{tabularx}\medskip

Giant draco lizards are able to extend their ribs and connected skin to
form a sort of wing, allowing them to fly for short distances (no more
than three rounds, and ascending is impossible). An average giant draco
lizard is 8' long, including its nearly
3' long tail.

\begin{center} \includegraphics[width=0.47\textwidth]{Pictures132/10000000000003D80000023290B50114FD83AE9A.png} \end{center}


\subsection*{Lizard, Giant Gecko}\index{Lizard, Giant Gecko}\label{lizard-giant-gecko}

\begin{tabularx}{0.50\textwidth}{@{}lX@{}}
Armor Class: & 15 \\\hline
Hit Dice: & 3+1 \\\hline
No. of Attacks: & 1 bite \\\hline
Damage: & 1d8 \\\hline
Movement: & 40' (special) \\\hline
No. Appearing: & 1d6, Wild 1d10 \\\hline
Save As: & Fighter: 2 \\\hline
Morale: & 7 \\\hline
Treasure Type: & None \\\hline
XP: & 145 \\\hline
\end{tabularx}\medskip

Giant gecko lizards range from 4' to 6' in length, and are generally green in color, though grey or white versions can be found underground. They can climb walls and even walk across ceilings at full movement rate due to their specialized toe pads. They are carnivores, typically attacking weaker prey from above.

\subsection*{Lizard, Giant Horned Chameleon}\index{Lizard, Giant Horned Chameleon}\label{lizard-giant-horned-chameleon}

\begin{tabularx}{0.50\textwidth}{@{}lX@{}}
Armor Class: & 18 \\\hline
Hit Dice: & 5 \\\hline
No. of Attacks: & 1 tongue or 1 bite \\\hline
Damage: & tongue grab or 2d6 bite \\\hline
Movement: & 40' (10') \\\hline
No. Appearing: & 1d3, Wild 1d6 \\\hline
Save As: & Fighter: 4 \\\hline
Morale: & 7 \\\hline
Treasure Type: & None \\\hline
XP: & 360 \\\hline
\end{tabularx}\medskip

Giant horned chameleons average 8' to 10' in length. They are typically green, but can change color to blend into their surroundings, allowing them to surprise prey on 1-4 on 1d6. Giant horned chameleon have very long tongues, able to spring out up to 20' forward; the sticky muscular ball on the end grabs on to the chameleon' s prey, and the chameleon then drags the prey to its mouth, doing bite damage automatically on the following round (and all subsequent rounds, until the chameleon is killed or fails a morale check, or until the prey is dead).

The horns of the giant horned chameleon are used only in mating rituals, not in combat.

\subsection*{Lizard, Giant Tuatara}\index{Lizard, Giant Tuatara}\label{lizard-giant-tuatara}

\begin{tabularx}{0.50\textwidth}{@{}lX@{}}
Armor Class: & 16 \\\hline
Hit Dice: & 6 \\\hline
No. of Attacks: & 2 claws, 1 bite \\\hline
Damage: & 1d4 claw, 2d6 bite \\\hline
Movement: & 40' (10') \\\hline
No. Appearing: & 1d2, Wild 1d4 \\\hline
Save As: & Fighter: 5 \\\hline
Morale: & 6 \\\hline
Treasure Type: & None \\\hline
XP: & 500 \\\hline
\end{tabularx}\medskip

Giant tuataras are large, being 10 to 12 feet long, and heavily built. They are predators with a powerful shearing bite. Giant tuataras are more resistant to cold than most lizards, and are thus sometimes found hunting deep underground. They are also known to hibernate in cold weather. Sages argue as to whether or not they are actually members of the lizard family, but the giant tuatara does not care about such things.

\subsection*{Lizard, Monitor}\index{Lizard, Monitor}\label{lizard-monitor}

\begin{tabularx}{0.50\textwidth}{@{}lXXX@{}}
& Large & Huge & Giant \\\hline
Armor Class: & 12 & 14 & 16 \\\hline
Hit Dice: & 3* & 5* & 7* \\\hline
No. of Attacks: &  \multicolumn{3}{c}{1 bite} \\\hline
Damage: & 1d4 + poison & 1d6 + poison & 1d8 + poison \\\hline
Movement: & \multicolumn{3}{c}{40' (special)}\\\hline
No. Appearing: &  \multicolumn{3}{c}{1d4, Wild 1d6}\\\hline
Save As: & Fighter: 3 & Fighter: 5 & Fighter: 7 \\\hline
Morale: & 7 & 8 & 9 \\\hline
Treasure Type: & \multicolumn{3}{c}{None} \\\hline
XP: & 175 & 405 & 735 \\\hline
\end{tabularx}\medskip

Monitor lizards are generally dark in color, but often have bright, colorful, lace-like patterns covering their skin. A monitor may rise up onto its hind legs to run at a rate of 60 feet per round; such movement must be in a straight line, ending with the creature on all fours again. However, unlike a "double move" running movement, the monitor may still attack after moving.

Large monitor lizards range from 4 to 7 feet in length, and include such creatures as the so-called Komodo Dragon. Their venom is slow; those who are bitten must save vs. Poison at +2, with failure resulting in the victim suffering 1d6 points of damage each turn for 2d4 turns. 

Huge monitors range from 8 to 11 feet in length. Their venom works faster than that of their smaller brethren; those bitten must save vs. Poison at +2 or suffer 1d6 points of damage each round for 2d4 rounds. 

Giant monitor lizards range from 12 to 15 feet in length. Those bitten by a giant monitor must save vs. Poison at +2 or die. 

All monitors are carnivores who hunt by running down their prey, and anything smaller than a monitor is considered prey.

\subsection*{Lizard Man}\index{Lizard Man}\label{lizard-man}

\begin{tabularx}{0.50\textwidth}{@{}lXX@{}}
& Common & Subterranean \\\hline
Armor Class: & 15 (12) & 15 \\\hline
Hit Dice: & 2 & 2 \\\hline
No. of Attacks: & 1 weapon & 2 claws, 1 bite \\\hline
Damage: & 1d6+1 or by weapon +1 & 1d4 claw,1d4 bite \\\hline
Movement: & 20' Unarmored 30' Swim 40' (no armor) & 30' Swim 40' \\\hline
No. Appearing: & 2d4, Wild 2d4, Lair 6d6 & 1d8, Lair 5d8 \\\hline
Save As: & Fighter: 2 & Fighter: 2 \\\hline
Morale: & 11 & 9 \\\hline
Treasure Type: & D & D \\\hline
XP: & 75 & 75 \\\hline
\end{tabularx}\medskip

\textbf{Common lizard men }are tall, generally 6 to 7 feet tall at adulthood and weighing up to 250 pounds. Males and females are basically the same size, and it is quite difficult for other races to tell them apart. Due to their great Strength they always receive a +1 to damage done with melee weapons. They wear leather armor and carry shields in battle.

Lizard men are excellent swimmers and can hold their breath for an extended period of time (up to a full turn). They cannot swim while wearing armor; however, they often hide in the water even while armored, standing on the bottom with just nose and eyes exposed (similar to a crocodile). When they are able to employ this maneuver, lizard men surprise on 1-4 on 1d6.

Lizard men are largely indifferent to other races, being primarily interested in their own survival. If aroused, however, they are fearsome warriors, using simple but sound tactics. 

Subterranean lizard men, also called troglodytes, are superficially very similar to the common variety. Their skin is paler, and their eyes are red and seem to glow in low light conditions. Individuals are shorter than the common variety, standing just 5 to 6 feet tall, due in part to their somewhat "hunched" stance. They weigh about as much as the common type.


\begin{center} \includegraphics[width=0.47\textwidth]{Pictures132/10000000000003D80000035B55782BFFA36BB7D2.png} \end{center}


These monsters can change color at will, allowing them to blend into underground environments so well that they gain surprise on a roll of 1-5 on 1d6. Furthermore, they gain a +2 attack bonus during any surprise round due to their excellent ambush skills.

Subterranean lizard men secrete a smelly oil that keeps their scaly skin supple. All mammals (including, of course, all the standard character races) find the scent repulsive, and those within 10 feet of one must make a saving throw versus poison. Those failing the save suffer a -2 penalty to attack rolls while they remain within 10 feet of the creature. Getting out of range negates the penalty, but renewed exposure reinstates the penalty without an additional saving throw. The results of the original save last a full 24 hours, after which a new save must be rolled.

\begin{center} \includegraphics[width=0.47\textwidth]{Pictures132/10000000000003D800000725B324B5B549812A46.png} \end{center}


\subsection*{Lycanthrope*}\index{Lycanthrope}\label{lycanthrope}

Lycanthropes are humans who transform into animals or animal-human hybrid forms; the exact nature of the transformation varies between specific types. They look like ordinary humans when not transformed, though lycanthropes who have been afflicted for a long time sometimes begin to resemble their animal form even when not transformed. An animal form will usually appear larger and stronger than normal animals of the sametype, and some say you can see the intelligence of a human in their  eyes, if you dare to get close enough.

This affliction is in fact a kind of magical disease, though it is not susceptible to the \textbf{cure disease} spell. Any human who loses half or more of their hit points due to lycanthrope bite and/or claw attacks will subsequently contract the same form of lycanthropy in 3d6 days. For non-Human characters or creatures, contracting the disease is fatal in the same time period. A \textbf{cure disease} cast before the onset is complete will stop the progress of the disease, but once the time has elapsed, the transformation is permanent.

When first infected, most lycanthropes cannot control their changes and will transform when stressed or under some other type-specific circumstances. After around two to three years, they gain the ability to change at will, and may attempt to resist involuntary transformation by means of a saving throw vs. Paralysis.

In animal or hybrid form lycanthropes may be hit only by silver or magical weapons. 

\subsection*{Lycanthrope, Werebear*}\index{Lycanthrope, Werebear}\label{lycanthrope-werebear}

\begin{tabularx}{0.50\textwidth}{@{}lX@{}}
Armor Class: & 18 (s) \\\hline
Hit Dice: & 6*  \\\hline
No. of Attacks: & 2 claws, 1 bite + hug \\\hline
Damage: & 2d4 claw, 2d8 bite, 2d8 hug \\\hline
Movement: & 40' \\\hline
No. Appearing: & 1d4, Wild 1d4, Lair 1d4 \\\hline
Save As: & Fighter: 6 \\\hline
Morale: & 10 \\\hline
Treasure Type: & C \\\hline
XP: & 555 \\\hline
\end{tabularx}\medskip

Werebears are humans that can transform into large bears. When in human form, they typically appear as well-muscled, imposing figures, with an abundance of thick hair. Werebears typically dwell in deep forests, far from civilization. They are distrustful of those that they do not know, but will ferociously defend those that they have befriended.

\subsection*{Lycanthrope, Wereboar*}\index{Lycanthrope, Wereboar}\label{lycanthrope-wereboar}

\begin{tabularx}{0.50\textwidth}{@{}lX@{}}
Armor Class: & 16 (s) \\\hline
Hit Dice: & 4* \\\hline
No. of Attacks: & 1 bite \\\hline
Damage: & 2d6 \\\hline
Movement: & 50' Human Form 40' \\\hline
No. Appearing: & 1d4, Wild 2d4, Lair 2d4 \\\hline
Save As: & Fighter: 4 \\\hline
Morale: & 9 \\\hline
Treasure Type: & C \\\hline
XP: & 280 \\\hline
\end{tabularx}\medskip

Wereboars gain full control of their transformations quickly, needing only about a year to achieve mastery. They have just two forms, that of a human and that of a particularly large wild boar or sow. 

In human form they tend to have "piggish" features such as gluttony and cunning, as well as an often strong physical resemblance to swine. Wereboars usually dislike both hard work and responsibility, but they are bullies who enjoy being in charge. They are easily bored with mundane things but are excited by violence (whether witnessing it or participating in it).

\subsection*{Lycanthrope, Wererat*}\index{Lycanthrope, Wererat}\label{lycanthrope-wererat}

\begin{tabularx}{0.50\textwidth}{@{}lX@{}}
Armor Class: & 13 (s) \\\hline
Hit Dice: & 3* \\\hline
No. of Attacks: & 1 bite or 1 weapon \\\hline
Damage: & 1d4 bite or 1d6 or by weapon \\\hline
Movement: & 40' \\\hline
No. Appearing: & 1d8, Wild 2d8, Lair 2d8 \\\hline
Save As: & Fighter: 3 \\\hline
Morale: & 8 \\\hline
Treasure Type: & C \\\hline
XP: & 175 \\\hline
\end{tabularx}\medskip

In addition to assuming the form of a giant rat, wererats can assume a hybrid form. This ratman form shares the animal form' s immunity to normal weapons and can deliver an identical bite, but in this form the wererat may use a normal weapon instead of biting. Note that the wererat in ratman form cannot bite and use a weapon in the same round.

In human form, wererats tend to be skinny, nervous-looking individuals with pointed noses and lank hair. Their hair is almost always a rat-like brown color; those with different hair color who become wererats usually undergo a slow change to this color.

Unlike most lycanthropes, wererats prefer to inhabit civilized areas, particularly cities. They frequently lair in sewers or other underground areas, coming out by night to steal from or kill city folk. The common stereotype of wererats as thieves is not unfounded; in fact, in many cities wererats lead the guild of thieves, sometimes secretly, sometimes not so much.

\subsection*{Lycanthrope, Weretiger*}\index{Lycanthrope, Weretiger}\label{lycanthrope-weretiger}

\begin{tabularx}{0.50\textwidth}{@{}lX@{}}
Armor Class: & 17 (s) \\\hline
Hit Dice: & 5* \\\hline
No. of Attacks: & 2 claws, 1 bite \\\hline
Damage: & 1d6 claw, 2d6 bite \\\hline
Movement: & 50' Human Form 40' \\\hline
No. Appearing: & 1d4, Wild 1d4, Lair 1d4 \\\hline
Save As: & Fighter: 5 \\\hline
Morale: & 9 \\\hline
Treasure Type: & C \\\hline
XP: & 405 \\\hline
\end{tabularx}\medskip

Weretigers are humans that can transform into tigers. In human form, they are generally tall, trim, and very agile. They tend to live and hunt close to human settlements, and are excellent trackers (5 in 6 chance to track prey in either form). Weretigers will typically only attack if provoked. They are capricious and arbitrary to deal with unless whatever offer is made to them is very attractive to them.

\begin{center} \includegraphics[width=0.47\textwidth]{Pictures132/10000001000003D8000002ACA5424ABDA6DD8900.png} \end{center}


\subsection*{Lycanthrope, Werewolf*}\index{Lycanthrope, Werewolf}\label{lycanthrope-werewolf}

\begin{tabularx}{0.50\textwidth}{@{}lX@{}}
Armor Class: & 15 (s) \\\hline
Hit Dice: & 4* \\\hline
No. of Attacks: & 1 bite (or 1 weapon, as given below) \\\hline
Damage: & 2d4 bite, 1d6 or by weapon \\\hline
Movement: & 60' Human Form 40' \\\hline
No. Appearing: & 1d6, Wild 2d6, Lair 2d6 \\\hline
Save As: & Fighter: 4 \\\hline
Morale: & 8 \\\hline
Treasure Type: & C \\\hline
XP: & 280 \\\hline
\end{tabularx}\medskip

Werewolves may be found anywhere humans are found. They are ferocious predators, equally willing to eat animal or human flesh. Unlike most lycanthropes, werewolves have no distinguishing features in human form, making them very hard indeed to identify.

Though they usually have only the usual human and animal forms, there are rumors of some who are also able to assume a wolfman form. In this form they may choose to either bite or use a weapon, and may change back and forth each round. As with the animal form, the wolfman form is hit only by silver or magical weapons.

\begin{center} \includegraphics[width=0.47\textwidth]{Pictures132/10000001000003D8000003B974B30CE33B4FEA0F.png} \end{center}


\subsection*{Mammoth (and Mastodon)}\index{Mammoth (and Mastodon)}\label{mammoth-and-mastodon}

\begin{tabularx}{0.50\textwidth}{@{}lXX@{}}
& Mammoth & Mastodon \\\hline
Armor Class: & 17 & 18 \\\hline
Hit Dice: & 15 (+11) & 13 (+10) \\\hline
No. of Attacks: & -- 2 tusks, 1 trunk grab, 2 tramples -- & \\\hline
Damage: & 3d6 tusk, 2d6 trunk, 2d8 trample & 2d6 tusk, 2d4 trunk, 2d8 trample \\\hline
Movement: & 40' (15') & 50' (15') \\\hline
No. Appearing: & Wild 1d12 & Wild 2d8 \\\hline
Save As: & Fighter: 15 & Fighter: 13 \\\hline
Morale: & 8 & 8 \\\hline
Treasure Type: & special & special \\\hline
XP: & 2,850 & 2,175 \\\hline
\end{tabularx}\medskip

\textbf{Mammoths} are huge, shaggy prehistoric relatives of the elephant. Though found in a variety of climates, they are most common in colder territories.

Like elephants, mammoths have five distinct attack modes (two tusks, a trunk grab, and two tramples with the front feet), but a single individual can apply no more than two of these attacks to any single opponent of small or medium size; large opponents may be targeted by three of these attacks in a round. However, one can attack multiple opponents in its immediate area at the same time.

A light load for a mammoth is 8,500 pounds; a heavy load, up to 17,000 pounds.

A mammoth has no treasure as such, but its tusks are worth 2d6 x 100 gp. 

Mastodons are a related species found in more temperate climates. These prehistoric relatives of the elephant are intelligent and able to communicate with each other in a rudimentary way. They are more aggressive than the common elephant and will attack any creature they see as a threat. 

They have the same basic attack modes as mammoths, with the same limitations as given above.

A light load for a mastodon is 8,000 pounds; a heavy load, up to 16,000 pounds.

A mastodon has no treasure as such, but its tusks are worth 1d10 x 100 gp.

\begin{center} \includegraphics[width=0.47\textwidth]{Pictures132/10000000000007AA0000056CBF65D8FD54ED52CD.png} \end{center}


\subsection*{Manticore}\index{Manticore}\label{manticore}

\begin{tabularx}{0.50\textwidth}{@{}lX@{}}
Armor Class: & 18 \\\hline
Hit Dice: & 6+1* \\\hline
No. of Attacks: & 2 claws, 1 bite or 1d8 spikes (180' range) \\\hline
Damage: & 1d4 claw, 2d4 bite, 1d6 spike \\\hline
Movement: & 40' Fly 60' (10') \\\hline
No. Appearing: & 1d2, Wild 1d4, Lair 1d4 \\\hline
Save As: & Fighter: 6 \\\hline
Morale: & 9 \\\hline
Treasure Type: & D \\\hline
XP: & 555 \\\hline
\end{tabularx}\medskip

Manticores look like an overgrown lion with thick leathery wings and an ugly bearded humanoid face, often like that of a human or dwarf. Their tail ends in an assortment of spikes, which the beast may fire as projectiles; a maximum of 24 are available, and the manticore will launch a random number (1d8, as shown above) each time it chooses to use this attack. The creature will regrow just 1d6 spikes per day after expending them, so they will often delay using them against weaker opponents who they think they can easily dispatch.

An adult manticore is big, with an average weight of 1,000 pounds and alength (not including tail) of around 8 feet.

Manticores are vicious carnivores with a preference for human flesh. They will use their ranged attacks to "soften up" larger or more dangerous-looking prey before closing to melee range.

\subsection*{Medusa}\index{Medusa}\label{medusa}

\begin{tabularx}{0.50\textwidth}{@{}lX@{}}
Armor Class: & 12 \\\hline
Hit Dice: & 4** \\\hline
No. of Attacks: & 1 snakebite, gaze \\\hline
Damage: & 1d6+poison bite, petrification gaze \\\hline
Movement: & 30' \\\hline
No. Appearing: & 1d3, Wild 1d4, Lair 1d4 \\\hline
Save As: & Fighter: 4 \\\hline
Morale: & 8 \\\hline
Treasure Type: & F \\\hline
XP: & 320 \\\hline
\end{tabularx}\medskip

A medusa appears to be a human female with vipers growing from her head instead of hair. The gaze of a medusa will petrify any creature who meets it unless a save vs. Petrify is made. In general, any creature surprised by the medusa will meet its gaze. Those who attempt to fight the monster while averting their eyes suffer penalties of -4 on attack rolls and -2 to AC. It is safe to view a medusa' s reflection in a mirror or other reflective surface; anyone using a mirror to fight a medusa suffers a penalty of -2 to attack and no penalty to AC. If a medusa sees its own reflection, it must save vs. Petrify itself; a petrified medusa is no longer able to petrify others, but the face of a medusa continues to possess the power to petrify even after death otherwise. Medusae instinctively avoid mirrors or other reflective surfaces, even drinking with their eyes closed, but if an attacker can manage to surprise the monster with a mirror she may see her reflection.

Further, the snakes growing from her head are poisonous (save vs. Poison or die in one turn). They attack as a group, not individually, once per round for 1d6 points of damage (plus the poison).

These creatures are well aware that, from the neck down anyway, they are quite visually pleasing to most humanoid males; they will thus often seek to show off their bodies in clinging clothes while wearing veils, hoods, scarves, and so on to hide their true nature. In this way they hope to get closer to potential victims and moreeasily surprise them.

Medusae are shy and reclusive, owing no doubt to the fact that, once the lair of one is found, any humans living nearby will not rest until she is slain. They are hateful creatures, however, and will seek to destroy as many humans as they can without being discovered.

\begin{center} \includegraphics[width=0.40\textwidth]{Pictures132/10000000000003D80000048F6B00BAB2FF62B36A.png} \end{center}


\subsection*{Mermaid}\index{Mermaid}\label{mermaid}

\begin{tabularx}{0.50\textwidth}{@{}lX@{}}
Armor Class: & 12 \\\hline
Hit Dice: & 1* \\\hline
No. of Attacks: & 1 weapon \\\hline
Damage: & 1d6 or by weapon \\\hline
Movement: & Swim 40' \\\hline
No. Appearing: & Wild 1d2 or 3d6 (see below) \\\hline
Save As: & Fighter: 1 \\\hline
Morale: & 8 \\\hline
Treasure Type: & A \\\hline
XP: & 37 \\\hline
\end{tabularx}\medskip

Mermaids have the upper bodies of women and the lower bodies of dolphins. Also called "sirens," mermaids often attempt to lure sailors or other men found near the sea. They accomplish this by means of their enchanting songs.

A mermaid' s song will attract any man within 100 yards, but generally has no effect on women. Men within the area of effect must save vs. Spells to resist, or else they will move toward the mermaid with amorous intent as directly as possible. If two mermaids are singing, apply a penalty of -4 to the save; more than two gives no extra benefit. Affected men will submit to anything the mermaid desires. When she tires of him, he might be freed or slain, depending on the mermaid' s temperament.


\begin{center} \includegraphics[width=0.40\textwidth]{Pictures132/10000000000003D8000007691AF12B442BC3E96E.png} \end{center}

Contrary to popular belief, mermaids are not fish (nor even half fish) and do not breathe water. They can hold their breath for up to an hour of light activity, or two turns (20 minutes) of strenuous action. However, being out of water more than two turns (20 minutes) causes the mermaid 1d4 points of damage per turn.

Mermaids can hear as well as dolphins, and can produce sounds ranging from the lowest frequency a normal human woman can produce up to the highest frequency of a dolphin. This means that mermaids can learn to communicate with dolphins and whales; at least 35\% of mermaids will know the language of one or the other, and 10\% can communicate with any such creature.

Three-quarters of mermaid births are female. Of the quarter which are male, most have legs rather than tails. Such will either be slain or put ashore to be adopted by humans, depending on the temperament of the mother. Mermen (those born with tails) are raised to be subservient to the females. A small mermaid community (3d6 including the male) will often form around such a merman and his mother, who becomes their leader. Such a group is called a pod.

A mermaid with a child will not generally be encountered, as they remain in the deeper parts of the ocean and avoid the attention of men. Pods of mermaids do likewise, and in fact any pod includes 2d4-2 children or juveniles (over and above the number rolled for Number Appearing). Adventurers generally meet mermaids only in groups of one or two.

Mermaids arm themselves with spears or daggers. They hunt fish and harvest kelp for food. Mermaids sometimes possess more than 1 hit die, and about 3\% have some Clerical abilities.

\subsection*{Minotaur}\index{Minotaur}\label{minotaur}


\begin{center} \includegraphics[width=0.40\textwidth]{Pictures132/10000000000003D800000616CFEC4685BDB7AF31.png} \end{center}

\begin{flushleft}
	\begin{tabularx}{0.50\textwidth}{@{}lX@{}}
Armor Class: & 14 (12) \\\hline
Hit Dice: & 6 \\\hline
No. of Attacks: & 1 gore, 1 bite or 1 weapon \\\hline
Damage: & 1d6 gore, 1d6 bite, 1d6+2 or by weapon +2 \\\hline
Movement: & 30' Unarmored 40' \\\hline
No. Appearing: & 1d6, Wild 1d8, Lair 1d8 \\\hline
Save As: & Fighter: 6 \\\hline
Morale: & 11 \\\hline
Treasure Type: & C \\\hline
XP: & 500 \\\hline
\end{tabularx}\medskip
\end{flushleft}

Minotaurs are huge bull-headed humanoid monsters, standing more than 7 feet tall and weighing nearly 500 pounds. Most minotaurs are very aggressive, and fly into a murderous rage if provoked or hungry. Although minotaurs are not especially intelligent, they possess innate cunning and logical ability. They never become lost (even in mazes), and can track enemies with 85\% accuracy. They gain +2 to damage when using melee weapons due to their great Strength. Minotaurs often wear toughened hides for armor. 

\subsection*{Moose}\index{Moose}\label{moose}

See \textbf{Antelope} on page \hyperlink{antelope}{\pageref{antelope}}.

\subsection*{Mountain Lion}\index{Mountain Lion}\label{mountain-lion}

\begin{tabularx}{0.50\textwidth}{@{}lX@{}}
Armor Class: & 14 \\\hline
Hit Dice: & 3+2 \\\hline
No. of Attacks: & 2 claws, 1 bite \\\hline
Damage: & 1d4 claw, 1d6 bite \\\hline
Movement: & 50' \\\hline
No. Appearing: & Wild 1d4, Lair 1d4 \\\hline
Save As: & Fighter: 3 \\\hline
Morale: & 8 \\\hline
Treasure Type: & None \\\hline
XP: & 145 \\\hline
\end{tabularx}\medskip

These great cats are about 7 feet long (from nose to tail-tip) and weigh about 140 pounds. They see well in darkness and may be found hunting day
or night.

\subsection*{Mummy*}\index{Mummy}\label{mummy}

\begin{tabularx}{0.50\textwidth}{@{}lX@{}} 
Armor Class: & 17 (m) (see below) \\\hline
Hit Dice: & 5** \\\hline
No. of Attacks: & 1 touch \\\hline
Damage: & 1d12 + disease \\\hline
Movement: & 20' \\\hline
No. Appearing: & 1d4, Lair 1d12 \\\hline
Save As: & Fighter: 5 \\\hline
Morale: & 12 \\\hline
Treasure Type: & D \\\hline
XP: & 450 \\\hline
\end{tabularx}\medskip

Mummies are \textbf{undead} monsters, linen-wrapped preserved corpses animated through the auspices of dark desert gods best forgotten. Mummies are normally man-sized, but due to their desiccation one will normally not weigh more than about 100 pounds.


\begin{center} \includegraphics[width=0.47\textwidth]{Pictures132/10000000000003D80000040BFC389C06EA51C5F8.png} \end{center}

As they are undead, mummies are immune to \textbf{sleep}, \textbf{charm}, and \textbf{hold} magic. They can only be injured by spells, fire, or magical weapons; furthermore, magic weapons do only half damage, while any sort of fire-based attack does double damage. Those injured by mummy attacks will contract \textbf{mummy rot}, a disease that prevents normal or magical healing; a \textbf{cure disease }spell must be applied to the victim before they may again regain hit points.

\subsection*{Nixie}\index{Nixie}\label{nixie}

\begin{tabularx}{0.50\textwidth}{@{}lX@{}}
Armor Class: & 16 \\\hline
Hit Dice: & 1* \\\hline
No. of Attacks: & 1 dagger \\\hline
Damage: & 1d4 \\\hline
Movement: & 40' Swim 40' \\\hline
No. Appearing: & Wild 2d20, Lair 2d20 \\\hline
Save As: & Fighter: 2 \\\hline
Morale: & 6 \\\hline
Treasure Type: & B \\\hline
XP: & 37 \\\hline
\end{tabularx}\medskip

\begin{wrapfigure}{l}{0.25\textwidth}
	\includegraphics[width=0.25\textwidth]{Pictures132/10000000000003D8000004265BBAC08166A65E91.png}
\end{wrapfigure}

Nixies are small water fairies, just 4 feet tall and about 40 pounds at most. As far as anyone knows, all nixies are female. Most nixies are slim and comely, with lightly scaled, pale green skin and dark green hair. They often twine shells and pearl strings in their hair and dress in wraps woven from colorful seaweed.

Ten or more nixies can work together to cast a powerful charm (similar to \textbf{charm person}). The charm lasts one year (unless dispelled). A save vs. Spells is allowed to resist. Each nixie can cast \textbf{water breathing }once per day, with a duration of one day. Finally, a group of nixies will often have a school of giant bass living nearby who can be called to their aid (see \textbf{Fish, Giant Bass} for details).

Nixies are fey creatures, and thus unpredictable. However, they are rarely malicious, attacking only when they feel threatened. They do not like leaving their safe, comfortable river or lake and will do so only in the most dire need.

\subsection*{Ochre Jelly}\index{Ochre Jelly}\label{ochre-jelly}

See \textbf{Jelly, Ochre }on page
\hyperlink{jelly-ochre}{\pageref{jelly-ochre}}.

\subsection*{Octopus, Giant}\index{Octopus, Giant}\label{octopus-giant}

\begin{tabularx}{0.50\textwidth}{@{}lX@{}}
Armor Class: & 19 \\\hline
Hit Dice: & 8 \\\hline
No. of Attacks: & 8 tentacles, 1 bite (see below) \\\hline
Damage: & 1d4 tentacle, 1d6 bite \\\hline
Movement: & Swim 30' \\\hline
No. Appearing: & Wild 1d2 \\\hline
Save As: & Fighter: 8 \\\hline
Morale: & 7 \\\hline
Treasure Type: & None \\\hline
XP: & 875 \\\hline
\end{tabularx}\medskip

The giant octopus is, obviously, an enormous version of the normal creature. They are physically powerful as well as being clever, which makes them a serious threat to seagoing vessels. 

In order to bite a creature, the giant octopus must hit with at least two tentacles first. Further, any time a giant octopus hits with at least one tentacle per each 100 pounds of weight of its prey, it has grabbed it; unless the victim can find a way to resist (using whatever method the player might think of and whatever rolls the GM may choose), they will be pulled into the water and thus be in danger of drowning. Don' t forget to account for the weight of armor worn!

If a giant octopus fails a morale check, it will squirt out a cloud of black "ink" 40' in diameter and then jet away at twice normal speed for 2d6 rounds. Any characters being held will normally be released at this point.

\subsection*{Ogre}\index{Ogre}\label{ogre}

\begin{tabularx}{0.50\textwidth}{@{}lX@{}}
Armor Class: & 15 (12) \\\hline
Hit Dice: & 4+1 \\\hline
No. of Attacks: & 1 huge weapon \\\hline
Damage: & 2d6 huge weapon \\\hline
Movement: & 30' Unarmored 40' \\\hline
No. Appearing: & 1d6, Wild 2d6, Lair 2d6 \\\hline
Save As: & Fighter: 4 \\\hline
Morale: & 10 \\\hline
Treasure Type: & C + 1d20x100 gp \\\hline
XP: & 240 \\\hline
\end{tabularx}\medskip

Ogres appear as large, very ugly humans. They are brutish and aggressive, but inherently lazy. They employ direct attacks in combat, typically using large clubs, axes, or pole arms, generally causing 2d6 damage. If normal weapons are employed, an ogre has a +3 bonus to damage due to strength. If an ogre fights bare-handed, it does 1d8 subduing damage per hit.

\begin{center} \includegraphics[width=0.4\textwidth]{Pictures132/10000000000003CF0000052143244EE5AC68953E.png} \end{center}

One out of every six ogres will be a pack leader of 6+1 Hit Dice (500 XP). Ogres gain a +1 bonus to their morale if they are led by a pack leader. In ogre lairs of 10 or greater, there will also be an ogre bully of 8+2 Hit Dice (875 XP), with an Armor Class of 17 (13) (movement 20') and having a +4 bonus to damage due to strength. Ogre bullies generally wire together pieces of chainmail to wear over their hides. Ogres gain +2 to morale so long as the ogre bully leads them.

\subsection*{Orc}\index{Orc}\label{orc}

\begin{tabularx}{0.50\textwidth}{@{}lX@{}}
Armor Class: & 14 (11) \\\hline
Hit Dice: & 1 \\\hline
No. of Attacks: & 1 weapon \\\hline
Damage: & 1d8 or by weapon \\\hline
Movement: & 30' Unarmored 40' \\\hline
No. Appearing: & 2d4, Wild 3d6, Lair 10d6 \\\hline
Save As: & Fighter: 1 \\\hline
Morale: & 8 \\\hline
Treasure Type: & Q, R each; D in lair \\\hline
XP: & 25 \\\hline
\end{tabularx}\medskip

Orcs are short humanoids (around 5' tall) with solidly-built bodies. Their upturned noses, wide pointed ears, and beady eyes give their faces a piglike appearance. An adult weighs about 200 pounds, and this weight does not differ much between males and females. Orcs utilize all manner of weapons and armor scavenged from battlefields.


\begin{center} \includegraphics[width=0.40\textwidth]{Pictures132/10000000000003D8000003D88ED2EAFC6E645859.png} \end{center}


Orcs have Darkvision to a range of 60'. They suffer an attack penalty of -1 in bright sunlight or within the radius of a spell causing magical \textbf{light}. They speak their own rough and simple language, but many also speak some Common or Goblin.

One out of every eight orcs will be a warrior of 2 Hit Dice (75 XP). Orcs gain a +1 bonus to their morale if they are led by a warrior. In orc lairs, one out of every twelve will be a chieftain of 4 Hit Dice (240 XP) in chainmail with an Armor Class of 15 (11), a movement 20', and having a +1 bonus to damage due to strength. In lairs of 30 or more, there will be an orc king of 6 Hit Dice (500 XP), with an Armor Class of 16 (11), in chainmail with a shield, movement 20', and having a +2 bonus to damage. In the lair, orcs never fail a morale check as long as the orc king is alive. In addition, a lair has a chance equal to 1-2 on 1d6 of a shaman being present. A shaman is equivalent to a warrior orc statistically, but has Clerical abilities at level 1d4+1.


\subsection*{Ostrich (and Emu)}\index{Ostrich (and Emu)}\label{ostrich-and-emu}

\begin{tabularx}{0.50\textwidth}{@{}llX@{}}
& Ostrich & Emu \\\hline
Armor Class: & 14 & 14 \\\hline
Hit Dice: & 3 & 2 \\\hline
No. of Attacks: & 1 kick & 1 kick \\\hline
Damage: & 1d6 & 1d4 \\\hline
Movement: & 60' & 50' \\\hline
No. Appearing: & Wild 1d6 & Wild 1d6 \\\hline
Save As: & Fighter: 3 & Fighter: 2 \\\hline
Morale: & 8 & 8 \\\hline
Treasure Type: & None & None \\\hline
XP: & 145 & 75 \\\hline
\end{tabularx}\medskip

These birds are sometimes raised (or hunted) as food. In addition, the large, decorative quills of ostriches are in demand in some social
circles.

Emus will not be found in most campaigns. Unlike ostriches, they have no particular value to humans except as meat.

\subsection*{Owl}\index{Owl}\label{owl}

\begin{tabularx}{0.50\textwidth}{@{}lX@{}}
Armor Class: & 15 \\\hline
Hit Dice: & 1 \\\hline
No. of Attacks: & 2 talons, 1 beak \\\hline
Damage: & 1d4 talon, 1d4 beak \\\hline
Movement: & 10' Fly 160' (10') \\\hline
No. Appearing: & 1, Wild 1d4 \\\hline
Save As: & Fighter: 1 \\\hline
Morale: & 8 \\\hline
Treasure Type: & None \\\hline
XP: & 25 \\\hline
\end{tabularx}\medskip

Owls are birds of prey with large eyes and the ability to fly without making a sound. They are nocturnal, and have superior Darkvision of 120' range. An owl will stand about 6 to 18 inches tall with a wingspan of 20 inches. An owl's vision is very sharp and comparable to that of a falcon. It can also hear very well, even to the point that a rodent creeping through grass will draw attention. The owl will fly over a field and listen and watch for movement and then dive for a kill with its talons.

\subsection*{Owlbear}\index{Owlbear}\label{owlbear}


\begin{center} \includegraphics[width=0.47\textwidth]{Pictures132/10000000000003D800000395C22E953B0BF82F41.png} \end{center}

\begin{tabularx}{0.50\textwidth}{@{}lX@{}}
Armor Class: & 15 \\\hline
Hit Dice: & 5 \\\hline
No. of Attacks: & 2 claws, 1 bite + 1 hug \\\hline
Damage: & 1d8 claw, 1d8 bite, 2d8 hug \\\hline
Movement: & 40' \\\hline
No. Appearing: & 1d4, Wild 1d4, Lair 1d4 \\\hline
Save As: & Fighter: 5 \\\hline
Morale: & 9 \\\hline
Treasure Type: & C \\\hline
XP: & 360 \\\hline
\end{tabularx}\medskip

Owlbear are among the most feared of the no turnal forest dwelling monsters, and for good reason for they are always hungry and always aggressive. They appear to be bears with owlish faces, including a large, sharp beak. They fight much as do bears, and as with normal bears an owlbear must hit with both claws in order to do the listed "hug" damage.

These monsters are known to hunt by day when particularly hungry, but they prefer to live nocturnally. They have superior Darkvision with a range of 120 feet, without the usual penalties for being in full sunlight. They are also very quiet, surprising on 1-4  on 1d6 in their nativeterritory.


\subsection*{Parrot (or Cockatiel)}\index{Parrot (or Cockatiel)}\label{parrot-or-cockatiel}

\begin{tabularx}{0.50\textwidth}{@{}lX@{}}
Armor Class: & 11 \\\hline
Hit Dice: & ½ (1d4 hit points) \\\hline
No. of Attacks: & 1 talon or 1 beak \\\hline
Damage: & 1d4 talon or 1d4 beak \\\hline
Movement: & 10' Fly 100' (10') \\\hline
No. Appearing: & Wild 1d4 \\\hline
Save As: & Normal Man \\\hline
Morale: & 6 \\\hline
Treasure Type: & None \\\hline
XP: & 10 \\\hline
\end{tabularx}\medskip

These are decorative birds about the size of a falcon, known for their ability to learn to mimic speech and other sounds. Parrots usually have green or blue feathers with multi-colored tail feathers. Cockatiels are white with crested heads. While these birds can learn to imitate human speech when raised in captivity, most cannot actually carry on a conversation.

\subsection*{Pegasus}\index{Pegasus}\label{pegasus}

\begin{tabularx}{0.50\textwidth}{@{}lX@{}}
Armor Class: & 15 \\\hline
Hit Dice: & 4 \\\hline
No. of Attacks: & 2 hooves \\\hline
Damage: & 1d6 hoof \\\hline
Movement: & 80' (10') Fly
160' (10') \\\hline
No. Appearing: & Wild 1d12 \\\hline
Save As: & Fighter: 2 \\\hline
Morale: & 8 \\\hline
Treasure Type: & None \\\hline
XP: & 240 \\\hline
\end{tabularx}\medskip

The pegasus is the winged horse of legend. They are prized as aerial steeds as they are the swiftest of fliers, but they are shy creatures who live in the highest mountains, making them rare indeed in captivity. 

An average female pegasus stands 5 feet high at the shoulder, weighs 1,200 pounds, and has a wingspan of 20 feet; males are somewhat larger, averaging 6 feet in height and weighing 1,400 pounds, with a wingspan of 22 feet. A light load for a pegasus is up to 400 pounds; a heavy load, up to 900 pounds.

 \begin{center}
 	\includegraphics[width=0.47\textwidth]{Pictures132/10000000000007E90000067D263DC91D6CB36315.png} 
 \end{center}

\subsection*{Pixie}\index{Pixie}\label{pixie}

\begin{tabularx}{0.50\textwidth}{@{}lX@{}}
Armor Class: & 17 \\\hline
Hit Dice: & 1* \\\hline
No. of Attacks: & 1 dagger \\\hline
Damage: & 1d4 \\\hline
Movement: & 30' Fly 60' \\\hline
No. Appearing: & 2d4, Wild 10d4, Lair 10d4 \\\hline
Save As: & Fighter: 1 (with Elf bonuses) \\\hline
Morale: & 7 \\\hline
Treasure Type: & R, S \\\hline
XP: & 37 \\\hline
\end{tabularx}\medskip

\begin{wrapfigure}{l}{0.25\textwidth}
 \includegraphics[width=0.25\textwidth]{Pictures132/10000001000003D80000054D22C08A792E16CEE6.png} 
\end{wrapfigure}

Pixies are winged fairies often found in forested areas. Like sprites, pixies love beauty. They dress in bright colors and favor clothing with flourishes like feathered caps, curly-tipped shoes, scarves, and so on. They are quite small, just 2½ feet in height and weighing no more than 30 pounds. Pixies can only fly for 3 turns maximum before requiring rest of at least one turn, during which time the pixie may walk at normal speed but may not fly.

A pixie can become \textbf{invisible }at will, as many times per day as itwishes, and can attack while remaining invisible. Anyone attacking an invisible pixie does so with an attack penalty of -4 unless the attacker can somehow detect invisible creatures. Pixies may ambush their foes while invisible; if they do so, they surprise on 1-5 on 1d6. 

Pixies are whimsical, enjoying nothing so much as a good joke or prank, especially at the expense of a "big person."

\subsection*{Purple Worm}\index{Purple Worm}\label{purple-worm}

\begin{tabularx}{0.50\textwidth}{@{}lXX@{}}
Armor Class: & 16 & 17 \\\hline
Hit Dice: & 11* (+9) to 15* (+11) & 16* (+12) to 20* (+13) \\\hline
No. of Attacks: & \multicolumn{2}{c}{1 bite, 1 sting} \\\hline
Damage: & 2d8 bite, 1d8 + poison sting & 3d8 bite, 1d10 + poison sting \\\hline
Movement: & \multicolumn{2}{c}{20' (15') Burrow 20' (15')}\\\hline
No. Appearing: &  \multicolumn{2}{c}{1d2, Wild 1d4}\\\hline
Save As: & \multicolumn{2}{c}{Fighter: 6 to 10 (½ of Hit Dice)} \\\hline
Morale: & 10 & 10 \\\hline
Treasure Type: & None & None \\\hline
XP: & 11 HD 1,670, 12 HD 1,975, 13 HD 2,285, 14 HD 2,615, 15 HD 2,975 & 16 HD 3,385, 17 HD 3,745, 18 HD 4,160, 19 HD 4,675, 20 HD 5,450  \\\hline
\end{tabularx}\medskip

Purple worms are gigantic subterranean monsters; they are rarely found above ground. Adult purple worms range from 5-9 feet in diameter and 50-100 feet long, with an average weight of about 40,000 pounds. 

The creature' s tail ends in a narrow point tipped with a poisonous stinger; those injured by it must save vs. Poison or die. Note that the purple worm' s movement is less than the monster' s length, so that, if attacking from out of a tunnel, it might not be able to use the stinger for several rounds.

Any time a purple worm successfully bites a man-sized or smaller opponent with a natural roll of 19 or 20, the opponent has been swallowed and will suffer 3d6 damage per round afterward due to being digested. A character who has been swallowed can only effectively attack with small cutting or stabbing weapons such as dagger or shortsword.

\begin{center} \includegraphics[width=0.47\textwidth]{Pictures132/10000000000003D800000409D68D45FB3AFF18AB.png} \end{center}


\subsection*{Rat (and Rat, Giant)}\index{Rat (and Rat, Giant)}\label{rat-and-rat-giant}

\begin{tabularx}{0.50\textwidth}{@{}lXX@{}}
& Normal & Giant \\\hline
Armor Class: & 11 & 13 \\\hline
Hit Dice: & 1 Hit Point & ½ (1d4 hit points) \\\hline
No. of Attacks: & 1 bite per pack & 1 bite \\\hline
Damage: & 1d6 + disease & 1d4 + disease \\\hline
Movement: & 20' Swim 10' &
40' Swim 20' \\\hline
No. Appearing: & 5d10, Wild 5d10,  Lair 5d10 & 3d6, Wild 3d10, Lair 3d10 \\\hline
Save As: & Normal Man & Fighter: 1 \\\hline
Morale: & 5 & 8 \\\hline
Treasure Type: & None & C \\\hline
XP: & 360* & 10 \\\hline
\end{tabularx}

Rats are omnivorous pests found in all climates where humans live.

Normal rats attack as a swarm; each point of damage done to the swarm reduces their numbers by one animal.


Giant rats are scavengers, but will attack to defend their nests and territories. A giant rat can grow to be up to 4 feet long and weigh over 50 pounds. A single giant rat, or a small group of up to four, will generally be shy, but larger packs attack fearlessly, biting and chewing with their sharp incisors.

Any rat bite has a 5\% chance of causing a disease. A character who suffers  one or more rat bites where the die roll indicates disease will sicken in 3d6 hours. The infected character will lose one point of Constitution per hour; after losing each point, the character is allowed a save vs. Death Ray (adjusted by the current Constitution bonus or penalty) to break the fever and end the disease. Any character reduced to zero Constitution is dead. See \textbf{Constitution Point Losses} in the \textbf{Encounter} section for details on regaining lost Constitution. 

* Note: The XP award for normal rats is for driving away or killing an entire pack of normal size. If the adventurers are forced to flee, the
GM should award 3 XP per rat slain.

\begin{center}
	\includegraphics[width=0.25\textwidth]{Pictures132/10000000000003D8000005DFD5FF9E14423C4818.png}
\end{center}


\subsection*{Rhagodessa, Giant}\index{Rhagodessa, Giant}\label{rhagodessa-giant}

\begin{tabularx}{0.50\textwidth}{@{}lX@{}}
Armor Class: & 16 \\\hline
Hit Dice: & 4 \\\hline
No. of Attacks: & 2 legs, 1 bite \\\hline
Damage: & leg grab, 2d8 bite \\\hline
Movement: & 50' \\\hline
No. Appearing: & 1d4, Wild 1d6, Lair 1d6 \\\hline
Save As: & Fighter: 4 \\\hline
Morale: & 9 \\\hline
Treasure Type: & U \\\hline
XP: & 240 \\\hline
\end{tabularx}\medskip

The rhagodessa is related to both spiders and scorpions, though it is not properly either. Rhagodessas have "pedipalps," an elongated pair of forelegs with sticky pads on them for capturing prey.

Giant rhagodessas are the size of a pony. Those found in desert terrain are generally marked in yellow, red, and brown, while those found underground may be black or white in color (those found in the deepest caverns are always white). Like spiders, they can climb walls, but they are unable to cross ceilings or otherwise climb entirely upside down.

A hit by a leg does no damage, but the victim is stuck fast and will be drawn to the rhagodessa' s mouth on the next round and automatically hit for 2d8 points of damage; this repeats each round, so long as the victim is held. Escaping from the sticky hold requires a successful roll to open doors. If both legs hit, this roll frees the victim from just one of them; a second roll is needed to fully escape, and of course the rhagodessa can simply attack again with the free leg on the next round. Alternately, victims may attack with small or medium elee weapons, and in fact gain a bonus of +2 is added to the attackoll if a small weapon is used. If the giant rhagodessa is slain, any held victim can be freed with the open doors roll mentioned above (and in this case another character can help, making the roll for the victim).

The rhagodessa seems unable to use its bite attack against a foe it has not captured in this way, and neither will it attack more than one foe with its legs. If threatened, a rhagodessa which has captured a victim will attempt to withdraw to consume its prey in peace.

\subsection*{Rhinoceros}\index{Rhinoceros}\label{rhinoceros}

\begin{tabularx}{0.50\textwidth}{@{}llX@{}}
& Black & Woolly \\\hline
Armor Class: & 17 & 19 \\\hline
Hit Dice: & 8 & 12 (+10) \\\hline
No. of Attacks: & 1 butt or 1 trample & \\\hline
Damage: & \vtop{\hbox{\strut 2d6 butt,}\hbox{\strut 2d8 trample}} &
\vtop{\hbox{\strut 2d8 butt,}\hbox{\strut 2d12 trample}} \\\hline
Movement: & 40' (15') &
40' (15') \\\hline
No. Appearing: & Wild 1d12 & Wild 1d8 \\\hline
Save As: & Fighter: 6 & Fighter: 8 \\\hline
Morale: & 6 & 6 \\\hline
Treasure Type: & None & None \\\hline
XP: & 875 & 1,875 \\\hline
\end{tabularx}\medskip

A rhinoceros (or "rhino") is a member of any of several species (including numerous supposedly extinct species) of odd-toed hooved mammals. They are found primarily in tropical and temperate grasslands. 

The woolly rhinoceros is a prehistoric beast with long fur, found in primitive "lost world" areas in colder territories. They behave much as the black rhino does.

\subsection*{Roc}\index{Roc}\label{roc}

\begin{tabularx}{0.50\textwidth}{@{}lXXX@{}}
& Normal & Large & Giant \\\hline
Armor Class: & 18 & 18 & 18 \\\hline
Hit Dice: & 6 & 12 (+10) & 32 (+16) \\\hline
No. of Attacks: &\multicolumn{3}{c}{2 claws, 1 bite} \\\hline
Damage: & 1d6 claw, 2d6 bite & 1d8 claw, 2d10 bite & 3d6 claw, 6d6 bite \\\hline
Movement: & \multicolumn{3}{c}{20' Fly~160'~(10')} \\\hline
No. Appearing: & Wild 1d12 & Wild 1d8 & Wild 1 \\\hline
Save As: & Fighter: 6 & Fighter: 12 & Fighter: 20 at +5 \\\hline
Morale: & 8 & 9 & 10 \\\hline
Treasure Type: & I & I & I \\\hline
XP: & 500 & 1,875 & 14,250 \\\hline
\end{tabularx}\medskip

Rocs are birds similar to eagles, but even a "normal" roc is huge, being about 9 feet long and having a wingspan of 24 feet. Large rocs are about 18 feet long and have wingspans of around 48 feet; giant rocs average 30 feet long and have massive wingspans of around 80 feet. A roc's plumage is either dark brown or golden from head to tail. Like most birds, the males have the brighter plumage, with females being duller in color and thus more easily hidden (if anything so large can even be hidden, that is).

A light load for a normal roc is 150 pounds, while a heavy load is 300 pounds. Obviously only the smallest characters can hope to ride upon a normal roc. For a large roc, a light load is up to 600 pounds and a heavy load up to 1,200.  Giant rocs can easily lift up to 3,000 pounds, and are heavily loaded when carrying up to 6,000 pounds. Tales of giant rocs carrying off full-grown elephants are somewhat exaggerated, but note that a young elephant would be reasonable prey for these monstrous birds. 


\begin{center} \includegraphics[width=0.40\textwidth]{Pictures132/10000000000003D8000005706FC87F0C286A2A39.png} \end{center}

A roc attacks from the air, swooping earthward to snatch prey in its powerful talons and carry it off for itself and its young to devour. Any successful hit with both claw (talon) attacks against a single creature results in that creature being carried off, unless of course the creature is too large for the roc to carry. While being carried, the victim will not be further attacked, so as to be as "fresh" as possible when given to the hatchlings (or consumed by the roc itself if it is solitary).

When rocs are encountered they are almost certainly hunting, and will generally attack creatures of horse size or less. If facing a large group the rocs may make passes close overhead to scatter them first before each chooses a single target to prey upon. Mated pairs found in their nests will fight to the death to protect their eggs or offspring (morale of 12 in this case). 


\subsection*{Rock Baboon}\index{Rock Baboon}\label{rock-baboon}

\begin{tabularx}{0.50\textwidth}{@{}lX@{}}

Armor Class: & 14 \\\hline
Hit Dice: & 2 \\\hline
No. of Attacks: & 1 club or 1 fist, 1 bite \\\hline
Damage: & 1d6 club or 1d4 fist, 1d4 bite \\\hline
Movement: & 40' \\\hline
No. Appearing: & 2d6, Wild 2d6, Lair 5d6 \\\hline
Save As: & Fighter: 2 \\\hline
Morale: & 8 \\\hline
Treasure Type: & None \\\hline
XP: & 75 \\\hline
\end{tabularx}\medskip

\begin{center} \includegraphics[width=0.4\textwidth]{Pictures132/10000000000003D80000052BF44D837D43B16075.png} \end{center}

Rock baboons are a large, particularly intelligent variety of baboon. An adult male rock baboon is 4' to 5' tall and weighs 200 to 250 pounds, with females being a bit smaller and lighter.

Rock baboons are omnivorous, but prefer meat. They are aggressive, naturally cruel creatures. They will prepare ambushes in rocky or forested terrain and attack any party they outnumber.



\subsection*{Rot Grub}\index{Rot Grub}\label{rot-grub}

\begin{tabularx}{0.50\textwidth}{@{}lX@{}}
Armor Class: & 10 \\\hline
Hit Dice: & 1 hp \\\hline
No. of Attacks: & 1 bite \\\hline
Damage: & special \\\hline
Movement: & 5' \\\hline
No. Appearing: & 5d4 \\\hline
Save As: & Fighter: 1 \\\hline
Morale: & 12 \\\hline
Treasure Type: & None \\\hline
XP: & 10 \\\hline
\end{tabularx}\medskip

Rot grubs are 1-inch long vermin found in carrion, dung, and other such garbage and organic material. Their skin color is white or brown. When a living creature contacts an area (dung heap, offal, etc) infested with rot grubs, the grubs will attack if they can come in contact with the victim's skin. A rot grub secretes an anesthetic when it bites and will burrow into the flesh. A burrowing grub can be noticed if the victim makes a successful save vs. Death Ray with Wisdom bonus applied in order to notice a strange rippling beneath their skin. Otherwise, the victim does not notice the grubs. During the first two rounds, a burrowing rot grub can be killed by applying fire to the infested skin or by cutting open the infested skin with any slashing weapon. Either method deals 1d8 points of damage to the victim, but kills the grubs. After the second round, only \textbf{cure disease} can kill the grubs before they burrow to the victim's heart and devour it in 1d3 turns.

\begin{center} \includegraphics[width=0.47\textwidth]{Pictures132/10000000000003D8000003EF11BAD6DC00FCF429.png} \end{center}


\subsection*{Rust Monster*}\index{Rust Monster}\label{rust-monster}

\begin{tabularx}{0.50\textwidth}{@{}lX@{}}
Armor Class: & 18 \\\hline
Hit Dice: & 5* \\\hline
No. of Attacks: & 1 touch \\\hline
Damage: & special \\\hline
Movement: & 40' \\\hline
No. Appearing: & 1d4 \\\hline
Save As: & Fighter: 5 \\\hline
Morale: & 7 \\\hline
Treasure Type: & None \\\hline
XP: & 405 \\\hline
\end{tabularx}\medskip

A rust monster (sometimes known as a \emph{corroder} or \emph{corrosion beast}) is a strange monster built like a huge turtle, with an insectoid head sporting large feather-like antennae and a thick tail with a hammer-like protrusion at the tip which seems to serve no purpose whatsoever.

The touch of any part of a rust monster' s body oxidizes metal objects instantly, turning them to rust, verdigris, or other oxides as appropriate. One attacks with its antennae, brushing them over metal items. Non-magical metal attacked by a rust monster, or that touches the monster (such as a sword used to attack it), is instantly ruined. A hit with a non-magical metal weapon inflicts half damage before the weapon is destroyed. Magic weapons or armor permanently lose ne "plus" each time they make contact with the monster. 

The metal oxides created by this monster are its food; a substantial mount of metal dropped in its path may cause it to cease pursuit of metal-armored characters. Use a morale check to determine this. Metals that do not normally oxidize, such as gold, are of no interest to a rust monster and will be ignored. While rust monsters will consume oxides of silver or copper, they have a strong preference for ferrous metals (iron or steel), preferring them over any other metal.

Whether the rust monster is in any way related to the rarer ironbane (as found on page \hyperlink{ironbane}{\pageref{ironbane}}) is unknown, but both monsters seem to have the exact same power.

\begin{center} \includegraphics[width=0.45\textwidth]{Pictures132/10000000000003D8000003522E0A3901C8479E59.png} \end{center}

\subsection*{Sabre-Tooth Cat}\index{Sabre-Tooth Cat}\label{sabre-tooth-cat}

\begin{tabularx}{0.50\textwidth}{@{}lX@{}}
Armor Class: & 14 \\\hline
Hit Dice: & 8 \\\hline
No. of Attacks: & 2 claws, 1 bite \\\hline
Damage: & 1d6 claw, 2d8 bite \\\hline
Movement: & 50' \\\hline
No. Appearing: & Wild 1d4, Lair 1d4 \\\hline
Save As: & Fighter: 8 \\\hline
Morale: & 10 \\\hline
Treasure Type: & None \\\hline
XP: & 875 \\\hline
\end{tabularx}\medskip

\begin{center} \includegraphics[width=0.45\textwidth]{Pictures132/10000000000003D800000434781C4555ECD5E0B9.png} \end{center}

The sabre-tooth cat, or \emph{smilodon}, is a prehistoric great cat with very large canine teeth. They are more robustly built than other great cats, with particularly well-developed forelimbs and exceptionally long upper canine teeth. Sabre-tooth cats are ambush predators, surprising on 1‑4 on 1d6 in their natural environment (forests and tall-grass prairies), where they prey primarily upon large herbivores.


\subsection*{Salamander*}\index{Salamander}\label{salamander}

Salamanders are large, lizard-like creatures from the elemental planes. They are sometimes found on the material plane; they can arrive through naturally-occurring dimensional rifts, or they may be summoned by high-level Magic-Users. Due to their highly magical nature, they cannot be harmed by non-magical weapons.

Flame, frost, and lightning salamanders hate each other, and each type will attack the others on sight in preference to any other nearby foe.

\subsection*{Salamander, Flame*}\index{Salamander, Flame}\label{salamander-flame}

\begin{tabularx}{0.50\textwidth}{@{}lX@{}}
Armor Class: & 19 (m) \\\hline
Hit Dice: & 8* \\\hline
No. of Attacks: & 2 claws, 1 bite + heat \\\hline
Damage: & 1d4 claw, 1d8 bite, 1d8/round heat \\\hline
Movement: & 40' \\\hline
No. Appearing: & 1d4+1, Wild 2d4, Lair 2d4 \\\hline
Save As: & Fighter: 8 \\\hline
Morale: & 8 \\\hline
Treasure Type: & F \\\hline
XP: & 945 \\\hline
\end{tabularx}\medskip


\begin{center} \includegraphics[width=0.45\textwidth]{Pictures132/10000000000003D800000314370BA8A5DA4EFD83.png} \end{center}

Flame salamanders come from the Elemental Plane of Fire. They look like giant snakes, more than 12' long, with dragon-like heads and lizard forelimbs. Their scales are all the colors of flame, red and orange and yellow. A flame salamander is flaming hot, and all non-fire-resistant creatures within 20' of the monster suffer 1d8 points of damage per round from the heat. They are immune to damage from any fire or heat attack. Flame salamanders are intelligent; they speak the language of the Plane of Fire, and many will also know Elvish, Common, and/or Dragon.


\subsection*{Salamander, Frost*}\index{Salamander, Frost}\label{salamander-frost}

\begin{tabularx}{0.50\textwidth}{@{}lX@{}}
Armor Class: & 21 (m) \\\hline
Hit Dice: & 12* (+10) \\\hline
No. of Attacks: & 4 claws, 1 bite + cold \\\hline
Damage: & 1d6 claw, 2d6 bite, 1d8/round cold \\\hline
Movement: & 40' \\\hline
No. Appearing: & 1d3, Wild 1d3, Lair 1d3 \\\hline
Save As: & Fighter: 12 \\\hline
Morale: & 9 \\\hline
Treasure Type: & E \\\hline
XP: & 1,975 \\\hline
\end{tabularx}\medskip

Frost salamanders come from the Elemental Plane of Water. They look like giant lizards with six legs. Their scales are the colors of ice, white, pale gray and pale blue. Frost salamanders are very cold, and all non-cold-resistant creatures within 20' suffer 1d8 points of damage per round from the cold. Frost salamanders are completely immune to all types of cold-based attacks. They are quite intelligent; all speak the language of the Plane of Water, and many also speak Common, Elvish, and/or Dragon.

Flame and frost salamanders hate each other, and each type will attack the other on sight, in preference over any other foe. If summoned by a Magic-User, a salamander is often assigned to protect a location, doorway, or treasure hoard; in such a case, the salamander will attack anyone attempting to gain unauthorized access to the protected area. Those which arrive through natural rifts may have any goals or motivations the GM wishes, and thus may choose to parley, fight, or even ignore adventurers.

\vfill

\begin{center} \includegraphics[width=0.47\textwidth]{Pictures132/10000000000003D80000029F6BEAF8BD1BA87EE2.png} \end{center}


\subsection*{Salamander, Lightning*}\index{Salamander, Lightning}\label{salamander-lightning}

\begin{tabularx}{0.50\textwidth}{@{}lX@{}}
Armor Class: & 20 (m) \\\hline
Hit Dice: & 10* (+9) \\\hline
No. of Attacks: & 2 bites + lightning \\\hline
Damage: & 2d4 bite, 1d8/round lightning \\\hline
Movement: & 40' \\\hline
No. Appearing: & 1d4, Wild 2d4, Lair 2d4 \\\hline
Save As: & Fighter: 10 \\\hline
Morale: & 8 \\\hline
Treasure Type: & E \\\hline
XP: & 1,390 \\\hline
\end{tabularx}\medskip

Lightning Salamanders come from the Elemental Plane of Air. A lightning salamander resembles a giant snake more than 12 feet long with two dragon-like heads (on short but flexible necks). Its scales are all the colors of lightning: white, blue, purple, and yellow. A lightning salamander constantly emits little bolts of lightning; all creatures within 20 feet of the salamander that are not lightning-resistant suffer 1d8 points of damage per round. A lightning salamander is immune to damage from any type of electrical or lightning attack. It is intelligent and can speak the language of the Plane of Air, and many will also know Elvish, Common, and/or Dragon.

Despite having two heads a lightning salamander has only one mind; either head may speak or both may, but it is very rare to meet a lightning salamander who can speak different words with each head at the same time (although those who can are known to sing duets with themselves, which may give away one' s location to those listening).

\vfill

\begin{center} \includegraphics[width=0.47\textwidth]{Pictures132/10000000000003D80000032ADC310D515714FD2B.png} \end{center}


\subsection*{Salamander, Sand*}\index{Salamander, Sand}\label{salamander-sand}

\begin{tabularx}{0.50\textwidth}{@{}lX@{}}
Armor Class: & 18 (m) \\\hline
Hit Dice: & 7* (+4) \\\hline
No. of Attacks: & 1 bite + special, see below \\\hline
Damage: & 1d6 + petrification \\\hline
Movement: & 20' \\\hline
No. Appearing: & 1d3, Wild 2d4, Lair 1d6 \\\hline
Save As: & Fighter: 7 \\\hline
Morale: & 8 \\\hline
Treasure Type: & L \\\hline
XP: & 735 \\\hline
\end{tabularx}

Sand salamanders come from the Elemental Plane of Earth. A sand salamander resembles a giant sea turtle with six flippers and a serpentine neck and head, with scales of varying shades of gray or brown.

The sand salamander' s most feared attack is its bite, for any living creature bitten by one must save vs. Petrify or be turned to stone. In addition to attacking, a sand salamander can temporarily transform any stone within a 20 foot radius into sand. Characters in the affected area must save vs. Paralysis each round in order to move through the sand, and if the save is successful, the character is still reduced to half their normal movement. Whenever the sand salamander moves out of range, the sand "congeals" back into stone, and any character in the affected area must save vs. Paralysis or become trapped. Extraction of a trapped person may take quite a long time, chipping and hammering at the stone to break it apart.

A sand salamander is immune to piercing attacks (such as spears or arrows) and suffers half damage from cutting attacks. It is intelligent and can speak the language of the Plane of Earth; many may also know Elvish, Common, or Dragon.

\begin{center} \includegraphics[width=0.47\textwidth]{Pictures132/10000000000003D8000004C00E4DE659EB41CAED.png} \end{center}


\subsection*{Scorpion, Giant}\index{Scorpion, Giant}\label{scorpion-giant}

\begin{tabularx}{0.50\textwidth}{@{}lX@{}}
Armor Class: & 15 \\\hline
Hit Dice: & 4* \\\hline
No. of Attacks: & 2 claws, 1 stinger \\\hline
Damage: & 1d10 claw, 1d6 + poison sting \\\hline
Movement: & 50' (10') \\\hline
No. Appearing: & 1d6, Wild 1d6 \\\hline
Save As: & Fighter: 2 \\\hline
Morale: & 11 \\\hline
Treasure Type: & None \\\hline
XP: & 280 \\\hline
\end{tabularx}\medskip

Giant scorpions are quite large, sometimes as large as a donkey. They are aggressive predators and generally attack on sight. If a claw attack hits, the giant scorpion receives a +2 attack bonus with its stinger (but two claw hits do not give a double bonus). Those hit by the stinger must save vs. Poison or die. Giant scorpions are most commonly found in desert areas or caverns.

\subsection*{Sea Serpent}\index{Sea Serpent}\label{sea-serpent}

\begin{tabularx}{0.50\textwidth}{@{}lX@{}}
Armor Class: & 17 \\\hline
Hit Dice: & 6 \\\hline
No. of Attacks: & 1 bite \\\hline
Damage: & 2d6 \\\hline
Movement: & Swim 50' (10') \\\hline
No. Appearing: & Wild 2d6 \\\hline
Save As: & Fighter: 6 \\\hline
Morale: & 8 \\\hline
Treasure Type: & None \\\hline
XP: & 500 \\\hline
\end{tabularx}\medskip

\begin{center} \includegraphics[width=0.45\textwidth]{Pictures132/10000000000003D8000004FBE2EF5191C0DC853A.png} \end{center}

Sea serpents are, obviously, serpentine monsters which live in the sea. They range from 20' to 40' long. A sea serpent can choose to wrap around a ship and constrict; in this case, roll 2d10 for damage to the vehicle, and reduce any effective Hardness by half..


\subsection*{Shadow*}\index{Shadow}\label{shadow}

\begin{tabularx}{0.50\textwidth}{@{}lX@{}}
Armor Class: & 13 (m) \\\hline
Hit Dice: & 2* \\\hline
No. of Attacks: & 1 touch \\\hline
Damage: & 1d4 + 1 point Strength loss \\\hline
Movement: & 30' \\\hline
No. Appearing: & 1d10, Wild 1d10, Lair 1d10 \\\hline
Save As: & Fighter: 2 \\\hline
Morale: & 12 \\\hline
Treasure Type: & F \\\hline
XP: & 100 \\\hline
\end{tabularx}\medskip

A shadow is an incorporeal monster, literally a kind of living shadow. Only magical weapons will harm a shadow. A shadow can be difficult to see in dark or gloomy areas but stands out starkly in brightly illuminated places. They are 5 to 6 feet tall, generally man-shaped, and weightless. Despite their appearance they are not \textbf{undead} monsters and thus do not share those creatures' weaknesses or powers; however, they are immune to \textbf{charm }and \textbf{sleep }magics.

A shadow' s attack does 1d4 points of cold damage and drains 1 point of Strength from the victim. Victims reduced to 2 or fewer points of Strength collapse and become unable to move; those reduced to 0 Strength die and rise as shadows a day later (at nightfall). Otherwise, Strength points lost to shadows are recovered at a rate of 1 point per turn.

\begin{center} \includegraphics[width=0.47\textwidth]{Pictures132/10000000000003D800000391359F9EB990D418E0.png} \end{center}


\subsection*{Shark, Bull}\index{Shark, Bull}\label{shark-bull}

\begin{tabularx}{0.50\textwidth}{@{}lX@{}}
Armor Class: & 13 \\\hline
Hit Dice: & 2 \\\hline
No. of Attacks: & 1 bite \\\hline
Damage: & 2d4 \\\hline
Movement: & Swim 60' (10') \\\hline
No. Appearing: & Wild 3d6 \\\hline
Save As: & Fighter: 2 \\\hline
Morale: & 7 \\\hline
Treasure Type: & None \\\hline
XP: & 75 \\\hline
\end{tabularx}\medskip

Bull sharks are so named because of their stocky, broad build. Male bull sharks can grow up to 7' long and weigh around 200 pounds, while females have been known to be up to 12' long, weighing up to 500 pounds. Bull sharks are able to tolerate fresh water, and often travel up rivers in search of prey.

\begin{center} \includegraphics[width=0.47\textwidth]{Pictures132/10000000000007E9000005C145675FEAFE1B81C2.png} \end{center}

\subsection*{Shark, Great White}\index{Shark, Great White}\label{shark-great-white}

\begin{tabularx}{0.50\textwidth}{@{}lX@{}}
Armor Class: & 19 \\\hline
Hit Dice: & 8 \\\hline
No. of Attacks: & 1 bite \\\hline
Damage: & 2d10 \\\hline
Movement: & Swim 60' (10') \\\hline
No. Appearing: & Wild 1d4 \\\hline
Save As: & Fighter: 8 \\\hline
Morale: & 8 \\\hline
Treasure Type: & None \\\hline
XP: & 875 \\\hline
\end{tabularx}

Great white sharks range from 12' to 15' in length on the average, though specimens ranging up to 30' in length have been reported. They are apex predators. Great white sharks have the ability to sense the electromagnetic fields of living creatures, allowing them to find prey even when light or water clarity are poor, and are able to smell blood at great distances.

\subsection*{Shark, Mako}\index{Shark, Mako}\label{shark-mako}

\begin{tabularx}{0.50\textwidth}{@{}lX@{}}
Armor Class: & 15 \\\hline
Hit Dice: & 4 \\\hline
No. of Attacks: & 1 bite \\\hline
Damage: & 2d6 \\\hline
Movement: & Swim 80' \\\hline
No. Appearing: & Wild 2d6 \\\hline
Save As: & Fighter: 4 \\\hline
Morale: & 7 \\\hline
Treasure Type: & None \\\hline
XP: & 240 \\\hline
\end{tabularx}\medskip

Mako sharks are fast-moving predators found in temperate and tropical seas. They average 9' to 13' in length and weigh up to 1,750 pounds. Mako sharks are known for their ability to leap out of the water; they are able to leap up to 20' in the air.

\subsection*{Shrew, Giant}\index{Shrew, Giant}\label{shrew-giant}

\begin{tabularx}{0.50\textwidth}{@{}lXX@{}}
& Common & Venomous \\\hline
Armor Class: & 16 & 16 \\\hline
Hit Dice: & 1 & 1* \\\hline
No. of Attacks: & 2 bites & 2 bites \\\hline
Damage: & 1d6 bite & 1d6 bite + poison \\\hline
Movement: & 60' & 60' \\\hline
No. Appearing: &  \multicolumn{2}{c}{1d4, Wild 1d8, Lair 1d8}\\\hline
Save As: & Fighter: 2 & Fighter: 2 \\\hline
Morale: & 10 & 10 \\\hline
Treasure Type: & None & None \\\hline
XP: & 25 & 37 \\\hline
\end{tabularx}\medskip

Giant shrews resemble giant rats, but are larger, being up to 6' long, and darker in color. They have a very fast metabolic rate and must eat almost constantly. Giant shrews are omnivorous, and aggressively defend their nests and the immediate territory around them.

Giant shrews move so swiftly that they are able to bite twice per round, and they may attack two different adjacent opponents in this way.

A few giant shrew species (generally no more than 5\% of those encountered) are venomous. The bite of such a giant shrew will kill the victim unless a save vs. Poison is made. A victim bitten twice in a round need only save once for that round, but of course will have to save again in subsequent rounds if bitten again.

\begin{center} \includegraphics[width=0.47\textwidth]{Pictures132/10000000000003BF0000025FD5AB97402AA58A1D.png} \end{center}


\subsection*{Shrieker (Wailing Morel)}\index{Shrieker (Wailing Morel)}\label{shrieker-wailing-morel}

\begin{tabularx}{0.50\textwidth}{@{}lX@{}}
Armor Class: & 13 \\\hline
Hit Dice: & 3 \\\hline
No. of Attacks: & Special \\\hline
Damage: & None \\\hline
Movement: & 5' \\\hline
No. Appearing: & 1d8 \\\hline
Save As: & Fighter: 1 \\\hline
Morale: & 12 \\\hline
Treasure Type: & None \\\hline
XP: & 145 \\\hline
\end{tabularx}\medskip

A shrieker, sometimes called a wailing morel,\textbf{ }is a large (3' to 5' tall and about the same size across), semi-mobile fungus that wails loudly as a defense mechanism when approached or threatened. Shriekers are found in underground areas such as caverns and dungeons. They are found in a variety of pale colors, most commonly white, gray, lavender, or red.  This monster does not attack directly; rather, its shrieking tends to attract the attention of other monsters in the nearby area. Movement or light within 10 feet, or causing any damage to one, will cause one to wail for 1d4 rounds.

In game terms, the GM should generally roll a wandering monster check each
round that this monster wails.

\begin{center} \includegraphics[width=0.47\textwidth]{Pictures132/10000000000003D8000002F91C5276CFDD7974A0.png} \end{center}


\subsection*{Skeleton}\index{Skeleton}\label{skeleton}

\begin{tabularx}{0.50\textwidth}{@{}lX@{}}
Armor Class: & 13 (special, see below) \\\hline
Hit Dice: & 1 \\\hline
No. of Attacks: & 1 weapon \\\hline
Damage: & 1d6 or by weapon \\\hline
Movement: & 40' \\\hline
No. Appearing: & 3d6, Wild 3d10 \\\hline
Save As: & Fighter: 1 \\\hline
Morale: & 12 \\\hline
Treasure Type: & None \\\hline
XP: & 25 \\\hline
\end{tabularx}\medskip

Skeletons
are mindless \textbf{undead} created by an evil Magic-User or Cleric, generally to guard a tomb or treasure hoard, or to act as guards for their creator. They take only ½ damage from edged weapons, and only a single point from arrows, bolts or sling stones (plus any magical bonus). As with all undead, they can be \textbf{Turned} by a Cleric, and are immune to \textbf{sleep, charm} or \textbf{hold} magic. As they are mindless, no form of \textbf{mind reading} is of any use against them. Skeletons never fail morale, and thus always fight until destroyed.

\begin{center} \includegraphics[width=0.47\textwidth]{Pictures132/10000000000003D80000045C35CAF2E9CD063D5A.png} \end{center}


\subsection*{Snake, Pit Viper (and Rattlesnake)}\index{Snake, Pit Viper (and Rattlesnake)}\label{snake-pit-viper-and-rattlesnake}

\begin{tabularx}{0.50\textwidth}{@{}XXX@{}}
& Normal & Giant \\\hline
Armor Class: & 14 & 15 \\\hline
Hit Dice: & 1* & 2* \\\hline
No. of Attacks: & 1 bite & 1 bite \\\hline
Damage: & 1d4 + poison & 1d8 + poison \\\hline
Movement: & 30' & 40' \\\hline
No. Appearing: & 1d4, Wild 1d4, Lair 1d4 & 1d2, Wild 1d2, Lair 1d2 \\\hline
Save As: & Fighter: 1 & Fighter: 2 \\\hline
Morale: & 7 & 8 \\\hline
Treasure Type: & None & None \\\hline
XP: & 37 & 100 \\\hline
\end{tabularx}

Pit vipers are highly venomous snakes. There are many varieties ranging in size from 2' to 12' at adulthood; the statistics above are for an "average" variety which reaches about 9' in length. Giant-sized pit vipers are much bigger, and range from 14' to 20' in length at adulthood.

Those bitten by a pit viper must save vs. Poison or die.

Pit vipers are named for the thermally sensitive "pits" between their eyes and nostrils. These are used to detect birds, mammals, and lizards, the natural prey of these snakes. Even though lizards are cold-blooded, pit vipers can still sense them because their temperature is often slightly higher or lower than their surroundings.

Rattlesnakes are a variety of pit viper; in addition to the details given above, a rattlesnake has a rattle (from which it gets its name) at the end of its tail. The rattle is used to warn away larger creatures. 


\begin{center} \includegraphics[width=0.47\textwidth]{Pictures132/10000000000003D80000053200DD61FC01A69780.png} \end{center}


\subsection*{Snake, Python}\index{Snake, Python}\label{snake-python}

\begin{tabularx}{0.50\textwidth}{@{}lX@{}}
Armor Class: & 14 \\\hline
Hit Dice: & 5* \\\hline
No. of Attacks: & 1 bite, 1 constrict (see below) \\\hline
Damage: & 1d4 bite, 2d4 constrict \\\hline
Movement: & 30' \\\hline
No. Appearing: & 1d3, Wild 1d3, Lair 1d3 \\\hline
Save As: & Fighter: 5 \\\hline
Morale: & 8 \\\hline
Treasure Type: & None \\\hline
XP: & 405 \\\hline
\end{tabularx}\medskip

After a successful bite attack, a python will wrap itself around the victim (in the same round), constricting for 2d4 points of damage plus an additional 2d4 per round thereafter. The hold may be broken on a roll of 1 on 1d6 (add the victim' s Strength bonus to the range, so a Strength of 16 would result in a range of 1-3 on 1d6); breaking the hold takes a full round.

\subsection*{Snake, Sea}\index{Snake, Sea}\label{snake-sea}

\begin{tabularx}{0.50\textwidth}{@{}lX@{}}
Armor Class: & 14 \\\hline
Hit Dice: & 3* \\\hline
No. of Attacks: & 1 bite \\\hline
Damage: & 1 + poison \\\hline
Movement: & 10' Swim 30' \\\hline
No. Appearing: & Wild 1d8 \\\hline
Save As: & Fighter: 3 \\\hline
Morale: & 7 \\\hline
Treasure Type: & None \\\hline
XP: & 175 \\\hline
\end{tabularx}\medskip

Sea snakes are relatively small; the largest varieties rarely exceed 6' in length. They have relatively small heads, and are very stealthy in the water. Their bite does so little damage that the creature bitten has only a 50\% chance to notice the attack, but their poison is terribly strong, such that any creature bitten must save vs. Poison at a penalty of -4 or die.

Fortunately, sea snakes rarely attack; only if molested (grabbed, stepped on, etc.) will they do so. They are very clumsy when out of the water.

\subsection*{Snake, Spitting Cobra}\index{Snake, Spitting Cobra}\label{snake-spitting-cobra}

\begin{tabularx}{0.50\textwidth}{@{}lX@{}}
Armor Class: & 13 \\\hline
Hit Dice: & 1* \\\hline
No. of Attacks: & 1 bite or 1 spit \\\hline
Damage: & 1d4 + poison bite, blindness spit \\\hline
Movement: & 30' \\\hline
No. Appearing: & 1d6, Wild 1d6, Lair 1d6 \\\hline
Save As: & Fighter: 1 \\\hline
Morale: & 7 \\\hline
Treasure Type: & None \\\hline
XP: & 37 \\\hline
\end{tabularx}\medskip

Spitting cobras average about 7' in length at adulthood. They use their spreading hood to warn other creatures not to bother them, and generally refrain from attacking if possible to allow larger creatures time to retreat. Failure to retreat from the spitting cobra will likely result in the cobra spitting venom; the cobra can project its venom up to 5', and any living creature hit must roll a save vs. Poison or be blinded permanently (though the \textbf{cure blindness} spell can heal this injury). If the cobra cannot deter a creature by spitting, it will attack using its bite. In this case, those hit must save vs. Poison or die.

\vfill

\begin{center} \includegraphics[width=0.47\textwidth]{Pictures132/10000000000003D80000054EFA61378F013FE3C8.png} \end{center}

\columnbreak


\subsection*{Spectre*}\index{Spectre}\label{spectre}

\begin{tabularx}{0.50\textwidth}{@{}lX@{}}
Armor Class: & 17 (m) \\\hline
Hit Dice: & 6** \\\hline
No. of Attacks: & 1 touch \\\hline
Damage: & Energy drain 2 levels/touch \\\hline
Movement: & Fly 100' \\\hline
No. Appearing: & 1d4, Lair 1d8 \\\hline
Save As: & Fighter: 6 \\\hline
Morale: & 11 \\\hline
Treasure Type: & E \\\hline
XP: & 610 \\\hline
\end{tabularx}\medskip

Spectres are incorporeal \textbf{undead} monsters. On any successful hit against a living creature, a spectre drains two life energy levels in addition to doing normal damage. Any character slain by a spectre will arise at the next sunset (but not sooner than 6 hours after death) as a spectre under the control of its killer.

A spectre will normally resemble the living creature it used to be. Most spectres are formed from humanoid creatures, but some may have other forms and sizes; statistically, most such creatures will be as given above, but of course the GM may create special types.

Like all \textbf{undead}, they may be Turned by Clerics and are immune to \textbf{sleep}, \textbf{charm},\textbf{ }and \textbf{hold} magics. Due to their incorporeal nature, they cannot be harmed by non-magical weapons.

\begin{center} \includegraphics[width=0.47\textwidth]{Pictures132/10000000000003D8000004A1D93F1B6AA0A16214.png} \end{center}


\subsection*{Spider, Giant Black Widow}\index{Spider, Giant Black Widow}\label{spider-giant-black-widow}

\begin{tabularx}{0.50\textwidth}{@{}lX@{}}
Armor Class: & 14 \\\hline
Hit Dice: & 3* \\\hline
No. of Attacks: & 1 bite \\\hline
Damage: & 2d6 + poison \\\hline
Movement: & 20' Web 40' \\\hline
No. Appearing: & 1d3, Wild 1d3, Lair 1d3 \\\hline
Save As: & Fighter: 3 \\\hline
Morale: & 8 \\\hline
Treasure Type: & None \\\hline
XP: & 175 \\\hline
\end{tabularx}\medskip

The giant black widow spider is a much enlarged version of the ordinary black widow; a full-grown male has a leg-span of 2 feet, while an adult female will be 3' or more across. Despite the size difference, both genders are statistically equal. Both genders are marked with an orange "hourglass" on the abdomen.

The venom of the giant black widow is strong, such that those bitten must save vs. Poison at a penalty of -2 or die. Giant black widow spiders spin strong, sticky, nearly invisible webs, usually across passageways or cave entrances, or sometimes between trees in the wilderness; those who stumble into these webs become stuck, and must roll to escape just as if opening a door. Any character stuck in such a web cannot effectively cast spells or use a weapon.

\subsection*{Spider, Giant Crab}\index{Spider, Giant Crab}\label{spider-giant-crab}

\begin{tabularx}{0.50\textwidth}{@{}lX@{}}
Armor Class: & 13 \\\hline
Hit Dice: & 2* \\\hline
No. of Attacks: & 1 bite \\\hline
Damage: & 1d8 + poison \\\hline
Movement: & 40' \\\hline
No. Appearing: & 1d4, Wild 1d4, Lair 1d4 \\\hline
Save As: & Fighter: 2 \\\hline
Morale: & 7 \\\hline
Treasure Type: & None \\\hline
XP: & 100 \\\hline
\end{tabularx}\medskip

Crab spiders are ambush predators, hiding using various forms of camouflage and leaping out to bite their surprised prey. Giant crab spiders are horribly enlarged, being around 3' in length. They can change color slowly (over the course of a few days), taking on the overall coloration of their preferred lair or ambush location. After this change is complete, the spider is able to surprise potential prey on 1-4 on 1d6 when in that preferred location. Anyone bitten by a giant crab spider must save vs. Poison or die. 


\subsection*{Spider, Giant Tarantula}\index{Spider, Giant Tarantula}\label{spider-giant-tarantula}

\begin{tabularx}{0.50\textwidth}{@{}lX@{}}
Armor Class: & 15 \\\hline
Hit Dice: & 4* \\\hline
No. of Attacks: & 1 bite \\\hline
Damage: & 1d8 + poison \\\hline
Movement: & 50' \\\hline
No. Appearing: & 1d3, Wild 1d3, Lair 1d3 \\\hline
Save As: & Fighter: 4 \\\hline
Morale: & 8 \\\hline
Treasure Type: & None \\\hline
XP: & 280 \\\hline
\end{tabularx}\medskip

Giant tarantulas are huge, hairy spiders, about the size of a pony. They run down their prey much as wolves do. The bite of the giant tarantula is poisonous; those bitten must save vs. Poison or be forced to dance wildly. 

The dance lasts 2d10 rounds, during which time the victim has a -4 penalty on attack and saving throw rolls. If the victim is a Thief, they cannot use any Thief abilities while dancing. Onlookers must save vs. Spells or begin dancing themselves; such "secondary" victims suffer the same penalties as above, but they will only dance for 2d4 rounds.

Each round the original victim dances, they must save vs. Poison again or take 1d4 points of damage. Secondary victims do not suffer this effect. \textbf{Neutralize poison} will cure the original victim, and \textbf{dispel magic} will stop the dance for all victims in the area of effect, whether they are original or secondary.

\vfill

\begin{center} \includegraphics[width=0.47\textwidth]{Pictures132/10000000000003D8000002D38A493C40B2E04DB0.png} \end{center}

\columnbreak

\subsection*{Sprite}\index{Sprite}\label{sprite}

\begin{tabularx}{0.50\textwidth}{@{}lX@{}}
Armor Class: & 15 \\\hline
Hit Dice: & ½ (1d4 hit points) * \\\hline
No. of Attacks: & 1 dagger or 1 spell \\\hline
Damage: & 1d4 dagger or by spell \\\hline
Movement: & 20' Fly 60' \\\hline
No. Appearing: & 3d6, Wild 3d6, Lair 5d8 \\\hline
Save As: & Magic-User: 4 (with Elf bonuses) \\\hline
Morale: & 7 \\\hline
Treasure Type: & S \\\hline
XP: & 13 \\\hline
\end{tabularx}\medskip

\begin{center}
\includegraphics[width=0.35\textwidth]{Pictures132/10000000000003D800000909E13335F66DE80750.png}
\end{center}

Sprites are reclusive fey creatures which resemble tiny elves just a foot tall with dragonfly-like wings. Though they may act as tricksters on occasion, they hate all forms of evil and ugliness, fighting all such foes with their tiny weapons and their magical abilities. Sprites are clever but they are not deep thinkers, and sometimes can be fooled into helping an evil creature or harming a good one due to their natural belief that evil is ugly and ugly is evil. 

Five sprites acting together can cast \textbf{remove curse}, or its reversed form \textbf{bestow curse}, once per day.

\subsection*{Squid, Giant}\index{Squid, Giant}\label{squid-giant}

\begin{tabularx}{0.50\textwidth}{@{}lXX@{}}
& Male & Female \\\hline
Armor Class: & 16 & 17 \\\hline
Hit Dice: & 6 & 7 \\\hline
No. of Attacks: & -- 8 tentacles, 1 bite -- & \\\hline
Damage: & 1d4 tentacle, 1d10 bite & 1d4 tentacle, 1d12 bite \\\hline
Movement: & -- Swim 40' -- & \\\hline
No. Appearing: & -- Wild 1d4 (see below) -- & \\\hline
Save As: & Fighter: 6 & Fighter: 7 \\\hline
Morale: & 8 & 8 \\\hline
Treasure Type: & None & None \\\hline
XP: & 500 & 670 \\\hline
\end{tabularx}\medskip

The
giant squid dwells in the deep ocean. One can grow to a tremendous size, to a maximum of around 40 feet for females and 33 feet for males. The mantle of the giant squid is about 6½ feet long (more for females, less for males). Their tentacles are studded with barbs and sharp-edged suckers.


\begin{center} \includegraphics[width=0.47\textwidth]{Pictures132/10000000000003D8000003AFABCC63C35D6CD26F.png} \end{center}

Members of any group of these creatures encountered are equally likely to be male or female. The GM may roll for this or may assign them as they see fit.

In order to bite a creature, the giant squid must hit with at least two tentacles first. Further, any time a giant squid hits with at least one tentacle per each 75 pounds of weight of its prey, it has grabbed it; unless the victim can find a way to resist (using whatever method the player might think of and whatever rolls the GM may choose), they will be pulled into the water and thus be in danger of drowning. Don' t forget to account for the weight of armor worn!

If a giant squid fails a morale check, it will squirt out a cloud of black "ink" 30' in diameter and then jet away at twice normal speed for 3d8 rounds. If a group fails a morale check they will move away in random directions in hopes that at least one will escape any pursuit.

\subsection*{Stirge}\index{Stirge}\label{stirge}

\begin{tabularx}{0.50\textwidth}{@{}lX@{}}
Armor Class: & 13 \\\hline
Hit Dice: & 1* \\\hline
No. of Attacks: & 1 bite \\\hline
Damage: & 1d4 bite, 1d4/round blood drain \\\hline
Movement: & 10' Fly 60' \\\hline
No. Appearing: & 1d10, Wild 3d12, Lair 3d12 \\\hline
Save As: & Fighter: 1 \\\hline
Morale: & 9 \\\hline
Treasure Type: & D \\\hline
XP: & 37 \\\hline
\end{tabularx}\medskip

Stirges are weird winged creatures that some say may have invaded from some other plane of existence. They are relatively small, just about 1 foot long with a wingspan of about 2 feet and an average weight of 1 pound. They vaguely resemble hairless bats with a rubbery tubular proboscis and no back legs (so that their body simply comes to a blunt point at the rear).


\begin{center} \includegraphics[width=0.47\textwidth]{Pictures132/10000000000003CF000004B78F837C00D5161E1B.jpg} \end{center}

If a stirge hits a living creature, it grabs on with hooked claws on its wing joints and quickly embeds its proboscis in the victim' s body. The proboscis has rows of tiny serrated teeth on the inside, and literally turns itself inside out as it carves a way into the victim' s body. This causes 1d4 points of damage, and the stirge then proceeds to suck the victim' s blood, inflicting an additional 1d4 points of damage each round.

Once attached, the creature can only be removed by killing it. The victim cannot use weapons larger than a dagger or hand axe to attack the creature, and cannot attack it at all if attacked from behind. Others may attack the creature with a bonus of +2 on the die roll, but any attack that misses hits the victim instead.

\subsection*{Strangle Vine}\index{Strangle Vine}\label{strangle-vine}

\begin{tabularx}{0.50\textwidth}{@{}lX@{}}
Armor Class: & 15 \\\hline
Hit Dice: & 6 \\\hline
No. of Attacks: & 1 entangle + special \\\hline
Damage: & 1d8 entangle + special \\\hline
Movement: & 5' \\\hline
No. Appearing: & 1d4+1 \\\hline
Save As: & Fighter: 6 \\\hline
Morale: & 12 \\\hline
Treasure Type: & U \\\hline
XP: & 500 \\\hline
\end{tabularx}\medskip

A strangle vine (sometimes called an assassin vine) is a strange animated plant found in temperate and tropical forests, particularly in areas with poor-quality soil. They fertilize their soil by entangling, constricting, and killing living creatures, then depositing the bodies in loose soil around the plant' s base.

Because it can lie very still indeed, a strangle vine surprises on a roll of 1-4 on 1d6. A successful hit inflicts 1d8 points of damage, and the victim becomes entangled, suffering an additional 1d8 points of damage thereafter. A victim may attempt to escape by rolling a saving throw vs. Death Ray with Strength bonus added; this is a full action, so the victim may not attempt this and also perform an attack. The plant will continue to crush its victim until one or the other is dead or the victim manages to escape.


\begin{center} \includegraphics[width=0.40\textwidth]{Pictures132/10000000000007E900000A419C1F144A0C3F5D1C.png} \end{center}

Strangle vines are actually mobile, able to uproot themselves and move slowly from place to place; one generally only does so to seek new hunting grounds. They have no visual organs but can sense foes within 30 feet by sound and vibration.

Each plant consists of a single long vine of up to 20 feet in length, with many smaller vines 5 feet or so in length packed closely, two vines per foot or thereabouts. The smaller vines are covered in leaves, and in the fall they bear clusters of reddish-purple berries which are tough and bitter but not poisonous.

There is a similar plant found in underground environments which has leaves the color of iron with pale shiny metallic veins. They grow near geothermal vents or springs, and the rotting flesh that surrounds them often supports mushrooms of various sizes and types. This fungal growth conceals the strangle vine, allowing it to surprise on 1-5 on 1d6 as does the above-ground variety of the plant.

\subsection*{Tentacle Worm}\index{Tentacle Worm}\label{tentacle-worm}

\begin{tabularx}{0.50\textwidth}{@{}lX@{}}
Armor Class: & 13 \\\hline
Hit Dice: & 3* \\\hline
No. of Attacks: & 6 tentacles \\\hline
Damage: & paralysis \\\hline
Movement: & 40' \\\hline
No. Appearing: & 1d3, Lair 1d3 \\\hline
Save As: & Fighter: 3 \\\hline
Morale: & 9 \\\hline
Treasure Type: & B \\\hline
XP: & 175 \\\hline
\end{tabularx}\medskip

Tentacle worms appear to be giant worms of some sort, averaging 6 to 8 feet long. Their heads are pasty white or grey, but their bodies vary from livid pink or purple to deep green in color. Their tentacles splay out from around the creature' s "neck." Some sages believe they are the larval form of some other monster, but this has never been proven.\medskip

\includegraphics[width=0.40\textwidth]{Pictures132/10000000000003CF000004EED0230BF46EADCAAC.png}

A tentacle worm can attack as many as three adjacent opponents. Those hit must save vs. Paralysis or be paralyzed 2d4 turns. No matter how many of a tentacle worm' s attacks hit an opponent in a given round, only one saving throw is required in each such round.

If all opponents of a tentacle worm are paralyzed, it will begin to feed upon the victims, doing 1 point of damage every 1d8 rounds until the victim is dead; if other paralyzed victims are still alive, the worm is 50\% likely to move on immediately to another still-living victim. Otherwise, it continues to eat the corpse of the slain victim for 1d4 turns.

\subsection*{Tiger}\index{Tiger}\label{tiger}

\begin{tabularx}{0.50\textwidth}{@{}lX@{}}
Armor Class: & 14 \\\hline
Hit Dice: & 6 \\\hline
No. of Attacks: & 2 claws, 1 bite \\\hline
Damage: & 1d6 claw, 2d6 bite \\\hline
Movement: & 50' \\\hline
No. Appearing: & Wild 1d3, Lair 1d3 \\\hline
Save As: & Fighter: 6 \\\hline
Morale: & 9 \\\hline
Treasure Type: & None \\\hline
XP: & 500 \\\hline
\end{tabularx}\medskip


The tiger is among the largest great cat species, with male specimens averaging 10 feet in length (including about 2½ feet of tail) and weighing over 400 pounds. Females are smaller, averaging about 8 feet long and an average of about 275 pounds. Tigers are most recognizable for their dark vertical stripes on orange fur with a white underside. 

Tigers are apex predators and prefer prey such as deer and wild boar. They are territorial and generally solitary but social predators, requiring large contiguous areas of habitat to support their requirements for prey.

\subsection*{Titanothere}\index{Titanothere}\label{titanothere}

\begin{tabularx}{0.50\textwidth}{@{}lX@{}}
Armor Class: & 15 \\\hline
Hit Dice: & 12 (+10) \\\hline
No. of Attacks: & 1 butt or 1 trample \\\hline
Damage: & 2d6 butt, 3d8 trample \\\hline
Movement: & 40' (10') \\\hline
No. Appearing: & Wild 1d6 \\\hline
Save As: & Fighter: 8 \\\hline
Morale: & 7 \\\hline
Treasure Type: & None \\\hline
XP: & 1,875 \\\hline
\end{tabularx}\medskip

A titanothere is a huge prehistoric animal that resembles the rhinoceros; adults average 10' tall and 13' long. They have large, forked horns rather than the pointed horns of rhinos. Like rhinos, they are herd animals, and males aggressively defend the herd; females only enter combat if all males are defeated or the attackers are very numerous. If a single titanothere is encountered, it will be a rogue male; they are bad tempered and prone to attacking smaller creatures that enter their territory.

\subsection*{Treant}\index{Treant}\label{treant}

\begin{tabularx}{0.50\textwidth}{@{}lX@{}}
Armor Class: & 19 \\\hline
Hit Dice: & 8* \\\hline
No. of Attacks: & 2 fists \\\hline
Damage: & 2d6 fist \\\hline
Movement: & 20' \\\hline
No. Appearing: & Wild 1d8, Lair 1d8 \\\hline
Save As: & Fighter: 8 \\\hline
Morale: & 9 \\\hline
Treasure Type: & C \\\hline
XP: & 945 \\\hline
\end{tabularx}\medskip

Treants are a race of large, roughly humanoid tree-people. When one stands still with its legs together it cannot be easily distinguished from a normal oak tree. Their leaves are nearly identical to oak leaves, and are green in spring and summer, turning orange, red, or yellow in the fall and winter. A treant' s leaves do not normally fall out in the winter, but as some oaks also retain their leaves into cold weather this may not help in identifying one.

\begin{center} \includegraphics[width=0.47\textwidth]{Pictures132/10000000000003CF0000049657DEA254985FBC30.png} \end{center}

Being actual trees, treants are big, averaging 25 to 35 feet in height and weighing 4,000 to 5,000 pounds. Their language is difficult for most other races to learn, though Elves are known to have an advantage in mastering it. Many treants know Elvish, and some who live near humans learn Common as well.

Treants are slow to act when potential enemies are nearby, preferring watch them carefully before taking any action. Because of their excellent camouflage treants gain surprise on 1-4 on 1d6.

All treants have the power to animate other normal trees in their area; up to two trees can be animated at once, but the treant can release one and animate another if needed. Trees to be animated must be within 180 feet of the treant, and must remain within that range or they will return to their normal state. An animated tree requires a full round to uproot itself before it can move around, and then can move at a rate of just 10' per round. Of course, if enemies are within reach the animated tree need not be uprooted. If the treant controlling an animated tree is slain or incapacitated, the tree returns to its normal state.

Note that animated trees will automatically root themselves when returning to their normal, dormant state; if such a tree finds itself on stone or some other surface it cannot penetrate, it will move around andomly for 1d6 rounds until it accidentally finds a suitable area, or failing that, until it topples and falls. Any character or creature nearby when this happens must save vs. Death Ray to scamper away; those who fail will suffer 2d6 points of damage and will be trapped under the fallen tree. The GM should consider battlefield conditions when deciding on the exact results of such an event.

\subsection*{Troglodyte}\index{Troglodyte}\label{troglodyte}

See \textbf{Lizard Man }on page
\hyperlink{lizard-man}{\pageref{lizard-man}}.

\subsection*{Troll (and Trollwife)}\index{Troll (and Trollwife)}\label{troll-and-trollwife}

\begin{tabularx}{0.50\textwidth}{@{}lXX@{}}
& Troll & Trollwife \\\hline
Armor Class: & 16 & 17 \\\hline
Hit Dice: & 6* & 7* \\\hline
No. of Attacks: & -- 2 claws, 1 bite -- & \\\hline
Damage: & 1d6 claw, 1d10 bite & 1d8 claw, 2d6 bite \\\hline
Movement: & 40' & 40' \\\hline
No. Appearing: & 1d8, Wild 1d8, Lair 1d8 & 1 (special, see below) \\\hline
Save As: & Fighter: 6 & Fighter: 7 \\\hline
Morale: & 10 (8) & 10 (8), see below \\\hline
Treasure Type: & D & D \\\hline
XP: & 555 & 735 \\\hline
\end{tabularx}\medskip

Trolls are huge, rangy humanoids with lumpy skin that is a dull grayish green in color. They stand up to 9 feet tall despite having a rather hunched posture, and may weigh as much as 600 pounds. Their skin is rubbery and slightly damp to the touch, and they have long sharp black claws and long sharp white teeth. Trolls have a disconcerting tendency to smile toothily most of the time, as if their brutal lives are the most entertaining thing imaginable.

Trolls have the power of regeneration; they heal 1 hit point of damage each round after being injured. A troll reduced to 0 hit points is not dead, but only disabled for 2d6 rounds, at which point it will regain 1 hit point. Note that the troll may "play dead" until it has regenerated further. Damage from fire and acid cannot be regenerated, and must heal at the normal rate; a troll can only be killed by this sort of damage. The lower morale rating (in parentheses) is used when the troll faces attackers armed with fire or acid.

The regenerative power of trolls is so great that limbs or other body parts (even a head!) can be reattached if severed simply by pressing the severed ends back together for a moment. Trolls in a group will generally help dismembered fellows to reassemble themselves, but only if it' s convenient. If the severed part is not restored, a new one will grow in its place in 1d4 turns. Note that a troll with a new head will not remember its former life, nor will it yet know how to speak; it will behave as would any confused and hostile animal.

Trolls speak a primitive language, and are often fluent in Goblin, Hobgoblin, Orc, Ogre, or Giant depending on which of these species live nearest them. A few (20\% or so) speak Common.

Trolls are hateful creatures, reveling in combat and bloodshed. Though trolls could easily use a variety of weapons, they much prefer the sensation of flesh being rent by their teeth and claws. 

A \textbf{trollwife }is a female troll; despite the name, there is no requirement that she be married (nor, in fact, do trolls normally engage in formal marriages). A typical adult trollwife stands 11 feet tall and weighs 700 pounds. They have no outward appearance of femininity, at least according to the standards of humans, elves, or even orcs; rather, a trollwife simply looks like an extraordinarily large troll. Like a normal male troll, a trollwife has lumpy skin that is a dull grayish green in color.

Trollwives have all the abilities and weaknesses of the males of the species; in particular, they \textbf{regenerate} exactly as do the males.

When encountered, a trollwife may be alone, cohabitating with a male (her "husband"), or raising a brood of trollkin. Roll 1d10; on a result of 1, she is living alone; on a roll of 2-3, she is raising her young; on 4 or higher, she is living with a male. If one has a mate or offspring, there is a 1-3 on 1d10 chance she is encountered alone, 4-7 that her mate or young are encountered in her absence, or 8-10 that all are present.

Add 1 to the trollwife' s morale score if she is with her mate, or 2 if she has young present. This means that, unless threatened with fire or acid, a trollwife will fight without checking morale while her offspring are present. If a trollwife' s mate or offspring are slain in her absence, she will track the killers unerringly, and upon finding them will attack with the same morale bonus.

Trollwives are solitary with respect to other adult trollwives, for they hate each other with a fierce passion. If forced together they will put aside their enmity until all non-troll enemies are dead (at which point they may well fight over who will eat the choicest of the remains). 

\end{multicols}

\begin{center} \includegraphics[width=1\textwidth]{Pictures132/100000000000079E0000034D6B80C634EF67774F.png} \end{center}

\begin{multicols}{2}
	
\subsection*{Trollkin}\index{Trollkin}\label{trollkin}

\begin{tabularx}{0.50\textwidth}{@{}lXXX@{}}
& \textbf{Infant} & \textbf{Juvenile} & \textbf{Adolescent} \\\hline
Armor Class: & 14 & 15 & 16 \\\hline
Hit Dice: & 1*-2* & 3*-4* & 5*-6* \\\hline
No. of Attacks: & \multicolumn{3}{c}{2 claws, 1 bite} \\\hline
Damage: & 1d4 claw, 1d4 bite & 1d4 claw, 1d6 bite & 1d6 claw, 1d6 bite \\\hline
Movement: & 30' & 50' & 40' \\\hline
No. Appearing: &  \multicolumn{3}{c}{special, see below}\\\hline
Save As: & Fighter:1-2 & Fighter:3-4 & Fighter:5-6 \\\hline
Morale: & \multicolumn{3}{c}{9 (7)} \\\hline
Treasure Type: & \multicolumn{3}{c}{None} \\\hline
XP: &1 HD 37  2 HD 100 & 3 HD 175, 4 HD 280 & 5 HD 405 6 HD 555 \\\hline
\end{tabularx}\medskip

Trollkin are young trolls. They have all the powers and weaknesses of trolls, and look exactly like smaller than normal adult trolls. Even an infant has the same ability to \textbf{regenerate} as an adult troll. 

When you encounter trollkin, you can rest assured that there is a trollwife nearby (unless, of course, you' ve already slain her). They are as bloodthirsty as their parents; as such, determining the number appearing is done in a particularly unusual fashion:

Roll 1d6 for the number of individuals, and 2d6 for the number of hit dice. Divide the number of hit dice by the number of individuals to arrive at the hit dice of each individual. Note that a trollkin won' t be encountered having more than 6 hit dice, so if only one individual is indicated by the 1d6 roll but the 2d6 roll totals more than 6, you must increase the number of individuals. The referee should feel free to round the number of hit dice up or down as they see fit, or to allocate them in an approximately equal fashion if desired. Trollkin broods are rolled in this way owing to the fact that bigger or tougher individuals are likely to kill and eat the weaker ones, generally when their mother is out hunting.

Refer to the entry for trolls for details regarding regeneration, morale checks, and so on; except as noted above, trollkins share all these features with the adults.

\subsection*{Turtle or Tortoise}\index{Turtle or Tortoise}\label{turtle-or-tortoise}

\begin{tabularx}{0.50\textwidth}{@{}lXX@{}}
& \textbf{Box Turtle} & \textbf{Snapping Turtle} \\\hline
Armor Class: & 15 & 16 \\\hline
Hit Dice: & ½ (1d4 hit points) & 1 \\\hline
No. of Attacks: & 1 bite & 1 bite \\\hline
Damage: & 1d2 & 1d6 \\\hline
Movement: & 5' Swim 20' &
5' Swim 20' \\\hline
No. Appearing: & Wild 1d4 & Wild 1d4 \\\hline
Save As: & Normal Man & Fighter: 1 \\\hline
Morale: & 5 & 6 \\\hline
Treasure Type: & None & None \\\hline
XP: & 10 & 25 \\\hline
\end{tabularx}\medskip

Turtles and tortoises are reptiles with a hard shell into which the animal can pull its head and legs if threatened. Turtles will be found in marshes and near rivers or ponds, while tortoises are terrestrial and typically found in arid regions. (The specific names given to these animals are often misleading, as a tortoise might be called a turtle and vice versa.) The statistics given above are representative, and can be used for other species as needed. These animals are well-camouflaged, gaining surprise on a roll of 1-3 on 1d6 in their natural habitat.

\subsection*{Unicorn (and Alicorn)}\index{Unicorn (and Alicorn)}\label{unicorn-and-alicorn}

\begin{tabularx}{0.50\textwidth}{@{}lXX@{}}
 & Unicorn & Alicorn \\\hline
Armor Class: & 19 & 19 \\\hline
Hit Dice: & 4* & 4* \\\hline
No. of Attacks: & 2 hooves, 1 horn (+3 attack bonus) & 2 hooves, 1 horn \\\hline
Damage: & 1d8 hoof, 1d6+3 horn & 2d4 hoof, 2d6 horn \\\hline
Movement: & 80' & 70' \\\hline
No. Appearing: & Wild 1d6 & Wild 1d8 \\\hline
Save As: & Fighter: 8 & Fighter: 6 \\\hline
Morale: & 7 & 9 \\\hline
Treasure Type: & None & None \\\hline
XP: & 280 & 280 \\\hline
\end{tabularx}\medskip

Unicorns are horse-like creatures having a single spirally-twisted horn in the middle of the forehead. This horn is in fact a magic weapon with an added attack and damage bonus of +3; if removed from the unicorn or the unicorn dies the horn loses this effect. A unicorn may perform a charging attack with its horn, but it cannot also attack with its hooves in the same round.

Most unicorns are white, sometimes with the faintest tinge of lavender, pink, or gold visible in sunlight. Adult male unicorns have beards similar to those of goats. The eyes of a unicorn are almost always a bright, often unlikely color such as gold, sea-blue, deep green, violet, or even a luminous-looking magenta. Adult male unicorns stand an average of 5 feet at the shoulder and weigh 1,100 to 1,300 pounds. Females are slightly smaller, generally 4¾ feet at the shoulder with a weight from 950 to 1,200 pounds.

Three times per day a unicorn can cast \textbf{cure light wounds }by a touch of its horn. Once per day a unicorn can transport itself 360' (as the spell \textbf{dimension door}), and can carry a full load (possibly including a rider) while doing so. A light load for a unicorn is up to 300 pounds; a heavy load, up to 550 pounds.

An alicorn resembles a unicorn in all details, save that they always have yellow, orange or red eyes, and (if one gets close enough to see) pronounced, sharp canine teeth. Alicorns are as evil as unicorns are good, using their razor-sharp horns and clawlike hooves as weapons. They attack weaker creatures for the sheer pleasure of killing, but will try to avoid stronger parties. They may make charging attacks just as unicorns do.

Alicorns cannot heal or transport themselves by magic as unicorns do. However, alicorns may become invisible at will, exactly as if wearing a \textbf{Ring of Invisibility}.

\begin{center} \includegraphics[width=0.47\textwidth]{Pictures132/10000000000003D80000055A93B54E74FCE700A6.png} \end{center}


\subsection*{Urgoblin}\index{Urgoblin}\label{urgoblin}

\begin{tabularx}{0.50\textwidth}{@{}lX@{}}
Armor Class: & 14 (11) \\\hline
Hit Dice: & 2* \\\hline
No. of Attacks: & 1 weapon \\\hline
Damage: & 1d8 or by weapon \\\hline
Movement: & 30' Unarmored 40' \\\hline
No. Appearing: & Special \\\hline
Save As: & Fighter: 2 \\\hline
Morale: & 9 \\\hline
Treasure Type: & Q, R, S each; special in lair \\\hline
XP: & 100 \\\hline
\end{tabularx}\medskip

These creatures appear to be normal \textbf{hobgoblins}, but urgoblins are actually a mutant subspecies. Urgoblins are able to regenerate much as do \textbf{trolls} (with the same limitations). All urgoblins are male; if an urgoblin mates with a female hobgoblin, any offspring will also be male, but only one in four such offspring will share their father' s gifts. Like hobgoblins, urgoblins wear toughened hides and carry wooden shields into battle, blending in perfectly.

Some hobgoblin tribes consider urgoblins an abomination, and kill them whenever they can be identified. Other hobgoblin tribes employ them as bodyguards for the chieftain, and accord them great honor. There are even rumors of a tribe entirely made up of urgoblins, with kidnapped hobgoblin females as their mates; reportedly they slit the throats of all infants born to their mates, so that only those who have the power of regeneration survive.

\subsection*{Vampire*}\index{Vampire}\label{vampire}

\begin{tabularx}{0.50\textwidth}{@{}lX@{}}
Armor Class: & 18 to 20 (m) \\\hline
Hit Dice: & 7** to 9** (+8) \\\hline
No. of Attacks: & 1 weapon or special \\\hline
Damage: & 1d8 or by weapon or special \\\hline
Movement: & 40' Fly 60' (as giant bat,
see below) \\\hline
No. Appearing: & 1d6, Wild 1d6, Lair 1d6 \\\hline
Save As: & Fighter: 7 to 9 (as Hit Dice) \\\hline
Morale: & 11 \\\hline
Treasure Type: & F \\\hline
XP: & 800 - 1,225 \\\hline
\end{tabularx}\medskip

Vampires are \textbf{undead} monsters. Though they may look just a bit more pale than when they were alive, they do appear to live and even breathe as mortals do. A vampire has all the memories and abilities it had in life, and is effectively immortal. Living forever in the shadows often leads to vampires being decadent, while their hunger for blood makes them cruel.

Vampires cast no reflections in silvered mirrors or water, though despite conventional wisdom to the contrary they do reflect in shiny base metals; the innate purity of silver and water simply will not reflect them.

A vampire can charm anyone who meets its gaze; a save vs. Spells is allowed to resist, but at a penalty of -2 due to the power of the charm. This charm is so powerful that the victim will not resist being bitten by the vampire.

The bite inflicts 1d3 points of damage, then each round thereafter one energy level is drained from the victim. The vampire regenerates (i.e. is healed) for up to 1d6 hit points for each energy level drained. If the victim dies from the energy drain, they will arise as a vampire at the next sunset (but not less than 12 hours later) and thereafter will be under the complete control of the "parent" vampire (but is freed if that vampire is ever destroyed).

If using the bite attack, the vampire suffers a penalty of -5 to Armor Class due to the vulnerable position it must assume. For this reason, the bite is rarely used in combat. Vampires have great Strength, gaining a bonus of +3 to damage when using melee weapons, and thus a vampire will generally choose to use a melee weapon (or even its bare hands) in combat rather than attempting to bite.

Vampires are unharmed by non-magical weapons, and like all undead are immune to \textbf{sleep}, \textbf{charm},\textbf{ }and \textbf{hold }spells. If reduced to 0 hit points in combat, the vampire is not destroyed, though it may appear to be. The vampire will begin to regenerate 1d8 hours later, recovering 1 hit point per turn, and resuming normal activity as soon as the first point is restored.

Creatures of the night obey vampires. Once per day a vampire can call forth 10d10 rats, 5d4 giant rats, 10d10 bats, 3d6 giant bats, or a pack of 3d6 wolves (assuming any such creatures are nearby). The creatures summoned arrive 2d6 rounds later and will obey the vampire for up to an hour before leaving. 

Vampires can also transform into the form of these creatures, assuming the shape of a giant bat (page \hyperlink{bat-and-bat-giant}{\pageref{bat-and-bat-giant}}), giant rat (page \hyperlink{rat-and-rat-giant}{\pageref{rat-and-rat-giant}}), or a dire wolf (page \hyperlink{wolf-and-wolf-dire}{\pageref{wolf-and-wolf-dire}}) at will. The transformation requires a single round during which the vampire cannot attack. The flying movement given above is for the giant bat form; the vampire can use the attack forms of the animal shape assumed, as given on the indicated pages. The vampire cannot activate any of its other powers while in animal form, but effects already active will remain so. 

Vampires are powerful indeed, but they do have several specific weaknesses.

First, a vampire can be \textbf{held at bay} by several things, including the smell of garlic, a silver or silvered mirror, or a holy symbol presented by a believer (GM' s discretion is advised here, but in general someone threatened by a vampire may be more devout than usual). In such a situation the vampire cannot approach within 5 feet of the repellent item or character, nor can it make any melee attacks or otherwise touch those within the warded area. 

\begin{center} \includegraphics[width=0.47\textwidth]{Pictures132/10000000000003D8000004CC3E43B823F6C9DFE2.png} \end{center}

\textbf{Running water} (such as a stream or river) acts as a barrier toa vampire; one cannot cross over any such waterway, neither by bridge or by waterway or even by flying. It is possible for a vampire to be carried across while lying in its own coffin with the lid closed; the presence of the water flowing beneath the coffin will force the vampire to remain dormant.

Finally, a vampire may not enter any private dwelling without being invited by someone who resides there, and then only if invited while the resident is actually inside the structure. Public buildings of any sort do not present this problem; even inns are fully accessible to a vampire.

\textbf{Destroying a vampire }is not an easy task. When reduced to zero hit points or less a vampire becomes incapacitated but is not slain (as previously explained). \textbf{If exposed to direct sunlight}, the vampire will be utterly destroyed in a single round if it cannot find a way to escape in that time. \textbf{Being immersed in running water} (as defined above) causes 3d8 points of damage each round, and if reduced to zero or fewer hit points in this way the vampire is destroyed. If the vampire is \textbf{cremated in a funeral pyre} while in any dormant state it will be destroyed.

The most dramatic method of defeating a vampire is actually the least reliable: \textbf{driving a wooden stake through its heart}. Doing this will instantly reduce the vampire to zero hit points, and it will remain "dead" for so long as the stake is present. Removal of the stake, however, allows the vampire to begin regenerating as described above. It is not normally possible to drive a stake through a vampire' s heart while it is actively resisting, but this can be done after reducing the monster to zero hit points in the usual  way. Of course, after staking a vampire it is possible to use sunlight, running water, or a funeral pyre as described above to complete its destruction.

\subsection*{Water Termite, Giant}\index{Water Termite, Giant}\label{water-termite-giant}

\begin{tabularx}{0.50\textwidth}{@{}lX@{}}
Armor Class: & 13 \\\hline
Hit Dice: & 1 to 4 \\\hline
No. of Attacks: & 1 spray \\\hline
Damage: & Stun \\\hline
Movement: & Swim 30' \\\hline
No. Appearing: & Wild 1d4 \\\hline
Save As: & Fighter: 1 to 4 (as Hit Dice) \\\hline
Morale: & 10 \\\hline
Treasure Type: & None \\\hline
XP: & \vtop{\hbox{\strut 1 HD 25, 2 HD 75}\hbox{\strut 3 HD 145, 4 HD
240}} \\\hline
\end{tabularx}\medskip

Giant water termites vary from 1' to 5' in length. They attack using a noxious spray with a range of 5' which stuns the target for a full turn on a hit; a save vs. Poison is allowed to avoid the effect. A stunned character can neither move nor take action for the remainder of the current round and all of the next one.  However, the primary concern regarding these monsters is the damage they can do to boats and ships. Each creature can do 2d4 points of damage to a ship' s hull per round (no roll required) for a number of rounds equal to 1d4 plus the creature' s hit dice total; after this time, the monster is full. They eat noisily.

These creatures are found in fresh and salt water as well as in swamps. The freshwater variety tend to be smaller, 1-2 hit dice, the saltwater variety 3-4 hit dice, and those found in swamps range from 2-3 hit dice.

\subsection*{Weasel, Normal and Giant (or Ferret)}\index{Weasel, Normal and Giant (or Ferret)}\label{weasel-normal-and-giant-or-ferret}

\begin{tabularx}{0.50\textwidth}{@{}lXX@{}}
& Normal & Giant \\\hline
Armor Class: & 14 & 17 \\\hline
Hit Dice: & 1d2 hit points & 5 \\\hline
No. of Attacks: & 1 bite + hold & 1 bite + hold \\\hline
Damage: & 1d4 + 1d4/round & 2d4 + 2d4/round \\\hline
Movement: & 40' & 50' \\\hline
No. Appearing: & 1d6, Wild 1d8, Lair 1d8 & 1d4, Wild 1d6, Lair 1d6 \\\hline
Save As: & Fighter: 1 & Fighter: 5 \\\hline
Morale: & 7 & 8 \\\hline
Treasure Type: & None & V \\\hline
XP: & 10 & 360 \\\hline
\end{tabularx}\medskip

Normal weasels (or ferrets, see below) are small mammals with long bodies, short legs, and pointed, toothy snouts. They are predatory animals, hunting those creatures smaller than themselves. They are cunning, crafty hunters, and gain surprise on 1-3 on 1d6. Once one bites prey (i.e. a living creature smaller than itself), it hangs on, rending with its teeth each round until the victim or the weasel is dead. Should  one fail a morale check it will release its victim and flee.

There are many varieties of normal-sized weasel, including several which are called ferrets; in some territories, the giant weasel is thus called a giant ferret. The only distinction is that those which are tamed are always called ferrets, though not all giant ferrets are tame. Various humanoid races as well as some fairy creatures are known to tame giant ferrets for use as guards or war-animals.

Giant weasels resemble their more normally sized cousins, and other than their greater size they behave in exactly the same fashion.

\subsection*{Whale, Killer}\index{Whale, Killer}\label{whale-killer}



\begin{center} \includegraphics[width=0.20\textwidth]{Pictures132/10000000000003D800000524199B66E8B5BC78AD.png} \end{center}


\begin{tabularx}{0.50\textwidth}{@{}lX@{}}
Armor Class: & 17 \\\hline
Hit Dice: & 6 \\\hline
No. of Attacks: & 1 bite \\\hline
Damage: & 2d10 \\\hline
Movement: & Swim 80' (10') \\\hline
No. Appearing: & Wild 1d6 \\\hline
Save As: & Fighter: 6 \\\hline
Morale: & 10 \\\hline
Treasure Type: & None \\\hline
XP: & 500 \\\hline
\end{tabularx}


Killer whales, also called "orca" (both singular and plural), are large predatory cetaceans related to both dolphins and true whales. Adult males range from 20 to 26 feet long, while females are smaller at 16 to 23 feet in length. All are strikingly marked in black and white, with prominent white patches that resemble eyes. Their real eyes are much smaller and located away from the fake eye-spots.

Killer whales are apex predators, meaning that they themselves have no natural predators. They hunt in groups like wolf packs, subsisting primarily on fish, cephalopods, mammals, seabirds, and sea turtles. However, they are not above consuming a meal of humanoid nature if such becomes available. 


\subsection*{Whale, Narwhal}\index{Whale, Narwhal}\label{whale-narwhal}

\begin{tabularx}{0.50\textwidth}{@{}lX@{}}
Armor Class: & 19 \\\hline
Hit Dice: & 12 (+10) \\\hline
No. of Attacks: & 1 horn \\\hline
Damage: & 2d6 \\\hline
Movement: & Swim 60' \\\hline
No. Appearing: & Wild 1d4 \\\hline
Save As: & Fighter: 6 \\\hline
Morale: & 8 \\\hline
Treasure Type: & Special \\\hline
XP: & 1,875 \\\hline
\end{tabularx}\medskip

Narwhals are aquatic mammals resembling large dolphins with a single (or rarely, double) tusk protruding straight forward from the mouth. The tusk is helical in shape, and they are sometimes cut short and sold as "unicorn horns." However, they have no particular magical value. Narwhals are found in cold northern seas. They are not particularly aggressive.

\subsection*{Whale, Sperm}\index{Whale, Sperm}\label{whale-sperm}

\begin{tabularx}{0.50\textwidth}{@{}lX@{}}
Armor Class: & 22 \\\hline
Hit Dice: & 36* (+16)  \\\hline
No. of Attacks: & 1 bite or special \\\hline
Damage: & 3d20 \\\hline
Movement: & Swim 60' (20') \\\hline
No. Appearing: & Wild 1d3 \\\hline
Save As: & Fighter: 8 \\\hline
Morale: & 7 \\\hline
Treasure Type: & None \\\hline
XP: & 17,850 \\\hline
\end{tabularx}

Sperm whales are huge creatures, with males averaging 52 feet long at adulthood while females typically reach a length of around 36 feet. They are predators, hunting primarily giant squid. Sperm whales can emit an invisible focused beam of sound 5'  wide up to a 50' range underwater. This blast of sound disorients target creatures, leaving them effectively stunned for 1d4 rounds. A stunned character can neither move nor take action for the indicated duration. No attack roll is required, but a save vs. Death Ray is allowed to resist. A sperm whale can emit as many such blasts of sound  as it desires, once per round, instead of biting.

\subsection*{Wight*}\index{Wight}\label{wight}

\begin{tabularx}{0.50\textwidth}{@{}lX@{}}
Armor Class: & 15 (s) \\\hline
Hit Dice: & 3* \\\hline
No. of Attacks: & 1 touch \\\hline
Damage: & Energy drain (1 level) \\\hline
Movement: & 30' \\\hline
No. Appearing: & 1d6, Wild 1d8, Lair 1d8 \\\hline
Save As: & Fighter: 3 \\\hline
Morale: & 12 \\\hline
Treasure Type: & B \\\hline
XP: & 175 \\\hline
\end{tabularx}\medskip


\begin{center} \includegraphics[width=0.47\textwidth]{Pictures132/10000001000003D8000004F998C9FEA713515B65.png} \end{center}

Wights are \textbf{undead }monsters who have been twisted and deformed by their transformation. Their eyes are entirely black, and their bodies radiate a coldness that living creatures can feel from several feet away. 

If a wight touches or is touched by a living creature, that creature suffers one level of \textbf{energy drain} (as described in the \textbf{Encounter} section). No saving throw is allowed. Striking a wight with a weapon does not count as "touching" it, but punching or kicking one does.

Any humanoid slain by a wight becomes a wight by the next sunset (but not less than 12 hours later). They are slaves to the wight who created them until and unless that wight is destroyed. Wights remember almost nothing from their previous life, though a few very close friends or loved ones might be recalled (and hated, and possibly hunted by the monster).

Like all undead, wights may be Turned by Clerics and are immune to \textbf{sleep}, \textbf{charm},\textbf{ }and \textbf{hold} magics. Wights are harmed only by silver or magical weapons, and take only half damage from burning oil.



\subsection*{Wolf (and Wolf, Dire)}\index{Wolf (and Wolf, Dire)}\label{wolf-and-wolf-dire}

\begin{tabularx}{0.50\textwidth}{@{}lXX@{}}
& Normal & Dire \\\hline
Armor Class: & 13 & 14 \\\hline
Hit Dice: & 2 & 4 \\\hline
No. of Attacks: & 1 bite & 1 bite \\\hline
Damage: & 1d6 & 2d4 \\\hline
Movement: & 60' & 50' \\\hline
No. Appearing: & 2d6, Wild 3d6, Lair 3d6 & 1d4, Wild 2d4, Lair 2d4 \\\hline
Save As: & Fighter: 2 & Fighter: 4 \\\hline
Morale: & 8 & 9 \\\hline
Treasure Type: & None & None \\\hline
XP: & 75 & 240 \\\hline
\end{tabularx}\medskip

The wolf is a large canine found in a broad range of habitats. They travel in packs consisting of a mated pair accompanied by their offspring. Wolves are also territorial, and fights over territory are among the principal causes of wolf mortality. The wolf is mainly a carnivore and feeds on large wild hoofed mammals as well as smaller animals, livestock, and carrion.

Dire wolves are huge relatives of the ordinary wolves, being as large as horses. They live and hunt in packs, and are sometimes tamed by smaller humanoids as battle steeds or by larger ones as pets.

\begin{center} \includegraphics[width=0.47\textwidth]{Pictures132/10000000000003D8000003605AECE75F652E6F5C.png} \end{center}


\subsection*{Wraith*}\index{Wraith}\label{wraith}

\begin{tabularx}{0.50\textwidth}{@{}lX@{}}
Armor Class: & 15 (m) \\\hline
Hit Dice: & 4** \\\hline
No. of Attacks: & 1 touch \\\hline
Damage: & 1d6 + energy drain (1 level) \\\hline
Movement: & Fly 80' \\\hline
No. Appearing: & 1d4, Lair 1d6 \\\hline
Save As: & Fighter: 4 \\\hline
Morale: & 12 \\\hline
Treasure Type: & E \\\hline
XP: & 320 \\\hline
\end{tabularx}\medskip

Wraiths are fearsome incorporeal monsters who drain the very life energy from their victims. Being incorporeal, they are immune to non-magical weapons and are effectively flying at all times. They are semitransparent and exude the same kind of coldness as wights. Wraiths usually appear as they were in life, including such things as clothing or armor, though such things do not affect the monsters actual armor class or other statistics.

Like all undead, they may be Turned by Clerics and are immune to \textbf{sleep}, \textbf{charm},\textbf{ }and \textbf{hold} magics. Due to their incorporeal nature, they cannot be harmed by non-magical weapons.

\begin{center} \includegraphics[width=0.47\textwidth]{Pictures132/10000000000003D8000005244F3FEFED907ACCBB.png} \end{center}

\subsection*{Wyvern}\index{Wyvern}\label{wyvern}

\begin{tabularx}{0.50\textwidth}{@{}lX@{}}
Armor Class: & 18 \\\hline
Hit Dice: & 7* \\\hline
No. of Attacks: & 1 bite, 1 stinger or 2 talons, 1 stinger \\\hline
Damage: & \vtop{\hbox{\strut 2d8 bite, 1d6 + poison
sting,}\hbox{\strut 1d10 talon}} \\\hline
Movement: & 30' (10') Fly
80' (15') \\\hline
No. Appearing: & Wild 1d6, Lair 1d6 \\\hline
Save As: & Fighter: 7 \\\hline
Morale: & 9 \\\hline
Treasure Type: & E \\\hline
XP: & 735 \\\hline
\end{tabularx}\medskip

Wyverns are large dragon-like monsters, though they are built more like bats than lizards, having two legs and two wings; contrast this with true dragons, which have four legs and two wings. They are of animal intelligence, but are excellent predators with good hunting abilities. 

Wyverns are aggressive and always hungry, and will attack almost anything smaller than themselves. Any living creature hit by the wyvern' s stinger must save vs. Poison or die. Wyverns can only use their talons to attack when landing or flying past an enemy; while on the ground they may only bite and sting. Further, a wyvern may not bite when flying past an opponent. The only situation in which all attacks can be made is when the monster is landing. 

If a wyvern hits with both its talons, it may attempt to carry off its victim; victims weighing 300 pounds or less can be carried off, and the wyvern can only carry a victim for at most 6 rounds. While flying with a victim, the wyvern cannot make any further attacks against it, but of course if the victim makes a nuisance of itself (such as by injuring the wyvern), it may be dropped.

\begin{center} \includegraphics[width=0.47\textwidth]{Pictures132/10000000000006F10000065459421699438C7377.png} \end{center}

\subsection*{Yellow Mold}\index{Yellow Mold}\label{yellow-mold}

\begin{tabularx}{0.50\textwidth}{@{}lX@{}}
Armor Class: & Can always be hit \\\hline
Hit Dice: & 2*  \\\hline
No. of Attacks: & Special \\\hline
Damage: & See below \\\hline
Movement: & 0 \\\hline
No. Appearing: & 1d8 \\\hline
Save As: & Normal Man \\\hline
Morale: & N/A \\\hline
Treasure Type: & None \\\hline
XP: & 100 \\\hline
\end{tabularx}

If disturbed, a patch of this mold bursts forth with a cloud of poisonous spores. Each patch covers from 10 to 25 square feet; several patches may grow adjacent to each other, and will appear to be a single patch in this case. Each patch can emit a cloud of spores once per day. All within 10 feet of the mold will be affected by the spores and must save vs. Death Ray or take 1d8 points of damage per round for 6 rounds. Sunlight renders yellow mold dormant.

\begin{center} \includegraphics[width=0.47\textwidth]{Pictures132/10000000000003D80000050BD4D6BEE94A428785.png} \end{center}

\subsection*{Zombie}\index{Zombie}\label{zombie}

\begin{tabularx}{0.50\textwidth}{@{}lX@{}}
Armor Class: & 12 (see below) \\\hline
Hit Dice: & 2 \\\hline
No. of Attacks: & 1 bludgeon or 1 weapon \\\hline
Damage: & 1d8 or by weapon \\\hline
Movement: & 20' \\\hline
No. Appearing: & 2d4, Wild 4d6 \\\hline
Save As: & Fighter: 2 \\\hline
Morale: & 12 \\\hline
Treasure Type: & None \\\hline
XP: & 75 \\\hline
\end{tabularx}\medskip

Zombies are the \textbf{undead} corpses of humanoid creatures. They are deathly slow, but they move silently, are very strong and must be literally hacked to pieces to "kill" them. They take only half damage from blunt weapons, and only a single point from arrows, bolts or sling stones (plus any magical bonus). A zombie never has Initiative and always acts last in any given round.

As zombies are strong and do not feel pain, they can bludgeon enemies with both fists for 1d8 points of damage. However, their creators often arm them, either for greater damage (as with a greatsword or polearm) or simply for effect.

Like all undead, they may be Turned by Clerics and are immune to \textbf{sleep}, \textbf{charm},\textbf{ }and \textbf{hold} magics. As they are mindless, no form of \textbf{mind reading} is of any use against them. Zombies never fail morale checks, and thus always fight until destroyed.

\begin{center} \includegraphics[width=0.47\textwidth]{Pictures132/10000000000003D800000388A2A31CAE5ADDAFBD.png} \end{center}

\subsection*{Zombraire (and Skeletaire)}\index{Zombraire (and Skeletaire)}\label{zombraire-and-skeletaire}

\begin{tabularx}{0.50\textwidth}{@{}lXX@{}}
& Zombraire & Skeletaire \\\hline
Armor Class: & 12 (see below) & 13 (see below) \\\hline
Hit Dice: & 2* (variable) & 1* (variable) \\\hline
No. of Attacks: & 1 dagger or 1 spell & 1 dagger or 1 spell \\\hline
Damage: & 1d4 or per spell & 1d4 or per spell \\\hline
Movement: & 20' & 40' \\\hline
No. Appearing: & 1 & 1 \\\hline
Save As: & Magic-User: by HD & Magic-User: by HD \\\hline
Morale: & 9 to 12 (see below) & 12 \\\hline
Treasure Type: & None & None \\\hline
XP: & 100 (variable) & 37 (variable) \\\hline
\end{tabularx}\medskip

A \textbf{Zombraire} is a free-willed undead Magic-User. Like the zombie it resembles, a zombraire moves silently, is very strong, and must be literally hacked to pieces to be destroyed. However, it does not suffer the initiative penalty common to ordinary zombies.

\begin{center}
		\includegraphics[width=0.47\textwidth]{Pictures132/10000000000003D80000050BC021C927371412E9.png} \end{center}

takes only half damage from blunt weapons, and only a single point from arrows, bolts, and sling stones (plus any magical bonus). It may be Turned by a Cleric (as a wight), and is immune to \textbf{sleep},\textbf{ charm},\textbf{ }and\textbf{ hold} spells.

A zombraire slowly rots away, and as it does it loses its sanity; this is represented by the variable morale listed. An insane zombraire fights to the death in hopes of being slain, thus ending its tortured existence.

The given statistics are for a zombraire formed from a 2nd-level Magic-user; the HD and saving throws of a zombraire are based on the level it had in life. A zombraire can cast spells as it did when living, but cannot learn new spells.

A \textbf{Skeletaire} is the final form of a zombraire which has rotted away completely. It takes only half damage from edged weapons, and only a single point from arrows, bolts, and sling stones (plus any magical bonus). It can be Turned by a Cleric (as a zombie), and is immune to \textbf{sleep},\textbf{ charm}, and \textbf{hold} spells. A skeletaire never fails morale, and thus always fights until destroyed.

The statistics above are for a skeletaire formed from a 2nd-level Magic-user. A skeletaire will have HD equal to the character' s level minus 1, and will save as a Magic-user of the level equal to its HD. The skeletaire cannot speak, but still retains the ability to prepare and cast spells as it did in life (but like a zombraire, it cannot learn new spells). 

The process of creating or becoming a zombraire are variable, and often involve cursed magic items (especially cursed scrolls).

\end{multicols}

 \begin{center}
 \includegraphics[width=0.8\textwidth]{Pictures132/10000000000003CF0000056B6DAAEABCF6C9A7D0.png}
 \end{center}

\pagebreak

\section{PART 7: TREASURE}\label{part-7-treasure}\index{Part 7: Threasure}

\textit{"Your idea was good, Darion," said Morningstar after we had turned the sarcophagus lid halfway around. "You just got in a bit of a hurry to try it."}

\textit{"Indeed," I said, feeling foolish. I looked into the sarcophagus; lying atop the mummified remains inside I saw a sword, held in both bony hands with the point toward the corpse' s feet. The golden blade was untarnished; the huge ruby set into the cross-guard shone in the torchlight.}

\textit{I didn' t reach for it, though I found myself sorely tempted. "Barthal," I said, "tell me, is it safe to reach within?" The Halfling climbed up onto the edge of the platform and leaned over the sarcophagus, peering around within.}

\textit{"I don' t see a trap," he remarked after careful study. "Do you think the dead one, there, will rise up if you touch his sword?"}

\textit{"He might," answered Apoqulis, "but I doubt it. The skeletons were likely the only guardians in this tomb."}

\textit{All eyes seemed to be on me. Reluctantly I reached within, grasping the sword by the blade with my mailed glove so as to avoid touching the corpse. The long-dead chieftain didn' t want to part with his weapon, but after a bit of twisting I worked it loose.}

\textit{Shaking bits of desiccated flesh from the hilt, I grasped the weapon properly and held it aloft. It felt good in my hand... I wondered what magic it might contain?}

\textit{Barthal' s voice shook me out of my contemplation. "Look here!" he said, and I looked down. He had opened a secret panel in the platform... and as I looked on, gold  and silver coins began to spill out. Barthal cried out gleefully, "Jackpot!"}

\begin{multicols}{2}

\subsection{Distribution of Treasure}\label{distribution-of-treasure}\index{Distribution of Treasure}

Some adventurers choose to adventure to battle evil, while others seek to attain glory or power... but others go in search of treasure, gold and jewels and magical items. Below is the information the Game Master will need to satisfy the greedy.

\subsection{Random Treasure Generation}\label{random-treasure-generation}\index{Random Treasure Generation}

The tables below describe the various treasure types assigned to monsters, as well as unguarded treasures appropriate to various dungeon levels. To generate a random treasure, find the indicated treasure type and read across; where a percentage chance  is given, roll percentile dice to see if that sort of treasure is found. If so, roll the indicated dice to determine how much.

Tables for the random generation of gems, jewelry (and bejeweled art pieces), and magic items are provided after the main treasure tables.

\subsection{Placed Treasures}\label{placed-treasures}\index{
Placed Treasures}

The Game Master is never required to roll for treasure; rather, treasure may be placed, or random treasures amended, as desired or needed for the purposes of the adventure. Special treasures are always placed; for example, a special magic item  needed to complete an adventure.

\subsection{Adjusting Treasure Awards}\label{adjusting-treasure-awards}\index{Adjusting Treasure Awards}

There will be many cases where random treasure generation is not the best method to employ. For instance, a larger than average treasure assigned to a smaller than average lair of monsters might need to be reduced. It is up to the Game Master to  decide how much treasure they wish to allow into the campaign. Too much gold (or other treasure which can be converted to gold) may make things too easy for the player characters. Similarly, too many magic items may also make things too easy.

If you are a novice Game Master, remember that you can always give more treasure, but it can be difficult to drain treasure from the party without angering the players... particularly if you use heavy-handed methods. Start small, and work up as you gain experience. 

\begin{flushleft} \includegraphics[width=0.47\textwidth]{Pictures132/10000000000003D8000003C6A6D8D8304BC35483.png}  \end{flushleft}

\end{multicols}

\subsection{Treasure Types}\label{treasure-types}\index{Treasure Types}

\subsubsection{Lair Treasures}\label{lair-treasures}\index{Lair Treasures}

\begin{tabularx}{1\textwidth}{@{}lXXXXXXX@{}}
&\textbf{100's of}&\textbf{100's of}&\textbf{100's of}&\textbf{100's of}&\textbf{100's of}&\textbf{Gems and}&\textbf{Magic}\\
\textbf{Type} & \textbf{Copper} & \textbf{Silver} & \textbf{Electrum} & \textbf{Gold} & \textbf{Platinum} & \textbf{Jewelry} & \textbf{Items} \\\hline
A & 50\% 5d6 & 60\% 5d6 &  40\% 5d4 & 70\% 10d6 & 50\% 1d10 & 50\% 6d6 50\% 6d6 &  30\% any 3  \\\toprule
B &  75\% 5d10  &  50\% 5d6  &  50\% 5d4  &  50\% 3d6  &  None  &  25\% 1d6    25\% 1d6  &  10\% 1 weapon or armor  \\\hline
C &  60\% 6d6  &  60\% 5d4  &  30\% 2d6  &  None  &  None  &  25\% 1d4    25\% 1d4  &  15\% any 1d2  \\ \hline
D &  30\% 4d6  &  45\% 6d6  &  None  &  90\% 5d8  &  None  &  30\% 1d8    30\% 1d8  &  20\% any 1d2    + 1 potion  \\\hline
E &  30\% 2d8  &  60\% 6d10  &  50\% 3d8  &  50\% 4d10  &  None  &  10\% 1d10    10\% 1d10  &  30\% any 1d4    + 1 scroll  \\\hline
F &  None  &  40\% 3d8  &  50\% 4d8  &  85\% 6d10  &  70\% 2d8  &  20\% 2d12    10\% 1d12  &  35\% any 1d4 except weapons + 1 potion + 1 scroll \\\hline
G & None & None & None & 90\% 4d6x10 & 75\% 5d8 & 25\% 3d6 25\% 1d10  &  50\% any 1d4  + 1 scroll \\\hline
H* &  * 8d10  &  * 6d10x10  &  * 3d10x10  &  * 5d8x10  &  * 9d8  &  * 1d100    * 10d4  &  * any 1d4  + 1 potion + 1 scroll \\\hline
I &  None & None  &  None  &  None  &  80\% 3d10  &  50\% 2d6  50\% 2d6  &  15\% any 1  \\\hline
J &  45\% 3d8  &  45\% 1d8  &  None  &  None  &  None  & None None  & None \\\hline
K &  None &  90\% 2d10  &  35\% 1d8  &  None  &  None  &  None None  & None \\\hline
L &  None  &  None  &  None   &  None  &  None  &  50\% 1d4 None  & None \\\hline
M &  None  &  None  &  None  &  90\% 4d10  &  90\% 2d8x10  &  55\% 5d4  45\% 2d6  & None \\\hline
N &  None  &  None  &  None  &  None  &  None  &  None  None  &  40\% 2d4 potions \\\hline
O & None &   None  &  None &  None &  None  & None None  & 50\% 1d4 scrolls  \\\bottomrule
\end{tabularx}

* Type H treasures are specifically dragon treasure; the chance of each type of monetary treasure ranges from 35\% at the second age category to 85\% at the seventh,  while the odds of gems, jewelry, and magic items are 5\% per hit die of the monster. Hatchlings do not usually have any treasure.

\subsubsection{Individual Treasures}\label{individual-treasures}\index{Individual Treasures}

\begin{tabular*}{0.97\linewidth}{@{\extracolsep{\fill}}llllllll}
&\textbf{100's of}&\textbf{100's of}&\textbf{100's of}&\textbf{100's of}&\textbf{100's of}&\textbf{Gems and}&\textbf{Magic}\\
\textbf{Type} & \textbf{Copper} & \textbf{Silver} & \textbf{Electrum} & \textbf{Gold} & \textbf{Platinum} & \textbf{Jewelry} & \textbf{Items} \\\toprule
P & 3d8  & None & None & None & None & None None & None\\\hline
Q & None & 3d6 & None & None & None  & None None  &  None \\\hline
R & None & None & 2d6 & None & None  & None None  &  None \\\hline
S & None & None & None & 2d4 & None & None None  & None \\\hline
T &  None & None & None & None & 1d6 & None None & None \\\hline
U & 50\% 1d20  & 50\% 1d20 & None & 25\% 1d20 & None & 5\% 1d4 5\% 1d4 & 2\% Any 1 \\\hline
V & None & 25\% 1d20 & 25\% 1d20 & 50\% 1d20 & 25\% 1d20 & 10\% 1d4 10\% 1d4 & 5\% Any 1 \\\bottomrule
\end{tabular*}

\subsubsection{Unguarded Treasures}\label{unguarded-treasures}\index{Unguarded Treasures}

%\begin{tabularx}{1\textwidth}{@{}lXXXXXXX@{}}
\begin{tabular*}{0.97\linewidth}{@{\extracolsep{\fill}}llllllll}
&\textbf{100's of}&\textbf{100's of}&\textbf{100's of}&\textbf{100's of}&\textbf{100's of}&\textbf{Gems and}&\textbf{Magic}\\
\textbf{Type} & \textbf{Copper} & \textbf{Silver} & \textbf{Electrum} & \textbf{Gold} & \textbf{Platinum} & \textbf{Jewelry} & \textbf{Items} \\\toprule
1 & 75\% 1d8  & 50\% 1d6 & 25\% 1d4 & 7\% 1d4 & 1\% 1d4 & 7\% 1d4 3\% 1d4 & 2\% Any 1 \\\hline
2 & 50\% 1d10 & 50\% 1d8 & 25\% 1d6 & 20\% 1d6 & 2\% 1d4 & 10\% 1d6 7\% 1d4 &  5\% Any 1 \\\hline
3 & 30\% 2d6 & 50\% 1d10 & 25\% 1d8 & 50\% 1d6 & 4\% 1d4 & 15\% 1d6 7\% 1d6 & 8\% Any 1 \\\hline
4-5 & 20\% 3d6 & 50\% 2d6 & 25\% 1d10 & 50\% 2d6 & 8\% 1d4 & 20\% 1d8 10\% 1d6  & 12\% Any 1 \\\hline
6-7 & 15\% 4d6 & 50\% 3d6 & 25\% 1d12 & 70\% 2d8 & 15\% 1d4 & 30\% 1d8 15\% 1d6 & 16\% Any 1 \\\hline
8+ & 10\% 5d6 & 50\% 5d6 & 25\% 2d8 & 75\% 4d6 & 30\% 1d4 & 40\% 1d8 30\% 1d8  &  20\% Any 1 \\\bottomrule
\end{tabular*}

Note: Unguarded treasures should be rare; see the \textbf{Game Master}
section, below, for advice on placement of unguarded treasure.

\begin{multicols}{2}
	
\subsection{Gems and Jewelry}\label{gems-and-jewelry}\index{Gems and Jewelry}

Use the tables below to determine the base value and number found when gems are indicated in a treasure hoard. If the number generated in the main table above is small, roll for each gem; but if the number is large (10 or more, at the GM' s option), after each roll for Type and Base Value, roll the indicated die to see how many such gems are in the hoard.\medskip

\begin{tabular*}{0.93\linewidth}{@{\extracolsep{\fill}}llll}
&  & \textbf{Base Values} & \textbf{Number}\\
\textbf{d\%} & \textbf{Type} & \textbf{in Gold Pieces} & \textbf{Found}\\\toprule
01-20 & Ornamental & 10 & 1d10 \\\hline
21-45 & Semiprecious & 50 & 1d8 \\\hline
46-75 & Fancy & 100 & 1d6 \\\hline
76-95 & Precious & 500 & 1d4 \\\hline
96-00 & Gem & 1,000 & 1d2 \\\hline
& Jewel & 5,000 & 1 \\\bottomrule
\end{tabular*}\medskip

The values of gems vary from the above for reasons of quality, size, etc. The GM may use the table below to adjust the values of the gems in the hoard, at their option. This is why there is no die result given in the table above for Jewel; on a roll of 12 on the table below, a Gem can become a Jewel.\medskip

\begin{tabular*}{0.93\linewidth}{@{\extracolsep{\fill}}ll}
\textbf{2d6} & \textbf{Value Adjustment} \\
2 & Next Lower Value Row \\\toprule
3 & 1/2 \\\hline
4 & 3/4 \\\hline
5-9 & Normal Value \\\hline
10 & 1.5 Times \\\hline
11 & 2 Times \\\hline
12 & Next Higher Value Row \\\bottomrule
\end{tabular*}\medskip

\begin{tabular*}{0.93\linewidth}{@{\extracolsep{\fill}}lllll}
\textbf{d\%} & \textbf{Gem Type} & & d\% & \textbf{Gem Type} \\
01-05 & Alexandrite & & 58-63 & Garnet \\\toprule
06-12 & Amethyst & & 64-68 & Heliotrope \\\hline
13-20 & Aventurine & & 69-78 & Malachite \\\hline
21-30 & Chlorastrolite & & 79-88 & Rhodonite \\\hline
31-40 & Diamond & & 89-91 & Ruby \\\hline
41-43 & Emerald & & 92-95 & Sapphire \\\hline
44-48 & Fire Opal & & 96-00 & Topaz \\\hline
49-57 & Fluorospar & & & \\\bottomrule
\end{tabular*}\medskip

Standard items of jewelry are valued at 2d8x100 gp value. The table below can be used to generate descriptions of the items themselves.\medskip

\begin{tabular*}{0.93\linewidth}{@{\extracolsep{\fill}}lllll}
\textbf{d\%} & \textbf{Type} & &\textbf{ d\%} & \textbf{Type} \\\toprule
01-06 & Anklet & & 56-62 & Earring \\\hline
07-12 & Belt & & 63-65 & Flagon \\\hline
13-14 & Bowl & & 66-68 & Goblet \\\hline
15-21 & Bracelet & & 69-73 & Knife \\\hline
22-27 & Brooch & & 74-77 & Letter Opener \\\hline
28-32 & Buckle & & 78-80 & Locket \\\hline
33-37 & Chain & & 81-82 & Medal \\\hline
38-40 & Choker & & 83-89 & Necklace \\\hline
41-42 & Circlet & & 90 & Plate \\\hline
43-47 & Clasp & & 91-95 & Pin \\\hline
48-51 & Comb & & 96 & Scepter \\\hline
52 & Crown & & 97-99 & Statuette \\\hline
53-55 & Cup & & 00 & Tiara \\\bottomrule
\end{tabular*}

\end{multicols}

\pagebreak

\subsection{Magic Item Generation}\label{magic-item-generation}\index{Magic Item Generation}

\begin{multicols}{2}

Determine the sort of item found by rolling on the following table:\medskip

\begin{tabularx}{0.45\textwidth}{@{}lXXl@{}}
&\multirow{2}{4em}{\textbf{Weapon or Armor}}&\multirow{2}{4em}{\textbf{Any Exc.  Weapons}}&\textbf{Type of Item}\\
\textbf{Any}&&&\\
&&&\\\toprule
01-25 & 01-70 & & Weapon \\\hline
26-35 & 71-00 & 01-12 & Armor \\\hline
36-55 & & 13-40 & Potion \\\hline
56-85 & & 41-79 & Scroll \\\hline
86-90 & & 80-86 & Wand, Staff, or Rod \\\hline
91-97 & & 87-96 & Miscellaneous Items \\\hline
98-00 & & 97-00 & Rare Items \\\bottomrule
\end{tabularx}

\subsection{Magic Weapons}\label{magic-weapons}\index{Magic Weapons}

First, roll d\% on the following table to determine the weapon type:

\begin{tabular*}{0.93\linewidth}{@{\extracolsep{\fill}}ll}
\textbf{d\%} & \textbf{Weapon Type} \\\toprule
01-02 & Great Axe \\\hline
03-09 & Battle Axe \\\hline
10-11 & Hand Axe \\\hline
12-19 & Shortbow \\\hline
20-27 & Shortbow Arrow \\\hline
28-31 & Longbow \\\hline
32-35 & Longbow Arrow \\\hline
36-43 & Light Quarrel \\\hline
44-47 & Heavy Quarrel \\\hline
48-59 & Dagger \\\hline
60-65 & Shortsword \\\hline
66-79 & Longsword \\\hline
80-81 & Scimitar \\\hline
82-83 & Two-Handed Sword \\\hline
84-86 & Warhammer \\\hline
87-94 & Mace \\\hline
95 & Maul \\\hline
96 & Pole Arm \\vv
97 & Sling Bullet \\
98-00 & Spear \\\bottomrule
\end{tabular*}\medskip

Next, roll on the Weapon Bonus tables. Follow the directions given if a roll on the Special Enemy or Special Ability tables are indicated; generally multiple rolls on the Special Ability table should be ignored when rolled.\medskip

\begin{tabular*}{0.93\linewidth}{@{\extracolsep{\fill}}lll}
\multicolumn{2}{c}{\textbf{d\% Roll}}&  \\
\textbf{Melee} & \textbf{Missile} & \textbf{Weapon Bonus} \\\toprule
01-40 & 01-46 & +1 \\\hline
41-50 & 47-58 & +2 \\\hline
51-55 & 59-64 & +3 \\\hline
56-57 & & +4 \\\hline
58 & & +5 \\\hline
59-75 & 65-82 & +1, +2 vs. Special Enemy \\\hline
76-85 & 83-94 & +1, +3 vs. Special Enemy \\\hline
86-95 & & Roll Again + Special Ability \\\hline
96-98 & 95-98 & Cursed, -1* \\\hline
99-00 & 99-00 & Cursed, -2* \\\bottomrule
\end{tabular*}\medskip

* If cursed weapons are rolled along with special abilities, ignore the
special ability roll.\medskip

\begin{tabular*}{0.93\linewidth}{@{\extracolsep{\fill}}lllll}
\textbf{1d6} & \textbf{Special Enemy} & & \textbf{1d6} & \textbf{Special Enemy} \\\toprule
1 & Dragons & & 4 & Regenerators \\\hline
2 & Enchanted & & 5 & Spell Users \\\hline
3 & Lycanthropes & & 6 & Undead \\\bottomrule
\end{tabular*}\medskip

\begin{tabular*}{0.93\linewidth}{@{\extracolsep{\fill}}ll}
\textbf{1d20} & \textbf{Special} \textbf{Ability} \\\toprule
01-09 & Casts Light on Command \\\hline
10-11 & Charm Person \\\hline
12 & Drains Energy \\\hline
13-16 & Flames on Command \\\hline
17-19 & Locate Objects \\\hline
20 & Wishes \\\bottomrule
\end{tabular*}

\subsection{Magic Armor}\label{magic-armor}\index{Magic Armor}

Generate the type and bonus of each item of magic armor on the tables
below.

\begin{tabular}[]{@{}lllll@{}}
\textbf{d\%} & \textbf{Armor Type} & & \textbf{d\%} & \textbf{Armor Bonus} \\\toprule
01-09 & Leather Armor & & 01-50 & +1 \\\hline
10-28 & Chain Mail & & 51-80 & +2 \\\hline
29-43 & Plate Mail & & 81-90 & +3 \\\hline
44-00 & Shield & & 91-95 & Cursed * \\\hline
& & & 96-00 & Cursed, AC 11 ** \\\bottomrule
\end{tabular}\medskip

* If Cursed armor is rolled, roll again and reverse the bonus (e.g., -1
instead of +1).

** This armor has AC 11 but appears to be +1 when tested.

\end{multicols}

\vfill

 \includegraphics[width=1\textwidth]{Pictures132/10000000000007E90000009E3C5FFAA43076CFF9.png}  

\pagebreak

\subsection{Potions}\label{potions}\index{Potions}

\begin{tabular*}{1\linewidth}{@{\extracolsep{\fill}}llllllll}
\textbf{d\%} & \textbf{Type} & & \textbf{d\%} & \textbf{Type} & & \textbf{d\%} & \textbf{Type} \\\toprule
01-03 & Clairaudience & & 26-32 & Delusion & & 64-68 & Invisibility \\\hline
04-06 & Clairvoyance & & 33-35 & Diminution & & 69-72 &
Invulnerability \\\hline
07-08 & Cold Resistance & & 36-39 & Fire Resistance & & 73-76 &
Levitation \\\hline
09-11 & Control Animal & & 40-43 & Flying & & 77-80 & Longevity \\\hline
12-13 & Control Dragon & & 44-47 & Gaseous Form & & 81-84 & Mind
Reading \\\hline
14-16 & Control Giant & & 48-51 & Giant Strength & & 85-86 & Poison \\\hline
17-19 & Control Human & & 52-55 & Growth & & 87-89 & Polymorph Self \\\hline
20-22 & Control Plant & & 56-59 & Healing & & 90-97 & Speed \\\hline
23-25 & Control Undead & & 60-63 & Heroism & & 98-00 & Treasure
Finding \\\bottomrule
\end{tabular*}

\begin{multicols}{2}

\subsection{Scrolls}\label{scrolls}\index{Scrolls}

\begin{tabular*}{0.93\linewidth}{@{\extracolsep{\fill}}ll}
\textbf{d\%} & \textbf{General Type} \\\toprule
01-03 & Cleric Spell Scroll (1 Spell) \\\hline
04-06 & Cleric Spell Scroll (2 Spells) \\\hline
09 & Cleric Spell Scroll (4 Spells) \\\hline
10-15 & Magic-User Spell Scroll (1 Spell) \\\hline
16-20 & Magic-User Spell Scroll (2 Spells) \\\hline
21-25 & Magic-User Spell Scroll (3 Spells) \\\hline
26-29 & Magic-User Spell Scroll (4 Spells) \\\hline
30-32 & Magic-User Spell Scroll (5 Spells) \\\hline
33-34 & Magic-User Spell Scroll (6 Spells) \\\hline
35 & Magic-User Spell Scroll (7 Spells) \\\hline
36-40 & Cursed Scroll \\\hline
41-46 & Protection from Elementals \\\hline
47-56 & Protection from Lycanthropes \\\hline
57-61 & Protection from Magic \\\hline
62-75 & Protection from Undead \\\hline
76-85 & Map to Treasure Type A \\\hline
86-89 & Map to Treasure Type E \\\hline
90-92 & Map to Treasure Type G \\\hline
93-00 & Map to 1d4 Magic Items \\\bottomrule
\end{tabular*}

\subsection{Wands, Staves and Rods}\label{wands-staves-and-rods}\index{Wands, Staves and Rods}

\begin{tabular*}{0.93\linewidth}{@{\extracolsep{\fill}}ll}
\textbf{d\%} & \textbf{Type} \\\toprule
01-08 & Rod of Cancellation \\\hline
09-13 & Snake Staff \\\hline
14-17 & Staff of Commanding \\\hline
18-28 & Staff of Healing \\\hline
29-30 & Staff of Power \\\hline
31-34 & Staff of Striking \\\hline
35 & Staff of Wizardry \\\hline
36-40 & Wand of Cold \\\hline
41-45 & Wand of Enemy Detection \\\hline
46-50 & Wand of Fear \\\hline
51-55 & Wand of Fireballs \\\hline
56-60 & Wand of Illusion \\\hline
61-65 & Wand of Lightning Bolts \\\hline
66-73 & Wand of Magic Detection \\\hline
74-79 & Wand of Paralysis \\\hline
80-84 & Wand of Polymorph \\\hline
85-92 & Wand of Secret Door Detection \\\hline
93-00 & Wand of Trap Detection \\\bottomrule
\end{tabular*}

\end{multicols}

\vfill

 \includegraphics[width=1\textwidth]{Pictures132/10000000000007E900000312CD38C589D27F757E.png} 
 
 \pagebreak
 
 \begin{multicols}{2}
 	
\subsection{Miscellaneous Items}\label{miscellaneous-items}\index{Miscellaneous Items}

\textbf{Effect Subtables}\medskip

\begin{tabular*}{0.93\linewidth}{@{\extracolsep{\fill}}ll}
\textbf{d\%} & \textbf{Subtable} \\\toprule
01-57 & Effect Subtable 1 \\\hline
58-00 & Effect Subtable 2 \\\bottomrule
\end{tabular*}\medskip

\textbf{Effect Subtable 1}\medskip

\begin{tabular*}{0.93\linewidth}{@{\extracolsep{\fill}}lll}
\textbf{d\%} & \textbf{Effect} & \textbf{Form} \\\toprule
01 & Blasting & G \\\hline
02-05 & Blending & F \\\hline
06-13 & Cold Resistance & F \\\hline
14-17 & Comprehension & E \\\hline
18-22 & Control Animal & C \\\hline
23-29 & Control Human & C \\\hline
30-35 & Control Plant & C \\\hline
36-37 & Courage & G \\\hline
38-40 & Deception & F \\\hline
41-52 & Delusion & A \\\hline
53-55 & Djinni Summoning & C \\\hline
56 & Doom & G \\\hline
57-67 & Fire Resistance & F \\\hline
68-80 & Invisibility & F \\\hline
81-85 & Levitation & B \\\hline
86-95 & Mind Reading & C \\\hline
96-97 & Panic & G \\\hline
98-00 & Penetrating Vision & D \\\bottomrule
\end{tabular*}\medskip

\textbf{Effect Subtable 2}\medskip

\begin{tabular*}{0.93\linewidth}{@{\extracolsep{\fill}}lll}
d\% & Effect & Form \\\toprule
01-07 & Protection +1 & F \\\hline
08-10 & Protection +2 & F \\\hline
11 & Protection +3 & F \\\hline
12-14 & Protection from Energy Drain & F \\\hline
15-20 & Protection from Scrying & F \\\hline
21-23 & Regeneration & C \\\hline
24-29 & Scrying & H \\\hline
30-32 & Scrying, Superior & H \\\hline
33-39 & Speed & B \\\hline
40-42 & Spell Storing & C \\\hline
43-50 & Spell Turning & F \\\hline
51-69 & Stealth & B \\\hline
70-72 & Telekinesis & C \\\hline
73-74 & Telepathy & C \\\hline
75-76 & Teleportation & C \\\hline
77-78 & True Seeing & D \\\hline
79-88 & Water Walking & B \\\hline
89-99 & Weakness & C \\\hline
00 & Wishes & C \\\bottomrule
\end{tabular*}\\

Roll on the first table above to select a subtable, then roll again on the selected subtable to determine the exact effect. Finally, roll on the specified column of the Form table to determine the physical form of the item.

\end{multicols}

\begin{tabular*}{1\linewidth}{@{\extracolsep{\fill}}lllllllll}
\textbf{Form of Item} & A & B & C & D & E & F & G & H \\\toprule
Bell (or Chime) & 01-02 & & & & & & 01-17 & \\\hline
Belt or Girdle & 03-05 & & & & & 01-07 & & \\\hline
Boots & 06-13 & 01-25 & & & & & & \\\hline
Bowl & 14-15 & & & & & & & 01-17 \\\hline
Cloak & 16-28 & & & & & 08-38 & & \\\hline
Crystal Ball or Orb & 29-31 & & & & & & & 18-67 \\\hline
Drums & 32-33 & & & & & & 18-50 & \\\hline
Helm & 34-38 & & & & 01-40 & & & \\\hline
Horn & 39-43 & & & & & & 51-00 & \\\hline
Lens & 44-46 & & & 01-17 & & & & \\\hline
Mirror & 47-49 & & & 18-17 & & & & 68-00 \\\hline
Pendant & 50-67 & 26-50 & 01-40 & 18-50 & 41-80 & 39-50 & & \\\hline
Ring & 68-00 & 51-00 & 41-00 & 51-00 & 81-00 & 51-00 & & \\\bottomrule
\end{tabular*}\\\medskip


\label{rare-items}\index{Rare Items}

\begin{tabular*}{1\linewidth}{@{\extracolsep{\fill}}ll}
\textbf{Rare Items} \textbf{d\%} & \textbf{Type} \\\toprule
01-05 & Bag of Devouring \\\hline
06-20 & Bag of Holding \\\hline
21-32 & Boots of Traveling and Leaping \\\hline
33-47 & Broom of Flying \\\hline
48-57 & Device of Summoning Elementals (see below) \\\hline
58-59 & Efreeti Bottle \\\hline
60-64 & Flying Carpet \\\hline
65-81 & Gauntlets of Ogre Power \\\hline
82-86 & Girdle of Giant Strength \\\hline
87-88 & Mirror of Imprisonment \\\hline
89-00 & Rope of Climbing \\\bottomrule
\end{tabular*}

\begin{multicols}{2}

\subsection{Devices of Summoning Elementals}\label{devices-of-summoning-elementals}\index{Devices of Summoning Elementals}

Review the information for Elementals on page \hyperlink{elemental}{\pageref{elemental}} before randomly selecting any of these items, as you may not wish to allow all types of elementals in your campaign.

\begin{tabular*}{0.93\linewidth}{@{\extracolsep{\fill}}ll}
\textbf{1d8} & \textbf{Type} \\\toprule
1 & Bowl of Summoning Water Elementals \\\hline
2 & Brazier of Summoning Fire Elementals \\\hline
3 & Censer of Summoning Air Elementals \\\hline
4 & Crucible of Summoning Metal Elementals \\\hline
5 & Mallet of Summoning Wood Elementals \\\hline
6 & Marble Plate of Summoning Cold Elementals \\\hline
7 & Rod of Summoning Lightning Elementals \\\hline
8 & Stone of Summoning Earth Elementals \\\bottomrule
\end{tabular*}

\end{multicols}

\subsection{Explanation of Magic Items}\label{explanation-of-magic-items}\index{Explanation of Magic Items}

\begin{multicols}{2}
	
\subsection{Using Magic Items}\label{using-magic-items}\index{Using Magic Items}

To use a magic item, it must be activated, although sometimes activation simply means putting a ring on your finger. Some items, once donned, function constantly.

Many items are activated just by using them. For instance, a character has to drink a potion, swing a sword, interpose a shield to deflect a blow in combat, wear a ring, or don a cloak. Activation of these items is generally straightforward and self-explanatory. This doesn't mean  that if you use such an item, you automatically know what it can do. You must know (or at least guess) what the item can do and then use the item in order to activate it, unless the benefit of the item comes automatically, such as from drinking a potion or swinging a sword.

If no activation method is suggested either in the magic item description or by the nature of the item, assume that a command word is needed to activate it. Command word activation means that a character speaks the word and the item activates. No other special knowledge is  needed.

A command word can be a real word, but when this is the case, the holder of the item runs the risk of activating the item accidentally by speaking the word in normal conversation. More often, the command word is some seemingly nonsensical word, or a word or phrase from an ancient language no longer in common use. Note that many magic items must be held in the hand (or otherwise specially handled or worn) to be used; the risk of accidental activation is less significant for such items.

Learning the command word for an item may be easy (sometimes the word is actually inscribed on the item) or it may be difficult, requiring the services of a powerful wizard or sage, or some other means of discovery. 

Only the character holding or wearing a magic item may activate it. A character who has been gagged or silenced may not activate a magic item which requires a command word.

When an article of magic armor, clothing or jewelry (including a ring) is discovered, size is not usually an issue. Such items magically adjust themselves for wearers from as small as Halflings to as large as Humans. This effect is called \textbf{accommodation}. The GM may create "primitive" items lacking this power if they wish. 

Generally only one magical item of a given type may be worn at the same time. For example, a character can normally only wear one suit of armor, wear one necklace and carry one shield at a time. In the case of rings, a character may wear one magical ring per hand. If a character wears more items of a given type than would normally be practical, the items will usually fail to function due to interference with one another; for instance, wearing two rings on the same hand normally results in both rings failing to operate. \textbf{Note, however, }that this limitation cannot be used to disable cursed magic items. For example, wearing a cursed ring would prevent another magic ring from being worn and used on that hand, but the curse would not be lifted by donning a second magic ring.

\subsection{Magic Weapons}\label{magic-weapons-1}\index{Magic Weapons}

Magic weapons are created with a variety of powers and will usually aid the wielder in combat. A magical weapon' s bonus is applied to all attack and damage rolls made with the weapon.

\textbf{Casts Light on Command:} By drawing the weapon and uttering a command word, the wielder may cause it to glow; it will then shed light with the same radius as a \textbf{light} spell. Sheathing or laying down the weapon, or speaking the command word again, dispels the effect. This power may be used as often as desired.

\textbf{Charm Person:} This power allows the wielder to cast  \textbf{charm person} once per day, as if by an 8\textsuperscript{th} level Magic-User, by brandishing the weapon, speaking a command word and gazing at the target creature. (The wielder' s gaze does not have to be met for the spell to be cast.) The target creature is allowed saving throws just as described in the spell description. 

\textbf{Drains Energy:} A weapon with this power drains one life energy level on a hit, as described under \textbf{Energy Drain }in the \textbf{Encounter} section; up to 2d4 levels can be drained by a weapon with this power, after which time the weapon loses this power but retains any other magical effects or bonuses.

\textbf{Flames on Command:} Upon command, the weapon will be sheathed in fire. The fire does not harm the wielder. The effect remains until the command is given again, or until the weapon is dropped or sheathed. While it flames, all damage done by the weapon is treated as fire damage, and an additional +1 bonus (in addition to the weapon' s normal bonus) is added to damage when fighting trolls, treant, and other creatures especially vulnerable to fire. It casts light and burns just as if it were a torch. 

\textbf{Locate Objects:} This power allows the wielder to cast the spell \textbf{locate object} once per day, as if by an 8\textsuperscript{th} level Magic-User.

\begin{wrapfigure}{l}{0.25\textwidth}
	\includegraphics[width=0.25\textwidth]{Pictures132/10000000000002DA0000080423581FFF58A0B468.png}\medskip
\end{wrapfigure}

\textbf{Special Enemy:} These weapons are created to combat a specific sort of creature, as rolled on the Special Ability table. When used against that specific enemy, the second listed bonus applies instead of the first; so a \textbf{Sword +1, +3 vs. Undead} would provide +1 attack and damage against giant rats, but +3 attack and damage rolls against zombies.

\textbf{Wishes:} Weapons with this power have the ability to grant 1d4 wishes. The GM must adjudicate all wishes, and instructions are given in the \textbf{Game Master} section regarding this. After these wishes have been made, the weapon loses this power, but retains any other bonuses and powers.

\textbf{Cursed Weapons} inflict a penalty to the wielder' s attack rolls, as rolled on the Weapon Bonus table. The curse causes the afflicted character to be unable to get rid of the weapon. There are two possible forms the curse may take: Obsession and Affliction. The GM may decide which to use at their option. 

\textbf{Obsession:} Regardless of how severe the penalty is, the character wielding the weapon will believe it is a bonus and refuse to use any other weapon in combat. A \textbf{remove curse} spell is the only way to rid a character of such a weapon; but as they will believe the weapon is the best magical weapon ever, the character receives a saving throw vs. Spells to resist.

\textbf{Affliction: } The character knows the weapon is cursed as soon as they use it in combat; however, any attempt to throw it away fails, as the weapon magically appears back in the character' s hand whenever they try to draw any other weapon. In this case, the \textbf{remove curse} spell needed to rid the character of the weapon will be unopposed (i.e. no saving throw).


\subsection{Magic Armor}\label{magic-armor-1}\index{Magic Armor}

Magic armor (including shields) offers improved, magical protection to the wearer. In general, magic armor grants the normal Armor Class for its type, plus the magical armor bonus, as rolled on the Magic Armor table; for example, Plate Mail +2 provides an Armor Class of 19.

There are two varieties of \textbf{cursed armor}: Cursed Armor -1 and Cursed Armor AC 11. The first variety' s AC is reduced by the rolled penalty; for example, Plate Mail -1 grants Armor Class 16. The second type is much worse, for regardless of the type, it only provides Armor Class 11. Dexterity and shield bonuses still apply.

Cursed armor cannot be removed from the wearer once the curse is proven, that is, once the wearer is hit in combat. Once the curse has taken effect, only a \textbf{remove curse} spell, or some more powerful magic (such as a wish), will enable the wearer to remove it. The armor will detect as magical, like any other magic armor; the curse cannot be detected by any means other than wearing the armor in combat.

\subsection{Potions}\label{potions-1}\index{Potions}

A potion is an elixir concocted with a spell-like effect that affects only the drinker. Unless otherwise noted, a potion grants its benefits for 1d6+6 turns (even if the duration of an associated spell is longer or shorter). Potions generally only affect living creatures.

\textbf{Clairaudience: }This potion enables the drinker to hear sounds in another area through the ears of a living creature in that area, up to a maximum 60' away. This effect otherwise functions just as the spell clairvoyance.

\textbf{Clairvoyance: }This potion grants the imbiber the effect of the clairvoyance spell.

\textbf{Cold Resistance:} This potion grants the imbiber the power of the spell resist cold.

\textbf{Control Animal: }This potion functions like a control human potion, but affects only normal, non-magical animals.

\textbf{Control Dragon: }This potion functions like a control human potion, but affects only dragons.

\textbf{Control Giant:} This potion functions like a control human potion, but affects only giants.

\textbf{Control Human:} This potion allows the drinker to charm any humanoid by gazing at it. The effect functions like the charm person spell. If the charm is resisted, the drinker can attempt to charm up to two more targets before the potion's benefit is exhausted.

\textbf{Control Plant: }This potion grants the drinker control over one or more plants or plant creatures within a 10' square area up to 50' away. Normal plants become animated, having a movement rate of 10', and obey the drinker' s commands. If ordered to attack, only the largest plants can do any real harm, attacking with a +0 attack bonus and inflicting 1d4 points of damage per hit. Affected plant creatures (who fail to save vs. Spells) can understand the drinker, and behave as if under a charm monster spell.

\textbf{Control Undead:} This potion grants the drinker command of 3d6 hit dice of undead monsters. A save vs. Spells is allowed to resist the effect. Mindless undead follow the drinker' s commands exactly; free-willed undead act as if under a charm person spell.

\textbf{Delusion: }This cursed potion will appear, if tested or analyzed, to be one of the other potions (other than poison). When imbibed, the drinker will briefly believe they have received the benefits of the "other" potion, but the illusion will be swiftly
exposed...

\textbf{Diminution:} This potion reduces the drinker and all items worn or carried to one-twelfth of their original height (so that a 6' tall character becomes 6" tall). The drinker' s weight is divided by 1,728; this makes an armed warrior weigh less than 2.5 ounces. The affected creature cannot make an effective attack against any creature bigger than a house cat, but may be able to slip under doors or into cracks and has a 90\% chance of moving about undetected (both in terms of sound and vision).

\textbf{Fire Resistance: }This potion grants the imbiber the power of the spell \textbf{resist fire}.

\textbf{Flying: }This potion grants the power of the spell fly.

\textbf{Gaseous Form:} The drinker and all of their gear become insubstantial, misty, and translucent. The drinker thus becomes immune to non-magical weapons, and has an Armor Class of 22 vs. magical weapons. The imbiber can't attack or cast spells while in gaseous form. The drinker also loses supernatural abilities while in gaseous form. A gaseous creature can fly at a speed of 10', and can pass through small holes or narrow openings, even mere cracks, as long as the potion persists. The gaseous creature is subject to the effects of wind, and can't enter water or other liquid. Objects cannot be manipulated in this form, even those brought along when the potion was imbibed. The drinker cannot resume material form at will, but must wait for the potion to expire; however, the potion may be quaffed in thirds, in which case each drink lasts 1d4+1 turns.

\textbf{Giant Strength: }This potion grants the imbiber the Strength of a giant. For the duration, the drinker gains a bonus of +5 on attack and damage rolls with melee or thrown weapons, and can throw large stones just as a stone giant can.

\textbf{Growth: }The drinker of this potion (with all equipment worn or carried) becomes twice normal height and eight times normal weight. The enlarged character is treated as having the Strength of a Stone Giant (but without the rock-throwing ability), gaining +5 on attack and damage rolls.

\textbf{Healing: }The imbiber of this potion receives 1d6+1 hit points of healing (as the spell cure light wounds).

\textbf{Heroism: } This potion improves the fighting ability of the drinker. Fighters of the 3rd level or less gain +3 to attack bonus as well as gaining 3 hit dice. Fighters of 4th or 5th level gain +2 to attack bonus and 2 hit dice. Fighters of 6th or 7th level gain +1 to attack bonus and 1 hit die. Fighters of 8th level or higher, as well as non-Fighter class characters, receive +1 to attack bonus but gain no hit dice. Hit dice gained are only temporary, and damage received is deducted from those hit dice first; any such points that remain when the potion expires are simply lost.

\textbf{Invisibility: }This potion makes the imbiber invisible (as the spell). This potion may be quaffed in thirds, in which case each drink lasts 1d4+1 turns.

\textbf{Invulnerability:} This potion grants a bonus of +2 to Armor Class.

\textbf{Levitation: }This potion grants the power of the spell levitate.

\textbf{Longevity: } The drinker of this potion becomes younger by 1d10
years.

\textbf{Mind Reading:} This potion grants the power of the spell of the same name.

\textbf{Poison: }This isn' t a potion at all, it' s a trap. The drinker must save vs. Poison or die, even if only a sip was imbibed.

\textbf{Polymorph Self:} This potion grants the power of the spell of the same name.

\textbf{Speed: } This potion gives the drinker the benefits of the spell
haste.

\textbf{Treasure Finding: }The imbiber of this potion will immediately know the direction and approximate distance to the largest treasure hoard in a 300' spherical radius. This potion specifically detects platinum, gold, electrum, silver, and copper; gemstones and magic items are not detected.

\subsection{Scrolls}\label{scrolls-1}\index{Scrolls}

Most scrolls contain some sort of magic which is activated when read, and which may only be used once; the characters burn away as the words are read.

\textbf{Spell Scrolls} are enchanted with one or more Cleric or Magic-User spells (never both sorts on the same scroll). Each spell can be used just once, though of course the same spell may appear multiple times on a single scroll. Use the table below to determine the spell level of each spell on a scroll. Only a Cleric can use a Clerical scroll, and only a Magic-User can use a Magic-User scroll.

Magic-Users must cast \textbf{read magic} on a spell scroll before being able to use it; each scroll needs to be treated in this way just once, and the effect lasts indefinitely thereafter. If a Magic-User attempts to cast a spell from a scroll, and they do not know that spell, there is a 10\% chance the spell will fail. If a spell on a scroll is of higher level than the highest level spell the Magic-User can cast, for eachspell level of difference, add 10\% to the chance of failure. For example, Aura the 3\textsuperscript{rd} level Magic-User attempts to cast \textbf{polymorph self} from a scroll. Aura is able to cast, at most, 2\textsuperscript{nd} level spells. \textbf{Polymorph self} is a 4\textsuperscript{th} level spell, so Aura has a chance of failure of 10\% (she doesn' t know the spell) plus 20\% (2\textsuperscript{nd} level maximum vs. 4\textsuperscript{th} level spell), for a total of 30\%.

Clerical scrolls are written in a normal language (being just specially enchanted prayers), so the Cleric merely needs to know the language in which the scroll is written in order to use it. Clerics suffer the same chance of failure as do Magic-Users, save that the 10\% penalty assigned for not knowing the spell does not apply. 

\begin{flushleft} \includegraphics[width=0.47\textwidth]{Pictures132/10000000000003FC000003CF041E16307FE1EBF7.png}  \end{flushleft}


\textbf{Spell Scrolls: Spell Level}

\begin{tabular*}{0.93\linewidth}{@{\extracolsep{\fill}}ll}
\textbf{d\%} & \textbf{Level of Spell} \\\toprule
01-30 & 1st \\\hline
31-55 & 2nd \\\hline
56-75 & 3rd \\\hline
76-88 & 4th \\\hline
89-97 & 5th \\\hline
98-00 & 6th \\\bottomrule
\end{tabular*}\medskip

A \textbf{Cursed Scroll} inflicts some curse upon whoever reads it. It need not be read completely; in fact, merely glancing at the text is enough to inflict the curse. A saving throw may or may not be allowed, as determined by the GM (though a save vs. Spells should usually be allowed). The GM is encouraged to be creative when creating curses; the spell \textbf{bestow curse} (the reverse of \textbf{remove curse}) can be used for inspiration, but cursed scrolls can contain more powerful or inventive curses at the GM' s discretion.

\textbf{Protection Scrolls} can be read by any character class, assuming the character can read the language the scroll is written in (see the notes under \textbf{Language} on page \hyperlink{languages}{\pageref{languages}} of the \textbf{Character} section for details). When read, a protection scroll creates a 10' radius protective circle around the reader; preventing the warded creatures from entering. The circle moves with the reader. Any creature other than the sort the scroll wards may enter, including of course the allies of the scroll-reader, who are themselves protected so long as they remain entirely within the circle. If any creature within the circle performs a melee attack against any of the warded creatures, the circle is broken and the warded creatures may freely attack. Normal protection scrolls last for 2 turns after being read.

\textbf{Protection from Magic} scrolls are special, as they protect against magic spells and items rather than creatures. No magical effect can cross the 10' circle of protection in either direction for 1d4 turns. As with the other protection scrolls, the circle created by this scroll moves with the reader.

\textbf{Treasure Maps} are generally non-magical. They must be created by the GM, although they may delay creating the map until the characters can actually use it. The treasure indicated on the map will normally be guarded by some sort of monster, determined by the GM as desired. 

\subsection{Wands, Staves and Rods}\label{wands-staves-and-rods-1}\index{Wands, Staves and Rods}

A wand is a short stick, generally 12 to 18 inches long, imbued with the power to cast a specific spell or spell-like effect. A newly created wand has 20 charges, and each use of the wand depletes one of those charges; a wand found in a treasure hoard will have 2d10 charges remaining. If a wand generates an effect equivalent to a spell, assume the spell functions as if cast by a 6\textsuperscript{th} level caster, or the lowest level caster who could cast that spell (whichever is higher), unless otherwise noted. Wands are generally usable only by Magic-Users. Saving throws are rolled as normal, but on the Magic Wands column rather than the Spells column.

A staff has a number of different (but often related) spell effects. A newly created staff has 30 charges, and each use of the staff depletes one or more of those charges. A staff found in a treasure hoard will have 3d10 charges remaining. Spell effects generated by a staff operate at 8\textsuperscript{th} level, or the lowest caster level the spell could be cast by, whichever is higher, unless otherwise stated. Staves are usable only by Magic-Users, except where noted. Saving throws against magic from a staff are rolled on the Spells column.

A rod is a scepter-like item with a special power unlike that of any known spell. Rods are normally usable by any class.

\textbf{Rod of Cancellation:} This rod may only be successfully used once. If struck against another magic item, it utterly destroys all enchantments within the item; this expends the rod' s magic. A roll to hit is required if the target item is held by another character or creature.

\textbf{Snake Staff:} This item is a walking staff +1. When used by a Cleric, the user may command the staff to transform into a constrictor snake (instead of causing damage) on a successful hit. The snake will wrap around a target up to man sized and hold them fast for 1d4 turns, unless a save vs. Spells is made. The snake does not attack in any other way, nor cause any damage. The snake may be recalled by the user at any point, in which case it returns to their hand and returns to staff form. It also returns in this way when the duration expires, or if the save is made. The snake has Armor Class 15, moves 20' per round and has 20 hit points; any hit points of damage taken are healed completely when the snake returns to staff form; if killed in snake form, the magic is destroyed and it turns into a broken stick. The staff may be used any number of times per day, and neither has nor uses charges.

\textbf{Staff of Commanding:} This staff can cast \textbf{charm person} and \textbf{charm monster }spells, and can grant a power equivalent to a \textbf{Potion of Plant Control}. Each function uses one charge.

\textbf{Staff of Healing:} This staff can heal 1d6+1 hit points per charge expended, as the spell \textbf{cure light wounds}. Alternately, with an expenditure of two charges, the staff can cast \textbf{cure disease}. This staff is only usable by a Cleric.

\textbf{Staff of Power:} This staff has the following powers costing one charge per use: \textbf{lightning bolt }(6d6 damage), \textbf{fireball} (6d6 damage), \textbf{cone of cold} (as the wand, for 6d6 damage), \textbf{continual light}, and \textbf{telekinesis} (as the ring, lasting at most 1d6 turns). The staff is also a \textbf{Walking Staff +2}, and can be used exactly as a \textbf{Staff of Striking}. A staff of power can be used for a \emph{\textbf{retributive strike}}, requiring it to be broken by its wielder. All charges currently in the staff are instantly released in a 30' radius, doing 1d6 points of damage per charge remaining (save vs. Spells for half damage). All within the area, including the wielder, are affected by this. Even after all charges have been used, it remains a \textbf{walking staff +2}.

\textbf{Staff of Striking:} This staff has no attack bonus, but is treated as a +1 weapon with respect to what sorts of monsters it can hit (and is usable by any class in that mode). This staff' s primary power may only be used if wielded by a Cleric: By uttering a command word, the Cleric may create an effect similar to the spell \textbf{striking}. Expenditure of one charge adds 1d6 points of damage to the weapon' s next strike; expenditure of two charges adds 2d6, and expenditure of three charges adds 3d6 damage. If the weapon is not successfully used after the command word has been spoken, the effect dissipates after one turn.

\textbf{Staff of Wizardry:} This staff is equivalent to the\textbf{ Staff of Power}, above, and has the following powers as well: \textbf{invisibility}, \textbf{passwall}, \textbf{web}, and \textbf{conjure elementals }(as the spell, but conjuring staff elementals as described in the \textbf{Monsters} section). These powers each use one charge when activated.

\textbf{Wand of Cold:} This wand generates a conical blast of cold doing 6d8 damage (save vs. Magic Wands for half damage). The cone spreads from the tip of the wand to a width of 30' at a distance of 40' away.

\textbf{Wand of Enemy Detection:} The effect of this wand is to make all enemies of the user within 60' glow with a greenish white light for one round. Even hidden or invisible enemies glow in this way, revealing them, but enemies completely out of sight (such as behind a wall) may not be seen by the user. An "enemy" is any creature which is thinking of or otherwise intending to harm the user; also, all undead monsters and animated constructs within range will glow in this way regardless of intent or thoughts (or lack thereof).

\textbf{Wand of Fear: } This wand generates the effect of the spell \textbf{cause fear }(the reverse of the spell \textbf{remove fear}).

\textbf{Wand of Fireballs: } This wand generates \textbf{fireballs}, exactly as the spell, doing 6d6 damage.

\textbf{Wand of Illusion: } This wand allows the user to create illusions equivalent to the spell \textbf{phantasmal force}.

\textbf{Wand of Lightning Bolts: } This wand generates \textbf{lightning bolts}, exactly as the spell, doing 6d6 damage.

\textbf{Wand of Magic Detection:} This wand grants the user a power equivalent to the spell \textbf{detect magic}.

\textbf{Wand of Paralysis: } This wand creates the effect of the spell \textbf{hold person}.

\textbf{Wand of Polymorph: } This wand can be used to cast either \textbf{polymorph self} or \textbf{polymorph other}.

\textbf{Wand of Secret Door Detection: } This wand grants the user a power similar to the spell \textbf{find traps}, but which reveals secret doors rather than traps.

\textbf{Wand of Trap Detection: } This wand grants the user a power equivalent to the spell \textbf{find traps}.

\subsection{Miscellaneous Items}\label{miscellaneous-items-1}\index{Miscellaneous Items}

These items may take any of several forms, and have a variety of effects; when randomly rolled, the effect is determined first (as explained with the tables above) and then the form is rolled on a separate table. Generally, such an item is written as a \textbf{{[}Form{]} of {[}Effect{]}}, for example, a \textbf{Cloak of Fire Resistance}.

Items with forms meant to be worn are limited in that only a normal number may be used at one time: at most two rings (one on each hand), one cloak, a pair of boots, one helm, and one pendant. If a character is wearing two items that grant the same continuous effect, such as \textbf{Cold Resistance} or \textbf{Protection}, only one such item will function. If the items have differing levels of effect, such as a \textbf{Cloak of Protection +1} and a \textbf{Ring of Protection +3}, only the more powerful item will operate (in this case, the ring).

While the tables on page \hyperlink{miscellaneous-items}{\pageref{miscellaneous-items}} provide standard forms for miscellaneous items, the Game Master is in no way limited to what appears on the Form of Item table; items may be created with forms not normally allowed there, or indeed even in forms that are not there at all.

\subsection{Miscellaneous Item Effects}\label{miscellaneous-item-effects}\index{Miscellaneous Item Effects}

\textbf{Blasting:} When this device is played (as appropriate for its form), it creates a powerful blast of sound filling a cone 10' long and 2' wide at the base. Those within the area of effect suffer 2d6 points of damage and are deafened for 1 full turn. A successful saving throw vs. Death Ray reduces damage by half and reduces the period of deafness to a single round.

The device can also be used to damage or destroy buildings, fortifications, etc. Double the damage listed above when this item is used against a structure. The \textbf{Stronghold} rules in the \textbf{Game Master} section contains further guidance on this.

\textbf{Blending: } The wearer of this item becomes nearly invisible, granting an 80\% chance that the wearer can move about unnoticed. If detected by onlookers, the wearer can be attacked without significant penalty.

\textbf{Cold Resistance:} The wearer of this device receives protection as the spell \textbf{resist cold}, but the protection works continually.

\textbf{Comprehension:} The wearer of this device is granted the ability to read any language, including any form of magical script. Being able to read magic does not confer magical abilities upon the wearer, but if worn by a magic-user it grants the effects of \textbf{read magic} constantly. Note that when reading non-magical texts, the limitations described under the spell \textbf{read languages} apply to this device also.

\textbf{Control Animal:} The wearer of this device can charm up to 6 hit dice of animals. The effect works much like a \textbf{charm person} spell, but only affects animals (including giant-sized animals, but excluding fantastic creatures as well as anything more intelligent than a dog or cat). The wearer can activate the power at will, targeting any animal within 60' that they can see. The wearer may choose to end the effect for one or more controlled creatures at any time, in order to "free" enough hit dice to control a new target.

\textbf{Control Human:} The wearer of this device may cast the spell \textbf{charm person} at any target they can see within 60'. The wearer can use this power once per round, at will. If more than one humanoid is to be affected, the wearer cannot control more than 6 hit dice or levels of humanoids at a time; however, the wearer may choose to end the effect for one or more controlled creatures at any time, in order to "free" enough hit dice or levels to control a new target.

\begin{center}
	\includegraphics[width=0.47\textwidth]{Pictures132/100000000000086F0000078AD2AAD2D10EF7477E.jpg}
\end{center}

\textbf{Control Plant:} The wearer of this device may create an effect equivalent to a \textbf{Potion of Plant Control} at will, affecting plants or plant creatures within 60' that they can see. The effect lasts as long as the wearer remains within 60' of the plants or plant creatures. A saving throw is allowed just as for the potion.

\textbf{Courage: }When this device is played, all characters and creatures friendly to the user within 60 feet are affected as by the spell \textbf{remove fear}.

\textbf{Deception:} This device grants to the wearer the power of the Deceiver (as described on page \hyperlink{deceiver-panther-hydra}{\pageref{deceiver-panther-hydra}}). Any attacker will believe the wearer is 3 feet from their true location, and the attacker' s first strike will always miss. Thereafter, the attacker suffers a penalty of -2 on all attack rolls. This ring does not affect mindless creatures, constructs such as golems or living statues, or any sort of undead. Living creatures which are not mindless will be affected even if they do not use sight to target the wearer.

\textbf{Delusion:} Whoever wears this device believes it is some other form of useful magical device, and behaves thus (so for example, if Darion believes he is wearing a \textbf{Ring of Invisibility} he will believe himself to actually be invisible). Unlike the potion of the same name, the device' s effect is not dispelled by the wearer suffering damage; in fact, the only way to rid a character of this cursed item is with the spell \textbf{remove curse}.

\textbf{Djinni Summoning:} Each device of this type has a specific djinni bound to it, which will be summoned to the wearer' s location when they rub the device while wearing it. The djinni appears in the next round and protects, serves, and obeys the wearer. The djinni will serve at most 1 hour per day, and can be summoned at most once per day. If the djinni bound to a device is ever slain, the ring loses all magical properties.

\textbf{Doom:} When this device is played (as appropriate for its form), this device will create animated skeletons or zombies as if by the spell \textbf{animate dead.} Up to 3d6 hit dice of undead monsters will be so created from remains within a 60' radius of the character who activated the device. If both skeletal and fleshy remains are available in the area of effect, skeletons will be animated in preference over zombies. If the user is a magic-user or cleric, the created undead may be controlled so long as that character retains the device. If played by a fighter or thief, the undead created will be uncontrolled, and will attack any living creatures nearby. The device may be used once per day, but no more than 18 hit dice of undead created by it may exist at any one time.

\textbf{Fire Resistance:} The wearer of this device receives protection as the spell \textbf{resist fire}, but the protection works continually.

\textbf{Invisibility:} The wearer of this device can become invisible (as the spell \textbf{invisibility}) on command. If the invisibility is dispelled (as described for the spell), the device may not be reactivated for one full turn. The invisibility effect otherwise lasts for 24 hours.

\textbf{Levitation:} The wearer of this device may \textbf{levitate} (as the spell) at will by concentrating. There is no limit to how long this device may be used.

\textbf{Mind Reading:} Whoever wears this device has access to a permanent form of the spell \textbf{mind reading}; it is always available but only activates when the wearer spends a full round concentrating upon it, and persists until the wearer ceases to use it. The effect can be activated as many times per day as the wearer wants.

\textbf{Panic:} When this device is played (as appropriate for its form), all creatures more than 20 feet from the user but not over 120 feet away must save vs. Spells or be affected as by the spell \textbf{cause fear}.

\textbf{Penetrating Vision:} On command, and for so long thereafter that the wearer concentrates on it, this device confers the power to see through solid matter as if it were transparent as glass. The effect extends at most 20 feet, and the wearer sees as if in normal light even if they are in fact in total darkness.

This effect is blocked by certain materials; the wearer can see through at most 3 feet of wood or soil, 1 foot of stone, or 1 inch of most metals. Gold or lead no thicker than foil will completely block the effect.

The device may be used three times per day, and each use lasts at most one turn.

\textbf{Protection:} The wearer of an item with this power receives the listed benefit (from +1 to +3) to their Armor Class for so long as the ring is worn. This bonus is also applied to the wearer' s saving throw die rolls.

\textbf{Protection from Energy Drain:} This device has the power to absorb and neutralize energy-draining attacks, death spells or effects (such as \textbf{slay living}), and curses that would otherwise affect the wearer. The device has 2d6 charges when found, and each negative level, curse, or spell absorbed consumes one charge\emph{.} When the device' s charges are exhausted it disintegrates into golden sparkles and disappears.

\textbf{Protection from Scrying:} The wearer of this item is immune to all forms of scrying (including crystal balls, clairvoyance, clairaudience, and any other means of location or spying at a distance) as well as any form of \textbf{mind reading}. Other characters who are within 30' of the wearer are also immune to scrying, but not to \textbf{mind reading}.

\textbf{Regeneration:} This device grants the wearer the power of regeneration, exactly as described in the description of the Troll on page \hyperlink{troll-and-trollwife}{\pageref{troll-and-trollwife}}, including the weakness with respect to acid and fire damage. Note that this device will only heal damage suffered while it is worn; pre-existing damage is not healed by putting on the device.

\textbf{Scrying:} This device can be used to spy upon other people or locations, regardless of distance. A scrying device may only be used by Magic-Users. It can be used three times per day, for up to a turn each time.

The chance of success when using a scrying device is as shown below. Total chances equal to or greater than 100\% do not require a roll.\medskip

\begin{tabular*}{0.93\linewidth}{@{\extracolsep{\fill}}ll}
Knowledge and Connection & Chance \\\toprule
Secondhand Knowledge (heard of) & 25\% \\\hline
Firsthand Knowledge (seen briefly) & 55\% \\\hline
Familiar (known well) & 95\% \\\hline
Possession or garment & +25\% \\\hline
Body part, lock of hair, bit of nail, etc. & +50\% \\\bottomrule
\end{tabular*}\medskip

The user of the device is the only one who will see the image. No sound will be heard normally. \textbf{Detect magic}, \textbf{detect evil}, and \textbf{mind reading} have a 3\% chance per level of the caster of operating correctly if used with a scrying device.

\textbf{Scrying, Superior:} This item works exactly like a standard scrying device, as described above, but also allows the user to hear any sounds in the location viewed as if they were there.

\textbf{Speed:} The wearer of this device can activate it with a command word (or by clicking their heels together, if the item takes the form of boots) and gain the effect of a \textbf{haste}\emph{ }spell, and can end he effect the same way. The effect can be used for a total of 10 rounds each day.

\textbf{Spell Storing:} These much sought-after devices each contain a number of spells which can be cast by the wearer. Most of them contain Magic-User spells, but 1 in 10 contains Clerical spells instead. \emph{No device may contain spells of both types! } Each spell stored in the device is cast as if by the lowest-level character who could normally cast the spell, but not less than 6\textsuperscript{th} level in any case.

Any class may wear and use this device, but it can only be recharged by casting the appropriate spell into it. A table is provided below to determine how many spells, and what levels they are. A spell storing device must be recharged with the same spells that were placed into it when it was made; so a \textbf{Pendant of Two Spell Storing} containing \textbf{fireball} and \textbf{fly }can only be recharged with those two spells.

The wearer of one of these devices automatically knows the names of the spells stored within it, but is not granted detailed knowledge of how each spell works. If the wearer is not a magic-user and no such character is available to advise the wearer, there may be some difficulty in using it successfully (and safely).

A device of this type found in a treasure hoard may be completely charged, or discharged, or partially charged, at the GM' s option.\medskip

\begin{tabular*}{0.93\linewidth}{@{\extracolsep{\fill}}lllll}
d\% & \# of Spells & & d\% & Level of Spell \\\toprule
01-24 & 1 & & 01-30 & 1st \\\hline
25-48 & 2 & & 31-55 & 2nd \\\hline
49-67 & 3 & & 56-75 & 3rd \\\hline
68-81 & 4 & & 76-85 & 4th \\\hline
82-91 & 5 & & 86-97 & 5th \\\hline
92-96 & 6 & & 98-00 & 6th \\\bottomrule
97-00 & 7 & & & \\\hline
\end{tabular*}\medskip

\textbf{Spell Turning:} This device reflects spells cast directly at the wearer, but not area effect spells, back at the caster; so a \textbf{hold person} spell would be reflected, but not a \textbf{fireball}. It will reflect up to 2d6 spells before its power is exhausted.

\textbf{Stealth:} The wearer of this device can move quietly in virtually any surroundings, granting a 90\% chance of success when moving silently (as the Thief ability of the same name).

\textbf{Telekinesis:} The wearer of this device can use the power of the spell \textbf{telekinesis}, as if cast by a 12\textsuperscript{th} level Magic-User. The effect may be used as many times per day as the wearer wishes, but lasts only as long as the wearer concentrates on it.

\textbf{Telepathy:} Three times per day this item can be activated by use of its command word, at which point it will grant the wearer the power of a special version of the \textbf{mind reading} spell which has a range of 90 feet and lasts for 1 turn. During this time the wearer can send thoughts to the mind of any creature whose thoughts the wearer is already reading, allowing communication.

\textbf{Teleportation:} Upon command the wearer of this item may cast the spell \textbf{teleportation} as if by a wizard of the 12\textsuperscript{th} level of ability. This power can be used up to 3 times per day.

\textbf{True Seeing:} Three times per day this device can grant the user the power of the spell \textbf{true seeing}. Each use lasts at most one turn.

\textbf{Water Walking:} This device allows the wearer to walk on any liquid as if it were firm ground. Mud, oil, snow, quicksand, running water, ice, and even lava can be traversed easily, since the wearer's feet hover an inch or two above the surface. Molten lava will still cause the wearer damage from the heat since they are still near it. The wearer can walk, run, or otherwise move across the surface as if it were normal ground.

\textbf{Weakness:} Whoever puts this device on is cursed; their Strength score is reduced immediately to 3. The device can only be removed with \textbf{remove curse}. 

\textbf{Wishes:} This device contains the power to grant wishes to the wearer. 1d4 wishes will remain within the ring when it is found. The GM must adjudicate all wishes, and instructions are given in the \textbf{Game} \textbf{Master} section regarding this. 


\subsection{Rare Items}\label{rare-items-1}\index{Rare Items}

\textbf{Bag of Devouring:} This device appears, to all tests, to be a normal \textbf{Bag of Holding}, and in fact it performs exactly like one at first. However, all items placed within disappear forever 1d6+6 turns later. The bag continues to weigh whatever it did after the items were placed within it (that is, one-tenth the total weight of the items), until it is opened and discovered to be empty.

\textbf{Bag of Holding:} This device is a bag which appears to be about 2 feet wide and 4 feet deep. It opens into an extradimensional space, and is able to hold more than should be possible: up to 500 pounds of weight, and up to 70 cubic feet of volume. The bag weighs one-tenth as much as the total of the objects held within. Any object to be stored in the bag must fit through the opening, which has a circumference of 4 feet.

Puncturing or tearing the bag will destroy its magic and cause all contents to be lost forever. If this item is turned inside out all contents are dumped. The bag is unharmed, but it will no longer work until it is turned right side out again. If living creatures are placed inside they will suffocate within a turn (with exceptions for creatures resistant to suffocation as determined by the GM). The bag' s volume cannot be overfilled (as excess items simply cannot be put inside), but if overloaded above 500 pounds and then lifted it will be torn.

Getting any particular item from the bag requires the bearer to spend a round searching during which no movement may be made.

\textbf{Boots of Traveling and Leaping:} These boots allow the wearer to make great leaps, jumping up to 10' high and/or 30' across. They improve the wearer' s movement so greatly that they also increase their movement rate on land by an additional 10' per round.

\textbf{Broom of Flying:} This broom is able to fly through the air for up to 9 hours per day (split up as its owner desires). The broom can carry 200 pounds and fly at a speed of 40 feet, or up to 400 pounds at a speed at 30 feet. In addition, the broom can travel alone to any destination named by the owner as long as they have a good idea of the location and layout of that destination. It comes to its owner from as far away as 300 yards when the command word is spoken.

\textbf{Efreeti Bottle:} This item may appear as an ornate bottle, or sometimes as a magic lamp. It can be activated once per day, by opening it if it takes the form of a bottle or by rubbing it if it takes the form of a lamp. When activated, smoke pours out and forms into an efreeti.

The efreeti released was trapped in the bottle and forced to serve whoever activates it. However, an efreeti who has spent too long in a bottle may have lost its mind, and if this happens the efreeti will begin a frenzied attack upon whoever activated the bottle, disappearing when either the user of the bottle or the efreeti is dead. There is, fortunately, only a 1 in 10 chance this will happen.

On the other hand, there is also a 1 in 10 chance that the efreeti of the bottle is able to grant three wishes to the user. If this is the case, the efreeti will perform no other service, and cannot return to the bottle after it is activated until the user makes a wish. Subsequent wishes require additional activations, and upon the final wish being granted the efreeti disappears for good.

Roll 1d10 when the bottle is first activated; on a roll of 1, the efreeti is insane, while on a roll of 0 the efreeti has three wishes to grant. If neither of these results is rolled, the efreeti will serve the user for up to one hour per day for 101 days, after which time it is freed and will disappear. Note that after the first activation, every day that passes is counted toward the 101 days whether or not the user activates the bottle.

\textbf{Flying Carpet:} This item appears to be a fancy rug of the sort found in the castle of a king. It has the power to fly, carrying those upon it as if they stand upon a solid surface. A flying carpet is typically 5' x 8' in size and can carry up to 500 pounds at a movement rate of 100' per round, or up to 1,000 pounds (its maximum capacity) at a rate of 50' per round. A flying carpet can fly at any speed up to its maximum, and can hover on command.

\textbf{Gauntlets of Ogre Power:} These thick leather gloves grant the wearer a Strength bonus of +4 (instead of their own Strength bonus). Both gauntlets must be worn for the magic to be effective.

\textbf{Girdle of Giant Strength:} This broad leather belt grants the wearer the strength of a giant. For so long as it is worn, the wearer gains a Strength bonus of +5 (instead of their own Strength bonus), and can throw large stones just as a stone giant does.

\textbf{Mirror of Imprisonment:} This item can appear as any style of full-length or larger mirror. It is a form of magical trap, which can be set or deactivated by speaking a command word followed by "activate" or "deactivate."

When in its active state, any character or creature who stands within 30 feet of the mirror and sees its reflection must save vs. Spells or be drawn bodily into the mirror, including all items worn or carried. The victim is placed within one of 20 metaphysical cells inside the mirror. Those trapped in the mirror are mere reflections and are unable to take any action. The last character who spoke the command word to the mirror is immune to its power, as are undead, constructs, and any creature that lacks eyes.

A character who speaks the command word and then calls the number of a cell will cause the reflection of the occupant of that cell to appear in the mirror. The trapped creature can move and speak, but cannot cast spells or take any other real action. The controlling character may interrogate the trapped victim if desired, though the mirror does not compel the victim to respond or to be truthful. The controller may at any time speak the command word again and say "return" and the victim will be returned to its cell; or, the controller can say the command word followed by "come out" to free the victim, who appears standing beside the controller.

Should all cells be full when a creature is drawn inside, one cell chosen at random will be emptied with the freed creature appearing where the trapped one had been standing. The freed creature is safe from the mirror for one turn thereafter before it can again be trapped. All victims will be freed instantly if the mirror is broken.

\textbf{Rope of Climbing:} This 50 foot long rope is about ½ inch in diameter, but is capable of supporting up to 3,000 pounds if tied to a secure anchor point. When the user holds one end of the rope and speaks the command word, the rope animates, moving like a snake at a rate of 10' per round in whatever direction the user commands. The rope is even able to move into a completely vertical position if so ordered. It can be commanded to tie itself to any anchor point within reach (since the user must continue holding one end of the rope, it can reach no more than 50 feet from that point). The rope has no real strength and thus cannot lift or support any weight if not tied to an anchor point.

\subsection{Devices of Summoning Elementals}\label{devices-of-summoning-elementals-1}\index{Devices of Summoning Elementals}

These devices all grant the power to summon and control an elemental. As noted, the GM is likely to choose to allow only certain types of Elementals, so not all of these devices may be available in your campaign.

When one of these devices is activated in accordance with the summoning rules described for the Elemental monster entry on page \hyperlink{elemental}{\pageref{elemental}}, an appropriate elemental appears and follows the summoner' s commands.

\textbf{Bowl of Summoning Water Elementals:} This device is about the size of a punch bowl, and will be made of precious metals set with jewels. To activate this device, it must be filled with fresh water (at least a quart) and the command words inscribed around its rim must be spoken, which requires a full round to complete.

\textbf{Brazier of Summoning Fire Elementals:} This device appears to be a small brazier, a metal bowl with legs and cage-like sides which stands about a 1½ feet tall. Braziers are meant to hold fire. To activate this device, a fire must be built in this brazier and the command words inscribed around its rim-band must be spoken, which requires a full round to complete.

\textbf{Censer of Summoning Air Elementals:} This device appears to be an ornate but ordinary censer, a kind of bowl with a perforated lid and a chain which is used to lift and swing the device. These are normally used in worship, but this device is made for a different purpose. To activate this device, incense must be placed in the bowl, lit, and covered, and the command words inscribed around the edge of the lid must be spoken, which requires a full round to complete.

\textbf{Crucible of Summoning Metal Elementals:} This device appears to be a small crucible, an open-topped vessel used for melting metal. To activate this device, it must be placed on a solid, adequately heat-resistant surface and the command words inscribed around the edge of the lid must be spoken, which requires a full round to complete. 

The crucible becomes very hot while the metal elemental is present, causing 2d8 points of damage to any character who dares to touch it barehanded and 2d6 points even if wearing armor or ordinary fabric or leather gloves. The crucible requires a full turn to cool after use.

\textbf{Mallet of Summoning Wood Elementals: }This is a wooden mallet of the sort used by builders to drive in pegs; though it could be used as a weapon, it is a poor excuse for one as it does just 1d4 points of damage on a hit and is not treated as a magic weapon. To activate this device, it must be held in the hand and the command words inscribed on one side of the hammer-head must then be spoken, which requires a full round to complete.

\textbf{Marble Plate of Summoning Cold Elementals:} This device is a circular plate of marble about 9 inches in diameter and about ¾ of an inch thick. To activate this device, it must be placed level on a solid, stable surface and a full skin of fresh water must be poured upon it; as the water touches the base it magically freezes into a small sculpture of a cold elemental. The command words inscribed around the rim of the plate must then be spoken, which requires a full round to complete.

\textbf{Rod of Summoning Lightning Elementals:} This device appears as a copper rod about ¾ inch thick and 2 feet long, tipped with a polished ball of amber. The rod will have a smooth, polished green surface (a thin coating of verdigris) but will not be pitted or damaged by the corrosion. To activate this device it must be held in the hand and the command words inscribed on the length of the rod must then be spoken, which requires a full round to complete.

\textbf{Stone of Summoning Earth Elementals:} This device is a roughly-cut but highly-polished stone in which a vein or cluster of precious or semi-precious stonesis  visible. To activate this device, it must be held in the user' s hands and the command words inscribed on the smoothest area of the stone must be spoken, which requires a full round to complete.

\end{multicols}

\pagebreak

\section{PART 8: GAME MASTER INFORMATION}\label{part-8-game-master-information}\index{Part 8: Game Master information}

\subsection{Wandering Monsters}\label{wandering-monsters}\index{Wandering Monsters}

\textit{We had the foresight to bring several large sacks with us, and we swiftly filled them with coins and gems from beneath the sarcophagus. Without further delay we moved out, intent upon reaching the stairs to the surface and then returning to Morgansfort. But it couldn' t be that easy...}

\textit{On the way in, Barthal scouted ahead and we took our time, constantly on the lookout for monsters. On the way out, we threw caution to the wind, moving at full speed with Barthal watching behind us. So it was that Morningstar and I turned a corner and practically stepped on the first rank of a goblin patrol!}

\textit{Once again I was caught flatfooted, but so were the goblins. Morningstar reacted more swiftly, striking down the first of the little monsters. You might think that parley would have been a better idea, but we had already tried that with these goblins without success... so I couldn' t blame the Elf for striking first and asking questions later.}

\textit{I raised the golden sword and waded into battle...}

\subsection{Dungeon Encounters}\label{dungeon-encounters}\index{Dungeon Encounters}

Besides "placed" monsters, dungeons usually contain wandering monsters. The Game Master may create special wandering monster tables for specific dungeons, or the general wandering monster tables (below) may be used. 

In an average dungeon, a wandering monster encounter will occur on a roll of 1 on 1d6; the Game Master should check once every 3 turns. The circumstances of a specific dungeon may call for higher odds or more frequent (or possibly less frequent) wandering monster checks.\medskip

\begin{tabular*}{1\linewidth}{@{\extracolsep{\fill}}llll}
\textbf{Roll 1d12} & \textbf{Level 1} & \textbf{Level 2} & \textbf{Level 3} \\
1 & Bee, Giant & Beetle, Giant Bombardier & Ant, Giant \\\toprule
2 & Goblin & Fly, Giant & Ape, Carnivorous \\\hline
3 & Jelly, Green* & Ghoul & Beetle, Giant Tiger \\\hline
4 & Kobold & Gnoll & Bugbear \\\hline
5 & NPC Party: Adventurer & Jelly, Gray & Doppleganger \\\hline
6 & NPC Party: Bandit & Hobgoblin & Gargoyle* \\\hline
7 & Orc & Lizard Man & Jelly, Glass \\\hline
8 & Stirge & NPC Party: Adventurer & Lycanthrope, Wererat* \\\hline
9 & Skeleton & Snake, Pit Viper & Ogre \\\hline
10 & Snake, Cobra & Spider, Giant Black Widow & Shadow* \\\hline
11 & Spider, Giant Crab & Lizard Man, Subterranean & Tentacle Worm \\\hline
12 & Wolf & Zombie & Wight* \\\bottomrule
& & & \\\hline
\textbf{Roll 1d12} & \textbf{Level 4-5} & \textbf{Level 6-7} & \textbf{Level 8+} \\
1 & Bear, Cave & Basilisk & Basilisk, Greater* \\\hline
2 & Caecilia, Giant & Jelly, Black & Chimera \\\hline
3 & Cockatrice & Caecilia, Giant & Deceiver, Greater \\\hline
4 & Doppleganger & Deceiver & Giant, Hill \\\hline
5 & Jelly, Gray & Hydra & Giant, Stone \\\hline
6 & Hellhound & Rust Monster* & Hydra \\\hline
7 & Rust Monster* & Lycanthrope, Weretiger* & Jelly, Black \\\hline
8 & Lycanthrope, Werewolf* & Mummy* & Lycanthrope, Wereboar* \\\hline
9 & Minotaur & Owlbear & Purple Worm \\\hline
10 & Jelly, Ruddy* & Scorpion, Giant & Salamander, Flame* \\\hline
11 & Owlbear & Spectre* & Salamander, Frost* \\\hline
12 & Wraith* & Troll & Vampire* \\\hline
\end{tabular*}

\subsection{Wilderness Encounters}\label{wilderness-encounters}\index{Wilderness Encounters}

The Game Master should check for random encounters in the wilderness about every four hours of game time; this translates nicely to three night checks and three daytime checks. If your players choose to stand three night watches, you simply check for each watch; in the daytime, check morning, afternoon, and evening.

To check for a wilderness encounter, roll 1d6; on a roll of 1, an encounter occurs. If a wilderness encounter is indicated, roll 2d8 on the appropriate table below. The Game Master should think carefully about how the encounter happens; check for surprise in advance, and if the monster is not surprised, it may be considered to have had time to set up an ambush (at the GM' s option).\medskip

\begin{tabular*}{1\linewidth}{@{\extracolsep{\fill}}llll}

\textbf{Roll 2d8} & \textbf{Desert or Barren} & \textbf{Grassland} & \textbf{Inhabited Territories} \\\toprule
2 & Dragon, Desert & Dragon, Plains & Dragon, Cloud \\\hline
3 & Hellhound & Troll & Ghoul \\\hline
4 & Giant, Fire & Fly, Giant & Bugbear \\\hline
5 & Purple Worm & Scorpion, Giant & Goblin \\\hline
6 & Fly, Giant & NPC Party: Bandit & Centaur \\\hline
7 & Scorpion, Giant & Lion & NPC Party: Bandit \\\hline
8 & Camel & Boar, Wild & NPC Party: Merchant \\\hline
9 & Spider, Giant Tarantula & NPC Party: Merchant & NPC Party:
Pilgrim \\\hline
10 & NPC Party: Merchant & Wolf & NPC Party: Noble \\\hline
11 & Hawk & Bee, Giant & Dog \\\hline
12 & NPC Party: Bandit & Gnoll & Gargoyle* \\\hline
13 & Ogre & Goblin & Gnoll \\\hline
14 & Griffon & Flicker Beast & Ogre \\\hline
15 & Gnoll & Wolf, Dire & Minotaur \\\hline
16 & Dragon, Mountain & Giant, Hill & Vampire* \\\hline
& & & \\\hline
\textbf{Roll 2d8} & \textbf{Jungle} & \textbf{Mountains or Hills} & \textbf{Ocean} \\\hline
2 & Dragon, Forest & Dragon, Ice & Dragon, Sea \\\hline
3 & NPC Party: Bandit & Roc (1d6: 1-3 Large,4-5 Huge,6~Giant) & Hydra \\\hline
4 & Goblin & Deceiver & Whale, Sperm \\\hline
5 & Hobgoblin & Lycanthrope, Werewolf* & Crocodile, Giant \\\hline
6 & Centipede, Giant & Mountain Lion & Crab, Giant \\\hline
7 & Snake, Giant Python & Wolf & Whale, Killer \\\hline
8 & Elephant & Spider, Giant Crab & Octopus, Giant \\\hline
9 & Antelope & Hawk & Shark, Mako \\\hline
10 & Jaguar & Orc & NPC Party: Merchant \\\hline
11 & Stirge & Bat, Giant & NPC Party: Buccaneer (Pirate) \\\hline
12 & Beetle, Giant Tiger & Hawk, Giant & Shark, Bull \\\hline
13 & Caecilia, Giant & Giant, Hill & Roc (1d8: 1-5 Huge, 6-8 Giant) \\\hline
14 & Shadow* & Chimera & Shark, Great White \\\hline
15 & NPC Party: Merchant & Wolf, Dire & Mermaid \\\hline
16 & Lycanthrope, Weretiger* & Dragon, Mountain & Sea Serpent \\\hline
& & & \\\hline
\textbf{Roll 2d8} & \textbf{River or Riverside} & \textbf{Swamp} & \textbf{Woods or Forest} \\
2 & Dragon, Swamp & Dragon, Swamp & Dragon, Forest \\\hline
3 & Fish, Giant Piranha & Shadow* & Alicorn (see Unicorn) \\\hline
4 & Stirge & Troll & Treant \\\hline
5 & Fish, Giant Bass & Lizard, Giant Draco & Orc \\\hline
6 & NPC Party: Merchant & Centipede, Giant & Boar, Wild \\\hline
7 & Lizard Man & Leech, Giant & Bear, Black \\\hline
8 & Crocodile & Lizard Man & Hawk, Giant \\\hline
9 & Frog, Giant & Crocodile & Antelope \\\hline
10 & Fish, Giant Catfish & Stirge & Wolf \\\hline
11 & NPC Party: Buccaneer & Orc & Ogre \\\hline
12 & Troll & Toad, Giant (see Frog, Giant) & Bear, Grizzly \\\hline
13 & Jaguar & Lizard Man, Subterranean & Wolf, Dire \\\hline
14 & Nixie & Blood Rose & Giant, Hill \\\hline
15 & Water Termite, Giant & Hangman Tree & Owlbear \\\hline
16 & Dragon, Forest & Basilisk & Unicorn \\\bottomrule
\end{tabular*}\medskip

\pagebreak

\begin{multicols}{2}
	
\subsection{City, Town or Village Encounters}\label{city-town-or-village-encounters}\index{City, Town or Village Encounters}

It' s important for the Game Master to remember that, unlike dungeon or wilderness environments, cities, towns and villages are busy places. During the day, most towns will have people on the streets more or less all the time; the absence of people on the streets is often an indication of something interesting. By night, much of the town will be dark and quiet, and encounters will be mostly Thieves or other unsavory types; but near popular eating or drinking establishments, people of all sorts are still likely to be encountered. The GM must make sure that their descriptions of the town environment make this clear; of course, this will also make it harder for the players to identify "real" encounters.

The GM is encouraged to create their own encounter tables for use in each city, town or village created (or assign encounters by other means if desired); however, a set of "generic" encounter tables are provided below for those times when such preparation has not been completed. Roll 2d6 on the table below to determine what sort of encounter occurs; a description of each type of encounter appears below the table.\medskip

\begin{tabular*}{0.93\linewidth}{@{\extracolsep{\fill}}lll}
\textbf{Die} & \textbf{Day} & \textbf{Night}\\
\textbf{Roll}&\textbf{Encounter}&\textbf{Encounter}\\\toprule
2 & Doppleganger & Doppleganger \\\hline
3 & Noble & Shadow* \\\hline
4 & Thief & Press Gang \\\hline
5 & Bully & Beggar \\\hline
6 & City Watch & Thief \\\hline
7 & Merchant & Bully \\\hline
8 & Beggar & Merchant \\\hline
9 & Priest & Giant Rat \\\hline
10 & Mercenary & City Watch \\\hline
11 & Wizard & Wizard \\\hline
12 & Lycanthrope, Wererat* & Lycanthrope, Wererat* \\\bottomrule
\end{tabular*}\medskip

\textbf{Beggar} encounters will often begin with a single beggar approaching the party, but there will generally be 2d4 beggars in the area, and if any party member gives anything to the first beggar, the others will descend on the party like flies. Each beggar is 90\% likely to be a normal man, and 10\% likely to be a 1\textsuperscript{st} level Thief, possibly scouting for the Thieves Guild or a local gang.


\textbf{Bully} encounters will be with 2d4 young toughs; each is 70\% likely to be a normal man, 30\% likely to be a 1\textsuperscript{st} level Fighter. Bullies generally appear unarmed, depending on their brawling ability in a fight (but keeping a dagger or shortsword hidden, to be used in case the fight is going against them). Bullies can be a bit unpredictable, such that the GM may want to use a reaction roll to determine the leader' s mood.

\textbf{City Watch }encounters will be with 2d6 watchmen, all 1\textsuperscript{st} level Fighters save for the squad leader, who will be from 2\textsuperscript{nd} through 4\textsuperscript{th} level. They will confront "suspicious-looking" characters, but generally will need a good reason before they attempt to arrest or otherwise interfere with player characters.


\begin{flushleft} \includegraphics[width=0.47\textwidth]{Pictures132/10000000000003CF000004844502DDC7ABCA473F.png}  \end{flushleft}

\textbf{Doppleganger} encounters will, of course, appear to be some other type of encounter; the GM should roll again to determine what the doppleganger is masquerading as. 1d6 dopplegangers will be encountered; any extra group members will be humans who do not know they are traveling in the company of shapeshifting monsters. If the party is "interesting" to the dopplegangers, one or more of the monsters will attempt to follow them and replace a party member (as described in the monster description). In many cases, player character parties will not discover the true nature of the encounter until much later. 

\textbf{Giant Rat} encounters will generally involve alleys, the docks, or other "low" places. Rats are generally not dangerous unless provoked, but if surprised they may attack. See the monster description for details of this encounter type.

\textbf{Lycanthrope, Wererat} encounters will appear to be some other type of encounter, either another sort of "normal" encounter or a giant rat encounter (depending on the circumstances). Wererats are cowardly and will not attack a party of equal or larger size.

\textbf{Mercenary} encounters will involve 2d6 members of a mercenary company, going about some business or other. A mercenary leader may offer a position to Fighter-classed player characters if they have any reputation at all.

\textbf{Merchants} are a common feature of towns, and may be encountered performing any sort of business. As with mercenary encounters, merchants may offer jobs to interesting player characters, particularly those with good reputations. See \textbf{Creating an NPC Party}, below, for details on this type of encounter. (A merchant in a town may not have a full entourage as described below; the GM should use their discretion in creating the encounter.)

\textbf{Nobles} encountered may also offer positions to player characters, or possibly offer a reward for some dangerous task. Player characters with bad reputations may be confronted, ordered to leave town, or even arrested if the noble is able to call for the city watch. (See \textbf{Creating an NPC Party}, below, for details on this type of encounter.) A noble in a town may not have a full entourage as described below; the GM should use their discretion in creating the encounter.

\textbf{Press Gangs} will consist of 2d6 Fighters, all 1\textsuperscript{st} level except for one or two leaders of 2\textsuperscript{nd} through 5\textsuperscript{th} level. They will be armed with blunt weapons or possibly will fight with their bare hands, since their goal is to capture rather than kill player characters; however, it is likely that at least some members of a press gang will have daggers or swords on their persons in case a serious fight breaks out. A press gang will not confront a party of equal or greater size unless the party is obviously weakened, drunk, etc. If the party loses, they will awaken aboard a ship at sea or in a military camp (depending on whether sailors or soldiers captured them), unarmed and at the mercy of their captors.

\textbf{Priest} encounters will usually be similar to a group of pilgrims (see \textbf{Creating an NPC Party}, below, for details), though the group encountered will not be as large as would be encountered in the wilderness. Generally, a single priest of 1\textsuperscript{st} through 4\textsuperscript{th} level will be encountered, accompanied by 1d4 of the faithful. 

\textbf{Shadow} encounters in a town will be much like the same encounter underground; see the monster description for details.

\textbf{Thief} encounters will be with a group of 1d6 Thieves, generally disguised as ordinary townsmen or sometimes as beggars. One Thief in the group will be from 2\textsuperscript{nd} to 4\textsuperscript{th} level, with the others being 1\textsuperscript{st} level only. They will seek to steal from the party, of course, unless watched very carefully. 

\textbf{Wizard }encounters will involve a Magic-User of 4\textsuperscript{th} through 7\textsuperscript{th} level, accompanied by 1d4-1 apprentices of 1\textsuperscript{st} level. The GM must decide on the temperament and mood of the wizard.

\subsection{Creating An NPC Party}\label{creating-an-npc-party}\index{Creating An NPC Party}

\textbf{Adventurers}

A party of NPC adventurers will usually consist of 4-8 characters, as follows: 1d3 Fighters, 1d2 Thieves, 1d2 Clerics, and 1d2-1 Magic-Users. Usually the characters will all be of similar levels; after deciding what average level the party should be, you may wish to make a few of the characters lower levels (to reflect the usual "replacements" brought in when some characters die).

The Game Master must choose the race or races of the NPC adventurers to suit the region where they are found (or come from). Probably 80\% or more of adventurers are Human, 10\% are Dwarves, 6\% are Halfling and the remaining 4\% Elvish. If the NPC adventurer party is evil, the GM may choose to replace some party members with humanoid monsters such as orcs, hobgoblins, or gnolls.

The party may be rivals with the player characters, vying for the same treasures, or they may actually be enemies, evil marauders that the player characters must defeat. It is, of course, possible that the NPC adventurers are allied or otherwise friendly with the player characters, but this may make things too easy for the players.

\begin{wrapfigure}{r}{0.25\textwidth}
	\includegraphics[width=0.25\textwidth]{Pictures132/10000000000003680000084870A32F874DA387F6.png} 
\end{wrapfigure}

\textbf{Bandits, Brigands, and Highwaymen}

A party of bandits will generally consist of 2d12 1\textsuperscript{st} level Fighters and 1d6 1\textsuperscript{st} level Thieves, led by a Fighter or Thief of 2\textsuperscript{nd} to 5\textsuperscript{th} level (1d4+1) or by one of each class (if there are 11 or more 1\textsuperscript{st} level members total). In the wilderness, bandits will generally have horses or other steeds appropriate to the terrain (stolen, of course) as well as light armor, swords and bows or crossbows. Determine magic items as given below for the leaders only; rank-and-file members will not normally have magic items.

In their lair or hideout, a party of bandits will generally have type A treasure (with magic items omitted since they will have been generated using the rules below).

\textbf{Buccaneers and Pirates}

The difference between buccaneers and pirates is largely a question of what they wish to be called; whatever you call them, they are waterborne equivalents of bandits, attacking other ships or raiding coastal towns for plunder.

A buccaneer party will consist of 3d8 1\textsuperscript{st} level Fighters, led by a Fighter of 3\textsuperscript{rd} to 6\textsuperscript{th} level (1d4+2) and 1d3 Fighters of 2\textsuperscript{nd} to 5\textsuperscript{th} level. All will be experienced at handling ships, of course. They will be unarmored or armored only in leather, and will be armed with swords and bows or crossbows.

Seagoing pirates may appear in larger numbers, but the number of leader-types will be similar to that given above. Generate magic items for leaders only as described below. A shipload of pirates or buccaneers will have a type A treasure, with magic items omitted (since magic items will already have been rolled for the NPCs); the treasure may not be aboard the ship, however, as pirates often bury their treasures on distant shores. In such a case, the Captain or one of his mates will have a map of some kind leading to the treasure.

\textbf{Merchants}

Merchants must often transport their wares through wilderness areas. About half of the time (50\%), a land-bound merchant party will be led by a single wealthy merchant; other merchant parties will consist of 1d4+1 less wealthy merchants who have banded together for their own safety. There will be 2d4 wagons (but at least one per merchant) drawn by horses or mules. Each wagon is driven by a teamster who is a normal man, usually unarmored and armed with a dagger or shortsword. The caravan will employ 1d4+2 first-level Fighters and 1d4 second-level Fighters as guards.

If encountered at sea, a merchant party will generally consist of a single ship owned or rented by a single merchant. The ship will have a crew of 2d8+8 regular crewmen, who are normal men, unarmored and armed with clubs, daggers or shortswords; the Captain, First Mate, and other officers are taken from this number. Large ships may require larger crews. 1d4+2 first-level Fighters and 1d4 second-level Fighters will be aboard as guards, just as with a caravan.

Besides the valuable but undoubtedly bulky trade goods transported by the merchant caravan or ship, such a party will also have a type A treasure, with magic items omitted; it may be in one chest, or spread out among the wagons.

\textbf{Nobles}

A noble party will consist of a noble (of course), possibly accompanied by a spouse (also a noble, of course) and/or one or more children. Each adult noble will have at least one attendant (assistant, lady-in-waiting, etc.).

Lower-ranking nobles (such as barons) will have a single wagon or carriage, drawn by fine horses; higher-ranking nobles will have two or more wagons. The noble may be mounted on a warhorse, though they may choose to ride in a carriage part of the time. Each carriage or wagon will have a teamster, who in this case will be a 1\textsuperscript{st} level Fighter in chainmail with a longsword. At least two mounted Fighters of 1\textsuperscript{st} through 4\textsuperscript{th} level will be with the noble as guards; again, higher ranking nobles will have more guards. Guards will generally be armed with longswords and possibly lances, armored in platemail, and their warhorses will usually be barded with chainmail. Determining the exact number of guards is left to the GM in this case. The normal chances for magic items apply, of course.

A noble will usually be traveling with a little spending money; a type A treasure should be rolled to represent this. In this case, do not omit the magic items, as nobles will generally be more wealthy than the average party of men.

Nobles are usually (70\%) normal men; otherwise, roll 1d10: 1-6 indicates a Fighter, 7-8 indicates a Magic-User, 9 indicates a Cleric, and 10 indicates a Thief. (Clerical "nobles" are bishops, archbishops, and the like.) Roll 2d4-1 for the level of each "classed" noble. 

\textbf{Pilgrims}

A party of pilgrims is on its way to (or from) a major religious locale or activity. Such a party will be led by a 1d4 Clerics of level 1-4 (roll for each).

The remainder of the party is rather random in nature; most pilgrim groups include 3d6 normal men (or women if the religion allows women to go on pilgrimages), 1d6 Fighters of level 1-4 (roll for each) with chainmail and longsword, and 1d4 Thieves of level 1-4 (each of whom may be a genuine devout person, or possibly just on the lam). There is also a 50\% chance of a single Magic-User of level 1-4 being with the party. 

Pilgrims usually travel light, carrying a single bag each and walking or riding mules or horses. The pilgrim party will most likely be bringing offerings of some sort to their destination; generate a type A treasure for this purpose. If magic items are indicated, they will most likely not be used by any of the NPCs as they have already been dedicated to the god or pantheon.

\textbf{Magic Items for NPCs}

NPCs will generally have magic items in proportion to their class and level; assume a 5\% chance per level that any given Fighter, Thief, or Cleric NPC will have a magic weapon or magic armor (roll for weapon and armor separately for each NPC). Regardless of level, a roll of 96-00 should be considered a failure. Magic-Users will have a \textbf{Ring of Protection} (roll the bonus as usual for the item) on a roll of 4\% per level, and a magic dagger or walking staff on a roll of 3\% per level.

In addition, assume a 2\% per level chance that any given character will have a potion, and 3\% per level that a Cleric or Magic-User will have a scroll of some sort.

Finally, add up the levels of all members of the party, and use this number as a percentage chance that a \textbf{Rare Item} (as found on the table on page \hyperlink{rare-items}{\pageref{rare-items}}) will be found among them. If the roll is made, divide the number by two and roll again; if the second roll is made, two such items are found. If the party has more than 3 members, you might wish to divide the number in half again and roll for a third such item. Assign the item or items to whichever party members seem most appropriate, or roll randomly if you can' t decide.

\textbf{Non-Human Parties}

It is assumed above that NPC parties will be Human, or predominantly so; but the Game Master may choose to present parties of Elves, Dwarves, or Halflings from time to time. In general, such a party will be homogeneous\ldots. an Elf party would consist of all Elves, for instance. If encountered in the territory of another race, the party might include a guide hired to lead them to their destination. For example, the Elf party mentioned above might hire a Human guide to help them when traveling through a Human country.

The Game Master may simply use the figures given above when generating such parties. One thing that the GM must decide is whether or not the "normal men" rules apply to non-humans... are there "normal elves" for instance? This decision is left to the GM. If there are such characters, they will have the same racial abilities as others of their kind, but will fight with an Attack Bonus of +0 just as normal men do. If there are no such characters in the campaign world, then simply substitute 1\textsuperscript{st} level Fighters for the normal men listed above.

\end{multicols}

\subsection{Dealing with Players}\label{dealing-with-players}\index{Dealing with Players}

\begin{multicols}{2}


\subsection{Character Creation Options}\label{character-creation-options}\index{Character Creation Options}

The standard character creation rules call for rolling 3d6 for each Ability Score in order. Players may complain that they can' t create the sort of characters they want to play. Here are several options you may choose from if you wish to make things easier for your players. Note that the players must not be allowed to demand these options; it' s purely the decision of the Game Master.

\textbf{Point Swapping:} Allow the player to "move" points from one Ability Score to another, at a rate of -2 to one score for each +1 added to the other. The maximum score is still 18 (or the racial maximum if lower), and the player should not be allowed to lower any score below 9. 

Score Swapping: Let the player exchange any two Ability Scores, once per character.

\textbf{The Full Shuffle:} Let the player arrange the six Ability Score values as they wish. This allows the most customization for the player, but on the other hand you mayfind that all player characters in your  campaign begin to look very much alike. It' s not uncommon for players to "dump" the lowest statistic in Charisma, for instance.

\subsection{Hopeless Characters}\label{hopeless-characters}\index{Hopeless Characters}

Sometimes a player will roll for ability scores, look at the ability scores rolled, and declare the character "hopeless." Of course, no player can be required to play a character with less than 9 in the first four scores, since all four classes would be unavailable to that character. However, you as the Game Master might choose to allow the player to reroll a character with scores that are overall below average even if the character isn' t as "hopeless" as this.

\begin{center}

\includegraphics[width=0.47\textwidth]{Pictures132/10000000000007E900000616167463C4E47EEBAD.png}  
\end{center}

Here' s an alternate suggestion: Allow the player to "flip" all of their scores by subtracting them from 21. This will turn a roll of 15 (+1 bonus) into a 6 (-1 penalty) but turn a roll of 3 (-3 penalty) into an 18 (+3 bonus). This will result in a character who previously had mostly penalties becoming one with mostly bonuses. If this is allowed, \textbf{all scores must be flipped}, not just the bad ones! Doing this ensures the character is playable while still allowing the possibility of some penalty scores.

It is of course possible to roll all average scores, such that the character has neither bonuses nor penalties and would not gain any benefit from flipping scores. The choice here should be the player' s; if they wish to reroll all scores, the GM should probably allow it.

\subsection{Acquisition of Spells}\label{acquisition-of-spells}\index{Acquisition of Spells}

Clerics have an obvious advantage over Magic-Users, in that, in theory, they have access to any spell of any level which they can cast. However, note that Clerics are limited in their spell selection based on their deity, faith or ethos; for instance, a Cleric of the goddess of healing should not be surprised that their deity refuses to grant reversed healing spells. If a Cleric prays for a spell that is not allowed, the Game Master may choose to grant the character a different spell, or optionally (if the deity is angered) no spell at all for that "slot."

Magic-Users begin play knowing two spells, \textbf{read magic} plus one other (unless the GM grants more starting spells). Each time the character gains a level, they gain the ability to cast more spells; in addition, every other level the Magic-User gains access to the next higher level spells (until all levels are available). However, gaining the ability to cast these spells does not necessarily mean the Magic-User instantly learns new spells.

Magic-Users may learn spells by being taught by another Magic-User, or by studying another Magic-User' s spellbook. If being taught, a spell can be learned in a single day; researching another Magic-User' s spellbook takes one day per spell level. In either case, the spell learned must be transcribed into the Magic-User' s own spellbook, at a cost of 500 gp per spell level transcribed.

A Magic-User may add a new spell of any level they may cast at any point; however, spells of higher levels may not be learned or added to the Magic-User' s spellbook. The Magic-User must find a teacher or acquire a reference work (such as another Magic-User' s spellbook) in order to learn new spells, and the cost of such is in addition to the costs given above. Often a Magic-User will maintain a relationship with their original master, who will teach the character new spells either for free or in return for services. Sometimes two Magic-Users will agree to exchange known spells. In many cases the only option available to a Magic-User will be to pay another Magic-User (often an NPC) anywhere from 100 gp to 1000 gp per spell level in return for such training.

Magic-Users may also create entirely new spells (or alter existing spells); see the Magic Research rules, below, for details.

\subsection{Weapon and Armor Restrictions}\label{weapon-and-armor-restrictions}\index{Weapon and Armor Restrictions}

Several races and classes have weapon and/or armor restrictions applied to them. What happens when a player declares that their character is going to use a prohibited weapon or wear prohibited armor?

Clerics: The prohibition against edged weapons is a matter of faith for Clerics. Therefore, if a Cleric uses a prohibited weapon, they immediately lose access to their spells as well as the power to Turn the Undead. A higher-level NPC Cleric of the same faith must assign some quest to the miscreant which must be completed in order for the fallen Cleric to atone and regain their powers. If unrepentant, the character is changed permanently from a Cleric to a Fighter. Re-calculate the character' s level, applying the current XP total to the Fighter table to determine this. Hit points and attack bonus remain the same; change the attack bonus only after a new level is gained as a Fighter, and roll Fighter hit dice as normal when levels are gained.

Magic-Users: These characters are simply untrained in any weapon other than those normally allowed to them, and should suffer a -5 attack penalty when using any prohibited weapon. A Magic-User in armor can' t cast spells at all; any attempt fails, and the spell is lost.

Thieves: Wearing armor heavier, more restrictive and/or noisier than leather armor prevents the use of any Thief ability, including the Sneak Attack ability. Thieves may choose to wear such armor, but this only makes them a poor excuse for a Fighter.

\textbf{Dwarves and Halflings:} These characters are prohibited from using large weapons, mainly due to their small stature and relatively low weight. It' s hard to swing a weapon when the weapon is trying to swing you. If such a character tries to use a prohibited weapon, the Game Master may either apply a -5 attack penalty based on the difficulty of using the weapon, or alternately declare the attempt unsuccessful, at their option.

\subsection{Judging Wishes}\label{judging-wishes}\index{Judging Wishe}

Wishes are one of the most potentially unbalancing things in the game.  With a carefully worded wish, a player character can make sweeping, dramatic changes in the game world, possibly even rewriting history. Before allowing the player characters in your game access to even one wish, think about how you will deal with it.

Wishes are granted by a variety of beings. Even when a wish comes from a device (a ring or a sword, for instance), some extradimensional being, god or devil or whatever, has placed that wish in the device. A wish will tend to further the goals of the granting being; if the granter is an evil efreeti, for instance, it will attempt to twist the meaning or intent of the wish so that it does not really accomplish what the player character wants. On the other hand, if the granter is one of the good powers, it will grant the wish as intended so long as the player character isn' t being greedy or spiteful.

\begin{flushleft} \includegraphics[width=0.47\textwidth]{Pictures132/10000000000003CF0000039565CC963BE90356AF.png}  \end{flushleft}

Game balance is the main issue that must be considered. Using a wish to heal the entire party, teleport everyone without error to a distant location, or to avoid or redo a catastrophic battle, is reasonable. A wish that a character be restored to life and health is reasonable, but a wish that not only restores but also improves the character is not.

In general, a wish is granted with at least literal accuracy... the words of the wish must be fulfilled. The exception is wishes that are unreasonable for game balance purposes; they are still at least literally interpreted, but may be only partially granted. In the last example above, for instance, the granting power would likely restore the character to life and health but ignore the "improvements" wished for.


\subsection{Optional Rules}\label{optional-rules}\index{Optional Rules}

\subsubsection{Death and Dying}\label{death-and-dying}\index{Death and Dying}

The rules state that, at zero hit points, the character is dead. If this is too harsh for you, here are several approaches to changing the situation:

\textbf{Raise Dead:} The first approach doesn' t change the rules a bit. Arrange matters so that characters killed in an adventure can be easily \textbf{raised} (but at a substantial cost). This not only "deals" with the mortality issue, it also soaks up excess treasure, preventing the player characters from becoming too rich to be interested in adventuring. It also tends to reward the cautious (since they get to keep their gold more often).

What if the characters don' t have enough money when they die to afford to be \textbf{raised}? Allow the local religious establishment to \textbf{raise} dead adventurers in return for their indenture... that is, the adventurers, upon being restored to life, owe the church or temple the money it would have cost to be \textbf{raised}, \emph{or an equivalent service}. Thus, the local religious leaders would have a ready pool of adventurers to undertake dangerous missions for them.

But the adventurer(s) are dead... how can they agree to the indenture? There are two options: the priests can use \textbf{speak with dead} to attain agreement, or the adventurers can sign an agreement with the church before leaving on the potentially dangerous adventure. The latter might even be considered a standard procedure in some places.

Save vs. Death: The first actual rule alteration is to allow characters reduced to zero hit points to save vs. Death Ray to avoid death. If the save is failed, the character is immediately dead, just as in the normal rules. If the save is made, the character remains alive for 2d10 rounds; if the character' s wounds are bound (or they receive healing magic) within this time frame, death is averted. The character remains unconscious for the full 2d10 rounds rolled, either dying if left untreated or awakening if their wounds are bound.

\begin{center}
	\includegraphics[width=0.47\textwidth]{Pictures132/10000000000003CF000004EEC25D74F91DEF21C4.png}
\end{center}

Binding the wounds of the dying character stabilizes them at zero hit points. The injured character may not move more than a few feet without help, nor fight, nor cast spells, until their hit points are again greater than zero. Non-magical healing will require a full week to restore the first hit point; after this, healing proceeds at the normal rate.

Magical healing will restore the character to whatever total is rolled on the healing die roll (up to the usual maximum of course).  Any spellcaster reduced to zero hit points who then survives loses all remaining prepared spells.

This rule might be combined with the suggestions under Raise Dead, above.

Negative Hit Points: Instead of stopping at zero hit points, keep track of the current negative figure. At the end of each round after they fall, the character loses an additional hit point. If a total of -10 is  reached, the character is dead. Before this point is reached, the character may have their wounds bound and/or receive magical healing, which will stabilize the character. This rule should \emph{not} be combined with the Save vs. Death option.

Just as with the \textbf{Save vs. Death} rule, characters with zero hit points may not move more than a few feet without help, nor fight, nor cast spells, and spellcasters who survive being reduced to zero or negative hit points lose all currently prepared spells.

As a further option, the GM may choose to use a negative number equal to the character' s Constitution score rather than a straight -10.

\subsubsection{"Save or Die" Poison}\label{save-or-die-poison}\index{"Save or Die" Poison}

Poisons, as described in the Encounter and Monster sections, kill characters instantly. Game Masters may find this makes the mortality rate of player characters a bit too high. On the other hand, poisons \textbf{should} be scary. Here' s an optional rule which makes things a bit easier without entirely removing the fear from poison:

Where a "save or die" poison is indicated, the victim must make a save vs. Poison or suffer 1d6 points of damage per round for 6 rounds, starting the round following the exposure to the poison; this is an average of 21 points of damage, but even a first level character might survive with a combination of luck and healing magic. The GM may create poisons which vary from these figures, of course. If the \textbf{Negative Hit Points} optional rule is being used, it is suggested to increase the poison duration to 10 rounds (an average 35 points).

\subsubsection{Ability Rolls}\label{ability-rolls}\index{Ability Rolls}

There will be times when a player character tries to do something in the game that seems to have no rule covering it. In some of those cases, the only attribute the PC has that seems appropriate may be an Ability Score. Here is a suggested method for making rolls against Ability Scores that still gives better odds to higher level characters:

The player rolls 1d20 and adds their Ability Bonus for the score the GM thinks is most appropriate, as well as any bonus or penalty the GM assigns. Consult the following table. If the total rolled is equal to or higher than the given Target number, the roll is a success.


\begin{flushleft}
	\begin{tabular*}{0.93\linewidth}{@{\extracolsep{\fill}}ll}
\textbf{Level} & \textbf{Target} \\\toprule
NM or 1 & 17 \\\hline
2-3 & 16 \\\hline
4-5 & 15 \\\hline
6-7 & 14 \\\hline
8-9 & 13 \\\hline
10-11 & 12 \\\hline
12-13 & 11 \\\hline
14-15 & 10 \\\hline
16-17 & 9 \\\hline
18-19 & 8 \\\hline
20 & 7 \\\bottomrule
\end{tabular*}
\end{flushleft}

\subsubsection{Human Experience Bonus Option}\label{human-experience-bonus-option}\index{Human Experience Bonus Option}

Some Game Masters find that the standard 10\% bonus to earned experience for Human player characters is not enough to encourage people to play as Humans. If you feel that this is the case in your campaign, there is absolutely no reason not to give double that bonus. Because of the exponential (doubling) nature of the character advancement tables from 1\textsuperscript{st} to 9\textsuperscript{th} levels, giving a larger bonus will not significantly unbalance your game in any way.


\subsubsection{Awarding Experience Points for Treasure Gained}\label{awarding-experience-points-for-treasure-gained}\index{Awarding Experience Points for Treasure Gained}

The Game Master may also assign experience points for treasure gained, at a rate of 1 GP = 1 XP. This is optional; GMs wishing to advance their players to higher levels more quickly may choose to do this, while those preferring a more leisurely pace should omit it. If experience is awarded for treasure, it should be awarded only for treasure acquired and returned to a place of safety. Alternately, the GM may require treasure to be spent on training in order to count it for experience. This is a highly effective way to remove excess treasure from the campaign.

\subsubsection{Thief Abilities}\label{thief-abilities-1}\index{Thief Abilities}

Some players of Thieves may wish to have more control over their Thief abilities. If you study the Thief Abilities table, you' ll discover its secret: from levels 2-9, the Thief improves 30 percentiles (total) each level; from levels 10-15, 20 percentiles; and from level 16 on, 10 percentiles. If you wish to allow Thief customization, simply let the player allocate these points as they wish rather than following the table. Allow no more than 10 percentiles to be added to any single Thief ability per level gain. Note also that no Thief ability may be raised above 99 percent.

 \begin{center}
 	\includegraphics[width=0.47\textwidth]{Pictures132/1000000000000777000005BA0B8F04EE41FCA874.png} 
 \end{center}

\subsubsection{Preparing Spells From Memory}\label{preparing-spells-from-memory}\index{Preparing Spells From Memory}

Sometimes a Magic-User will want to prepare spells, but their spellbook may be unavailable; this includes when the book has been destroyed or stolen as well as times when the Magic-User has been captured or trapped.

A Magic-User can always prepare \textbf{read magic} from memory. Other spells require an Intelligence ability roll, as described above, with the spell level as a penalty on the die roll.

Failure exhausts the spell slot being prepared, just as if it had been successfully prepared and then cast; so if a 5\textsuperscript{th} level Magic-User attempts to prepare \textbf{fireball} from memory, and fails, they will have no 3\textsuperscript{rd} level spells for the day.

\begin{center}
	\includegraphics[width=0.47\textwidth]{Pictures132/10000000000003D80000026E45042F3D7C67FB6C.png}
\end{center}

\subsection{Magical Research}\label{magical-research}\index{Magical Research}

\subsubsection{General Rules for Research}\label{general-rules-for-research}\index{General Rules for Research}

At some point a Magic-User or Cleric may wish to start creating magic items or inventing spells. This is termed \textbf{magical research}. For any research, a Magic-User must have a tower or laboratory, while a Cleric requires a properly consecrated temple or church of their faith. In addition, there will be a cost for the creation of each item, a minimum time required to create it, and a given chance of success. If the roll fails, generally the time and money are wasted and the procedure must be started again from the beginning; however, consult the detailed rules below for exceptions.

\emph{In almost all cases, the Game Master should make this roll in secret.} There are many situations where the character (or the player) should not know whether the roll has actually failed, or whether the GM has decided the research is impossible for the character. The GM may decide to tell the player that the research is impossible if the roll succeeds; if the roll is a failure, that is all the player should be told.

In general, Clerics may only create magic items reproducing the effects of Clerical spells; Clerics may also make enchanted weapons and armor, even those sorts which they may not use themselves (since they may be creating weapons or armor for other followers of their faith). Magic-Users may create any sort of magic item except for those reproducing Clerical spells for which no equivalent Magic-User spell exists.

Time spent doing magical research must be eight-hour days with interruptions lasting no more than two days. Longer interruptions result in failure of the project.

The GM may wish to grant Experience Points to characters who successfully complete magical research. It is suggested that the rate of such awards be 1 XP per 10 gp spent on the research. This award may be granted for all research, or only for creation of magic items, or not at all if the GM prefers to emphasize adventuring for advancement purposes.

\subsubsection{Spell Research}\label{spell-research}\index{Spell Research}

Researching new spells is the most common type of magical research. A Magic-User may research a standard spell, removing the need for a teacher or reference; alternately, a Cleric or Magic-User may invent an entirely new spell. No character may invent or research a spell of a higher level than they can cast.

If the character is inventing a new spell, the GM must determine the spell' s level and judge whether or not the spell is possible "as is." The GM does not have to tell the player whether the spell is possible, and in fact this may be preferable.

The cost to research a spell is 1,000 gp per spell level for "standard" spells, or 2,000 gp per spell level for newly invented spells; in either case, one week is required per spell level to complete the research. The chance of success is 25\%, plus 5\% per level of the character, minus 10\% per level of the spell; the maximum chance of success is 95\%.

If the research roll is successful, the character may add the spell to their spellbook (if a Magic-User) or may subsequently pray for the spell (if a Cleric). On a failure, the money and time are spent to no avail. Clerics of the same deity, faith or ethos may teach each other the prayers required to access new spells; this takes one hour per spell level. The procedure to exchange spells with other Magic-Users has already been explained (under Acquisition of Spells, above).

As mentioned above, the GM may decide that a proposed new spell is not "correct" for their campaign; too powerful, too low in level, etc. Rather than tell the player this, there are two strategies that may be used.

First, the Game Master may decide to revise the spell. If the roll is a success, the GM then presents the player with a revised writeup of the spell, adjusted however the GM feels necessary for game balance purposes.

The alternative, more appropriate when the GM believes the spell should be higher level than the player character can cast, is to make the roll anyway. If the roll fails, that is all the player needs to know; but if it succeeds, the GM should then show the player the revised version of the spell and explain that the character may try again when they attain a high enough level to cast it. In this case, the GM may allow the character to reduce either the time or the cost by half when the research is attempted again at the new level.

\subsubsection{Magic Item Research}\label{magic-item-research}\index{Magic Item Research}

Any character who wishes to create magical items must know all (if any) spells to be imbued in the item. Items that produce effects not matching any known spell may require additional research (to devise the unknown spell) if the GM so desires.

Some magic items require one or more special components that cannot usually be bought. Special components can only be used once on such a project. For example, the GM might require the skin of a deceiver to create a \textbf{Cloak of Deception}, or mountain dragon saliva to create a \textbf{Wand of Fireballs}. Note that there are specific rules for components under \textbf{Other Magic Items}, below.

Special component requirements are entirely at the option of the Game Master, and are usually employed to slow the creation of powerful magic items that might tend to unbalance the campaign. It' s also a way to lead the spellcaster (and their party) to new adventures.

\subsubsection{Chance of Success}\label{chance-of-success}\index{Chance of Success}

Unless given differently below, the base chance of success creating a magic item is 15\% plus 5\% per level of the spellcaster, plus the spellcaster' s full Intelligence (if a Magic-User) or Wisdom (if a Cleric). Thus, a 9\textsuperscript{th} level spellcaster with a 15 Prime Requisite has a base chance of 75\%.

\subsection{Spell Scrolls}\label{spell-scrolls}\index{Spell Scrolls}

A spellcaster may create a scroll containing any spell they have access to (for a Magic-User, spells in their spellbook; for a Cleric, any spell the character might successfully pray for). The cost is 50 gp per spell level, and the time required is 1 day per spell level. Reduce the chance of success based on the level of the spell being inscribed, at a rate of -10\% per level.

\begin{center}
	\includegraphics[width=0.47\textwidth]{Pictures132/10000000000003CF000002E2057B4027AF25ECD6.png}
\end{center}

If the roll fails, the enchantment of the scroll has failed; however, if the caster tries again to inscribe the same spell, either the cost or the time is reduced by half (at the player' s option, if a player character is involved).

\subsection{Other Single-Use Items}\label{other-single-use-items}\index{Other Single-Use Items}

Scrolls (other than spell scrolls), potions, and a few other items (such as the \textbf{Rod of Cancellation}) are single-use items. These items may be created by Magic-Users or Clerics of the 7\textsuperscript{th} level or higher.

The chance of success is as given for scrolls, above, when the item being created reproduces a known spell (or when the GM decides a spell must be created, as described above). For other types of items, the GM should assign a spell level as they see fit, and the cost and time required is doubled (making up for the spell research or knowledge required for spell-reproducing items). The time required is one week plus one day per spell level (or equivalent), and the cost to enchant the item is 50 gp per spell level, per day.

\begin{center}
	\includegraphics[width=0.47\textwidth]{Pictures132/10000000000003D80000041A2372CB23C620F45D.png}
\end{center}

Potions are a special case; the character creating a potion may make a large batch, consisting of several doses, which may be bottled in separate vials or combined in a larger flask. For each additional dose created at the same time, reduce the chance of success by 5\% and increase the time required by one day. Note that increasing the time required will directly increase the cost, but the cost is for the entire batch, not per dose, effectively giving a discount to the cost per dose. If the roll to create the item fails, the entire batch is spoiled.

The assistance of an alchemist (found in the \textbf{Specialists} section on page \hyperlink{Alchemistux20Entry}{\pageref{Alchemistux20Entry}}) adds 15\% to the chance of success of a Magic-User or Cleric creating a potion.

\subsection{Permanent Magic Items}\label{permanent-magic-items}\index{Permanent Magic Items}

Creating permanent magic items (rings, weapons, wands, staves, and most miscellaneous magic items) requires a Magic-User or Cleric of the 9\textsuperscript{th} level or higher.

When enchanting an item with multiple abilities, each ability of the item requires a separate roll for success; the first failed roll ends the enchantment process. Such an item will still perform the powers or effects already successfully enchanted into it, but no further enchantment is possible.

Permanent magic items, including weapons (described in detail below), must be created from high-quality items. The cost of such items will generally be ten times the normal cost for such an item.

\subsection{Enchanting Weapons and Armor}\label{enchanting-weapons-and-armor}\index{Enchanting Weapons and Armor}

The base cost of enchanting a weapon or armor is 1,000 gp per point of bonus. For weapons with two bonuses, divide the larger bonus in half (don' t round) and add the smaller bonus; thus, a \textbf{Sword +1, +3 vs. Dragons} would cost 2,500 gp to enchant. Enchanting a weapon takes one week plus two days per point of bonus; thus, the sword described would require twelve days to enchant. 

Reduce the chance of success by 10\% times the bonus; so, a \textbf{Sword +1} would reduce the base chance 10\%, while the \textbf{Sword +1, +3 vs. Dragons} described above would reduce the base chance 25\%. Further, the chance of success may be increased 25\% by doubling the cost and time required (this decision must be announced before the roll is made).

For weapons having additional powers, combine the rules above with the rules for creating permanent items. All enchantments must be applied in a single enchantment "session."

At the GM' s option, hiring a master armorer or weaponsmith (one good enough to charge twice normal rates as given in the \textbf{Specialists }subsection on page \hyperlink{Weaponsmithux20Entry}{\pageref{Weaponsmithux20Entry}}) might add 15\% to the chance of success.

\subsection{Other Magic Items}\label{other-magic-items}\index{Other Magic Items}

Magic items can have several \textbf{features}. Each feature added to a magic item increases the cost and the time required, and decreases the chance of success. The features are as follows:

\textbf{Creates a spell or spell-like effect}: This is the basic feature of all non-weapon magic items. The base cost of this enchantment is 500 gp per spell level; time required is five days plus two days per level. If the magic item has multiple spell or spell-like effects, add the cost and time figures together. The chance of success is reduced 5\% per spell level.

\textbf{Has multiple charges}: This includes, of course, wands and staffs, but several other magic items would also have charges. Each spell or spell-like effect normally has a separate pool of charges (but see next). The table below shows the various maximum charge levels and the associated cost, time and chance adjustments:\medskip

\begin{tabular*}{0.93\linewidth}{@{\extracolsep{\fill}}llll}
\textbf{Charge} & \textbf{Cost per} & \textbf{Charges per} & \textbf{Chance} \\
\textbf{Level} & \textbf{Charge} & \textbf{Day} &  \\\toprule
2-3 & +150 gp & 1 & - 5\% \\\hline
4-7 & +125 gp & 2 & - 10\% \\\hline
8-20 & +100 gp & 3 & - 20\% \\\hline
21-30 & +75 gp & 4 & - 30\% \\\bottomrule
\end{tabular*}\medskip

When using the table above, don' t count the first charge for cost or time purposes. Note that each separate pool of charges in the item must be figured separately.

\textbf{Item can be recharged:} Figure the additional cost and time, and the penalty to the chance of success, for rechargeable items as being exactly twice the figures from the table above; so, creating a rechargeable item with 3 charges costs 600 gp more rather than 300 gp more, and takes two days per charge (or four extra days); the chance of success is lowered 10\% rather than 5\%.

\begin{center}
	\includegraphics[width=0.37\textwidth]{Pictures132/10000000000003D800000357E1299D49C6B03181.png}
\end{center}

\textbf{Item recharges itself:} Creating a self-recharging item is expensive; apply the following adjustments to the charge cost, time and chance for items that recharge automatically. Note that self-recharging items are never "rechargeable" in that they may not be recharged other than by themselves.\medskip

\begin{tabular*}{0.93\linewidth}{@{\extracolsep{\fill}}llll}

\textbf{Charging Rate} & \textbf{Cost} & \textbf{Time} & \textbf{Chance} \\\toprule
1 per day & x 3 & x 2 & - 10\% \\\hline
All per day & x 5 & x 3 & - 30\% \\\hline
All per week & x 4 & x 2 & - 20\% \\\bottomrule
\end{tabular*}\medskip

\textbf{Charges are generic:} This means that all the effects of the item draw power from the same pool of charges; most Magic-User staffs are in this category. Items with generic charges are always rechargeable; don' t apply the normal costs for this feature. Instead, combine the normal costs for the charge pools of each effect (which must all have the same number of charges), and then divide the charge cost, time and chance adjustments by two. Thus, two effects sharing one pool costs the same as a single effect with a single pool. 

\textbf{Item may be used by any class:} By default, magic items may only be used by the class that created them; so a \textbf{Wand of Fireballs} is normally usable only by Magic-Users, or a \textbf{Staff of Healing} only by Clerics. This feature allows the item to be used by any class of character, and involves assigning simple command words and gestures to the item. Adding this feature costs 1,000 gp per effect. Note that all the item' s effects do not have to be covered; it is possible to create an item where some effects may be used by any class, but other effects may only be used by the creator' s class.


\textbf{Item operates continuously or automatically:} This feature supersedes both the charges and item use features. The item works whenever properly worn, or activates automatically when required. A \textbf{Ring of Fire Resistance} is a good example; also, the \textbf{Ring of Invisibility} is in this category. Adding this feature multiplies the final cost and time figures by 3 and applies a 20\% penalty to the chance of success. 

Each feature above applied to a magic item will require a valuable, rare and/or magical material to support the enchantment. For example, a wand of fireballs has a spell effect that is powered by charges; these are two relatively ordinary features, so the Magic-User creating the item proposes a rare wood for the shaft and a 1,000~gp value ruby for the tip. The GM may, of course, require something more rare or valuable if the magic item is particularly powerful.

The base cost of a spell effect feature can be reduced by 25\% by applying limits to the ability. For example, a \textbf{Ring of Charm Dryad} is an example of limited \textbf{charm person} spell effect, which would qualify for the deduction. This does not affect the chance of success or the time required.

Weapons which are to be enchanted with additional powers other than the normal bonus require combining the standard weapon enchantment rules with the rules given above. Perform the weapon enchantment first; if it is successful, then the character enchanting the weapon must immediately (within two days, as previously explained) begin the spell or spell-like power enchantment process. Failure of the second procedure does not spoil the weapon enchantment.


\begin{center}
	\includegraphics[width=0.42\textwidth]{Pictures132/10000001000003D8000004F994A1153C48E14BFF.png}
\end{center}

\subsection{Cursed Items}\label{cursed-items}\index{Cursed Items}

Some cursed items, such as cursed scrolls, are created that way specifically by the spellcaster. The difficulty of creating such an item is roughly the same as the difficulty of creating a spell scroll of \textbf{bestow curse}.

Other cursed magic items may be the result of a failed attempt to create a useful item. The GM must decide whether or not a failed research project will actually create a cursed item.


\subsection{Creating a Dungeon Adventure}\label{creating-a-dungeon-adventure}\index{Creating a Dungeon Adventure}

\subsubsection{1. Think About Why}\label{think-about-why}

When creating a dungeon, the first question you must answer is: Why will your player characters risk going into this dangerous dungeon full of monsters and traps?

Here are some possible scenarios:

\textbf{To Explore the Unknown:} This is common in pulp fiction. One or more of the player characters has heard of some ancient site, and wishes to explore purely for knowledge. Possibly some of the other player characters are involved for other reasons.

\textbf{To Battle An Evil Incursion:} Goblins are raiding farms in the area, and the Baron has offered a reward for stopping the raids; the player characters are happy to help.

\textbf{To Rescue A Kidnapped Victim:} Some friend of the player characters has been kidnapped, and they must sneak into or storm the villain' s tower/cave/dungeon to rescue the victim. Or, perhaps, the victim is the son or daughter of the local Baron or a wealthy merchant who offers a reward for the safe return of their offspring.

\textbf{To Fulfill A Quest:} The local church, to whom the player characters owe a favor, would like an ancient relic recovered from a lost mountain fortress, and the High Priest asks them to look into it; or some similar task might be assigned, depending on who the player characters owe a favor.

\textbf{To Get Loot:} This is a surprisingly common scenario (well, perhaps not so surprising). The dungeon is rumored to contain a hidden treasure of great value, and the first characters to find it will be rich! Of course, the treasure might not be that huge, and might be guarded by any number of horrific monsters...

\textbf{To Escape Confinement:} The player characters have been captured by an enemy, and find themselves incarcerated without their weapons, armor, or equipment. This scenario must be used with care, as the GM must not be seen to be "railroading" the characters into the adventure. 

There are many other possible scenarios, and each has many variations. Knowing the answer to this question will make the next questions easier to answer.

\subsubsection{2. What Kind Of Setting Is It?}\label{what-kind-of-setting-is-it}

Is the dungeon beneath a ruined fortress, or an ancient wizard' s tower? Or perhaps it' s a natural cave, which has been expanded by kobolds... or the tomb of an ancient barbarian warlord, guarded by undead monsters... there are many possibilities.

\subsubsection{3. Choose Special Monsters}\label{choose-special-monsters}

Now you know why the player characters want to go there (or why they will, when they learn of the dungeon), and you know what sort of place it is. Next, decide what special monsters you will place within. For instance, the natural cave expanded by kobolds contains kobolds, obviously, while the warlord' s tomb contains some undead, zombies and skeletons perhaps.

\begin{center}
	\includegraphics[width=0.47\textwidth]{Pictures132/10000000000003CF0000033CB74ABAE754D152B5.png}
\end{center}

\subsubsection{4. Draw The Dungeon Map}\label{draw-the-dungeon-map}

Dungeon maps can be drawn on graph paper in pencil, or created on the computer with any of a broad variety of dungeon-drawing programs. (If you like the design of the maps in the official Basic Fantasy RPG adventure modules, be sure to visit www.basicfantasy.org and try out our map designer, MapMatic +2.) When creating a dungeon for personal use, there is certainly no good reason not to use pencil and paper. Below is an example of a hand-drawn dungeon map, with the various symbols noted:


\subsubsection{5. Stock The Dungeon}\label{stock-the-dungeon}

"Stocking" the dungeon refers to assigning contents to each room. There are several possibilities; a room might contain a monster (which might or might not have treasure), a trap (which might guard a treasure, or might not), an "unguarded" treasure, a "special" (something other than a monster, trap, or treasure; often a puzzle of some sort), or be "empty."

The GM may choose the contents of any room, or may roll on the table below:\medskip


\begin{tabular*}{0.93\linewidth}{@{\extracolsep{\fill}}ll}
\textbf{d\%} & \textbf{Contents} \\\toprule
01-16 & Empty \\\hline
17-20 & Unguarded Treasure \\\hline
21-60 & Monster \\\hline
61-84 & Monster with Treasure \\\hline
85-88 & Special \\\hline
89-96 & Trap \\\hline
97-00 & Trap with Treasure \\\bottomrule
\end{tabular*}\medskip

An \textbf{unguarded treasure} will generally be hidden (such as in a secret room, inside an unusual container, etc.) or protected by a trap (a poison needle in the lock of a chest, or a poison gas canister that explodes if the container is opened, or something similar); such a treasure might even be hidden \textbf{and }trapped! Again, some sort of saving throw should be allowed if a trap is used. It' s not a bad idea to hide a treasure so well that the player characters are unlikely to find it; don' t be concerned if they don' t. If you give away the location of all your unguarded treasures, your players will not appreciate it properly when they manage to find one by cleverness or luck. 

A \textbf{monster} might be selected by the GM or rolled on the random encounter tables. It' s traditional that the first level (below ground) contains monsters of 1 hit die or less, the second level contains monsters of around 2 hit dice, and so on, but the GM may choose to arrange their dungeon in any way desired.

A \textbf{monster with treasure} might indicate a lair, or it might be a group of monsters carrying loot, possibly camping for some reason before moving on. 

\begin{center}
	\includegraphics[width=0.47\textwidth]{Pictures132/10000000000003D800000571573B6243AB7B0D4D.png}  
\end{center}

A \textbf{trap} is, obviously, some sort of device intended to harm the player characters, including such things as pendulum blades, hidden pits, spear-chucking devices, and so forth. A \textbf{trap with treasure} is such a trap protecting a treasure, which might be in the room beyond the trap or actually within it (such as in a pit). See the Traps section, below, for more information.

A \textbf{special} might be a puzzle of some sort, such as a door that can only be opened by a combination (hidden elsewhere in the dungeon); or perhaps an oracle that answers questions about the dungeon (but possibly it lies). The classic "magic fountain" that randomly changes the ability scores of the drinker is another possibility; if this is done, some sort of limit should be imposed (such as, the device only affects a given creature once, or the device causes harm more often than it gives aid) to prevent abuse. In general, a "special" room is any room containing something that either interests or obstructs the player characters but is not a monster, trap, or unguarded treasure.

Empty rooms contain no monsters, traps, unguarded treasures, or specials. This does not mean that they are truly "empty;" a room might contain a fireplace, upholstered chairs, side tables, torch sconces, and curtains, and still be considered empty. Hide a treasure in a secret drawer in a side table, and it becomes an unguarded treasure room; in other words,  to be empty there has to be basically nothing of serious interest to the player characters in the room.


\subsubsection{6. Finishing Touches}\label{finishing-touches}

The GM may wish to create one or more custom wandering monster tables for the dungeon; monster patrols, if any, may need to be described; and possibly some locations may have unusual sounds, smells, graffiti, etc. which need to be noted. Don' t spend too much time on this, though.

Remember, if you only detail the "interesting" things, your players will begin to guess what might be in a room. Some extra description will help make things uncertain for the players. For instance, a room with an unguarded treasure:

Game Master:\emph{ This room contains a chest, centered against the far wall.}

Player 1:\emph{ We look for monsters, and if we don' t see any, the thief will check the chest for traps.}

Kind of boring, right? This might be better:

Game Master: \emph{In this room you see a comfortable-looking upholstered chair, a side table with a drawer and a foot stool. Two burned-out torches are held by sconces on each wall.}

Player 1: \emph{If we don' t see any monsters, the thief will check the table and the footstool for traps and see if anything is hidden inside them, while the rest of us check for secret doors... one of those sconces might open one.}

A little extra detail can add a lot to the adventure.

\subsection{Traps}\label{traps-1}

Some suggestions of typical traps are listed below, to assist the GM. Deadlier traps can be created by combining simple traps, by making their effects harder to avoid, or by making them capable of dealing more damage.

Traps are not necessarily reliable; the GM may choose to make a roll of some sort for each potential victim until the trap is sprung (say, 1-2 on 1d6). Or, a trap door might not open until a given weight is placed on it, so that a lightly loaded thief might cross without difficulty, only to see his heavily armored warrior ally fall victim to it.

\textbf{Alarm:} Everyone within a 30' radius must save vs. Spells or be deafened for 1d8 turns by the loud noise. The GM should check immediately for wandering monsters, which, if indicated, will arrive in 2d10 rounds.

\textbf{Arrow Trap:} A hidden, mounted crossbow attacks at AB +1, doing 1d6+1 points of damage on a successful hit.

\textbf{Chute:} These are usually covered with a hidden trap door. The triggering character must save vs. Death Ray (with Dexterity bonus added) or tumble down to lower level of the dungeon. Chutes usually do little or no damage to the victim.

\textbf{Falling stones or bricks:} Rocks fall from the ceiling. The triggering character must save vs. Paralysis or Petrify (with Dexterity bonus added) or take 1d10 points of damage. 

\textbf{Flashing Light:} With a loud snap, a bright light goes off in the face of the character that triggered the trap. That character, and anyone else looking directly at it, must save vs. Spells or be blinded for 1d8 turns.

\textbf{Monster-Attracting Spray:} A strong-smelling but harmless liquid is sprayed on the triggering character. The smell attracts predatory creatures, doubling the chances of wandering monsters for 1d6 hours or until washed off.

\textbf{Oil Slick:} Oil is sprayed onto the floor of the room. Anyone trying to walk through the oil must save vs. Death Ray (with Dexterity bonus added) or fall prone. Oil is highly flammable and may be ignited by torches or other flame sources held by characters who slip and fall into it.

\textbf{Pit Trap:} Usually hidden with a breakable cover, trap door, or illusion. The victim must save vs. Death Ray (with Dexterity bonus added) or fall into the pit, taking damage according to the distance fallen (see "Falling Damage"). A pit trap can be made deadlier by placing spikes, acid, or dangerous creatures at the bottom, or partly filling it with water to represent a drowning hazard.

\textbf{Poison Dart Trap:} A spring-loaded dart launcher attacks at AB +1 for 1d4 points of damage, and the victim must save vs. Poison or die.

\textbf{Poison Gas:} Gas emerges from vents to fill the room. All within the affected area must save vs. Poison or die. Poison gases are sometimes highly flammable and may be ignited by torches or other flame sources, doing perhaps 1d6 points of damage to each character in the area of effect (with a save vs. Dragon Breath allowed to avoid the damage).

\textbf{Poison Needle Trap:} A tiny, spring-loaded needle pops out of a keyhole or other small aperture and injects poison into the finger of the character who triggered the trap (most likely, a Thief trying to pick the lock), who must save vs. Poison or die.

\textbf{Portcullis:} A falling gate blocks the passage. The character who triggered the trap must save vs. Death Ray or take 3d6 points of damage.

\textbf{Rolling Boulder Trap:} A spherical or cylindrical rock rolls down a slanting corridor. Anyone in its path must save vs. Death Ray (with Dexterity bonus added) or take 2d6 points of damage. Alternately, if the corridor has no other place for the character to escape to (that is, no room for the character to step out of the path of the rock), it may be necessary to outrun the rock to avoid the damage.

\begin{center}
	\includegraphics[width=0.47\textwidth]{Pictures132/10000000000003D8000002492A29E8B22D7F0984.png}
\end{center}

\textbf{Blade Trap:} A blade or spear drops down from the ceiling or pops out of the wall and attacks at AB +1 for 1d8 points of damage. Particularly large blades might attack everyone along a 10' or 20' line.

\textbf{Triggered Spell:} When activated, a spell of the GM' s choice is cast, targeting or centered on the character who triggered it. Popular choices include curses, illusions, or a \textbf{wall of fire}.

\subsection{Designing a Wilderness Adventure}\label{designing-a-wilderness-adventure}\index{Designing a Wilderness Adventure}

\subsubsection{1. Think About Why}\label{think-about-why-1}

This is much the same task as was described above. The player characters may enter a particular area looking for a town to resupply from, a church or temple to provide healing services, or for many other reasons. Once in the area, the Game Master can make the player characters aware of adventuring opportunities in the area, by means of rumors, posted bounties (such as for raiding humanoids), quests offered by local clergy, and so forth.

\subsubsection{2. What Kind Of Setting Is It?}\label{what-kind-of-setting-is-it-1}

Decide whether the area is deep in the wilderness, or in more inhabited territories, what sort of climate will be found there, how many towns, and of what size, are present, and so on.

You may choose to design a new territory based on the goals of the player characters in your campaign. For example, if the player characters decide to seek their fortunes in the richest city in the world, you could decide where this is and begin to describe it by providing rumors of its wealth and splendor told by far-wandering merchants. If these descriptions intrigue the characters and they travel toward the city, you will have time to decide what terrain -- and dangers -- lie in their path.

On the other hand, your setting should make sense, which will help players make meaningful choices when traveling. For example, areas under human control will be settled, with signs of civilization such as cleared land for agriculture, roads, strongholds, etc. Areas dominated by humanoid monsters, or which are being raided by wandering humanoids, will be battle-scarred and will not have food or other goods available. A valley that was settled many years ago but abandoned after a dragon attacked could contain ruined buildings, their walls likely still bearing the marks of flame and claw, and fields grown high with saplings.

\subsubsection{3. Draw An Area Map}\label{draw-an-area-map}

Now it' s time to draw the area map. Some Game Masters prefer to draw maps freehand, while others like to use hex or graph paper; of course, programs are available to create maps on a computer as well. It is a good idea to provide a \textbf{scale} for the map, which can be whatever best fits the map and the area you want to depict. A scale of 18 miles per square or hex is a good choice for a large-scale map, as this is the distance that a group of humans can cover in a day in clear terrain (see \textbf{Wilderness Movement Rates}), which makes it easy to determine travel times.

Rivers and coastline, hills and mountains, forests and plains must be clear on the map. All of these areas should have an appropriate climate: for example, the windward side of a mountain range will usually receive a great deal of rain, while the other side will be dry. You may choose to create an area with abnormal weather for its location, such as a sandy desert in the midst of a rain forest, but this should be unusual, a tip to observant players that strange magic is involved.

Go ahead and place any interesting sites such as towns, ruins, and significant monster lairs. Remember, in most cases your party of adventurers will need some base of operations, be it a city, town, village, or border fortress.

\subsubsection{4. Detail Interesting Sites And People}\label{detail-interesting-sites-and-people}

Describe at least the base town, and the dungeon you expect the party to visit first. Also detail any set or placed encounters you laid out in the step above. There is a lot of room for creativity here: a distant, unfamiliar town may have different laws, traditions, or currency. You should also describe key NPCs and their connections to each other. NPCs have their own goals and plans, which may or may not involve the PCs, and the actions of player characters toward one person will often influence how others treat them. Don' t go overboard trying to detail every single place on the map... leave some room for expansion later, after you have a feel for your players and their characters.

\subsubsection{5. Create Encounter Tables}\label{create-encounter-tables}

When designing a wilderness area, one touch that will really set it apart is a custom encounter table. Choose those monsters that seem most appropriate to the area, using the standard encounter tables as a guide. If you have placed humanoid lairs or encampments, you may wish to include their patrols on the custom table.

Another alternative is to roll six or eight or ten random encounters using the "generic" encounter table for the relevant terrain type, and use that list as your random encounter table for the area. When doing this, you probably should re-roll duplicates.

\subsection{Strongholds}\label{strongholds}\index{Strongholds}

Many player characters, upon reaching higher levels, choose to settle down and build a \textbf{stronghold}. Generally this is allowed when a character reaches 9th level or higher. The player character must obtain land on which to build. In some lands, frontier territory may be made available to any freeman (or freewoman) who can tame it; in others, land may be available for someone with enough gold; while in other cases the character will need to petition the local Count, Duke or King for a land grant.

Usually, Fighters build \textbf{castles}, Magic-Users build \textbf{towers}, Clerics build \textbf{temples} and Thieves build \textbf{guildhouses}, but this is not always so. Any character who builds a stronghold suitable to their class will attract 1st level followers of the same class as follows:\medskip

\begin{tabular*}{0.93\linewidth}{@{\extracolsep{\fill}}ll}
\textbf{Class} & \textbf{Number of Followers} \\\toprule
Fighter & 3d6 \\\hline
Magic-User & 1d8 \\\hline
Cleric & 2d8 \\\hline
Thief & 2d6 \\\bottomrule
\end{tabular*}\medskip

These followers will assist the character, but will not go on adventures away from the stronghold in most cases (especially dangerous dungeon adventures). Their living comes from the income generated by the stronghold. The primary sources of this income are taxes on peasants for castles, fees for magical services and students' tuition for towers, tithing from the faithful for temples, and criminal activities for guildhouses. A stronghold must have 200 square feet of living space for each follower, as well as quarters for guests, stables for horses, and so on.

A player who wants to build a stronghold should draw its floor plan. Each story is usually 10 feet tall. The construction costs for the stronghold are determined by the square footage of its walls, floors and roofs, the materials used, and the thickness of the walls.

Make sure not to double-count corners on walls that are 5' thick or thicker; count the length of only one face. When determining wall length for round walls and towers, approximate pi by 3, since the inner face of the wall has a shorter circumference. The table below gives costs in gp for each 10 foot square section of wall. The number by the material is its \textbf{hardness} which is deducted from damage to the wall, just as with \textbf{Attacking a Vehicle} as explained on page \hyperlink{attacking-a-vehicle}{\pageref{attacking-a-vehicle}}.

Also note that a building over 40' high must have a solid foundation, and if over 60' high, it must rest on bedrock.\medskip

\begin{tabularx}{0.45\textwidth}{@{}lXXXX@{}}
\textbf{Wall material} & \textbf{1' thick} & \textbf{5' thick} &
\textbf{10' thick} & \textbf{15' thick} \\\toprule
Maximum height & 40' & 60' & 80' & 100' \\\hline
Wood (H 6) & 10 gp & n/a & n/a & n/a \\\hline
Brick (H 8) & 20 gp & 50 gp & n/a & n/a \\\hline
Soft stone (H 12) & 30 gp & 70 gp & 200 gp & n/a \\\hline
Hard stone (H 16) & 40 gp & 90 gp & 260 gp & 350 gp \\\bottomrule
\end{tabularx}\medskip

A 1' thick wall is made of solid pieces of material held with mortar (or pegs and ropes for wooden walls); such walls may be at most 40' tall. A 5' thick wall consists of two 1' thick walls sandwiching 3' of earth and rubble; such a wall may be at most 60' tall. A 10' thick wall consists of a 4' thick outer wall and a 2' thick inner wall sandwiching 4' of earth and rubble, and may be built up to 80' tall. A 15' thick wall consists of a 6' thick outer wall and a 2' thick inner wall sandwiching 7' of earth and rubble; these walls may be built up to 100' tall. To attain the maximum height, thinner walls can be used on upper stories. For example, an 80 ft. tower must have at least 20' of 10' thick walls at the base, but more could be used.

The character will have to pay engineering costs for designing the stronghold, and tall structures are more difficult to design and to build. For each portion of the  stronghold (wall, tower, and so on), each 10' of height adds 10\% to the costs in both time and money. The GM should feel free to add a multiplier to reflect the difficulties of building in a remote area, obtaining materials, etc. In particular, if materials need to be transported, they require 1 ton of cargo space per 5 gp of wood or stone construction. (The increased weight of stone compensates for its compactness compared to wood.)

A stronghold requires one worker-day of construction labor for every gp it costs to build. Adding more workers reduces construction time, but the time cannot be reduced below the square root of the time for one worker to build the stronghold. Assume that there are 140 working days per year (seven months of 20 working days each) in temperate climates.

Floors and thatched roofs cost as much and take as long to build as it would take to build the square footage of their bases of 1' thick wood walls. Wood-shingled roofs cost twice this amount and take twice as long to build, while slate-shingled roofs cost four times as much and take four times as long. (You don' t need to calculate the greater surface area of a pitched roof, since the increased height increases construction costs enough to cover this.)

These costs include normal features such as stairs, doors and windows. Interior walls are not included; they are usually 1' thick. \textbf{Parapets}, which provide cover for defenders atop castle walls and towers, are usually 1' thick and 5' high (so they are half-cost).

Note that guildhouses are almost always built in cities and thus are usually built with 1' thick exterior walls, but they cost twice as much to build due to the traps and secret passageways that are designed into them. A Magic-User' s tower costs three times as much to build, due to the need for ancient books, alchemical equipment, and other supplies for conducting research.

For example, Sir Percy, a 9th-level Fighter, desires to build a 60' tall square keep (50' walls with a 10' peaked slate-shingled roof) that is 50' square. The keep will have four stories and an attic, and the first story, which will contain the great hall, will be 20' high. Sir Percy wishes his keep to be strongly built, so he tells his architect to build with hard stone and use 10' thick walls for the first two stories and 5' thick walls for the rest. The first and second floors will thus be 30' square or 900 square feet, and the third and fourth floors will be 40' square or 1,600 square feet. With a total floor area of 5,000 square feet, Sir Percy's keep will house him and up to 24 other people (or animals such as horses, which during an attack may be stabled in the great hall!) in acceptable comfort. Its floor plans are shown on the next page.

The first floor has 30 (= 5 {[}for 50' length{]} x 2 {[}for 20' height{]} x 4 walls, minus 8 sections double-counted at the corners and 2 sections for the entrance) 10' square sections of 10' thick hard stone walls, which cost 7,800 gp, and 9 10' square sections of floor, which cost 90 gp, for a total cost of 7,890 gp. The second floor is the same as the first, except that the walls are 10' high and there is no deduction for an entrance, giving a cost of 4,250 gp. The third and fourth floors each require 18 sections of 5' thick hard stone walls, costing 1,620 gp, and 16 sections of floor, costing 160 gp, for a total of 1,780 gp per floor. The 50' square roof costs 4 x 25 x 10 = 1,000 gp, and the 40' square attic floor adds 160 gp. The design calls for a total of 770' of 1' thick interior walls and doors, which would cost 30,800 gp if made of hard stone; Sir Percy uses wood, which costs only 7,700 gp. These costs total 24,560 gp, but since the keep is 60' high, its cost is increased by 60\% to 39,296 gp. The keep will require 39,296 worker-days. Sir Percy may employ up to 198 workers to build the keep, in which case it will take 198 working days to build, or a year and three months' time. Keep in mind what might happen in this time, given that the area is dangerous enough to warrant building a castle.

\textbf{Dungeons: }A stronghold may also have a dungeon excavated under it. A dungeon is an excellent place to store perishable supplies, a good shelter if the castle is overrun, and often incorporates an escape route if all is lost for the castle's defenders or a secret way out for when a raid is desired. Magic-Users sometimes encourage monsters to take up residence in their dungeons, as they provide a convenient source of supplies for magical research and help keep away unwanted guests. Use the following figures for skilled workers, such as dwarves or goblins, to create dungeons; double the times for less skilled miners.\medskip

\begin{tabularx}{0.45\textwidth}{lX}
\textbf{Material} & \textbf{Time for one worker to excavate a 5' cube} \\\toprule
{Earth} & {5 days (supports are required)} \\\hline
{Soft stone} & {10 days} \\\hline
{Hard stone} & {20 days} \\\bottomrule
\end{tabularx}\medskip

\textbf{Structural strength and breaches:} A section of stronghold wall has as many hit points as its base cost in gp (for example, a section of 10' thick soft stone wall has 200 hit points). Stone and brick walls only take damage from crushing blows, while wood walls are also affected by fire and chopping attacks. If a given section of wall loses all of its hit points, it is breached, allowing attackers to pass through. If a breach occurs on a lower course of wall, there is a 40\% chance that the 10' section above it will be breached by collapse, and a 20\% chance that the section below it will be breached. These secondary breaches have the same chances of affecting the next 10' section above or below them, and so on until the top or bottom course of wall is reached. If a breach occurs on a right or acute corner (90 degrees or less), the chances of breaches double in each direction.

\textbf{Attacking a Castle: } Siege engines (as listed on page \hyperlink{siege-engines}{\pageref{siege-engines}}) are difficult to aim, but as castles don' t dodge around, each successive shot by a given siege engine with a given crew has an increasing chance of hitting. To reflect this, the first attack on a castle' s walls is made against Armor Class 20; each subsequent attack by that weapon, fired by that crew, at that same point in the wall, is made against an Armor Class one lower than the previous shot, to a minimum AC of 11.

Attacks on a castle' s defenders are at -4 on the attack roll if they are standing on the parapets, and at -10 if they are behind arrow slits. Since characters defending the castle do move around, the odds of hitting them with a siege engine do not improve from shot to shot. There is an additional -2 on the attack roll for missile attacks if the defenders are more than 20' higher than the attackers; this is not specifically due to altitude, but rather because the defenders can use more of the wall for cover. The defenders can take advantage of their height by dropping objects on attackers near the castle' s base; these missiles do 2d10 points of damage, but they have a -2 attack penalty if dropped from a height of 30' or more.

Siege engines can damage several adjacent characters; roll damage separately for each character in the 10' square hit by the missile. Of course, the attack roll must be high enough to damage each one; a roll of 19 against characters having Armor Classes of 18 and 20 would hit the former but not the latter.

A castle may also be attacked by \textbf{mining}. This method of attack involves tunneling under the castle wall, then setting fire to the supports of the tunnel to cause the wall to collapse. It is also slow, and if the castle has a moat, the tunnel must avoid it; this requires that it be dug deeper, requiring twice the time. A mine is dug like a dungeon, and once its supports are fired, the wall above is breached; if the mine is only 5' wide, there is only a 50\% chance of causing a breach.

Finally, a \textbf{screw} may be used to attack a stronghold. This device is used to bore through castle walls. A crew of at least eight is required to operate it. It is only used at the base of a wall, and it is usually operated under a \textbf{sow}, or portable roof, as it is slow. The device does 1d8 points of damage per turn, but it ignores hardness. A breach caused by a screw is small, so it has only half the usual chance of spreading to the next course of wall, unless widened by miners. As a defensive structure, a sow has hardness 12 and 200 hit points. 
\end{multicols}

\pagebreak

\begin{center}
	\includegraphics[width=1\textwidth]{Pictures132/100000000000079E00000735667EF20DA4FFA222.png} 
\end{center}

\pagebreak

\subsection{Getting Started with Basic Fantasy RPG}\label{getting-started-with-basic-fantasy-rpg}\index{Getting Started with Basic Fantasy RPG}

So now you' ve read the rules, and they sound like fun;
what next? This page will hopefully get you started.

\subsection{Starting Adventures}\label{starting-adventures}\index{Starting Adventures}

As a Game Master you may of course create your own adventures for your players; in fact, adventure creation takes less time for Basic Fantasy RPG than for many more modern games. But not everyone feels confident or has the time to create their ownadventures. But, you' re  in luck! The Basic Fantasy Project has a large and varied collection of adventures you can grab from the website. Here are just a few of them to get you started:

\textbf{BF1 Morgansfort: The Western Lands Campaign}

A small campaign area centered around Morgansfort, a detailed ``home base'' for a starting adventuring party along with surrounding multi-level dungeons to explore.

\textbf{JN1 The Chaotic Caves}

\begin{wrapfigure}{c}{0.2\textwidth}
	 \includegraphics[width=0.3\textwidth]{Pictures132/100000000000017200000172FA7D8F240ED536AA.png}  
\end{wrapfigure}


An excellent traditional adventure setting consisting of a group of monster-infested caves and an abandoned manor-fort, for beginning player characters.

\textbf{DC1 Tales From the Laughing Dragon} :A starter module for a group of beginning player characters, presented as three sequential adventures set in the Dragonclaw Barony.

\textbf{KH1 The Blackapple Brugh}: Blackapple is a small, remote village on the edge of the woods that provides a good base town for low level player characters and hub for adventures involving several encounters and mini-adventures around town as well as one involving a mysterious cruel elf lord of the Brugh itself.

\textbf{The Adventure Anthology Series}: Our Adventure Anthology series of books contain many varied short adventures for characters of various levels. For those just starting out with Basic Fantasy RPG, here are a few suggestions from the AA series based on responses in our forum and our Facebook group:

\textbf{Gold in the Hills (AA1)} - \textbf{Beneath Brymassen (AA1)} -  \textbf{Kidnapped (AA2)} - \textbf{The Bear Dungeon (AA2)}


\subsection{Community}\label{community}\index{Community}

There are a few ways to connect with the Basic Fantasy RPG community if you have any questions or need help with your game. Links to all our official resources can be found on our main website:

\href{www.basicfantasy.org}{www.basicfantasy.org}

\textbf{The Forum:} The Forum can be found by following the link at the top of the website home page. We do all development of Basic Fantasy RPG game materials on the forum, in the area called the Workshop; questions are welcome in General Discussion, and artists who wish to contribute are encouraged to visit the Artwork subforum.

\textbf{A note, to make things easier:} When you join the forum, you' ll be asked a question about a spell you shouldn' t use when fighting a flesh golem. The answer to the question is \textbf{lightning bolt}. Now you don' t have to go back and look it up!

The Basic Fantasy Project also maintains a \textbf{Discord} server and a \textbf{Facebook} group; while official game materials aren' t generally created or shared on those sites, they can be great places to ask questions or share "war stories." Links for both can be found on the home page.

\subsection{Helpful Supplements}\label{helpful-supplements}\index{Helpful Supplements}\index{Helpful Supplements}

Several of our other books and supplements can be very helpful in running your game. All can be found on the \textbf{Downloads} page, linked from the top of the home page, including the Core Rules (this book) as well as handy download-only items. Recommended items include:

The \textbf{Beginner's Essentials} book, a stripped-down rules booklet for players. It' s great for beginners, and is available in many languages!

\textbf{The Basic Fantasy Field Guide \textbf{Series}:} These books contain many additional monsters to supplement those found in this book. 

\textbf{Character Sheets:} We have too many to list, starting with our "Standard" character sheet. Heck, download them all and pick whichever one suits you and your players best!

\pagebreak
\printindex

\pagebreak

\end{document}